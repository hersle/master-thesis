\appendix

\chapter{General relativity}

\section{Derivation of Einstein's field equations from Hilbert action}
\label{sec:einstein_derivation}

Hilbert action
\begin{equation}
	S = \int \dif^n x \sqrt{-g} R
\end{equation}
Will set $\variation{S} = \int \dif^n x \sqrt{-g} ( \ldots ) \variation{g^{\mu \nu}} = 0$.
$g^{\mu \lambda} g_{\lambda \nu} = \delta^\mu_\nu$, so by the product rule
\begin{equation}
	\variation{g_{\mu \nu}} = -g_{\mu \rho} g_{\nu \sigma} \variation{g^{\rho \sigma}} ,
\end{equation}
meaning stationary points of upper/lower indices are the same.
Using $R = R\indices{^\mu_\mu} = g^{\mu \nu} R_{\mu \nu}$ and varying the action, we get
\begin{equation}
\begin{split}
	\variation{S} &= \int \dif^n x \sqrt{-g} g^{\mu \nu} \variation{R_{\mu \nu}} \\
	              &+ \int \dif^n x \sqrt{-g} R_{\mu \nu} \variation{g^{\mu \nu}} \\
	              &+ \int \dif^n x R \variation{\sqrt{-g}} \\
\end{split}
\end{equation}
The Ricci tensor $R_{\mu \nu}$ is the contraction of the Riemann tensor
\begin{equation}
	R\indices{^\rho_\mu_\lambda_\nu} = \partial_\lambda \Gamma^{\rho}_{\nu \mu} + \Gamma^\rho_{\lambda \sigma} \Gamma^\sigma_{\nu \mu} - (\lambda \leftrightarrow \nu)
\end{equation}
The variation of the connection $\variation{\Gamma^\rho_{\nu \mu}}$ transforms as a tensor, since it is the difference between two connections (SOURCE).
\begin{equation}
	\nabla_\lambda \variation{\Gamma^\rho_{\nu \mu}} = \partial_\lambda \variation{\Gamma^\rho_{\nu \mu}} 
	                                                 + \Gamma^\rho_{\lambda \sigma} \variation{\Gamma{^\sigma_{\nu \mu}}} 
	                                                 - \Gamma^\sigma_{\lambda \nu} \variation{\Gamma{^\rho_{\sigma \mu}}} 
	                                                 - \Gamma^\sigma_{\lambda \mu} \variation{\Gamma{^\rho_{\nu \sigma}}} 
\end{equation}
This gives
\begin{equation}
	\variation{R\indices{^\rho_\mu_\lambda_\nu}} = \nabla_\lambda \variation{\Gamma^\rho_{\nu \mu}}
	                                             - \nabla_\nu \variation{\Gamma^\rho_{\lambda \mu}}
\end{equation}
With this, we get
\begin{equation}
\begin{split}
	\variation{S}_1  = \int \dif^n x \sqrt{-g} g^{\mu \nu} \left( \nabla_\lambda \variation{\Gamma^\lambda_{\mu \nu}} - \nabla_\nu \variation{\Gamma^\lambda_{\lambda \mu}} \right) \\
	                &= \int \dif^n x \sqrt{-g} \nabla_\sigma \left( g^{\mu \nu} \variation{\Gamma^\sigma_{\mu \nu}} - g^{\mu \sigma} \variation{\Gamma^\lambda_{\lambda \mu}} \right) \\
\end{split}
\end{equation}
From definition of $\Gamma$ (DEF), we get
\begin{equation}
	\variation{\Gamma^\sigma_{\mu \nu}} = -\frac{1}{2} \left( 
		g_{\lambda \mu} \nabla_\nu \variation{g^{\lambda \sigma}} +
		g_{\lambda \nu} \nabla_\mu \variation{g^{\lambda \sigma}} -
		g_{\mu \alpha} g_{\nu \beta} \nabla^\sigma \variation{g^{\alpha \beta}}
	\right)
\end{equation}
With all of this, we get
\begin{equation}
	\variation{S}_1 = \int \dif^n x \sqrt{-g} \nabla_\sigma \left( g_{\mu \nu} \nabla^\sigma \variation{g^{\mu \nu}} - \nabla_\lambda \variation{g^{\sigma \lambda}} \right) = 0
\end{equation}
by Stokes theorem!

Matrix identity:
\begin{equation}
	\log{\det{M}} = \trace{\log{M}}
\end{equation}
Variation of it gives
\begin{equation}
	\frac{\variation{\det{M}}}{\det{M}} = \trace{M^{-1} \variation{M}}
\end{equation}
Taking $M = g$, we get
\begin{equation}
	\variation{\sqrt{-g}} = \frac{-1}{2 \sqrt{-g}} \variation{g} = \frac{-1}{2} \sqrt{-g} g_{\mu \nu} \variation{g^{\mu \nu}}
\end{equation}
Thus, $\variation{S}_3 = \ldots$, and
\begin{equation}
	\variation{S} = \int \dif^n x \sqrt{-g} \left( R_{\mu \nu} - \frac{1}{2} R g_{\mu \nu} \right) \variation{g^{\mu \nu}} = 0
\end{equation}
gives Einstein's field equations in vacuum
\begin{equation}
	 R_{\mu \nu} - \frac{1}{2} R g_{\mu \nu} = 0
\end{equation}

To get Einstein's field equations in non-vacuum, we add a matter action $S_M$:
\begin{equation}
	S \rightarrow \frac{1}{16 \pi G} S + S_M
\end{equation}
where we normalize the action to get the right result.
This gives
\begin{equation}
	\frac{1}{\sqrt{-g}} \frac{\variation{S}}{\variation{g^{\mu \nu}}} = \frac{1}{16 \pi G} \left( R_{\mu \nu} - \frac{1}{2} R g_{\mu \nu} \right) + \frac{1}{\sqrt{-g}} \frac{\variation{S_m}}{\variation{g^{\mu \nu}}} = 0
\end{equation}
Defining the energy-momentum tensor
\begin{equation}
	T_{\mu \nu} = \frac{-1}{2} \frac{1}{\sqrt{-g}} \frac{\variation{S_M}}{\variation{g^{\mu \nu}}} ,
\end{equation}
we get Einstein's field equations in non-vacuum,
\begin{equation}
	R_{\mu \nu} - \frac{1}{2} R g_{\mu \nu} = 8 \pi G T_{\mu \nu}
\end{equation}

\section{Summary of important quantities}
\label{chap:gr_summary} % TODO: chap -> sec

TODO: derive from Einstein-Hilbert action?

In general relativity, the geometry of spacetime is described by the \textbf{metric} $g\indices{_\mu_\nu}$ and the \textbf{line element}
\begin{equation}
	\dif s^2 = g\indices{_\mu_\nu} \dif x^\mu \dif x^\nu .
	\label{eq:def_line_elem}
\end{equation}

From the metric, we can construct the \textbf{Christoffel symbols}
\begin{equation}
	\Gamma^\sigma_{\mu \nu} = \frac{1}{2} g\indices{^\sigma^\epsilon} \left(
		\partial\indices{_\mu} g\indices{_\nu_\epsilon} +
		\partial\indices{_\nu} g\indices{_\epsilon_\mu} +
		\partial\indices{_\epsilon} g\indices{_\mu_\nu}
	\right) .
	\label{eq:def_christoffel}
\end{equation}

They allow us to generalize the partial derivative of a contravariant vector $V^\nu$ or a covariant vector $V_\nu$ to the \textbf{covariant derivative}
\begin{equation*}
	\nabla_\mu V^\nu = \partial_\mu V^\nu + \Gamma_{\sigma \mu}^\nu V^\sigma
	\qquad \text{or} \qquad
	\nabla_\mu V_\nu = \partial_\mu V_\nu - \Gamma_{\nu \mu}^\sigma V_\sigma
	.
\end{equation*}
For a general type $(r,s)$ tensor $T^{\alpha_1 \ldots \alpha_r}_{\beta_1 \ldots \beta_s}$ (where we suppress the order of the indices),
\begin{equation}
\begin{split}
	\nabla_\mu T^{\alpha_1 \ldots \alpha_r}_{\beta_1 \ldots \beta_s} &= \partial_\mu T^{\alpha_1 \ldots \alpha_r}_{\beta_1 \ldots \beta_s} \\
	                                                                 &+ \Gamma^{\alpha_1}_{\sigma\mu} T^{\sigma \alpha_2 \ldots \alpha_r}_{\beta_1 \ldots \beta_s} + \dots + \Gamma^{\alpha_r}_{\sigma\mu} T^{\alpha_1 \ldots \alpha_{r-1}\sigma}_{\beta_1 \ldots \beta_s} \\
	                                                                 &- \Gamma^\sigma_{\beta_1 \mu} T^{\alpha_1 \ldots \alpha_r}_{\sigma \beta_2 \ldots \beta_s} - \cdots - \Gamma^\sigma_{\beta_s \mu} T^{\alpha_1 \ldots \alpha_r}_{\beta_1 \ldots \beta_{s-1} \sigma}.
	\label{eq:def_cov_deriv}
\end{split}
\end{equation}
\iffalse
\begin{align}
	\nabla_c T\indices{^{a_1 \ldots a_r}_{b_1 \ldots b_s}} &= \partial_c {T^{a_1 \ldots a_r}}_{b_1 \ldots b_s} \\
	                                                       &+ \Gamma^{a_1}_{dc} T\indices{^{d a_2 \ldots a_r}_{b_1 \ldots b_s}} + \dots + \Gamma^{a_r}_{dc} T\indices{^{a_1 \ldots a_{r-1}d}_{b_1 \ldots b_s}} \\
	                                                       &- {\Gamma^d}_{b_1 c} {T^{a_1 \ldots a_r}}_{d b_2 \ldots b_s} - \cdots - {\Gamma^d}_{b_s c} {T^{a_1 \ldots a_r}}_{b_1 \ldots b_{s-1} d}.
	\label{eq:def_cov_deriv}
\end{align}
\fi
That is, for each upper index $\alpha_i$, add $+\Gamma^{\alpha_i}_{\sigma \mu} T^{\alpha_1 \ldots \alpha_{i-1} \sigma \alpha_{i+1} \ldots \alpha_r}_{\beta_1 \ldots \beta_s}$,
and for each lower index $\beta_i$, add $-\Gamma^{\sigma}_{\beta_i \mu} T^{\alpha_1 \ldots \alpha_r}_{\beta_1 \ldots \beta_{i-1} \sigma \beta_{i+1} \ldots \beta_s}$,
As the name and notation suggests, $\nabla_\mu$ transforms covariantly, so the covariant derivative of a tensor is independent of coordinate system.

The curvature of spacetime is expressed through the \textbf{Riemann curvature tensor}
\begin{equation}
	R\indices{^\epsilon_\sigma_\mu_\nu} =
	\partial\indices{_\mu} \Gamma^\epsilon_{\nu \sigma} -
	\partial\indices{_\nu} \Gamma^{\epsilon}_{\mu \sigma} +
	\Gamma^\epsilon_{\mu \lambda} \Gamma^{\lambda}_{\nu \sigma} -
	\Gamma^\epsilon_{\nu \lambda} \Gamma^{\lambda}_{\mu \sigma} .
	\label{eq:def_riemann_tensor}
\end{equation}

The curvature tensor can be contracted to form the \textbf{Ricci tensor}
\begin{equation}
	R\indices{_\mu_\nu} = R\indices{^\lambda_\mu_\lambda_\nu},
	\label{eq:def_ricci_tensor}
\end{equation}
whose trace is known as the \textbf{Ricci scalar}
\begin{equation}
	R = R\indices{^\mu_\mu} .
	\label{eq:def_ricci_scalar}
\end{equation}
For more details, consult an introductory textbook on general relativity like \cite{ref:carroll}.

\chapter{Code}

\section{Derivation of the Tolman-Oppenheimer-Volkoff equation \texorpdfstring{\\}{} without using energy-momentum conservation}
\label{sec:tov_cas_derivation}

When deriving \cref{eq:tov} analytically, we made use of energy-momentum conservation $\nabla_\mu T\indices{^\mu^\nu} = 0$ instead of substituting our results into the unused \cref{eq:einstein_to_tov:thetatheta}.
Here, we do the latter in the computer algebra system SAGE.

\inputminted{python}{../code/einstein_to_tov/ein.sage}

The output matches \cref{eq:tov} precisely.
