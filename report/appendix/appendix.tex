\appendix

\chapter{General relativity}

\section{Least-action derivation of the Einstein field equations}
\label{sec:einstein_derivation}

Following \cite[section 4.3]{ref:carroll}, we will derive the Einstein field equations
\begin{equation}
	R_{\mu \nu} - \frac{1}{2} R g_{\mu \nu} = \frac{8 \pi G}{c^4} T_{\mu \nu}
\end{equation}
from the principle of least action.
We will \emph{postulate} the action
\begin{equation}
	S[g_{\mu \nu}, \nabla_\sigma g_{\mu \nu}] = \int \dif^n x \lagr(g_{\mu \nu}, \nabla_\sigma g_{\mu \nu})
	                                          = \int \dif^n x \sqrt{-\det{g}} \hat{\lagr}(g_{\mu \nu}, \nabla_\sigma g_{\mu \nu})
\end{equation}
that, when varied with respect to the metric $g_{\mu \nu}$ and subject to the principle of least action $\variation{S} = 0$, yields the Einstein field equations.
As the strategy simply involves \emph{guessing} the correct action that produces the desired equations, this derivation is not based on any physical first principles, so its consequences would ultimately have to be experimentally verified.
Nevertheless, \cite[page 160-161]{ref:carroll} explains how one can at the very least narrow down the choice of action based on scalar quantities that are relevant for describing curved space.

We postulate the \textbf{Hilbert action}
\begin{equation}
	% i have x = (ct, x1, x2, x3), so I have a c "already" in the first component
	% this is normal! see e.g. https://physics.stackexchange.com/a/322055/299916
	% remember [R] = 1/m^2
	S_H = \frac{c^3}{16 \pi G} \int \dif^n x \sqrt{-\det{g}} \, R .
	\label{eq:einstein_derivation:hilbert_action}
\end{equation}
Had it been written directly in terms of the metric and its covariant derivatives, we could get the corresponding equation of motion by simply plugging it into the Euler-Lagrange equations
\begin{equation}
	\diffp{\hat{\lagr}}{\phi} - \nabla_\mu \left( \diffp{\hat{\lagr}}{{\left(\nabla_\mu \phi\right)}} \right) = 0 .
\end{equation}
But we are not that fortunate.
Instead, we will vary it with respect to the metric and express the variation in the form 
\begin{equation}
	\variation{S_H} = \int \dif^n x \sqrt{-\det{g}} F(g_{\mu \nu}, \nabla_\sigma g_{\mu \nu}) \, \variation{g^{\mu \nu}} = 0 .
	\label{eq:einstein_derivation:action_form}
\end{equation}
Then we can conclude that the equation of motion is $F(g_{\mu \nu}, \nabla_\sigma g_{\mu \nu}) = 0$.

It may sound more natural to express the variation in terms of the ordinary metric $g_{\mu \nu}$ instead of its inverse $g^{\mu \nu}$, like we did above.
But since $g^{\mu \lambda} g_{\lambda \nu} = \delta^\mu_\nu$, varying both sides with the product rule gives
\begin{equation}
	\variation{g_{\mu \nu}} = -g_{\mu \rho} g_{\nu \sigma} \variation{g^{\rho \sigma}} ,
	\label{eq:einstein_derivation:var_g_ginv}
\end{equation}
expressing the variations of the metric and its inverse in terms of each other.
Thus, the stationary points are the same regardless of which one we vary with respect to.
We vary with respect to the inverse metric, as it makes the derivation flow a little more naturally.

Using $R = R\indices{^\mu_\mu} = g^{\mu \nu} R_{\mu \nu}$ and varying the action \eqref{eq:einstein_derivation:hilbert_action} with the product rule, we get
\begin{equation}
	\variation{S_H} = \frac{c^3}{16 \pi G} \left(
	                  \underbrace{\int \dif^n x \sqrt{-\det{g}} \, g^{\mu \nu} \variation{R_{\mu \nu}}}_{\textstyle \variation{S}_1}
	                + \underbrace{\int \dif^n x \sqrt{-\det{g}} \, R_{\mu \nu} \variation{g^{\mu \nu}}}_{\textstyle \variation{S}_2}
	                + \underbrace{\int \dif^n x \, R \, \variation{\sqrt{-\det{g}}}                      }_{\textstyle \variation{S}_3}
					\right) .
%\begin{split}
%	                                                                                               \variation{S}_1 &= \int \dif^n x \sqrt{-\det{g}} g^{\mu \nu} \variation{R_{\mu \nu}} \\
%	\variation{S} = \variation{S}_1 + \variation{S}_2 + \variation{S}_3 , \quad \text{where} \quad \variation{S}_2 &= \int \dif^n x \sqrt{-\det{g}} R_{\mu \nu} \variation{g^{\mu \nu}} \\
%	                                                                                               \variation{S}_3 &= \int \dif^n x R \variation{\sqrt{-\det{g}}} \\
%\end{split}
%\begin{split}
%	\variation{S} &= \int \dif^n x \sqrt{-\det{g}} g^{\mu \nu} \variation{R_{\mu \nu}} \\
%	              &+ \int \dif^n x \sqrt{-\det{g}} R_{\mu \nu} \variation{g^{\mu \nu}} \\
%	              &+ \int \dif^n x R \variation{\sqrt{-\det{g}}} \\
%\end{split}
	\label{eq:einstein_derivation:ds_split}
\end{equation}
The second term $\variation{S}_2$ is already on the desired form \eqref{eq:einstein_derivation:action_form}, but we must do some work to bring $\variation{S}_1$ and $\variation{S}_3$ to the same form.

% TODO: latex package glossary?

First, let us take care of $\variation{S}_1$ by reexpressing $\variation{R_{\mu \nu}}$ in terms metric variations in a top-down manner.
The Ricci tensor $R_{\mu \nu} = R\indices{^\lambda_\mu_\lambda_\nu}$ is the contraction of the Riemann tensor \eqref{eq:def_riemann_tensor}.
Varying it, we get
% TODO: do more intelligently by writing (\mu <-> \nu), etc.
\begin{equation}
	\variation{R\indices{^\rho_\sigma_\mu_\nu}} = \partial_\mu \variation{\Gamma^\rho_{\nu \sigma}}
	                                            - \partial_\nu \variation{\Gamma^\rho_{\mu \sigma}}
												+ \left(\variation{\Gamma^\rho_{\mu \lambda}}\right) \Gamma^\lambda_{\nu \sigma}
												+ \Gamma^\rho_{\mu \lambda} \left(\variation{\Gamma^\lambda_{\nu \sigma}\right)}
												- \left(\variation{\Gamma^\rho_{\nu \lambda}}\right) \Gamma^\lambda_{\mu \sigma}
												- \Gamma^\rho_{\nu \lambda} \left(\variation{\Gamma^\lambda_{\mu \sigma}\right)} .
	\label{eq:einstein_derivation:var_riemann}
\end{equation}
Now reexpress the variations of the Christoffel symbols.
Instead of hammering straight through their definition \eqref{eq:def_christoffel}, we observe that while single Christoffel symbols do not transform as a tensor, their \emph{variation} is the difference between two Christoffel symbols and \emph{do}. \cite[page 96,98]{ref:carroll}
It is therefore meaningful to take its covariant derivative using \cref{eq:def_cov_deriv}, giving
\begin{equation}
	\nabla_\lambda \variation{\Gamma^\rho_{\nu \mu}} = \partial_\lambda \variation{\Gamma^\rho_{\nu \mu}} 
	                                                 + \Gamma^\rho_{\lambda \sigma} \variation{\Gamma{^\sigma_{\nu \mu}}} 
	                                                 - \Gamma^\sigma_{\lambda \nu} \variation{\Gamma{^\rho_{\sigma \mu}}} 
	                                                 - \Gamma^\sigma_{\lambda \mu} \variation{\Gamma{^\rho_{\nu \sigma}}} .
	\label{eq:einstein_derivation:christoffel_cov_deriv}
\end{equation}
Flipping this equation around for $\partial_\lambda \variation{\Gamma^\rho_{\nu \mu}}$ and substituting it into the variation of the Riemann tensor \eqref{eq:einstein_derivation:var_riemann}, we witness an avalance of cancellations, leaving only the two terms
\begin{equation}
	\variation{R\indices{^\rho_\mu_\lambda_\nu}} = \nabla_\lambda \variation{\Gamma^\rho_{\nu \mu}}
	                                             - \nabla_\nu \variation{\Gamma^\rho_{\lambda \mu}} .
\end{equation}
The Ricci tensor follows by contracting $\rho$ and $\lambda$. 
Then the first term in the variation of the action becomes
\begin{equation}
\begin{split}
	\variation{S}_1 &= \int \dif^n x \sqrt{-\det{g}} \, g^{\mu \nu} \left( \nabla_\lambda \variation{\Gamma^\lambda_{\mu \nu}} - \nabla_\nu \variation{\Gamma^\lambda_{\lambda \mu}} \right) \\
	                &= \int \dif^n x \sqrt{-\det{g}} \, \nabla_\sigma \left( g^{\mu \nu} \variation{\Gamma^\sigma_{\mu \nu}} - g^{\mu \sigma} \variation{\Gamma^\lambda_{\lambda \mu}} \right) . \\
	\label{eq:einstein_derivation:ds1_intermediate}
\end{split}
\end{equation}
To bring the metric variation into play, we vary the Christoffel symbols from their definition \eqref{eq:def_christoffel} with the product rule.
Converting metric variations to inverse metric variations with \eqref{eq:einstein_derivation:var_g_ginv} and recognizing covariant derivatives from \cref{eq:def_cov_deriv}, we find
\begin{equation}
	\variation{\Gamma^\sigma_{\mu \nu}} = -\frac{1}{2} \left( 
		g_{\lambda \mu} \nabla_\nu \variation{g^{\lambda \sigma}} +
		g_{\lambda \nu} \nabla_\mu \variation{g^{\lambda \sigma}} -
		g_{\mu \alpha} g_{\nu \beta} \nabla^\sigma \variation{g^{\alpha \beta}}
	\right) .
\end{equation}
Inserting this into our expression for $\variation{S}_1$, we get
\begin{equation}
	\variation{S}_1 = \int \dif^n x \sqrt{-\det{g}} \, \nabla_\sigma \left( g_{\mu \nu} \nabla^\sigma \variation{g^{\mu \nu}} - \nabla_\lambda \variation{g^{\sigma \lambda}} \right).
\end{equation}
Our hard work has paid off -- by \textbf{Stokes theorem} \cite[equation 3.35]{ref:carroll}
\begin{equation}
	\int_M \dif^n x \sqrt{\abs{g}} \nabla_\mu V^\mu = \int_{\partial M} \dif^{n-1} \sqrt{\abs{\gamma}} n_\mu V^\mu ,
\end{equation}
our integral for $\variation{S}_1$ over $4$-space can be converted into a boundary integral over $3$-space at infinity.
But the variational method that we have used here asserts that there is no variation at the boundary, so
\begin{equation}
	\variation{S}_1 = 0 !
\end{equation}
(SPØRSMÅL: kan jeg ikke si dette allerede etter \eqref{eq:einstein_derivation:ds1_intermediate}?)

Let us now express $\variation{S}_3$ in terms of $\variation{g^{\mu \nu}}$.
We will need the matrix identity
\begin{equation}
	\log{\Big( \det{M} \Big)} = \trace{\Big(\log{\big(M\big)}\Big)} .
	\label{eq:matrix_log_det_trace}
\end{equation}
This is trivial for diagonal matrices $M$.
By using the property $\det{AB} = \det{A} \det{B}$, we can easily extend it to diagonalizable matrices $M = P D P^{-1}$.
Varying both sides of \cref{eq:matrix_log_det_trace}, we get \cite{ref:matrix_ln_det_tr_exercise}
\begin{equation}
	\frac{\variation{\det{M}}}{\det{M}} = \trace(M^{-1} \variation{M}) .
\end{equation}
Taking $M$ to be the metric $g_{\mu \nu}$ and $M^{-1}$ its inverse $g^{\mu \nu}$, we find
\begin{equation}
	\variation{\det{g}} = \det{g} g^{\mu \nu} \variation{g_{\mu \nu}} = -\det{g} g_{\mu \nu} \variation{g^{\mu \nu}} ,
\end{equation}
where we used \cref{eq:einstein_derivation:var_g_ginv} to convert $\variation{g_{\mu \nu}}$ to $\variation{g^{\mu \nu}}$.
Now the chain rule gives
\begin{equation}
	\variation{\sqrt{-\det{g}}} = -\frac{1}{2} \frac{\variation{\det{g}}}{\sqrt{-\det{g}}} = -\frac{1}{2} \sqrt{-\det{g}} g_{\mu \nu} \variation{g^{\mu \nu}},
\end{equation}
so the third contribution to the variation of the action \eqref{eq:einstein_derivation:ds_split} is
\begin{equation}
	\variation{S}_3 = \int \dif^n x \sqrt{-\det{g}} \left( \frac{-1}{2} R g_{\mu \nu} \right) \variation{g^{\mu \nu}} .
\end{equation}

At last, we have brought the variation of the action to the form \eqref{eq:einstein_derivation:action_form} with
\begin{equation}
	\variation{S_H} = \frac{c^3}{16 \pi G} \int \dif^n x \sqrt{-\det{g}} \left( R_{\mu \nu} - \frac{1}{2} R g_{\mu \nu} \right) \variation{g^{\mu \nu}} = 0 .
\end{equation}
The variation of the integral can only vanish if the integrand vanishes, so we have found the \textbf{Einstein field equations in vacuum},
\begin{equation}
	 \frac{c^3}{16 \pi G} \frac{1}{\sqrt{-\det{g}}} \diffv{S_H}{g^{\mu \nu}} = R_{\mu \nu} - \frac{1}{2} R g_{\mu \nu} = 0 .
\end{equation}
To get the Einstein field equations in matter, we add another contribution $S_M$ to the action that represents matter, so the total action is
\begin{equation}
	S = S_H + S_M .
\end{equation}
Repeating the same procedure as above, we get the equation of motion
\begin{equation}
	\frac{c^3}{16 \pi G} \frac{1}{\sqrt{-\det{g}}} \diffv{S}{g^{\mu \nu}} = \left( R_{\mu \nu} - \frac{1}{2} R g_{\mu \nu} \right) + \frac{c^3}{16 \pi G} \frac{1}{\sqrt{-\det{g}}} \frac{\variation{S_m}}{\variation{g^{\mu \nu}}} = 0 .
\end{equation}
If we now \emph{define} the energy-momentum tensor
\begin{equation}
	T_{\mu \nu} = \frac{-c}{2 \sqrt{-\det{g}}} \frac{\variation{S_M}}{\variation{g^{\mu \nu}}} ,
\end{equation}
we get the \textbf{Einstein field equations in matter},
\begin{equation}
	R_{\mu \nu} - \frac{1}{2} R g_{\mu \nu} = \frac{8 \pi G}{c^4} T_{\mu \nu} .
\end{equation}

\section{Summary of important quantities}
\label{chap:gr_summary} % TODO: chap -> sec

TODO: rewrite / extend

In general relativity, the geometry of spacetime is described by the \textbf{metric} $g\indices{_\mu_\nu}$ and the \textbf{line element}
\begin{equation}
	\dif s^2 = g\indices{_\mu_\nu} \dif x^\mu \dif x^\nu .
	\label{eq:def_line_elem}
\end{equation}

From the metric, we can construct the \textbf{Christoffel symbols}
\begin{equation}
	\Gamma^\sigma_{\mu \nu} = \frac{1}{2} g\indices{^\sigma^\rho} \left(
		\partial\indices{_\mu} g\indices{_\nu_\rho} +
		\partial\indices{_\nu} g\indices{_\rho_\mu} -
		\partial\indices{_\rho} g\indices{_\mu_\nu}
	\right) .
	\label{eq:def_christoffel}
\end{equation}

They allow us to generalize the partial derivative of a contravariant vector $V^\nu$ or a covariant vector $V_\nu$ to the \textbf{covariant derivative}
\begin{equation*}
	\nabla_\mu V^\nu = \partial_\mu V^\nu + \Gamma_{\sigma \mu}^\nu V^\sigma
	\qquad \text{or} \qquad
	\nabla_\mu V_\nu = \partial_\mu V_\nu - \Gamma_{\nu \mu}^\sigma V_\sigma
	.
\end{equation*}
For a general type $(r,s)$ tensor $T^{\alpha_1 \ldots \alpha_r}_{\beta_1 \ldots \beta_s}$ (where we suppress the order of the indices),
\begin{equation}
\begin{split}
	\nabla_\mu T^{\alpha_1 \ldots \alpha_r}_{\beta_1 \ldots \beta_s} &= \partial_\mu T^{\alpha_1 \ldots \alpha_r}_{\beta_1 \ldots \beta_s} \\
	                                                                 &+ \Gamma^{\alpha_1}_{\sigma\mu} T^{\sigma \alpha_2 \ldots \alpha_r}_{\beta_1 \ldots \beta_s} + \dots + \Gamma^{\alpha_r}_{\sigma\mu} T^{\alpha_1 \ldots \alpha_{r-1}\sigma}_{\beta_1 \ldots \beta_s} \\
	                                                                 &- \Gamma^\sigma_{\beta_1 \mu} T^{\alpha_1 \ldots \alpha_r}_{\sigma \beta_2 \ldots \beta_s} - \cdots - \Gamma^\sigma_{\beta_s \mu} T^{\alpha_1 \ldots \alpha_r}_{\beta_1 \ldots \beta_{s-1} \sigma}.
	\label{eq:def_cov_deriv}
\end{split}
\end{equation}
\iffalse
\begin{align}
	\nabla_c T\indices{^{a_1 \ldots a_r}_{b_1 \ldots b_s}} &= \partial_c {T^{a_1 \ldots a_r}}_{b_1 \ldots b_s} \\
	                                                       &+ \Gamma^{a_1}_{dc} T\indices{^{d a_2 \ldots a_r}_{b_1 \ldots b_s}} + \dots + \Gamma^{a_r}_{dc} T\indices{^{a_1 \ldots a_{r-1}d}_{b_1 \ldots b_s}} \\
	                                                       &- {\Gamma^d}_{b_1 c} {T^{a_1 \ldots a_r}}_{d b_2 \ldots b_s} - \cdots - {\Gamma^d}_{b_s c} {T^{a_1 \ldots a_r}}_{b_1 \ldots b_{s-1} d}.
	\label{eq:def_cov_deriv}
\end{align}
\fi
That is, for each upper index $\alpha_i$, add $+\Gamma^{\alpha_i}_{\sigma \mu} T^{\alpha_1 \ldots \alpha_{i-1} \sigma \alpha_{i+1} \ldots \alpha_r}_{\beta_1 \ldots \beta_s}$,
and for each lower index $\beta_i$, add $-\Gamma^{\sigma}_{\beta_i \mu} T^{\alpha_1 \ldots \alpha_r}_{\beta_1 \ldots \beta_{i-1} \sigma \beta_{i+1} \ldots \beta_s}$,
As the name and notation suggests, $\nabla_\mu$ transforms covariantly, so the covariant derivative of a tensor is independent of coordinate system.

The curvature of spacetime is expressed through the \textbf{Riemann curvature tensor}
\begin{equation}
	R\indices{^\rho_\sigma_\mu_\nu} =
	\partial\indices{_\mu} \Gamma^\rho_{\nu \sigma} -
	\partial\indices{_\nu} \Gamma^{\rho}_{\mu \sigma} +
	\Gamma^\rho_{\mu \lambda} \Gamma^{\lambda}_{\nu \sigma} -
	\Gamma^\rho_{\nu \lambda} \Gamma^{\lambda}_{\mu \sigma} .
	\label{eq:def_riemann_tensor}
\end{equation}

The curvature tensor can be contracted to form the \textbf{Ricci tensor}
\begin{equation}
	R\indices{_\mu_\nu} = R\indices{^\lambda_\mu_\lambda_\nu},
	\label{eq:def_ricci_tensor}
\end{equation}
whose trace is known as the \textbf{Ricci scalar}
\begin{equation}
	R = R\indices{^\mu_\mu} .
	\label{eq:def_ricci_scalar}
\end{equation}
For more details, consult an introductory textbook on general relativity like \cite{ref:carroll}.

\chapter{Code}

\section{Derivation of the Tolman-Oppenheimer-Volkoff equation \texorpdfstring{\\}{} without using energy-momentum conservation}
\label{sec:tov_cas_derivation}

When deriving \cref{eq:tov} analytically, we made use of energy-momentum conservation $\nabla_\mu T\indices{^\mu^\nu} = 0$ instead of substituting our results into the unused \cref{eq:einstein_to_tov:thetatheta}.
Here, we do the latter in the computer algebra system SAGE.

\inputminted{python}{../code/einstein_to_tov/ein.sage}

The output matches \cref{eq:tov} precisely.
