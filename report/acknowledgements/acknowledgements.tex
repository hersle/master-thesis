\chapter*{Disclaimer (to the examiner)}

\iffalse
\vspace{-40pt}
\begin{center}
\tikzsetnextfilename{disclaimer-warning}
\begin{tikzpicture}
	\fill[black] (0, 0) rectangle (\textwidth, 20pt);
	\fill[pattern={Lines[angle=45, line width=15pt, distance=30pt]}, pattern color=yellow] (0, 0) rectangle (\textwidth, 20pt);
\end{tikzpicture}
\end{center}
\fi

%\begin{tcolorbox}[fonttitle=\bfseries, title=Disclaimer regarding evaluation]
This document joins the work of a project and master thesis and is split in three parts:
\begin{enumerate}
\item \Cref{part:project} comprises the project thesis.
\item \Cref{part:master} comprises the master thesis and builds upon the work of \cref{part:project}.
\item \Cref{part:appendices} comprises appendices for both \cref{part:project} and \cref{part:master}.
      Appendices \ref{chap:gr}, \ref{chap:relfluid}, \ref{chap:matsum} and \ref{chap:integrals} are part of the project thesis,
      while appendices \ref{chap:lsm3fpotential} and \ref{chap:code} are part of the master thesis.
\end{enumerate}

\textbf{During examination, only \cref{part:master} and \cref{chap:lsm3fpotential,chap:code} should be considered part of the master thesis.}

This structure has been chosen in order to avoid self-plagiarism by clearly separating the two,
while still maintaining the continuity and progression from \cref{part:project} to \cref{part:master}.
\Cref{part:master} makes use of several results from \cref{part:project} and frequently refers to it for readers who seek more background information,
but such references are always accompanied by short summaries for those who rather wish to read the master thesis independently of the project thesis.

\mbox{}\vfill
\begin{center}
\tikzsetnextfilename{disclaimer-warning}
\begin{tikzpicture}
	\fill[black] (0, 0) rectangle (\textwidth, 20pt);
	\fill[pattern={Lines[angle=45, line width=15pt, distance=30pt]}, pattern color=yellow] (0, 0) rectangle (\textwidth, 20pt);
\end{tikzpicture}
\end{center}

\chapter*{Acknowledgements}

This thesis marks the end of five great years of studying physics at NTNU in Trondheim.

My supervisor, Professor Jens Oluf Andersen,
has regularly given me thoughtful and invaluable feedback on my work.
When I first met him,
I remember that I was a little lost as he sketched the plan for this thesis.
Over the coming two semesters, he guided me through the steps of this plan at an understanding and adaptive pace.
Looking back, I see very clearly how well-thought-out his plan was,
and how he has challenged me to take daunting steps into unknown territory along the way,
but always had realistic expectations of my work.
His friendliness makes him a breeze to cooperate with,
and his flexibility has encouraged me to work under freedom and to steer relevant pieces of the work into my interests and skills.
I am sincerely grateful for the time and effort he has devoted to me!

Last, but not least, I am very grateful to the large group of friends with whom I have spent my time in Trondheim.
Without them life here would be very monotonous.

\null\hfill \textit{-- Trondheim, 16th of May 2022}
