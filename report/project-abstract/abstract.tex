\chapter{Abstract}
%\addcontentsline{toc}{chapter}{Abstract} % but still display in TOC (see https://tex.stackexchange.com/a/222961)

\section*{Project thesis (\cref{part:project})}

General relativity and quantum field theory are indispensable for studying compact stars that are composed of subatomic particles with extreme density.
In this project, we solve the Tolman-Oppenheimer-Volkoff equations for a cold Fermi gas composed of free neutrons, producing a mass-radius curve for ideal neutron stars parametrized by central pressure, then finally analyze their stability.
First, we derive the Tolman-Oppenheimer-Volkoff equations from the Einstein field equations in a radially symmetric metric for a perfect fluid in equilibrium.
Second, we present thermal field theory and use it to express the partition function of a free Fermi gas as a path integral.
Next, we combine these two results by numerically integrating the Tolman-Oppenheimer-Volkoff equations with the equation of state that follows from the partition function, yielding the mass-radius curve.
Finally, we apply perturbation theory to the initial equilibrium analysis of general relativity to find a Sturm-Liouville problem that determines normal radial vibration modes of stars out of equilibrium, then solve it with the shooting method to analyze the stability of the stars.
%Along the way, we review general relativity, discover the Buchdal limit for stars, take a particularly detailed look on fermionic coherent states and demonstrate the technique of Matsubara energy summation.
%\TODO{should I omit the preceeding sentence?}
Our mass-radius curve reproduces the upper mass limit of 0.71 solar masses for neutron stars originally calculated by Oppenheimer and Volkoff in 1939.
Likewise, our quantitative stability analysis confirms the correctness of a set of qualitative rules based on curvature and extrema in the mass-radius diagram.
Many observations have been made of neutron stars around 2 solar masses, so the model is too simple for describing real neutron stars.
Nevertheless, this project establishes a broad and solid base platform from which one can continue to study more advanced models for compact stars.
%for a fundamental understanding of compact stars and continued study of more advanced stellar models.


\section*{Master thesis (\cref{part:master})}

According to quantum chromodynamics,
hadron-confined quarks break free into a state of deconfined quark matter at high density.
Recent observations of the massive $2 M_\odot$-pulsars PSR J1614$-$2230 and PSR J0348$+$0432
suggest that the density in neutron stars could reach sufficiently high levels
for formation of small cores of deconfined quark matter in what is then referred to as hybrid stars.
Even pure quark stars consisting only of deconfined quark matter have been hypothesized, but not decisively observed.
After reviewing the conventional MIT bag model,
we model quark stars using the effective quark-meson model of quantum chromodynamics,
calculating its grand potential to one fermion loop in the mean-field approximation for bosons,
which is consistent in large-$N_c$ approximation scheme.
We find maximum masses $M \leq 2.0 \, M_\odot$ and $M \leq 1.8 \, M_\odot$ with the two-flavor and three-flavor models, respectively.
In particular, we struggle to fit measured masses 
of the model's $\sigma$ meson to its grand potential at tree-level,
but resolve this using recent work of Adhikari and others that consistently fits parameters at one loop level.
Finally, we assemble hybrid stars by joining the quark-meson model with the hadronic Akmal-Pandharipande-Ravenhall equation of state,
generating short branches of stable hybrid stars with plausible maximum masses $1.9 M_\odot \leq M \leq 2.1 M_\odot$
and small two-flavor and three-flavor quark cores around only $0.12 M_\odot$ and $0.02 M_\odot$, respectively.
A discontinuous phase transition destabilizes stars with heavier quark cores.
The results agree with other work using variations of the quark-meson model and Nambu-Jona-Lasinio model.
