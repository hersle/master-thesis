\chapter{Conclusion and outlook}
\label{chap:conclusion}

\TODO{do all of this ``complicated stuff'' in the outlook instead, keep introduction simple and safe?}
Theory: (glendenning p 231+)
\begin{itemize}
\item matter of neutron stars: ``neutron star matter''
\item 1939 pure free neutrons
\item nstar is bound by gravity, not by the strong force, since the density is so great. gravity is weaker, but ``compensates'' by acting on longer range.
\item not pure neutrons. for example, neutrons will beta decay, reaching equilibrium between protons, neutrons and electrons. (5.3.3)
\item neutron star matter is complex. (fig 5.17) low density: mostly neutrons, some protons and electrons. increasing density: hadrons?
\item one improvement: $\sigma-\omega$ model (lagr. 5.6 in caroline)
\item second improvement: scalar self interactions (caroline 5.62)
\item third improvement: isospin force (caroline 5.70)
\item fourth improvement: electrons and muons (caroline 5.94)
\item all together: neutron star matter, Glendenning (5.38)
\item the precise core matter is an unknown and important problem in physics ( see e.g. abstract of review \url{https://arxiv.org/abs/1603.02698v1})
\end{itemize}

guided by fig 5.17:
\begin{enumerate}
\item neutron star is just formed. very hot. suppose there is mostly neutrons. then neutron star quickly cools down.
\item neutrons can beta decay into protons and neutrons (see fig 5.17)
\item nucleons (protons and neutrons) can react to form lambda hyperons and kaons (5.29), kaons can decay into muons, neutrinos etc (5.30), then it can form $\Sigma^-$
\item 
\end{enumerate}
star energy/temperature goes down > fermi energy falls > many reactions become irreversible due to Pauli blocking (reaction cannot occur due to Pauli exclusion principle)
