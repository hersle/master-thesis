\chapter{Thermal field theory}

\newcommand{\transampl}{\braket{\phi_B | e^{- i \hat{H} T} | \phi_A}}

\section{Summary of quantization of quantum fields}

Consider a quantum field theory with Schrödinger-picture field operators $\hat{\phi}(\vec{x})$ and conjugate momenta $\hat{\pi}(\vec{x})$ and Hamiltonian operator
\begin{equation}
	\hat{H} = \int \dif^3 x \, \ham(\hat{\pi}(\vec{x}), \hat{\phi}(\vec{x})) .
\end{equation}
In analogy with position $x$ and momentum $p$ in classical mechanics, we will refer to $\hat\phi(\vec{x})$ and $\hat\pi(\vec{x})$ as operators in ``position-space'' and ``momentum-space'', too.
Whether ``position'' refers to $\vec{x}$ or $\phi(\vec{x})$ will therefore depend on context.

The field operators $\hat{\phi}(\vec{x})$ and $\hat{\pi}(\vec{x})$ have eigenstates $\ket{\phi}$ and $\ket{\pi}$ with corresponding eigenvalues $\phi(\vec{x})$ and $\pi(\vec{x})$ at every point $\vec{x}$, as expressed by the eigenvalue equations
\begin{equation}
	\hat{\phi}(\vec{x}) \ket{\phi} = \phi(\vec{x}) \ket{\phi}
	\qquad \text{and} \qquad
	\hat{\pi}(\vec{x}) \ket{\pi} = \pi(\vec{x}) \ket{\pi} .
\label{eq:tft:field_eigenvalue_equations}
\end{equation}

By assumption, the field and the momentum satisfy the commutation relations
\begin{equation}
	\comm{\hat{\phi}(\vec{x})}{\hat{\pi}(\vec{y})} = i \hbar \delta(\vec{x} - \vec{y})
	\qquad \text{and} \qquad
	\comm{\hat{\phi}(\vec{x})}{\hat{\phi}(\vec{y})} = 
	\comm{\hat{\pi}(\vec{x})}{\hat{\pi}(\vec{y)}} = 
	0
\label{eq:tft:boson_field_commutators}
\end{equation}
for bosons, while for fermions they instead satisfy the anticommutation relations
\begin{equation}
	\acomm{\hat{\phi}(\vec{x})}{\hat{\pi}(\vec{y})} = i \hbar \delta(\vec{x} - \vec{y})
	\qquad \text{and} \qquad
	\acomm{\hat{\phi}(\vec{x})}{\hat{\phi}(\vec{y})} = 
	\acomm{\hat{\pi}(\vec{x})}{\hat{\pi}(\vec{y)}} = 
	0 .
\label{eq:tft:fermion_field_anticommutators}
\end{equation}
% peskin eq. 2.20: in Heisenberg picture, these hold at *equal times*

The position-space eigenstates are orthogonal and complete in the sense
\begin{equation}
	\braket{\phi | \phi'} = \prod_{\vec{x}} \delta(\phi(\vec{x}) - \phi'(\vec{x}))
	\qquad \text{and} \qquad
	\int \dif \phi \ket{\phi} \bra{\phi} = 1 .
	\label{eq:tft:orthogonality_completeness_position}
\end{equation}

\newcommand{\posmom}[2]{\exp \left(  \frac{i}{\hbar} \int \dif^3 x \, #2(\vec{x}) #1(\vec{x}) \right)}
\newcommand{\mompos}[2]{\exp \left( -\frac{i}{\hbar} \int \dif^3 x \, #1(\vec{x}) #2(\vec{x}) \right)}
If we find the inner product $\braket{\phi | \pi}$, we can use it together with the completeness relation \eqref{eq:tft:orthogonality_completeness_position} to express position-space states and momentum-space states in terms of each other through
\begin{equation}
	\ket\pi = \int \dif \phi \ket\phi \braket{\phi | \pi} %= \int \dif \phi \posmom{\phi}{\pi}
	\qquad \text{or} \qquad
	\ket\phi = \int \dif \pi \ket\pi \braket{\pi | \phi} . %= \int \dif \phi \mompos{\pi}{\phi} .
\end{equation}
To do so, let us use the position-space representation $\hat{\pi} = -i \hbar \fdv{}/{\phi}$ of the momentum operator.
This gives us a first-order differential equation
\begin{equation}
	\braket{\phi | \hat{\pi} | \pi} = \pi(\vec{x}) \braket{\phi | \pi} = \frac{\hbar}{i} \fdv*{\braket{\phi | \pi}}{\phi}
\end{equation}
for the inner product $\braket{\phi | \pi}$.
Choosing the solution with prefactor $1$, we obtain
\begin{equation}
	\braket{\phi | \pi} = \posmom{\phi}{\pi} .
	\label{eq:tft:inner_product_position_momentum}
\end{equation}

The momentum states are also orthogonal and complete, but with slightly different factors.
Due to our Fourier transformation convention (TODO: ref), orthogonality takes the form
\begin{equation}
\begin{split}
	\braket{\pi_a | \pi_b} &= \int \dif \phi \braket{\pi_a | \phi} \braket{\phi | \pi_b} \\
	                       &= \int \dif \phi \exp \left( i \int \dif^3 x \, (\pi_b(\vec{x}) - \pi_a(\vec{x})) \phi(\vec{x}) / \hbar \right) \\
						   &= 2 \pi \hbar \delta(\pi_a(\vec{x}) - \pi_b(\vec{x})) .
\end{split}
\end{equation}
To find the completeness relation, we postulate it up to a constant $B$.
Inserting a complete set of both position and momentum states and using the inner product \eqref{eq:tft:inner_product_position_momentum}, consider
\begin{equation}
\begin{split}
	1 &= \int \frac{\dif \pi(\vec{x})}{B} \ket{\pi} \bra{\pi} \\
	  &= \int \frac{\dif \pi(\vec{x})}{B} \ket{\pi} \int \frac{\dif \pi'(\vec{x})}{B} \int \dif \phi(\vec{x}) \braket{\pi | \phi} \braket{\phi | \pi'} \bra{\pi'} \\
	  &= \int \frac{\dif \pi(\vec{x})}{B} \ket{\pi} \int \frac{\dif \pi'(\vec{x})}{B} \underbrace{\int \dif \phi(\vec{x}) \exp \left( \frac{i}{\hbar} \int \dif^3 x \, \left( \pi'(\vec{x}) - \pi(\vec{x}) \right) \phi(\vec{x}) \right)}_{2 \pi \hbar \, \delta(\pi'(\vec{x}) - \pi(\vec{x}))} \bra{\pi'} \\
	  &= \frac{2 \pi \hbar}{B} \underbrace{\int \frac{\dif \pi(\vec{x})}{B} \ket{\pi} \bra{\pi}}_{1} .
\end{split}
\end{equation}
This would be inconsistent unless $B = 2 \pi \hbar$, so completeness in momentum-space is
\begin{equation}
	\int \frac{\dif \pi(\vec{x})}{2 \pi \hbar} \ket{\pi} \bra{\pi} = 1 .
\end{equation}

\section{Path integral formulation}

(TODO: am I assuming $H$ to be time independent (which is always the case in statistical mechanics)?)

During the time $T$, a system evolves from the state $\ket{\phi_A}$ to the state $e^{-i \hat{H} T} \ket{\phi_A}$. (TODO: cite Sakurai on Hamiltonian being the time evolution operator)
The transition amplitude for going from the state $\ket{\phi_A}$ to a different state $\ket{\phi_B}$ is therefore
\begin{equation}
	\transampl \qquad (A \rightarrow B) .
	\label{eq:tft:transition_amplitude_intro}
\end{equation}
Let us demonstrate how this transition amplitude can be written as a path integral.
First, split the time interval $T$ into $N \gg 1$ tiny intervals $\Delta t = T / N$, and decompose the evolution operator $e^{- i \hat{H} T}$ into equally many products of $e^{- i \hat{H} \Delta t}$ to write
\newcommand\pointarrow[1]{\underset{\underset{\displaystyle #1}{\displaystyle \uparrow}}{}}
\begin{equation}
	\transampl = \braket{\phi_B | e^{- i \hat{H} \Delta t} \cdots e^{- i \hat{H} \Delta t} \cdots e^{- i \hat{H} \Delta t} | \phi_A} .
\end{equation}
%\transampl = \braket{\phi_b | \pointarrow{4} e^{- i H \Delta t} \,\, \cdots \pointarrow{3} e^{- i H \Delta t} \pointarrow{2} \cdots \,\, e^{- i H \Delta t} \pointarrow{1} | \phi_a}
Now take a deep breath and do the following.
\begin{itemize}
\item Insert $N$ complete sets of \emph{momentum} states $1 = \int \dif \pi_n / (2 \pi \hbar) \ket{\pi_n} \bra{\pi_n}$ to the \emph{left} of every exponential, including the rightmost one, with $n$ increasing from right to left.
\item Insert $N-1$ complete sets of \emph{position} states $1 = \int \dif \phi_n \ket{\phi_n} \bra{\phi_n}$ to the \emph{right} of every exponential, except the rightmost one, with $n$ increasing from right to left.
\end{itemize}
Now exhale.
With this trick, the transition amplitude is simply the product
\begin{equation}
	\transampl = \prod_{n=0}^{N} \int \frac{\dif \phi_n \dif \pi_n}{2 \pi \hbar} 
	             \braket{\phi_{n+1} | \pi_n} \braket{\pi_n | e^{- i \hat{H} \Delta t} | \phi_n} ,
\end{equation}
where we have defined $\ket{\phi_0} = \ket{\phi_A}$ and $\ket{\phi_{N+1}} = \ket{\phi_B}$.
The inner products $\braket{\phi_{n+1} | \pi_n}$ can simply be replaced by the exponential \eqref{eq:tft:inner_product_position_momentum}, so let us turn our attention to the matrix elements $\braket{\pi_n | e^{- i \hat{H} \Delta t} | \phi_n}$.
Since the time step $\Delta t$ is assumed small, we can expand the exponential $e^{- i \hat{H} \Delta t} \taylor 1 - i \hat{H} \Delta t$ to first order in time.
Under the assumption that the Hamiltonian $\hat{H}$ is a sum of terms with all \emph{position}-space operators $\hat{\phi}$ on the \emph{right} and all \emph{momentum}-space operators $\hat{\pi}$ on the \emph{left}, we can pull it out of the product at the additional benefit of replacing its operators by their eigenvalues.
Hence, we obtain
\begin{equation}
\begin{split}
	\braket{\pi_n | e^{- i \hat{H} \Delta t} | \phi_n} &\taylor \braket{\pi_n | (1 - i \hat{H} \Delta t) | \phi_n} \\
	                                                   &=       \braket{\pi_n | \phi_n} (1 - i H_n \Delta t) \\
	                                                   &\taylor \braket{\pi_n | \phi_n} e^{- i H_n \Delta t}, \\
\end{split}
\label{eq:tft:path_integral_hamiltonian_assumption}
\end{equation}
where we now have the \emph{scalar} Hamiltonian at the $n$-th timestep
\begin{equation}
	H_n = \int \dif^3 x \, \ham(\pi_n(\vec{x}), \phi_n(\vec{x})) .
\label{eq:tft:hamiltonian_eigenvalues}
\end{equation}
Note the importance of expanding the exponential to first order only.
If the Hamiltonian contained \emph{any} mixed sequence of operators such as $H \propto \hat{\pi} \hat{\phi}$, then higher powers like $\hat{H}^2 \propto \hat{\pi} \hat{\phi} \hat{\pi} \hat{\phi}$ in the power series expansion of the exponential would not be in the correct order.

(TODO: If the Hamiltonian is \emph{not} in the order we assumed above, we could always bring it into this order by commuting operators using the commutation relation \eqref{eq:tft:boson_field_commutators}.
But such terms would appear only as a constant phase on the right side of \eqref{eq:tft:path_integral_hamiltonian_assumption} that would not depend on the dynamics of the process.
AHence, we can relax this assumption by absorbing this physically irrelevant phase into the transition amplitude. ER DETTE RIKTIG? INGEN LÆREBØKER NEVNER DETTE.)

(TODO: if this argument holds, then it can be applied for any $\hat{H}^n$, and there is no reason to expand to first order in time, either)

Substituting \cref{eq:tft:inner_product_position_momentum,eq:tft:path_integral_hamiltonian_assumption,eq:tft:hamiltonian_eigenvalues}, the transition amplitude becomes
\begin{equation}
\begin{split}
	\transampl &=      \left( \prod_{n=1}^N \int \frac{\dif \phi_n \dif \pi_n}{2 \pi \hbar} \right) \\
	           &\times \exp \left( i \Delta t \sum_{n=1}^N \int \dif^3 x \left( \pi_n(\vec{x}) \frac{\phi_{n+1}(\vec{x}) - \phi_n(\vec{x})}{\Delta t} - \ham(\pi_n(\vec{x}), \phi_n(\vec{x})) \right)
	\right)
\end{split}
\end{equation}
Finally, we take the continuum limit by sending $N \rightarrow \infty$.
It is then natural to define
$\phi(\vec{x}, t_n) = \phi_n(\vec{x}, t_n)$
and
$\pi(\vec{x}, t_n) = \pi_n(\vec{x}, t_n)$
to be the values of the fields at each timestep $t_n$.
Both become continuous functions of time in the continuum limit.
We also use the finite difference definition of the derivative to turn the fraction in the exponential into a partial derivative $\dot{\phi}(\vec{x},t) = \pdv{\phi(\vec{x},t)}/{t}$.
Similarly, we use the Riemann sum definition of the integral to turn the sum $\sum \Delta t$ into an integral $\int \dif t$.
The integrals over the fields now become \textbf{functional integrals} defined as
\begin{equation}
	\int \pathintdif \phi = \lim_{N \rightarrow \infty} \prod_{n=1}^{N} \int \dif \phi_n
	\qquad \text{and} \qquad
	\int \pathintdif \pi = \lim_{N \rightarrow \infty} \prod_{n=1}^{N} \int \frac{\dif \pi_n}{2 \pi \hbar} .
\end{equation}
We have swept the diverging factor $(2 \pi \hbar)^N$ under the rug, arguing that it contains no physical information about the dynamics of the process and can be ignored, unlike the other factors.
(TODO: is this important)
(TODO: can integrate out momentum with gaussian integral if it appears as $p^2$)
With all of these steps, the transition amplitude takes the form of the path integral
\begin{equation}
\begin{split}
	\transampl &=      \int \pathintdif \pi \int_{\phi_A(\vec{x}, 0)}^{\phi_B(\vec{x}, T)} \pathintdif \phi \\
	           &\times \exp \left( i \int_0^T \dif t \int \dif^3 x \left( \pi(\vec{x}, t) \dot{\phi}(\vec{x}, t) - \ham(\pi(\vec{x}, t), \phi(\vec{x}, t)) \right) \right) .
\end{split}
\label{eq:tft:path_integral_hamiltonian}
\end{equation}
Note that the combination of the Hamiltonian and the fields in the exponential is precisely the Legendre transformation that converts between the Hamiltonian density $\ham$ and the Lagrangian density $\lagr$.
Thus, we might as well express the transition amplitude as the \textbf{path integral}
%	\transampl = \int \pathintdif \pi \int_{\phi_A(\vec{x}, 0)}^{\phi_B(\vec{x}, T)} \pathintdif \phi \\
%	             \exp \left( i \int_0^T \dif t \int \dif^3 x \, \left( \lagr(\pi(\vec{x}, t), \phi(\vec{x}, t)) \right) \right) .
\begin{equation}
	\transampl = \int \pathintdif \pi \int_{\phi_A(\vec{x}, 0)}^{\phi_B(\vec{x}, T)} \pathintdif \phi \, \exp \big( i S \left[ \pi(\vec{x}, t), \phi(\vec{x}, t) \right] \big) ,
\label{eq:tft:path_integral_lagrangian}
\end{equation}
where the \textbf{action} is
\begin{equation}
	S \left[ \pi(\vec{x}, t), \phi(\vec{x}, t) \right] = \int_0^T \dif t \int \dif^3 x \, \lagr \left( \pi(\vec{x}, t), \phi(\vec{x}, t) \right) .
\end{equation}
The path integral expresses the transition amplitude for the process $A \rightarrow B$ as a sum over all possible paths through phase space, each weighted by the value on the unit circle with phase corresponding to the action of the path.
Note that the position integral $\int \pathintdif \phi$ is constrained at the start and the end, but the momentum integral $\int \pathintdif \pi$ is not.
