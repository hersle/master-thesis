\chapter*{Notation and conventions} % do not number
\addcontentsline{toc}{chapter}{Notation and conventions} % but still display in TOC (see https://tex.stackexchange.com/a/222961)
%\markboth{NOTATION AND CONVENTIONS}{} % but still display correct header (see https://tex.stackexchange.com/a/78090)

\section*{Metric signature}

We use the $(+,-,-,-)$ metric signature.

\section*{Summation convention}

We use the Einstein summation convention, in which an index that appears once as a superscript and again as a subscript in the same term is to be summed over.
If the index is roman, the sum runs from $1$ to $3$, and if it is greek, it also runs over $0$.
For example,
\begin{equation*}
	T\indices{^\mu_\mu} = \sum_{\mu=0}^3 T\indices{^\mu_\mu}
	\quad \text{and} \quad
	T\indices{^i_i} = \sum_{i=1}^3 T\indices{^i_i}
	.
\end{equation*}

\iffalse
\section*{Fourier transformation}

We use the Fourier transformation convention
\begin{equation}
	f(k) = \int \dif x \, e^{i k x} f(x)
	\qquad \text{and} \qquad
	f(x) = \int \frac{\dif k}{2 \pi} \, e^{-i k x} f(k) .
\end{equation}
With this convention, the delta function $\delta(x' - x)$ is given by the highlighted part of
\begin{equation}
	f(x) = \int \frac{\dif k}{2 \pi} \, e^{-i k x} f(k)
	     = \int \dif x' \underbrace{\int \frac{\dif k}{2 \pi} \, e^{i k (x'-x)}}_{\displaystyle \delta(x'-x)} f(k) .
\label{eq:pre:delta_function}
\end{equation}
\fi

\TODO{big F in Figure, A in Appendix, etc?}
