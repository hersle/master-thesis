\chapter{MIT bag model}
\label{chap:mit}

To illustrate some of the concepts in \cref{chap:master_intro} and gently familiarize ourselves with additional ones,
let us first model quark stars consisting of a free Fermi gas of quarks with the \textbf{MIT bag model}.
Originally introduced by \cite{ref:mit_bag_model_original} at MIT,
this simple phenomenological model accounts for confinement of quarks in hadrons by adding a \textbf{bag constant} $B$ on top of a normal, deconfined ideal Fermi gas.
However, we will see that quark stars in this model consist of \emph{deconfined} quark matter.
Moreover, we will discuss the \textbf{strange matter hypothesis} and how it can be used to determine an interval of acceptable values of $B$.

We will start by simply ignoring the gluon fields $A_\mu^a$ from the quantum chromodynamics Lagrangian \eqref{eq:qcd:lagrangian}.
In addition, we couple the conserved vector current $j^\mu_f = \bar{q}_f \gamma^\mu q_f$ to a chemical potential $\mu_f$ for each quark flavor,
allowing us to tune the densities \eqref{eq:master_intro:densities} of quarks.%
\footnote{For an explanation of how the chemical potentials can be thought of as an analogy to constant gauge fields,
the interested reader can consult the discussion surrounding equations \eqref{eq:tft:conserved_current}--\eqref{eq:tft:chemical_potential}.}
With these modifications, the Lagrangian becomes
\begin{equation}
	\lagr = \bar{q} (i \slashed\partial + \mu \gamma^0 - m) q
	      = \sum_{c=1}^{N_c} \sum_{f=\{u,d,s\}} \bar{q}_{f,c} (i \slashed\partial + \mu_f \gamma^0 - m_f) q_{f,c} .
\label{eq:mit:lagrangian}
\end{equation}
In this model the two-flavor and three-flavor analysis is very similar, so we consider them in parallel.
Unless $N_f$ is specified explicitly, we perform the general analysis with $N_f=3$
and simply drop terms or factors indexed by the strange quark $s$ to get expressions for $N_f = 2$.
As we have simply omitted the gluons, we will continue to use the lone quark masses in \cref{tab:qcd:quark_properties}.

\textit{This chapter is inspired by reference \cite{ref:glendenning} and \cite{ref:quark_bag_model}.}

\section{Grand potential and equation of state}
\label{sec:mit:eos}

With the Euclidean version of the Lagrangian density \eqref{eq:mit:lagrangian},
the partition function \eqref{eq:master_intro:partition_function} reads
\begin{equation}
	Z = \prod_{c=1}^{N_c} \prod_{f=\{u,d,s\}} \oint_- \pathintdif \bar{q}_{f,c} \oint_- \pathintdif q_{f,c} \exp \bigg\{ \int_0^\beta \dif \tau \int_V \dif^3 x \lagr_E [q_{f,c},\bar{q}_{f,c}] \bigg\} .
\end{equation}
It decouples into a product of path integrals \eqref{eq:tft:dirac_partition_function_first} that we encountered back in \cref{chap:tft}
and simplified to the form \eqref{eq:tft:dirac_partition_function} for arbitrary temperature.
Then we neglected the divergent vacuum contribution from the first term and calculated it explicitly in the zero-temperature approximation,
arriving at the pressure \eqref{eq:nstars:pressure_zeroT} that is related to the grand potential density \eqref{eq:master_intro:grand_potential} by a simple sign flip.
Adding a background of free electrons and reinstating the Fermi momenta $p_f = \sqrt{\smash[b]{\mu_f^2-m_f^2}}$ for each particle species,
we can write the grand potential as
\begin{equation}
\begin{split}
	\Omega(\vec{\mu}) = &-\smashoperator{\sum_{\vphantom{\big|} f=\{u,d,s\}}} \frac{N_c}{24 \pi^2} \left[ \left( 2 \mu_f^2 - 5 m_f^2 \right) \mu_f \sqrt{\mu_f^2 - m_f^2} + 3 m_f^4 \asinh \left( \sqrt{\frac{\mu_{\smash{f}}^2}{m_f^2}-1} \right) \right] \\
	                    &-\phantom{\sum} \, \frac{1}{24 \pi^2} \left[ \left( 2 \mu_e^2 - 5 m_e^2 \right) \mu_e \sqrt{\mu_e^2 - m_e^2} \, \, + \, 3 m_e^4 \asinh \left( \sqrt{\frac{\mu_{\smash{e}}^2}{m_e^2}-1} \right) \right].
\label{eq:mit:grand_potential}
\end{split}
\end{equation}
This expression is only valid when $\mu_i \geq m_i$,
as we assumed so during calculation of the zero-temperature pressure integral \eqref{eq:nstars:pressure_zeroT} containing the step function $\Theta(\mu_i-\sqrt{\smash[b]{m_i^2-p^2}})$.
In the opposite case $\mu_i < m_i$, the step function would be turned off for all $p$ and make the integral vanish.
Equivalently, the grand potential above is valid for \emph{all} $\mu_i$ if we implicitly take its \emph{real part}.
Using this convention, the corresponding quark and electron densities \eqref{eq:master_intro:densities} are
\begin{equation}
	n_f = -\pdv{\Omega}{\mu_f} = \frac{N_c}{3 \pi^2} \Big( \mu_f^2 - m_f^2 \Big)^{\frac32}
	\qquad \text{and} \qquad
	n_e = -\pdv{\Omega}{\mu_e} = \frac{  1}{3 \pi^2} \Big( \mu_e^2 - m_e^2 \Big)^{\frac32},
\label{eq:mit:particle_densities}%
\end{equation}
and the pressure \eqref{eq:master_intro:pressure} and energy density \eqref{eq:master_intro:energy_density} follow.

We now see explicitly that the grand potential and hence the pressure and energy density are functions of the four chemical potentials $\mu_u$, $\mu_d$, $\mu_s$ and $\mu_e$.
As explained in \cref{sec:master_intro:tft},
we reduce them to a single independent chemical potential with the three constraints \eqref{eq:lsm:chemical_equilibrium} and \eqref{eq:lsm:charge_neutrality},
and take this to be the quark chemical potential $\mu$ defined in equation \eqref{eq:master_intro:chemical_potentials_transformed}.
With the quark charges in \cref{tab:qcd:quark_properties}, the densities \eqref{eq:mit:particle_densities} and for a given value of $\mu$,
we must then find the electron chemical potential $\mu_e$ that solves
\begin{equation}
	2 \Big[\mu_u^2-m_u^2\Big]^\frac32
	- \Big[(\mu_u+\mu_e)^2-m_d^2\Big]^\frac32 
	- \Big[(\mu_u+\mu_e)^2-m_s^2\Big]^\frac32 
	- \Big[\mu_e^2-m_e^2\Big]^\frac32 = 0.
\label{eq:mit:charge_neutrality_explicit}
\end{equation}
Having chosen a value of $\mu$ and determined a value of $\mu_e$,
the chemical potentials $\mu_u$, $\mu_d$ and $\mu_s$ follow from definition \eqref{eq:master_intro:chemical_potentials_transformed} and the chemical equilibrium constraint \eqref{eq:lsm:chemical_equilibrium}. 
Elimination of three of four chemical potentials yields two functions $P(\mu)$ and $\epsilon(\mu)$,
and the former can finally be inverted to yield the equation of state $\epsilon(P)$.

\begin{figure}
\centering
\tikzsetnextfilename{mit-eos}
\begin{tikzpicture}
\tikzset{declare function={
	muQ(\muu,\mud)=(\muu+\mud)/2;
	muu(\muQ)=2/(1+2^(1/3))*\muQ;
	mud(\muQ)=2/(1+2^(-1/3))*\muQ;
	mue(\muQ)=2*(2^(1/3)-1)/(2^(1/3)+1)*\muQ;
	nq(\mu)=3/(3*pi^2)*(\mu)^3;
	ne(\mu)=1/(3*pi^2)*(\mu)^3;
	nconv=1.29619e-7;
}};
\begin{groupplot}[
	group style={group size={1 by 3}, vertical sep=2.0cm},
	width=13cm, height=7cm,
	extra tick style={grid=major, grid style={dashed}},
	minor tick num=9,
]
\nextgroupplot[
	xlabel={$\mu \, / \, \si{\mega\electronvolt}$}, ylabel={$\mu_i \, / \, \si{\mega\electronvolt}$},
	%xmin=0, xmax=600, ymax=500, xtick distance=100, ytick distance=100, minor x tick num=9,
	xmin=0, xmax=700, xtick distance=100, minor x tick num=9,
	ymin=0, ymax=700, ytick distance=100, 
	%ymax=600, 
	title={\subcaption{\label{fig:mit:eos-parametrization}Parametrization of solutions}},
	legend cell align=right, legend pos=north west,
];
% fake legend
\addplot+ [black, solid, opacity=0.3] {-100}; \addlegendentry{$N_f=2$};
\addplot+ [black, solid, opacity=0.8] {-100}; \addlegendentry{$N_f=3$};
\addplot+ [red, solid, opacity=1.0] {-100}; \addlegendentry{$i=u$};
\addplot+ [darkgreen, solid, opacity=1.0] {-100}; \addlegendentry{$i=d$};
\addplot+ [purple, solid, opacity=1.0] {-100}; \addlegendentry{$i=s$};
\addplot+ [blue, solid, opacity=1.0] {-100}; \addlegendentry{$i=e$};

% 2-flavor
\addplot+ [red,       solid,  opacity=0.3] table [x expr={muQ(\thisrow{muu},\thisrow{mud})}, y=muu] {../code/data/MIT2F/eos_sigma_800.dat};
\addplot+ [darkgreen, solid,  opacity=0.3] table [x expr={muQ(\thisrow{muu},\thisrow{mud})}, y=mud] {../code/data/MIT2F/eos_sigma_800.dat};
\addplot+ [blue,      solid,  opacity=0.3] table [x expr={muQ(\thisrow{muu},\thisrow{mud})}, y=mue] {../code/data/MIT2F/eos_sigma_800.dat};

% 3-flavor
\addplot+ [red,       solid,  opacity=0.8] table [x expr={muQ(\thisrow{muu},\thisrow{mud})}, y=muu] {../code/data/MIT3F/eos_sigma_800.dat};
\addplot+ [darkgreen, solid,  opacity=0.8] table [x expr={muQ(\thisrow{muu},\thisrow{mud})}, y=mud] {../code/data/MIT3F/eos_sigma_800.dat};
\addplot+ [purple,    dashed, opacity=0.8] table [x expr={muQ(\thisrow{muu},\thisrow{mud})}, y=mus] {../code/data/MIT3F/eos_sigma_800.dat};
\addplot+ [blue,      solid,  opacity=0.8] table [x expr={muQ(\thisrow{muu},\thisrow{mud})}, y=mue] {../code/data/MIT3F/eos_sigma_800.dat};

\nextgroupplot[
	xlabel={$\mu \, / \, \si{\mega\electronvolt}$}, ylabel={$n_i \, / \, (1/\si{\femto\meter\cubed})$},
	xmin=0, xmax=700, xtick distance=100, minor x tick num=9,
	ymin=-0.2, ymax=5.0, ytick distance=1.0, minor y tick num=4, restrict y to domain=-10:10,
	title={\subcaption{\label{fig:mit:eos-density}Particle number densities}},
	legend cell align=right, legend pos=north west,
];
% fake legend
\addplot+ [black, solid, opacity=0.3] {-10}; \addlegendentry{$N_f=2$};
\addplot+ [black, solid, opacity=0.8] {-10}; \addlegendentry{$N_f=3$};
\addplot+ [red, solid, opacity=1.0] {-10}; \addlegendentry{$i=u$};
\addplot+ [darkgreen, solid, opacity=1.0] {-10}; \addlegendentry{$i=d$};
\addplot+ [purple, solid, opacity=1.0] {-10}; \addlegendentry{$i=s$};
\addplot+ [blue, solid, opacity=1.0] {-10}; \addlegendentry{$i=e$};

% 2-flavor
\addplot+ [red,       solid, opacity=0.3] table [x expr={muQ(\thisrow{muu},\thisrow{mud})}, y=nu] {../code/data/MIT2F/eos_sigma_800.dat};
\addplot+ [darkgreen, solid, opacity=0.3] table [x expr={muQ(\thisrow{muu},\thisrow{mud})}, y=nd] {../code/data/MIT2F/eos_sigma_800.dat};
\addplot+ [blue,      solid, opacity=0.3] table [x expr={muQ(\thisrow{muu},\thisrow{mud})}, y=ne] {../code/data/MIT2F/eos_sigma_800.dat};

% 3-flavor
\addplot+ [red,       solid, opacity=0.8] table [x expr={muQ(\thisrow{muu},\thisrow{mud})}, y=nu] {../code/data/MIT3F/eos_sigma_800.dat};
\addplot+ [darkgreen, solid, opacity=0.8] table [x expr={muQ(\thisrow{muu},\thisrow{mud})}, y=nd] {../code/data/MIT3F/eos_sigma_800.dat};
\addplot+ [purple,    solid, opacity=0.8] table [x expr={muQ(\thisrow{muu},\thisrow{mud})}, y=ns] {../code/data/MIT3F/eos_sigma_800.dat};
\addplot+ [blue,      solid, opacity=0.8] table [x expr={muQ(\thisrow{muu},\thisrow{mud})}, y=ne] {../code/data/MIT3F/eos_sigma_800.dat};

\nextgroupplot[
	xlabel={$P        \, / \, (\si{\giga\electronvolt\per\femto\meter\cubed})$},
	ylabel={$\epsilon \, / \, (\si{\giga\electronvolt\per\femto\meter\cubed})$},
	xmin=0, xmax=0.30, ymin=0, ymax=1.0, xtick distance=0.10, minor x tick num=9, ytick distance=0.5, minor y tick num=4, restrict y to domain=-1:+1,
	title={\subcaption{\label{fig:mit:eos-eos}Equation of state}},
	legend cell align=left, legend pos=north west,
];
\addplot+ [black, solid, opacity=0.3] table [x=P,y=epsilon] {../code/data/MIT2F/eos_sigma_800.dat}; \addlegendentry{$N_f=2$};
\addplot+ [black, solid, opacity=0.8] table [x=P,y=epsilon] {../code/data/MIT3F/eos_sigma_800.dat}; \addlegendentry{$N_f=3$};
\end{groupplot}
\end{tikzpicture}
\caption{\label{fig:mit:eos}%
Properties of electrically charge neutral two-flavor (weak lines) and three-flavor (strong lines) MIT bag model quark matter in $\beta$-equilibrium parametrized by the common quark chemical potential $\mu = (\mu_u+\mu_d)/2$.
Upper panel \subref{fig:mit:eos-parametrization} shows the relations between chemical potentials due to the constraints \eqref{eq:lsm:chemical_equilibrium} and \eqref{eq:lsm:charge_neutrality},
middle panel \subref{fig:mit:eos-density} the corresponding particle number densities \eqref{eq:mit:particle_densities} and
lower panel \subref{fig:mit:eos-eos} the resulting equation of state.
}
\end{figure}

\subsubsection{Ultra-relativistic limit}

Before tackling the general solution numerically,
it is instructive and in fact not far from accurate to solve this problem analytically in the ultra-relativistic limit $m_i \rightarrow 0$.
The grand potential \eqref{eq:mit:grand_potential} then reduces to
\begin{equation}
	\Omega(\vec{\mu}) = -\frac{N_c \mu_u^4}{12 \pi^2} - \frac{N_c \mu_d^4}{12 \pi^2} - \frac{N_c \mu_s^4}{12 \pi^2} - \frac{\mu_e^4}{12 \pi^2},
\label{eq:mit:grand_potential_massless}
\end{equation}
while the densities \eqref{eq:mit:particle_densities} become
\begin{equation}
	n_f = -\pdv{\Omega}{\mu_f} = \frac{N_c \mu_f^3}{3 \pi^2}
	\text{ for $f \in \{u,d,s\}$}
	\quad \text{and} \quad
	n_e = -\pdv{\Omega}{\mu_e} = \frac{    \mu_e^3}{3 \pi^2}.
\label{eq:mit:densities_massless}
\end{equation}
\emph{Regardless} of the relations between the chemical potentials,
the equation of state is then simply
\begin{equation}
	\epsilon = -P + \sum_i \mu_i n_i
	         = -P + \frac{1}{3 \pi^2} \Big(N_c \mu_u^4 + N_c \mu_d^4 + N_c \mu_s^4 + \mu_e^4\Big)
	         = -P - 4 \Omega 
	         = -P + 4 P 
	         = 3 P \, !
\label{eq:mit:eos_ur}
\end{equation}
We can also find approximate analytical relations between the chemical potentials in the ultra-relativistic limit
if we also assume that the electron density in the charge neutrality condition \eqref{eq:mit:charge_neutrality_explicit} can be neglected.
This assumption is subject to a self-consistency check down the line.
In the two-flavor case, this simplifies the condition to $n_d = 2 n_u$ with the solution $\mu_d = 2^{1/3} \mu_u$.
According to the $\beta$-equilibrium condition \eqref{eq:lsm:chemical_equilibrium},
the chemical potential of the electrons is then $\mu_e = \mu_d - \mu_u \approx (2^{1/3}-1) \mu_u$
and corresponds to the density $n_e = (2^{1/3}-1)^3 n_u / N_c = 0.006 \, n_u \ll n_u < n_d$,
so neglecting the electron density is a self-consistent approximation.
Expressing the chemical potentials in terms of the quark chemical potential \eqref{eq:master_intro:chemical_potentials_transformed},
we then have
\begin{equation}
	\mu_u = \underbrace{\frac{2}{1+2^{1/3}}}_{0.88} \mu , \quad
	\mu_d = \underbrace{\frac{2}{1+2^{-1/3}}}_{1.12} \mu \quad \text{and} \quad
	\mu_e = \underbrace{2 \frac{2^{1/3}-1}{2^{1/3}+1}}_{0.29} \mu
	\qquad (N_f = 2) .
\label{eq:mit:chemical_potentials_massless_2f}
\end{equation}
With three flavors, the charge neutrality condition \eqref{eq:mit:charge_neutrality_explicit} becomes $2 n_u - n_d - n_s = 0$
and has the very simple solution $n_u = n_d = n_s$ with
\begin{equation}
	\mu_u = \mu_d = \mu_s = \mu
	\qquad \text{and} \qquad
	\mu_e = 0
	\qquad (N_f = 3) .
\label{eq:mit:chemical_potentials_massless_3f}
\end{equation}
As $\mu_e = \mu_d - \mu_u = 0$, neglecting the electron contribution is exact in this case.

\subsubsection{General solution}

We now calculate the general solution with massive quarks using the program in \cref{sec:num:qstars2f},
solving equation \eqref{eq:mit:charge_neutrality_explicit} numerically for different quark chemical potentials $\mu$.
We obtain the chemical potentials, densities and equation of state shown in \cref{fig:mit:eos}.
Note that the electron density is very small, but nonzero.
Indeed, we see that they are very close to the ultra-relativistic relations
\eqref{eq:mit:densities_massless}-\eqref{eq:mit:chemical_potentials_massless_3f}
as $\mu_i \gg m_i$, thereby verifying our numerical calculations.


\section{Bag constant and strange matter hypothesis}

\TODO{elsewhere: reflect that all quark stars are pure quark stars with no confinement}

For chemical potentials $\mu_i < m_i$ the (real part of the) densities \eqref{eq:mit:particle_densities} vanish,
so we refer to this region of the phase diagram as the \textbf{vacuum phase}.
So far we have modeled the quarks as a free Fermi gas of \emph{deconfined} quarks.
We know, however, that quarks are \emph{confined} in hadrons at low energies.
Let us see how the MIT bag model attempts to incorporate confinement with a so-called bag constant.

The pressure $P = -\Omega$ due to the grand potential density \eqref{eq:mit:grand_potential} is normalized to $P = 0$ in the vacuum.
Suppose that we shift $\Omega \rightarrow \Omega + B$ by a \textbf{vacuum constant} $B$.
This in turn shifts the pressure \eqref{eq:master_intro:pressure} and energy density \eqref{eq:master_intro:energy_density} to
\begin{equation}
	P(\mu) \rightarrow P(\mu) - B
	\qquad \text{and} \qquad
	\epsilon(\mu) \rightarrow \epsilon(\mu) + B,
\label{eq:mit:bag_shift}
\end{equation}
effectively moving the equation of state in \cref{fig:mit:eos-eos} north-west in $P$-$\epsilon$-space.
The vacuum constant is hence a phenomenological parameter
that adjusts the normalization of the grand potential in vacuum,
and hence the zero-point energy and pressure, and the equation of state.

\begin{figure}[t]
\centering

\pgfmathdeclarerandomlist{randcolors}{{red}{green}{blue}}
\pgfmathdeclarerandomlist{randflavors}{{u}{d}{s}}
\newcommand{\fillrandomly}[6]{

	\def\xlist{4} % TODO: ???
	\def\ylist{4}
    \pgfmathsetmacro\diameter{#5*2}
    %\draw (0,0) rectangle (#3,#4);
    \foreach \i in {1,...,#6}{
        \pgfmathsetmacro\x{#1+rnd*#3}
        \pgfmathsetmacro\y{#2+rnd*#4}
        \xdef\collision{0}
        \foreach \element [count=\i] in \xlist{
            \pgfmathtruncatemacro\j{\i-1}
            \pgfmathsetmacro\checkdistance{ sqrt( ({\xlist}[\j]-(\x))^2 + ({\ylist}[\j]-(\y))^2 ) }
            \ifdim\checkdistance pt<\diameter pt
                \xdef\collision{1}
                \breakforeach
            \fi
        }
        \ifnum\collision=0
            \xdef\xlist{\xlist,\x}
            \xdef\ylist{\ylist,\y}
			\pgfmathrandomitem\thecolor{randcolors}
			\edef\thecolor{\thecolor}%
			\pgfmathrandomitem\theflavor{randflavors}
			\edef\theflavor{\theflavor}%
			\draw [draw=black, fill=\thecolor] (\x, \y) circle (#5) node [text=white] {$\theflavor$};
        \fi

    }
}

\subcaptionbox{\label{fig:mit:bag-constant-confined}confined phase}{
\centering
\tikzsetnextfilename{bag-constant-confined}
\begin{tikzpicture}
\draw [draw=black, fill=gray] (-3, -3) rectangle (+3, +3);

\draw [draw=black, fill=white] (0, 0) circle (2);
%\begin{scope}[blend group=screen]
	\draw [draw=black, fill=green] ( 30:0.9) circle (0.40) node [text=white] {$d$};
	\draw [draw=black, fill=red]   (150:0.9) circle (0.40) node [text=white] {$u$};
	\draw [draw=black, fill=blue]  (270:0.9) circle (0.40) node [text=white] {$d$};

	% kinetic pressure
	\draw [gray, very thick, -Latex] ( 30:1.4) -- ( 30:1.9);
	\draw [gray, very thick, -Latex] (150:1.4) -- (150:1.9);
	\draw [gray, very thick, -Latex] (270:1.4) -- (270:1.9);

	% bag pressure
	\draw [very thick, -Latex] ( 30:2.6) -- ( 30:2.1);
	\draw [very thick, -Latex] (150:2.6) -- (150:2.1);
	\draw [very thick, -Latex] (270:2.6) -- (270:2.1);
%\end{scope}
\node at (0, -0.25) {$P_\text{vac}=-B$};
\node [anchor=south west] at (-2.7, -2.7) {$P_\text{vac}=0$};

\node at (90:1.3) {  confined vacuum};
\node at (90:2.5) {deconfined vacuum};

\end{tikzpicture}
}
\subcaptionbox{\label{fig:mit:bag-constant-deconfined}deconfined phase}{
\centering
\tikzsetnextfilename{bag-constant-deconfined}
\begin{tikzpicture}
\draw [draw=black, fill=gray] (-3, -3) rectangle (+3, +3);
\pgfmathsetseed{93209302}
\node at (90:2.5) {deconfined vacuum};
\fillrandomly{-2.6}{-2.6}{5.2}{4.5}{0.40}{80}
\end{tikzpicture}
}
\caption{\label{fig:mit:bag_constant}%
	The bag constant $B$ can be interpreted as the pressure difference between the deconfined and confined vacua of quantum chromodynamics.
	In the confined phase \subref{fig:mit:bag-constant-confined},
	the surrounding quark-forbidden deconfined medium may pressurize hadronic ``bags''
	inside which quarks are trapped and stabilize the bag by exerting a counteracting kinetic pressure.
	In the deconfined phase \subref{fig:mit:bag-constant-deconfined} there is no bag pressure,
	so the bags open up and the quarks break free into deconfined quark matter.
}
\end{figure}


A positive vacuum constant $B$ creates a negative pressure $P = -B$ in vacuum.
In a sense, it can therefore be interpreted as a mechanism that \emph{confines} the quarks to ``bags'' that resemble hadrons.
Due to this physical interpretation, $B$ is more often called the \textbf{bag constant}.
From inside the bag, the quarks exert an outwards kinetic pressure that counteracts the external bag pressure and stabilizes the hadron.
The bag constant can therefore be interpreted as the pressure difference between two phases of confined and deconfined vacuum,
as illustrated in \cref{fig:mit:bag_constant}.

As an analogy, imagine a swimming pool full of water representing the deconfined phase:
it would cost an amount $BV$ of energy to expel water from a volume $V$ and create a bag of vacuum in the pool,
and a plastic bag enclosing this volume would have to generate some internal pressure to survive against the pressure from the surrounding water.

Despite the interpretation of $B$ as a confinement mechanism,
it is very important to note that the quark stars we will model 
will be integrated from a positive central pressure to a vanishing pressure at the surface,
and therefore have \emph{non-negative pressure everywhere} and \emph{only contain deconfined quark matter}.
The negative-pressure regime in the equation of state that arises due to the bag shift \eqref{eq:mit:bag_shift}
\emph{is outside reach of the pressure range within the quark stars}.
We should therefore not take the bag constant's interpretation as a confinement mechanism too literally in our case,
and it is perhaps better to think of it only as a phenomenological parameter that we can vary to adjust the normalization of the grand potential
and -- as we will see now -- a parameter that determines the stability relationship of quark matter compared to hadronic matter.
\TODO{spør hva JO tenker om ny bag-tolkning}

What values can the bag constant take?
Let us perform a heuristic calculation where we picture quarks to be confined in a spherical bag of radius $R$, as depicted in \cref{fig:mit:bag-constant-confined}.
The vacuum energy associated with the mere existence of the bag is then $E_V = 4 \pi R^3 B / 3$.
By Heisenberg's uncertainty principle $\Delta p \Delta x \geq \hbar/2$, a quark confined to a region extending $\Delta x = 2R$ in one dimension with mean momentum $\avg{p} = 0$ has momentum $p \lesssim \hbar/4R \propto 1/R$.
In the ultra-relativistic regime where we can neglect the particles' mass, the kinetic energy inside the bag is therefore $E_K = pc = C / R$ for some constant $C$.
As a function of $R$, the total energy $E = E_V + E_K$ has a minimum at $R = (C/4 \pi B)^{1/4}$ for which the bag is stable.
Solving for $C = 4 \pi B R^4$, the energy of this stable configuration is $E = 16 \pi R^3 B / 3$.
Assuming a neutron with total energy $E = m c^2 = \SI{940}{\giga\electronvolt}$ and radius $R \approx \SI{1}{\femto\meter} = 1 / \SI{197}{\mega\electronvolt}$,
the bag constant is
\begin{equation}
	B = (\SI{143.9}{\mega\electronvolt})^4.
\label{eq:mit:bag_constant_optimal}
\end{equation}

Another approach lets us determine a range of values for the bag constant.
In the early days with high density and temperature, the universe likely passed through a phase of deconfined quark-gluon plasma.
Today, two-flavor quark matter is accreting in nuclei through fusion towards iron-56 in stars, which seemingly represent the ground state of nuclear matter.
However, the \textbf{strange matter hypothesis} of \cite{ref:strange_hypothesis_bodmer} and \cite{ref:strange_hypothesis_witten} conjectures that this state is only \emph{metastable}
and could decay to three-flavor quark matter consisting of up, down and strange quarks.
If true, the universe could turn into a very \emph{strange} place some day.

Iron-56 is the most stable nuclide with an energy of $E/N_B = (E/V) / (N_B/V) = \epsilon/n_B = \SI{930}{\mega\electronvolt}$ per baryon.
If we denote the energy densities of two-flavor and three-flavor quark matter by $\epsilon_2$ and $\epsilon_3$,
then the instability of two-flavor quark matter and hypothesized stability of three-flavor quark matter at zero pressure $P=0$,
which characterizes stability of the system,
imply the inequality
\begin{equation}
	\frac{\epsilon_3(P=0)}{n_B} < \SI{930}{\mega\electronvolt} < \frac{\epsilon_2(P=0)}{n_B} .
\label{eq:mit:bag_stability}
\end{equation}
We will shortly see that this yields a range of bag constants that we call a \textbf{bag window}.

\subsubsection{Ultra-relativistic bag window}

Let us first examine how the ultra-relativistic equation of state is affected by the addition of a bag constant
and the corresponding bag window generated by inequality \eqref{eq:mit:bag_stability}.
The bag shift \eqref{eq:mit:bag_shift} changes the equation of state \eqref{eq:mit:eos_ur} to $\epsilon = 3 P + 4 B$.
Neglecting electrons again, the baryon density is $n_B = n_u = \mu_u^3 / \pi^2$,
since $n_d = 2 n_u$ and $n_s=0$ in the two-flavor case and $n_u=n_d=n_s$ in the three-flavor case.
With two flavors we also had $\mu_d = 2^{1/3} \mu_u$,
so the grand potential density \eqref{eq:mit:grand_potential_massless}
yields the pressure $P = -\Omega - B = (1 + 2^{4/3}) \mu_u^4 / 4 \pi^2 - B$ after ``bagging''.
With three flavors we had $\mu_s = \mu_d = \mu_u$ and instead find $P = 3 \mu_u^4 / 4 \pi^2 - B$.
Setting $P=0$, eliminating $\mu_u$ in favor of $B$ and inserting these relations into inequality \eqref{eq:mit:bag_stability},
it becomes exactly solvable and gives the bag window
\begin{equation}
	%\frac{4 B}{(4 \pi^2 B / 3)^\frac34 / \pi^2} < \SI{930}{\mega\electronvolt} < \frac{4 B}{(4 \pi^2 B / (1+2^\frac43))^\frac34 / \pi^2},
	%\quad \text{or} \quad
	\SI{144.4}{\mega\electronvolt} < B^\frac14 < \SI{162.8}{\mega\electronvolt}.
\label{eq:mit:bag_window_ur}
\end{equation}

\subsubsection{General bag window}

Returning to the general case with nonzero masses,
the program in \cref{sec:num:qstars2f} solves inequality \eqref{eq:mit:bag_stability} numerically.
It calculates the baryon density $n_B = (n_u+n_d+n_s)/3$
and performs the shift \eqref{eq:mit:bag_shift} of the pressure and energy density for different bag constants $B$
until it finds the one for which the inequality is barely satisfied.
For the two-flavor and three-flavor equations of state in \cref{fig:mit:eos}, it reports the bag window
\begin{equation}
	\SI{144.3}{\mega\electronvolt} < B^\frac14 < \SI{154.9}{\mega\electronvolt} .
\label{eq:mit:bag_window}
\end{equation}
In all cases, the lower bound comes from the two-flavor equation of state and the upper bound from the three-flavor equation of state.
Note that the lower bounds in the analytical and numerical bag windows \eqref{eq:mit:bag_window_ur} and \eqref{eq:mit:bag_window}
agree not only with each other, but also with the ``optimal'' bag constant \eqref{eq:mit:bag_constant_optimal}.
In contrast, the upper bounds disagree due to the massless approximation being worse with the addition of the heavy strange quark.


\section{Mass-radius solutions}

Having established the bag window \eqref{eq:mit:bag_window},
different equations of state are finally available to us by making the shift \eqref{eq:mit:bag_shift}
of the equation of state in \cref{fig:mit:eos-eos} with varying bag constants inside the window.
We then integrate the Tolman-Oppenheimer-Volkoff equation numerically as described in \cref{sec:master_intro:tov},
obtaining the quark stars in \cref{fig:mit:mass_radius}:
\begin{figure}[t]
\centering
\tikzsetnextfilename{mit-mass-radius}
\tikzset{
	Bpin/.style={gray, sloped, allow upside down=true, rotate=180, yshift=+0.4cm, font=\small},
}
\begin{tikzpicture}
\begin{groupplot}[
	group style={group size={2 by 1}, vertical sep=0cm, horizontal sep=0.5cm},
	width=8cm, height=8cm,
	xmin=5, xmax=20, ymin=0.5, ymax=2.5, xtick distance=5, ytick distance=0.5, minor tick num=4, grid=major,
	point meta=explicit, point meta min=33, point meta max=36,
	%colorbar horizontal, colormap name=plasmarev, colorbar style={xlabel=$\log_{10} (P_c \, / \, \si{\pascal})$, xtick distance=1, minor x tick num=9, at={(0.5,1.03)}, anchor=south, xticklabel pos=upper},
	/tikz/declare function={
		e0 = 4.266500881855304e+37;
	},
]
\tikzset{
	Bpin/.style={gray, sloped, allow upside down=true, rotate=180, yshift=+0.4cm, font=\small},
}
\nextgroupplot[
	xlabel={$R \, / \, \si{\kilo\meter}$},
	ylabel={$M \, / \, M_\odot$}, %title={Mass-radius diagram for 2-flavor quark stars }, title style={yshift=2.0cm},
	title = {$N_f=2$}, title style={text width=5cm},
];
\addplot+ [solid, mesh] table [x=R, y=M, meta expr={log10(\thisrow{P}*e0)}] {../code/data/MIT2F/stars_sigma_800_B14_145.dat}; % node [Bpin, pos=0.920] {$B = (\SI{27}{\mega\electronvolt})^4$};
\addplot+ [solid, mesh] table [x=R, y=M, meta expr={log10(\thisrow{P}*e0)}] {../code/data/MIT2F/stars_sigma_800_B14_150.dat}; % node [Bpin, pos=0.920] {$B = (\SI{27}{\mega\electronvolt})^4$};
\addplot+ [solid, mesh] table [x=R, y=M, meta expr={log10(\thisrow{P}*e0)}] {../code/data/MIT2F/stars_sigma_800_B14_155.dat}; % node [Bpin, pos=0.920] {$B = (\SI{27}{\mega\electronvolt})^4$};

\nextgroupplot[
	xlabel={$R \, / \, \si{\kilo\meter}$},
	yticklabels={,,},
	title = {$N_f=3$}, title style={text width=5cm},
	colorbar horizontal, colormap name=plasmarev, colorbar style={width=11cm, ylabel=$\log_{10} (P_c \, / \, \si{\pascal})$, ylabel style={rotate=-90}, xtick distance=1, minor x tick num=9, at={(0.87,-0.25)}, anchor=north east, xticklabel pos=lower},
];
\addplot+ [solid, mesh] table [x=R, y=M, meta expr={log10(\thisrow{P}*e0)}] {../code/data/MIT3F/stars_sigma_800_B14_145.dat}; % node [Bpin, pos=0.920] {$B = (\SI{27}{\mega\electronvolt})^4$};
\addplot+ [solid, mesh] table [x=R, y=M, meta expr={log10(\thisrow{P}*e0)}] {../code/data/MIT3F/stars_sigma_800_B14_150.dat}; % node [Bpin, pos=0.920] {$B = (\SI{27}{\mega\electronvolt})^4$};
\addplot+ [solid, mesh] table [x=R, y=M, meta expr={log10(\thisrow{P}*e0)}] {../code/data/MIT3F/stars_sigma_800_B14_155.dat}; % node [Bpin, pos=0.920] {$B = (\SI{27}{\mega\electronvolt})^4$};

\end{groupplot}
\end{tikzpicture}
\caption{\label{fig:mit:mass_radius}%
Mass-radius solutions of the Tolman-Oppenheimer-Volkoff equation \eqref{eq:master_intro:tov} parametrized by the central pressure $P_c$,
obtained with the two-flavor and three-flavor MIT bag model equations of state from \cref{fig:mit:eos-eos}
modified by the bag shift \eqref{eq:mit:bag_shift} with three bag constants $B = \{145,150,155\} \, \si{\mega\electronvolt}$ in the bag window \eqref{eq:mit:bag_window}.
Lower bag constants correspond to greater maximum masses.
\TODO{add bag constant labels?}
}
\end{figure}
\begin{itemize}
\item As mentioned in \cref{sec:master_intro:tov},
      stars with central pressure exceeding that of the maximum mass star are unstable,
      so we cut off the curve not long after the mass peak.
\item Using bag constants inside the bag window \eqref{eq:mit:bag_window},
      the MIT bag model realizes quark stars with maximum masses
      $1.7 \, M_\odot \leq M \leq 2.0 \, M_\odot$ with $N_f=2$ flavors
      and $1.6 \, M_\odot \leq M \leq 1.9 \, M_\odot$ with $N_f=3$ flavors,
      in both cases with corresponding radii $\SI{9}{\kilo\meter} \leq R \leq \SI{11}{\kilo\meter}$.
\item The smaller the bag constant, the larger and more massive the star.
      The bag shift \eqref{eq:mit:bag_shift} lifts the equation of state in the $P$-$\epsilon$-diagram up and to the left
      so that a given pressure corresponds to a greater energy density.
      Viewed the other way around, a given density corresponds to a lower pressure and makes the material easier to compress,
      so we say that the equation of state is \emph{softened}.
      Conversely, an equation of state where a given pressure corresponds to a lower energy density, and a given density to a higher pressure,
      describes material that is harder to compress and is said to be \emph{stiffer}.
      In other words, lower bag constants produce stiffer equations of state and larger and more massive stars.

      The increase in size is easy to understand:
      integrating the Tolman-Oppenheimer-Volkoff equations \eqref{eq:master_intro:tov} from a fixed central pressure $P_c$,
      a stiffer equation of state yields a strictly smaller mass gradient $\odv{m}/{r}$ and pressure gradient magnitude $\abs{\odv{P}/{r}}$,
      so the surface $P(R)=0$ is reached for a smaller radius.
      The increase in mass can then only be explained by the greater radius outweighing the smaller mass gradient.
\item Notice the qualitatively similar shapes of mass-radius curves corresponding to different bag constants.
      As mentionde in \cite[equation 8.29]{ref:glendenning},
      it is in fact possible to show that with the bagged ultra-relativistic equation of state $\epsilon = 3P + 4B$,
      the Tolman-Oppenheimer-Volkoff equation \eqref{eq:master_intro:tov} admits scaling solutions
      where the masses $M(B)$ and $M(B')$ and radii $R(B)$ and $R(B')$ corresponding to two different bag constants $B$ and $B'$
      are related by $M(B') = \sqrt{B/B'} \, M(B)$ and $R(B') = \sqrt{B/B'} \, R(B)$.
      These relations hold only approximately in our massive case.
\end{itemize}

\section{Summary}

In this chapter we have reviewed the simplest and most well-known model of quark stars
as a free Fermi gas featuring a phenomenological bag constant $B$.
We described a method of determining a window of acceptable bag constants by assuming that at zero pressure,
two-flavor quark matter is unstable and can decay to hadronic matter,
which in turn can decay to three-flavor quark matter according to the strange matter hypothesis.
By constraining the chemical potentials associated with each particle species,
we saw how to determine the equation of state $\epsilon(P)$,
which in turn was used to integrate the Tolman-Oppenheimer-Volkoff equations \eqref{eq:master_intro:tov},
yielding solutions corresponding to stars with given masses and radii.
We obtained maximum masses $1.7 \, M_\odot \leq M \leq 2.0 M_\odot$ with $N_f=2$ flavors
and $1.6 \, M_\odot \leq M \leq 1.9 \, M_\odot$ with $N_f=3$ flavors.
This concludes our introduction to the basic concepts to be used in the remainder of this thesis.
