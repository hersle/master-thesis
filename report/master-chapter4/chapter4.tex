\chapter{A hybrid quark-hadron model}
\label{chap:hybrid}

\TODO{try two-flavor hybrid star!}

\TODO{reflect changes with denser mass-radius plot in the text}

\begin{figure}[th!]
\centering
\tikzsetnextfilename{hybrid-star-illustration}
\begin{tikzpicture}
\draw[draw=black, fill=blue!  0!purple, circular glow={fill=blue!0!purple}] circle [radius=90pt]; % \node [rotate=45] at (135:37.5pt) {H}; \node [] at ( 90:37.5pt) {$\cdots$}; \node [, rotate=-45] at (45:37.5pt) {Fe};
\node [align=center] at (0pt, 65pt) {hadronic phase\\($n,p,\ldots$)};

\draw[draw=none, fill=green!60!black, circular glow={fill=green!60!black}] circle [radius=40pt];
\node [align=center] at (0pt, 0pt) {quark phase\\($u,d,s$)};

\draw[draw=none] (-120pt, -120pt) rectangle (+120pt, +120pt);
\end{tikzpicture}
\caption{\label{fig:hybrid:illustration}%
	A hybrid star composed of a core with strange quark matter surrounded by a hadronic phase.
	Quark cores are thought to be very small compared to the hadronic region, and the drawing is not to scale in this regard!
}
\end{figure}

In the preceding chapters we have modeled stars consisting of pure quark matter all the way from the center to the surface.
These models have accounted for the chiral symmetry of quantum chromodynamics,
but confinement has been imitated poorly by a mere bag constant $B$.
As one approaches the surface of a star,
the effects of confinement becomes increasingly important,
making hadrons the essential components instead of individual quarks.
The models we have used are therefore reliable only in the dense core of the stars,
and it is unrealistic to see the pure quark stars we have modeled in nature.

Moreover, one crucial aspect of modeling pure quark stars is inconsistent.
To calculate the lower bag constant bound, we assumed that two-flavor quark matter is unstable compared to hadronic matter at zero pressure.
But all the quark stars we modeled consisted of two-flavor quark matter close to and at the surface,
where the pressure vanishes by definition!
This is incompatible with the original assumption,
so a physical quark star must really be have a crust of hadronic matter near the surface.

To mitigate these problems it is common to model \textbf{hybrid stars} that consist of
quark matter only in the core where the density is high, 
surrounded by hadronic matter closer to the surface where the density is low.
In our final piece of work
we assemble a hybrid star modeled with the \textbf{three-flavor quark-meson model} from \cref{chap:lsm3f} in the core,
and with the representative \textbf{Akmal-Pandharipande-Ravenhall} (\textbf{APR}) hadronic equation of state from \cite{ref:apr} in the outer region.
Both equations of state are found at zero temperature, in chemical equilibrium and $\beta$-equilibrium and subject to global electrical charge neutrality.

\section{Construction of the hybrid equation of state}
\label{sec:hybrid:construction}

The individual equations of state for the two models are available to us from two different sources:
the three-flavor quark-meson model from the results in \cref{chap:lsm3f},
and the APR equation of state from data points at \cite{ref:apr_data}.
Data sets from both sources include the baryon chemical potential $\mu_B$, baryon density $n_B$, pressure $P$ and energy density $\epsilon$.

To join the two equations of state for the hadron phase $H$ and quark phase $Q$ into one hybrid version,
we apply the coarsest procedure outlined in \cite[section V-C]{ref:quark_star_review}:
\begin{enumerate}
\item \label{step:hybrid:one}%
      Plot the pressures $P_H(\mu_B)$ and $P_Q(\mu_B)$ as functions of the baryon chemical potential $\mu_B$.
\item \label{step:hybrid:two}%
      Find the chemical potential $\mu_B^0$ at which the curves $P_Q(\mu_B^0) = P_H(\mu_B^0)$ intersect.
      It typically corresponds to baryon densities $n_B^0 = n_B(\mu_B^0)$ in the range $2 n_\text{sat} \lesssim n_B^0 \lesssim 5 n_\text{sat}$.
\item \label{step:hybrid:three}%
      Compute the hybrid pressures and energy densities
      \begin{equation}
          P(\mu_B) = \begin{cases} P_H(\mu_B) & (\mu_B \leq \mu_B^0) \\ P_Q(\mu_B) & (\mu_B \geq \mu_B^0)  \end{cases}
          \quad \text{and} \quad
          \epsilon(\mu_B) = \begin{cases} \epsilon_H(\mu_B) & (\mu_B \leq \mu_B^0) \\ \epsilon_Q(\mu_B) & (\mu_B \geq \mu_B^0) \end{cases}.
      \label{eq:hybrid:pressure_energy_density}
      \end{equation}
\item \label{step:hybrid:four}%
      Eliminate $\mu_B$ from $P(\mu_B)$ and $\epsilon(\mu_B)$ to obtain the hybrid equation of state $\epsilon(P)$.
\end{enumerate}
The philosophy behind this method is that the preferable phase at any baryon chemical potential $\mu_B$ is the one with the greater pressure $P$, or lowest grand potential $\Omega = -P$.

\begin{figure}
\centering
\tikzsetnextfilename{hybrid-flavor-eos}
\begin{tikzpicture}
\begin{groupplot}[
	group style={group size={1 by 3}, vertical sep=1.5cm},
	width=13cm, height=7cm,
	extra tick style={grid=major, grid style={dashed}},
	minor tick num=9,
	legend style={xshift=-0.2cm, fill=none},
]
\nextgroupplot[
	xlabel={$\mu_B \, / \, \si{\mega\electronvolt}$},
	ylabel={$P \, / \, (\si{\giga\electronvolt\per\femto\meter\cubed})$},
	xmin=900, xmax=1700, restrict x to domain=800:2100, 
	ymin=0, ymax=0.4, restrict y to domain=0:5,
	title={Hybrid equation of state },
	title style={yshift=+0.9cm},
	legend style={anchor=south, at={(0.5,1.04)}}, legend columns=3,
];
\addplot+ [red, solid, very thick] table [x=muB2,y=P] {../code/data/LSM3F_APR/eos_sigma_600.dat}; \addlegendentry{quark phase \quad};
\addplot+ [blue!50!cyan, solid, very thick] table [x=muB1,y=P1] {../code/data/LSM3F_APR/eos_sigma_600.dat}; \addlegendentry{hadron phase \quad};
\addplot+ [black, dashed, very thick] table [x=muB,y=P] {../code/data/LSM3F_APR/eos_sigma_600.dat}; \addlegendentry{hybrid phase \quad};
\node at (1305, 0.2) [circle, fill, inner sep=1.5pt] {};

\nextgroupplot[
	xlabel={$\mu_B \, / \, \si{\mega\electronvolt}$},
	ylabel={$n_B \, / \, (\si{\per\femto\meter\cubed})$},
	xmin=900, xmax=1700, restrict x to domain=800:2100, 
	ymin=0, ymax=10, ytick distance=1, restrict y to domain=-1:12,
];
\addplot+ [red, solid, very thick] table [x=muB,y expr={\thisrow{nB2}/0.165}] {../code/data/LSM3F_APR/eos_sigma_600.dat};
\addplot+ [blue!50!cyan, solid, very thick] table [x=muB1,y expr={\thisrow{nB1}/0.165}] {../code/data/LSM3F_APR/eos_sigma_600.dat};
\addplot+ [black, dashed, very thick] table [x=muB,y expr={\thisrow{nB}/0.165}] {../code/data/LSM3F_APR/eos_sigma_600.dat};

\nextgroupplot[
	xlabel={$P        \, / \, (\si{\giga\electronvolt\per\femto\meter\cubed})$},
	ylabel={$\epsilon \, / \, (\si{\giga\electronvolt\per\femto\meter\cubed})$},
	xmin=0, xmax=0.4, ymin=0, ymax=2.0, xtick distance=0.1, ytick distance=1.0, minor y tick num=9, restrict x to domain=0:5, restrict y to domain=0:5,
];
\addplot+ [red, solid, very thick] table [x=P,y=epsilon2] {../code/data/LSM3F_APR/eos_sigma_600.dat};
\addplot+ [blue!50!cyan, solid, very thick] table [x=P1,y=epsilon1] {../code/data/LSM3F_APR/eos_sigma_600.dat};
\addplot+ [black, dashed, very thick] table [x=P,y=epsilon] {../code/data/LSM3F_APR/eos_sigma_600.dat};
\end{groupplot}
\end{tikzpicture}
\caption{\label{fig:hybrid:eos}%
Hybrid equation of state obtained
by combining the three-flavor quark-meson model (with $m_\sigma=\SI{600}{\mega\electronvolt}$ and $B^\frac14 = \SI{27}{\mega\electronvolt}$) for the core
with the Akmal-Pandharipande-Ravenhall hadronic equation of state for the outer region,
according to steps \ref{step:hybrid:one}--\ref{step:hybrid:four} in \cref{sec:hybrid:construction}.
}
\end{figure}

\Cref{fig:hybrid:eos} shows the step-by-step construction of a hybrid equation of state that joints the three-flavor quark-meson model and hadronic APR equation of state.
In this case, where we have used $m_\sigma=\SI{600}{\mega\electronvolt}$ and $B^\frac14 = \SI{111}{\mega\electronvolt}$ in the quark-meson model,
the pressures intersect at $\mu_B \approx \SI{1300}{\mega\electronvolt}$, which corresponds to a density $n_B \approx 4 n_\text{sat}$ in the hadronic phase and $n_B \approx 5.5 n_\text{sat}$ in the quark phase.
Due to the jump in density, a discontinuous phase transition between the two develops and is also visible at $\SI{0.19}{\giga\electronvolt\per\femto\meter\cubed} \lesssim P \lesssim \SI{0.20}{\giga\electronvolt\per\femto\meter\cubed}$ in the final hybrid equation of state $\epsilon(P)$.
The transition takes the form of a line of constant vapor pressure, like in the Maxwell construction.

The biggest problem with this rough splicing method is that it assumes the equations of state for both phases to be valid near the intersection point.
In reality it is unreliable to compare them across the entire range of densities from hadronic to quark matter.
A more sophisticated approach, also described in \cite[section V-F]{ref:quark_star_review},
is to restrict each equation of state to its domain of validity,
and then smoothly interpolate between them in the intermediate range where neither of them are trustworthy.
As already mentioned, we restrict ourselves to the simplest method,
wishing only to illustrate the mere possibility of using the three-flavor quark-meson model in a hybrid equation of state.

In our case,
the problems related to comparison of the two phases is emphasized by the presence of a \emph{second} intersection point at a lower chemical potential.
This means that the hadron phase does not have the greatest pressure for all chemical potentials below the intersection point!
Assuming that the quark matter equation of state is unreliable in this regime anyway, we simply gloss over this small surprise.

The detailed numerical implementation can be found in \cref{sec:num:qstars2f}.

\section{Stellar solutions}

\TODO{add band of two stars in \cite{ref:quark_hybrid_additional_ref}}

\begin{figure}
\centering

\tikzsetnextfilename{hybrid-flavor-mass-radius}
\begin{tikzpicture}
\begin{groupplot}[
	group style={group size={2 by 1}, vertical sep=0cm, horizontal sep=1.0cm},
	height=7cm,
	point meta=explicit, point meta min=33, point meta max=36,
	%colorbar horizontal, colormap name=plasmarev, colorbar style={xlabel=$\log_{10} (P_c \, / \, \si{\pascal})$, xtick distance=1, minor x tick num=9, at={(0.5,1.03)}, anchor=south, xticklabel pos=upper},
	/tikz/declare function={
		e0 = 4.266500881855304e+37;
	},
]
\tikzset{
	Bpin/.style={gray, sloped, allow upside down=true, rotate=180, yshift=+0.4cm, font=\small},
}
\nextgroupplot[
	width=10cm, 
	xlabel={$R \, / \, \si{\kilo\meter}$ },
	ylabel={$M \, / \, M_\odot$}, %title={Mass-radius diagram for 2-flavor quark stars }, title style={yshift=2.0cm},
	xmin=5, xmax=35, ymin=0.5, ymax=2.5, xtick distance=5, ytick distance=0.5, minor tick num=4, grid=major,
	%title = {$m_\sigma = \SI{600}{\mega\electronvolt}$, $B^\frac14 = \SI{111}{\mega\electronvolt}$ },
	legend columns=1, legend pos=south east, legend cell align=right,
	colorbar horizontal, colormap name=plasmarev, colorbar style={width=11cm, ylabel=$\log_{10} (P_c \, / \, \si{\pascal})$, ylabel style={rotate=-90}, xtick distance=1, minor x tick num=9, at={(0.8,-0.3)}, anchor=north, xticklabel pos=lower},
];
\addplot+ [solid, gray] table [x=R, y=M, meta expr={log10(\thisrow{P}*e0)}] {../code/data/LSM3F_APR/stars_hadron.dat}; \addlegendentry{neutron stars};
\addplot+ [solid, each nth point=8, mesh, mesh line legend] table [x=R, y=M, meta expr={log10(\thisrow{P}*e0)}] {../code/data/LSM3F_APR/stars_sigma_600_B14_111.dat}; \addlegendentry{hybrid stars};
\addplot+ [solid, each nth point=8, mesh] table [x=R, y=M, meta expr={log10(\thisrow{P}*e0)}] {../code/data/LSM3F_APR/stars_sigma_700_B14_68.dat}; % node [Bpin, pos=0.920] {$B = (\SI{27}{\mega\electronvolt})^4$};
\addplot+ [solid, each nth point=8, mesh] table [x=R, y=M, meta expr={log10(\thisrow{P}*e0)}] {../code/data/LSM3F_APR/stars_sigma_800_B14_27.dat}; % node [Bpin, pos=0.920] {$B = (\SI{27}{\mega\electronvolt})^4$};
\draw [Latex-] (11.45,2.07) -- (15.0,2.20) node [anchor=west] {$m_\sigma=\SI{800}{\mega\electronvolt}$, $\smash{B^\frac14} = \SI{27}{\mega\electronvolt}$};
\draw [Latex-] (11.55,1.96) -- (15.0,2.00) node [anchor=west] {$m_\sigma=\SI{700}{\mega\electronvolt}$, $\smash{B^\frac14} = \SI{68}{\mega\electronvolt}$};
\draw [Latex-] (11.65,1.90) -- (15.0,1.80) node [anchor=west] {$m_\sigma=\SI{600}{\mega\electronvolt}$, $\smash{B^\frac14} = \SI{111}{\mega\electronvolt}$};
\node[scale=0.75] at (11.226, 2.07) {\goldenstar};

\nextgroupplot[
	width=5cm, 
	xlabel={$R \, / \, \si{\kilo\meter}$},
	xmin=11, xmax=12, ymin=1.8, ymax=2.1, xtick distance=1, ytick distance=0.1, minor tick num=9, %grid=major,
	extra y ticks={2}, extra y tick labels={,,}, extra tick style={grid=major},
	%title = {$m_\sigma = \SI{600}{\mega\electronvolt}$, $B^\frac14 = \SI{111}{\mega\electronvolt}$ },
	legend columns=2, legend style={anchor=north, at={(0.5, 0.96)}},
];
\addplot+ [solid, gray] table [x=R, y=M, meta expr={log10(\thisrow{P}*e0)}] {../code/data/LSM3F_APR/stars_hadron.dat}; % node [Bpin, pos=0.920] {$B = (\SI{27}{\mega\electronvolt})^4$};
\addplot+ [solid, each nth point=8, mesh] table [x=R, y=M, meta expr={log10(\thisrow{P}*e0)}] {../code/data/LSM3F_APR/stars_sigma_600_B14_111.dat}; % node [Bpin, pos=0.920] {$B = (\SI{27}{\mega\electronvolt})^4$};
\addplot+ [solid, each nth point=8, mesh] table [x=R, y=M, meta expr={log10(\thisrow{P}*e0)}] {../code/data/LSM3F_APR/stars_sigma_700_B14_68.dat}; % node [Bpin, pos=0.920] {$B = (\SI{27}{\mega\electronvolt})^4$};
\addplot+ [solid, each nth point=8, mesh] table [x=R, y=M, meta expr={log10(\thisrow{P}*e0)}] {../code/data/LSM3F_APR/stars_sigma_800_B14_27.dat}; % node [Bpin, pos=0.920] {$B = (\SI{27}{\mega\electronvolt})^4$};
\node[scale=0.75] at (11.226, 2.07) {\goldenstar};
\end{groupplot}
\end{tikzpicture}

\caption{\label{fig:hybrid:mass-radius}%
	Mass-radius solutions of the Tolman-Oppenheimer-Volkoff equations \eqref{eq:tov:tovsys} parametrized by the central pressure $P_c$,
	obtained with three hybrid equations of state like the one in \cref{fig:hybrid:eos} with different values of $m_\sigma$ and $B$ in the three-flavor quark-meson model.
}

\bigskip

\tikzsetnextfilename{hybrid-flavor-extreme-star}
\begin{tikzpicture}
\begin{groupplot}[
	group style={group size={2 by 2}, horizontal sep=1.2cm, vertical sep=0.4cm},
	width=8cm, height=6cm,
	ylabel style={yshift=-0.2cm},
	enlargelimits=false, xtick distance=1.0, minor xtick={0,0.1,...,11.6},
	legend cell align=left,
]
\nextgroupplot[
	xticklabels={,,},
	ymax=2.0, ytick distance=0.5, minor y tick num=4,
	ylabel={$\{\epsilon,P\} \, / \, (\si{\giga\electronvolt\per\femto\meter\cubed})$ },
];
\addplot+ [blue] table [x=r, y=epsilon] {../code/data/LSM3F_APR/star_sigma_800_B14_27_Pc_0.0012587.dat}; \addlegendentry{$\epsilon$};
\addplot+ [red] table [x=r, y=P] {../code/data/LSM3F_APR/star_sigma_800_B14_27_Pc_0.0012587.dat}; \addlegendentry{$P$};
\nextgroupplot[
	ylabel=$m \, / \, M_\odot$,
	xticklabels={,,},
	ymax=2.1, ytick distance=0.5, minor y tick num=4,
];
\addplot+ [black] table [x=r, y=m] {../code/data/LSM3F_APR/star_sigma_800_B14_27_Pc_0.0012587.dat};
\nextgroupplot[
	xlabel=$r \, / \, \si{\kilo\meter}$,
	ylabel=$\mu_B \, / \, \si{\giga\electronvolt}$,
	ymin=0.9, ymax=1.5, ytick distance=0.1, minor y tick num=4, restrict y to domain=0.8:1.6,
];
\addplot+ [black] table [x=r, y expr={\thisrow{muQ}*3.0/1000.0}] {../code/data/LSM3F_APR/star_sigma_800_B14_27_Pc_0.0012587.dat};
\nextgroupplot[
	xlabel=$r \, / \, \si{\kilo\meter}$,
	ylabel=$n_i \, / \, n_\text{sat}$,
	ymax=15, ytick distance=5, minor y tick num=4,
];
\addplot+ [red, x filter/.expression={\thisrow{r}<1.34 ? x : nan}] table [x=r, y=nu] {../code/data/LSM3F_APR/star_sigma_800_B14_27_Pc_0.0012587.dat}; \addlegendentry{$n_u$};
\addplot+ [darkgreen, x filter/.expression={\thisrow{r}<1.34 ? x : nan}] table [x=r, y=nd] {../code/data/LSM3F_APR/star_sigma_800_B14_27_Pc_0.0012587.dat}; \addlegendentry{$n_d$};
\addplot+ [purple, x filter/.expression={\thisrow{r}<1.34 ? x : nan}] table [x=r, y=ns] {../code/data/LSM3F_APR/star_sigma_800_B14_27_Pc_0.0012587.dat}; \addlegendentry{$n_s$};
%\addplot+ [blue, x filter/.expression={\thisrow{r}<1.34}] table [x=r, y=ne] {../code/data/LSM3F_APR/star_sigma_800_B14_27_Pc_0.0012587.dat}; \addlegendentry{$n_e$};
\addplot+ [black, dashed, forget plot] table [x=r, y expr={(\thisrow{nu}+\thisrow{nd}+\thisrow{ns})/3}] {../code/data/LSM3F_APR/star_sigma_800_B14_27_Pc_0.0012587.dat};
\addplot+ [black, solid, forget plot, x filter/.expression={\thisrow{r}>=1.33 ? x : nan}] table [x=r, y expr={(\thisrow{nu}+\thisrow{nd}+\thisrow{ns})/3}] {../code/data/LSM3F_APR/star_sigma_800_B14_27_Pc_0.0012587.dat};
% cheat legend
\addplot+ [draw=none, black, solid] table [x=r, y expr={(\thisrow{nu}+\thisrow{nd}+\thisrow{ns})/3}] {../code/data/LSM3F_APR/star_sigma_800_B14_27_Pc_0.0012587.dat}; \addlegendentry{$n_B$};
\end{groupplot}
\node (title) at ($(group c1r1.north)!0.5!(group c2r1.north)$) [above, yshift=\pgfkeysvalueof{/pgfplots/every axis title shift}] {\goldenstar Maximum mass star profile ($m_\sigma=\SI{800}{\mega\electronvolt}$, $B^\frac14 = \SI{27}{\mega\electronvolt}$, $P_c=10^{34.730} \, \si{\pascal}$)};
\end{tikzpicture}
\caption{\label{fig:hybrid:star}%
	Radial profiles for the
	pressure $P$,
	energy density $\epsilon$,
	cumulative mass $m$,
	baryon chemical potential $\mu_B$
	and particle densities $n_i$
	for the maximum mass hybrid star \goldenstar in \cref{fig:hybrid:mass-radius}.
}

\end{figure}

Solving the Tolman-Oppenheimer-Volkoff with hybrid equations of state obtained in this manner
for $m_\sigma=\{600,700,800\} \, \si{\mega\electronvolt}$ and the lowermost bag constants \eqref{eq:lsm:bag_lower_bound} that respect instability of two-flavor quark matter,
we obtain the mass-radius curves in \cref{fig:hybrid:mass-radius}:
\begin{itemize}
\item After onset of the quark phase, the mass-radius curve develops new branches below the maximum hadron-only neutron star
      with maximum masses $1.90 M_\odot \leq M \lesssim 2.07 \leq M_\odot$.
\item At first glance, the new branches seem to plummet in mass straight after the branch point and indicate that all the hybrid stars are unstable.
      However, close inspection of the right panel in \cref{fig:hybrid:mass-radius} shows that the branches begin with a \emph{tiny} segment of stable stars whose masses increase up to the maximum.
      This supports the general claim in \cite{ref:quark_star_review} that quark cores are rare and have a very small size and mass, if at all existent.
\end{itemize}

Qualitatively, the radial profiles of the three maximum mass stars are very similar.
We take a detailed look at the one for $m_\sigma=\SI{800}{\mega\electronvolt}$ and $B^\frac14 = \SI{27}{\mega\electronvolt}$ in \cref{fig:hybrid:star}:
\begin{itemize}
\item The quark core has a central baryon density $n_B = 7 n_\text{sat}$ and extends $R'=\SI{1.3}{\kilo\meter} = 0.12 R$ from the center with a mass of only $M' = m(R') = 0.01 M_\odot = 0.005 M$.
      It contains three-flavor quark matter with up, down and strange quarks.
\item The phase transition from quark to hadronic matter takes place at $R'=\SI{1.3}{\kilo\meter}$ and entails huge drops $\Delta\epsilon = \SI{0.8}{\giga\electronvolt\per\femto\meter\cubed}$ and $\Delta n_B = 2 n_\text{sat}$ in energy and baryon density.
      Physically, stars are simply not able to support these dramatic discontinuous phase transitions, and the hybrid branches are therefore almost purely unstable.
\item The hadronic phase dominates, making up $\SI{99.8}{\percent}$ of the star's volume and $\SI{99.5}{\percent}$ of its mass.
\end{itemize}

\section{Summary}

In this chapter we have constructed hybrid stars modeled with the three-flavor quark-meson model in the core
and the Akmal-Pandharipande-Ravenhall hadronic equation of state in the outer region.
We saw that the hadronic mass-radius relation developed a very short branch of stable hybrid stars,
but a discontinuous transition between the two phases destabilized the stars immediately thereafter.
The hybrid branch realized maximum masses $1.90 M_\odot \lesssim M \lesssim 2.07 M_\odot$ depending on the $\sigma$-meson mass $m_\sigma$ used in the quark-meson model.
These results are in good agreement with hybrid stars modeled with the hadronic Walecka model and Nambu-Jona-Lasinio model for quark matter by \cite{ref:hybrid_stars_njl}.
They also find that the stability of the hybrid stars increases if the phase transition is smoothened out with the Gibbs construction,
making it very interesting to see how a more sophisticated construction of the hybrid equation of state would play out with our model.
This should certainly be studied in further work.
