\chapter{A Hybrid Quark-Hadron Model}
\label{chap:hybrid}

\begin{figure}[th!]
\centering
\tikzsetnextfilename{hybrid-star-illustration}
\begin{tikzpicture}
\draw[draw=black, fill=blue!  0!purple, circular glow={fill=blue!0!purple}] circle [radius=80pt]; % \node [rotate=45] at (135:37.5pt) {H}; \node [] at ( 90:37.5pt) {$\cdots$}; \node [, rotate=-45] at (45:37.5pt) {Fe};
\node [align=center] at (0pt, 55pt) {hadronic phase\\($n,p,\ldots$)};

\draw[draw=none, fill=green!60!black, circular glow={fill=green!60!black}] circle [radius=30pt];
\node [align=center] at (0pt, 0pt) {quark phase\\($u,d,s$)};

\draw[draw=none] (-100pt, -100pt) rectangle (+100pt, +100pt);
\end{tikzpicture}
\caption{\label{fig:hybrid:illustration}%
	A hybrid star with a strange quark core surrounded by a hadronic envelope.
	Compared to the hadronic phase,
	quark cores are thought to be much smaller than shown here.
}
\end{figure}

In the preceding chapters we modeled pure quark stars
consisting of deconfined quark matter out to the surface.
These models accounted for the chiral symmetry breaking of quantum chromodynamics.
They also featured a bag constant
that generally allowed for negative pressure in the equation of state,
therefore often interpreted as a confinement mechanism.
However, the pressure in the stars were restricted to \emph{non-negative} values,
so they contained only \emph{deconfined} quarks.
According to quantum chromodynamics,
quarks are confined in hadrons at low density and temperature,
so with the possible exception of pure strange quark stars,
they are unlikely to be seen in nature.

\iffalse
To calculate the lower bag constant bound \eqref{eq:lsm:bag_lower_bound},
we even assumed that two-flavor quark matter is unstable compared to hadronic matter at zero pressure.
But the surface has zero pressure \emph{by definition},
so it is inconsistent that two-flavor quark matter exists there!
\fi

More realistically,
quark matter can be found in high-density cores of \textbf{hybrid (neutron) stars}
surrounded by envelopes of low-density hadronic matter.
The recent observations \cite{ref:antoniadis,ref:arzoumanian,ref:fonseca}
of the heavy pulsars PSR J0348$+$0432, PSR J1614$-$2230 and PSR J0740$+$6620
with masses $M = \{2.01 \pm 0.04, 1.91 \pm 0.02, 2.08 \pm 0.07\} \, M_\odot$ 
suggest that real neutron stars could indeed reach sufficient densities
for deconfined quark matter to form in the core.
Due to these observations, hybrid star models are benchmarked by their ability to pass the $2 M_\odot$-limit.

In our final undertaking,
we assemble hybrid stars with the \textbf{quark-meson model} from \cref{chap:lsm2f,chap:lsm3f} in the core,
surrounded by a phase governed by the representative \textbf{Akmal-Pandharipande-Ravenhall} (\textbf{APR}) hadronic equation of state from \cite{ref:apr}.
All equations of state are found at zero temperature, in $\beta$-equilibrium and subject to local charge neutrality.

\textit{This chapter is inspired by reference \cite{ref:quark_star_review}.}

\section{Construction of the hybrid equation of state}
\label{sec:hybrid:construction}

The individual equations of state for the quark and hadronic phases are available to us from two different sources:
the two-flavor and three-flavor quark-meson model from \cref{chap:lsm2f,chap:lsm3f},
and the APR equation of state from data points at \cite{ref:apr_data}.
Both sources give us the baryon chemical potential $\mu_B$, baryon density $n_B$, pressure $P$ and energy density $\epsilon$.

To join the two equations of state for the hadron phase $H$ and quark phase $Q$ into one hybrid version,
we apply the coarsest procedure outlined in \cite[section V-C]{ref:quark_star_review}:
\begin{enumerate}
\item \label{step:hybrid:one}%
      Plot the pressures $P_H(\mu_B)$ and $P_Q(\mu_B)$ as functions of the baryon chemical potential $\mu_B$.
\item \label{step:hybrid:two}%
      Find the chemical potential $\mu_B^0$ at which the curves $P_Q(\mu_B^0) = P_H(\mu_B^0)$ intersect.
      It typically corresponds to baryon densities $n_B^0 = n_B(\mu_B^0)$ in the range $2 n_\text{sat} \lesssim n_B^0 \lesssim 5 n_\text{sat}$.
\item \label{step:hybrid:three}%
      Compute the hybrid pressures and energy densities
      \begin{equation}
          P(\mu_B) = \begin{cases} P_H(\mu_B) & (\mu_B \leq \mu_B^0) \\ P_Q(\mu_B) & (\mu_B > \mu_B^0)  \end{cases}
          \quad \text{and} \quad
          \epsilon(\mu_B) = \begin{cases} \epsilon_H(\mu_B) & (\mu_B \leq \mu_B^0) \\ \epsilon_Q(\mu_B) & (\mu_B > \mu_B^0) \end{cases}.
      \label{eq:hybrid:pressure_energy_density}
      \end{equation}
\item \label{step:hybrid:four}%
      Eliminate $\mu_B$ from $P(\mu_B)$ and $\epsilon(\mu_B)$ to obtain the hybrid equation of state $\epsilon(P)$.
\end{enumerate}
Effectively, the equations of state $\epsilon_H(P)$ and $\epsilon_Q(P)$ of each phase are concatenated into one equation of state $\epsilon(P)$
that prefers the phase with greater pressure for a given baryon chemical potential $\mu_B$.
The philosophy behind the method is that at any baryon chemical potential $\mu_B$,
nature prefers the phase with lower grand potential $\Omega = -P$.

\iffalse
\begin{figure}
\centering
\tikzsetnextfilename{hybrid-flavor-eos-2}
\begin{tikzpicture}
\begin{groupplot}[
	group style={group size={1 by 3}, vertical sep=1.5cm},
	width=13cm, height=7.5cm,
	extra tick style={grid=major, grid style={dashed}},
	minor tick num=9,
	legend columns=1, legend pos=north west, legend cell align=left,
]
\nextgroupplot[
	xlabel={$\mu_B \, / \, \si{\mega\electronvolt}$},
	ylabel={$P \, / \, (\si{\giga\electronvolt\per\femto\meter\cubed})$},
	xmin=900, xmax=1700, restrict x to domain=800:2100, 
	ymin=0, ymax=0.4, restrict y to domain=0:5,
	title={Two-flavor hybrid equation of state}, % ($m_\sigma=\SI{600}{\mega\electronvolt}$, $B^\frac14 = \SI{111}{\mega\electronvolt}$)},
];
\addplot+ [red, solid, very thick] table [x=muB2,y=P] {../code/data/LSM2F_APR/eos_sigma_600.dat}; \addlegendentry{quark phase \quad};
\addplot+ [blue!50!cyan, solid, very thick] table [x=muB1,y=P1] {../code/data/LSM2F_APR/eos_sigma_600.dat}; \addlegendentry{hadron phase \quad};
\addplot+ [black, dashed, very thick] table [x=muB,y=P] {../code/data/LSM2F_APR/eos_sigma_600.dat}; \addlegendentry{hybrid phase \quad};
\node at (1359, 0.241) [circle, fill, inner sep=1.5pt, label={[anchor=north west, font=\small]below:$\big(\mu_B^0,P^0\big)$}] {};

\nextgroupplot[
	xlabel={$\mu_B \, / \, \si{\mega\electronvolt}$},
	ylabel={$n_B \, / \, n_\text{sat}$},
	xmin=900, xmax=1700, restrict x to domain=800:2100, 
	ymin=0, ymax=10, ytick distance=1, restrict y to domain=-1:12,
];
\addplot+ [red, solid, very thick] table [x=muB,y expr={\thisrow{nB2}/0.165}] {../code/data/LSM2F_APR/eos_sigma_600.dat};
\addplot+ [blue!50!cyan, solid, very thick] table [x=muB1,y expr={\thisrow{nB1}/0.165}] {../code/data/LSM2F_APR/eos_sigma_600.dat};
\addplot+ [black, dashed, very thick] table [x=muB,y expr={\thisrow{nB}/0.165}] {../code/data/LSM2F_APR/eos_sigma_600.dat};
\node at (1359, 0.73/0.165) [circle, fill, inner sep=1.5pt, label={[anchor=north west, font=\small]below:$\big(\mu_B^0,n_{BH}^0\big)$}] {};
\node at (1359, 0.85/0.165) [circle, fill, inner sep=1.5pt, label={[anchor=south east, font=\small]above:$\big(\mu_B^0,n_{BQ}^0\big)$}] {};

\nextgroupplot[
	xlabel={$P        \, / \, (\si{\giga\electronvolt\per\femto\meter\cubed})$},
	ylabel={$\epsilon \, / \, (\si{\giga\electronvolt\per\femto\meter\cubed})$},
	xmin=0, xmax=0.4, ymin=0, ymax=2.0, xtick distance=0.1, ytick distance=1.0, minor y tick num=9, restrict x to domain=0:5, restrict y to domain=0:5,
];
\addplot+ [red, solid, very thick] table [x=P,y=epsilon2] {../code/data/LSM2F_APR/eos_sigma_600.dat};
\addplot+ [blue!50!cyan, solid, very thick] table [x=P1,y=epsilon1] {../code/data/LSM2F_APR/eos_sigma_600.dat};
\addplot+ [black, dashed, very thick] table [x=P,y=epsilon] {../code/data/LSM2F_APR/eos_sigma_600.dat};
\node at (0.241, 0.762) [circle, fill, inner sep=1.5pt, label={[anchor=north west, font=\small]below:$\big(P^0,\epsilon_H^0\big)$}] {};
\node at (0.241, 0.953) [circle, fill, inner sep=1.5pt, label={[anchor=south east, font=\small]above:$\big(P^0,\epsilon_Q^0\big)$}] {};
\end{groupplot}
\end{tikzpicture}
\end{figure}
\fi

\begin{figure}%\ContinuedFloat
\centering
\tikzsetnextfilename{hybrid-flavor-eos-3}
\begin{tikzpicture}
\begin{groupplot}[
	group style={group size={1 by 3}, vertical sep=1.5cm},
	width=13cm, height=7.5cm,
	extra tick style={grid=major, grid style={dashed}},
	minor tick num=9,
	legend columns=1, legend pos=north west, legend cell align=left,
]
\nextgroupplot[
	xlabel={$\mu_B \, / \, \si{\mega\electronvolt}$},
	ylabel={$P \, / \, (\si{\giga\electronvolt\per\femto\meter\cubed})$},
	xmin=900, xmax=1700, restrict x to domain=800:2100, 
	ymin=0, ymax=0.4, restrict y to domain=0:5,
	title={Construction of a three-flavor hybrid equation of state}, % ($m_\sigma=\SI{600}{\mega\electronvolt}$, $B^\frac14 = \SI{111}{\mega\electronvolt}$)},
];
\addplot+ [red, solid, very thick] table [x=muB2,y=P] {../code/data/LSM3F_APR/eos_sigma_600.dat}; \addlegendentry{quark phase \quad};
\addplot+ [blue!50!cyan, solid, very thick] table [x=muB1,y=P1] {../code/data/LSM3F_APR/eos_sigma_600.dat}; \addlegendentry{hadron phase \quad};
\addplot+ [black, dashed, very thick] table [x=muB,y=P] {../code/data/LSM3F_APR/eos_sigma_600.dat}; \addlegendentry{hybrid phase \quad};
\node at (1305, 0.2) [circle, fill, inner sep=1.5pt, label={[anchor=north west, font=\small]below:$\big(\mu_B^0,P^0\big)$}] {};

\nextgroupplot[
	xlabel={$\mu_B \, / \, \si{\mega\electronvolt}$},
	ylabel={$n_B \, / \, n_\text{sat}$},
	xmin=900, xmax=1700, restrict x to domain=800:2100, 
	ymin=0, ymax=10, ytick distance=1, restrict y to domain=-1:12,
];
\addplot+ [red, solid, very thick] table [x=muB,y expr={\thisrow{nB2}/0.165}] {../code/data/LSM3F_APR/eos_sigma_600.dat};
\addplot+ [blue!50!cyan, solid, very thick] table [x=muB1,y expr={\thisrow{nB1}/0.165}] {../code/data/LSM3F_APR/eos_sigma_600.dat};
\addplot+ [black, dashed, very thick] table [x=muB,y expr={\thisrow{nB}/0.165}] {../code/data/LSM3F_APR/eos_sigma_600.dat};
\node at (1307, 0.693/0.165) [circle, fill, inner sep=1.5pt, label={[anchor=north west, font=\small]below:$\big(\mu_B^0,n_{BH}^0\big)$}] {};
\node at (1307, 0.918/0.165) [circle, fill, inner sep=1.5pt, label={[anchor=south east, font=\small]above:$\big(\mu_B^0,n_{BQ}^0\big)$}] {};

\nextgroupplot[
	xlabel={$P        \, / \, (\si{\giga\electronvolt\per\femto\meter\cubed})$},
	ylabel={$\epsilon \, / \, (\si{\giga\electronvolt\per\femto\meter\cubed})$},
	xmin=0, xmax=0.4, ymin=0, ymax=2.0, xtick distance=0.1, ytick distance=1.0, minor y tick num=9, restrict x to domain=0:5, restrict y to domain=0:5,
];
\addplot+ [red, solid, very thick] table [x=P,y=epsilon2] {../code/data/LSM3F_APR/eos_sigma_600.dat};
\addplot+ [blue!50!cyan, solid, very thick] table [x=P1,y=epsilon1] {../code/data/LSM3F_APR/eos_sigma_600.dat};
\addplot+ [black, dashed, very thick] table [x=P,y=epsilon] {../code/data/LSM3F_APR/eos_sigma_600.dat};
\node at (0.202, 0.713) [circle, fill, inner sep=1.5pt, label={[anchor=north west, font=\small]below:$\big(P^0,\epsilon_H^0\big)$}] {};
\node at (0.202, 1.020) [circle, fill, inner sep=1.5pt, label={[anchor=south east, font=\small]above:$\big(P^0,\epsilon_Q^0\big)$}] {};
\end{groupplot}
\end{tikzpicture}
\caption{\label{fig:hybrid:eos}%
Hybrid equation of state built from
the three-flavor quark-meson model
with $m_\sigma=\SI{600}{\mega\electronvolt}$ and $B^\frac14 = \SI{111}{\mega\electronvolt}$
and the hadronic Akmal-Pandharipande-Ravenhall equation of state
with steps \ref{step:hybrid:one}--\ref{step:hybrid:four} in \cref{sec:hybrid:construction}.
}
\end{figure}

\begin{figure}
\centering
\tikzsetnextfilename{hybrid-flavor-eos-all}
\begin{tikzpicture}
\begin{groupplot}[
	group style={group size={2 by 1}, vertical sep=1cm},
	width=8cm, height=7cm,
	minor tick num=9,
	xlabel = {$P \, / \, (\si{\giga\electronvolt\per\femto\meter\cubed})$},
	xmin=0, ymin=0, xmax=0.5, ymax=3.5, restrict x to domain=-1:+4, restrict y to domain=-1:+4,
	legend columns=1, legend style={anchor=north, at={(0.5, 0.98)}},
	grid = both, minor grid style = {ultra thin, gray!10!white},
];
\nextgroupplot[
	title = {\subcaption{\label{fig:hybrid:eos-all-2}Two-flavor hybrid equations of state}},
	ylabel = {$\epsilon \, / \, (\si{\giga\electronvolt\per\femto\meter\cubed})$},
];
\addplot [red, solid] table [x=P, y=epsilon] {../code/data/LSM2F_APR/eos_sigma_600.dat};
\addlegendentry{$m_\sigma=\SI{600}{\mega\electronvolt}$, $\vphantom{\big)} \smash{B^\frac14} = \SI{111}{\mega\electronvolt}$};
\addplot [darkgreen, dashed] table [x=P, y=epsilon] {../code/data/LSM2F_APR/eos_sigma_700.dat};
\addlegendentry{$m_\sigma=\SI{700}{\mega\electronvolt}$, $\vphantom{\big)} \smash{B^\frac14} = \phantom{0}\SI{68}{\mega\electronvolt}$};
\addplot [blue, dotted] table [x=P, y=epsilon] {../code/data/LSM2F_APR/eos_sigma_800.dat};
\addlegendentry{$m_\sigma=\SI{800}{\mega\electronvolt}$, $\vphantom{\big)} \smash{B^\frac14} = \phantom{0}\SI{27}{\mega\electronvolt}$};

\nextgroupplot[
	title = {\subcaption{\label{fig:hybrid:eos-all-3}Three-flavor hybrid equations of state}},
];
\addplot [red, solid] table [x=P, y=epsilon] {../code/data/LSM3F_APR/eos_sigma_600.dat};
\addlegendentry{$m_\sigma=\SI{600}{\mega\electronvolt}$, $\vphantom{\big)} \smash{B^\frac14} = \SI{111}{\mega\electronvolt}$};
\addplot [darkgreen, dashed] table [x=P, y=epsilon] {../code/data/LSM3F_APR/eos_sigma_700.dat};
\addlegendentry{$m_\sigma=\SI{700}{\mega\electronvolt}$, $\vphantom{\big)} \smash{B^\frac14} = \phantom{0}\SI{68}{\mega\electronvolt}$};
\addplot [blue, dotted] table [x=P, y=epsilon] {../code/data/LSM3F_APR/eos_sigma_800.dat};
\addlegendentry{$m_\sigma=\SI{800}{\mega\electronvolt}$, $\vphantom{\big)} \smash{B^\frac14} = \phantom{0}\SI{27}{\mega\electronvolt}$};
\end{groupplot}
\end{tikzpicture}
\caption{\label{fig:hybrid:eos-all}%
	Six hybrid equations of state constructed like the one in \cref{fig:hybrid:eos}
	using the three parameter sets that generated the most massive quark stars
	with the two-flavor and three-flavor quark-meson models in \cref{chap:lsm2f,chap:lsm3f}.
}
\end{figure}

\Cref{fig:hybrid:eos} shows an example of the step-by-step construction of a hybrid equation of state
that joins the three-flavor quark-meson model
with the hadronic APR equation of state.
The pressures intersect at $\mu_B \approx \SI{1300}{\mega\electronvolt}$,
which corresponds to the density $n_{BH}^0 \approx 4 n_\text{sat}$ in the hadronic phase
and $n_{BQ}^0 \approx 5.5 n_\text{sat}$ in the quark phase.
The energy density drops by $\Delta \epsilon = \epsilon_Q^0 - \epsilon_H^0 = \SI{0.3}{\giga\electronvolt\per\femto\meter\cubed}$
in a discontinuous phase transition at $P = \SI{0.2}{\giga\electronvolt\per\femto\meter\cubed}$.
The transition takes the form of a line of constant vapor pressure, like in a Maxwell construction.
The construction is virtually identical with the two-flavor quark-meson model,
only with a lower drop $\Delta \epsilon$.

The biggest problem with this rough splicing method is that it assumes the equations of state for both phases to be valid near the intersection point.
In reality it is unreliable to compare them across the entire range of densities from hadronic to quark matter.
A more sophisticated approach described in \cite[section V-F]{ref:quark_star_review}
is to restrict each equation of state to its domain of validity,
and then smoothly interpolate between them in a clever way
in the intermediate range where neither of them are trustworthy.
We restrict ourselves to the simpler method above,
aiming only to investigate the mere possibility of using the quark-meson model in a hybrid star.

In our case,
the problems related to comparison of the two phases is emphasized
by the presence of a \emph{second} intersection point at a lower baryon chemical potential $\mu_B \approx \SI{1100}{\mega\electronvolt}$.
This means that the hadron phase does not have the greatest pressure for all chemical potentials below the first intersection point!
Assuming that the quark equation of state is unreliable in this regime anyway,
we simply gloss over this small inconsistency.
It would be utterly insane to go back to the quark phase near the surface.

In \cref{fig:hybrid:eos-all} we gather six equations of state constructed in this way
for $N_f = \{2,3\}$ and $m_\sigma = \{600,700,800\} \, \si{\mega\electronvolt}$
with the corresponding lowest bag constants satisfying the lower bounds \eqref{eq:lsm:bag_lower_bound}.
These are the same parameters we considered in earlier chapters that generated the most massive quark stars.
Note that the phase transition generally becomes more dramatic
with three quark flavors
and increasing $m_\sigma$,
accompanied by decreasing $B$.

The detailed numerical implementation can be found in \cref{sec:num:qstars2f}.

\section{Stellar solutions}

\begin{figure}[t!]
\centering

\tikzsetnextfilename{hybrid-2-flavor-mass-radius}
\begin{tikzpicture}
\begin{groupplot}[
	group style={group size={2 by 1}, vertical sep=0cm, horizontal sep=1.0cm},
	height=7cm,
	point meta=explicit, point meta min=33, point meta max=36,
	%colorbar horizontal, colormap name=plasmarev, colorbar style={xlabel=$\log_{10} (P_c \, / \, \si{\pascal})$, xtick distance=1, minor x tick num=9, at={(0.5,1.03)}, anchor=south, xticklabel pos=upper},
	/tikz/declare function={
		e0 = 4.266500881855304e+37;
	},
]
\tikzset{
	Bpin/.style={gray, sloped, allow upside down=true, rotate=180, yshift=+0.4cm, font=\small},
}
\nextgroupplot[
	width=10cm, 
	xlabel={$R \, / \, \si{\kilo\meter}$ },
	ylabel={$M \, / \, M_\odot$}, %title={Mass-radius diagram for 2-flavor quark stars }, title style={yshift=2.0cm},
	xmin=5, xmax=35, ymin=0.5, ymax=2.5, xtick distance=5, ytick distance=0.5, minor tick num=4, grid=major,
	legend columns=5, legend pos=south east, legend cell align=right, legend transposed,
	%colorbar horizontal, colormap name=plasmarev, colorbar style={width=11cm, ylabel=$\log_{10} (P_c \, / \, \si{\pascal})$, ylabel style={rotate=-90}, xtick distance=1, minor x tick num=9, at={(0.8,-0.3)}, anchor=north, xticklabel pos=lower},
];
\addplot [name path=J0748lo, draw=none, forget plot, domain=0:40] {2.08 - 0.07};
\addplot [name path=J0748hi, draw=none, forget plot, domain=0:40] {2.08 + 0.07};
\addplot [red, opacity=0.4] fill between [of=J0748lo and J0748hi]; \addlegendentry{PSR J0748$+$6620};

\addplot [name path=J0348lo, draw=none, forget plot, domain=0:40] {2.01 - 0.04};
\addplot [name path=J0348hi, draw=none, forget plot, domain=0:40] {2.01 + 0.04};
\addplot [green, opacity=0.4] fill between [of=J0348lo and J0348hi]; \addlegendentry{PSR J0348$+$0432};

\addplot [name path=J1614lo, draw=none, forget plot, domain=0:40] {1.91 - 0.02};
\addplot [name path=J1614hi, draw=none, forget plot, domain=0:40] {1.91 + 0.02};
\addplot [yellow, opacity=0.4] fill between [of=J1614lo and J1614hi]; \addlegendentry{PSR J1614$-$2230};

\addplot+ [solid, gray] table [x=R, y=M, meta expr={log10(\thisrow{P}*e0)}] {../code/data/LSM3F_APR/stars_hadron.dat}; \addlegendentry{neutron stars};
\addplot+ [solid, mesh, mesh line legend] table [x=R, y=M, meta expr={log10(\thisrow{P}*e0)}] {../code/data/LSM2F_APR/stars_sigma_600_B14_111.dat}; \addlegendentry{hybrid stars};
\addplot+ [solid, mesh] table [x=R, y=M, meta expr={log10(\thisrow{P}*e0)}] {../code/data/LSM2F_APR/stars_sigma_700_B14_68.dat}; % node [Bpin, pos=0.920] {$B = (\SI{27}{\mega\electronvolt})^4$};
\addplot+ [solid, mesh] table [x=R, y=M, meta expr={log10(\thisrow{P}*e0)}] {../code/data/LSM2F_APR/stars_sigma_800_B14_27.dat}; % node [Bpin, pos=0.920] {$B = (\SI{27}{\mega\electronvolt})^4$};
\draw [Latex-] (11.4,2.09) -- (15.0,2.20) node [anchor=west] {$m_\sigma=\SI{800}{\mega\electronvolt}$, $\smash{B^\frac14} = \SI{27}{\mega\electronvolt}$};
\draw [Latex-] (11.5,2.00) -- (15.0,2.00) node [anchor=west] {$m_\sigma=\SI{700}{\mega\electronvolt}$, $\smash{B^\frac14} = \SI{68}{\mega\electronvolt}$};
\draw [Latex-] (11.5,1.98) -- (15.0,1.80) node [anchor=west] {$m_\sigma=\SI{600}{\mega\electronvolt}$, $\smash{B^\frac14} = \SI{111}{\mega\electronvolt}$};
\node[scale=0.75] at (11.223, 1.999) {\goldenstar};

\nextgroupplot[
	width=5cm, 
	xlabel={$R \, / \, \si{\kilo\meter}$},
	xmin=11, xmax=12, ymin=1.85, ymax=2.15, xtick distance=1, ytick distance=0.1, minor tick num=9, %grid=major,
	extra y ticks={2}, extra y tick labels={,,}, extra tick style={grid=major},
];
\addplot [name path=J0748lo, draw=none, forget plot, domain=0:40] {2.08 - 0.07};
\addplot [name path=J0748hi, draw=none, forget plot, domain=0:40] {2.08 + 0.07};
\addplot [red, opacity=0.4] fill between [of=J0748lo and J0748hi];

\addplot [name path=J0348lo, draw=none, forget plot, domain=0:40] {2.01 - 0.04};
\addplot [name path=J0348hi, draw=none, forget plot, domain=0:40] {2.01 + 0.04};
\addplot [green, opacity=0.4] fill between [of=J0348lo and J0348hi];

\addplot [name path=J1614lo, draw=none, forget plot, domain=0:40] {1.91 - 0.02};
\addplot [name path=J1614hi, draw=none, forget plot, domain=0:40] {1.91 + 0.02};
\addplot [yellow, opacity=0.4] fill between [of=J1614lo and J1614hi];

\addplot+ [solid, gray] table [x=R, y=M, meta expr={log10(\thisrow{P}*e0)}] {../code/data/LSM3F_APR/stars_hadron.dat}; % node [Bpin, pos=0.920] {$B = (\SI{27}{\mega\electronvolt})^4$};
\addplot+ [solid, mesh] table [x=R, y=M, meta expr={log10(\thisrow{P}*e0)}] {../code/data/LSM2F_APR/stars_sigma_600_B14_111.dat}; % node [Bpin, pos=0.920] {$B = (\SI{27}{\mega\electronvolt})^4$};
\addplot+ [solid, mesh] table [x=R, y=M, meta expr={log10(\thisrow{P}*e0)}] {../code/data/LSM2F_APR/stars_sigma_700_B14_68.dat}; % node [Bpin, pos=0.920] {$B = (\SI{27}{\mega\electronvolt})^4$};
\addplot+ [solid, mesh] table [x=R, y=M, meta expr={log10(\thisrow{P}*e0)}] {../code/data/LSM2F_APR/stars_sigma_800_B14_27.dat}; % node [Bpin, pos=0.920] {$B = (\SI{27}{\mega\electronvolt})^4$};
\draw [Latex-] (11.4,2.09) -- (15.0,2.20) node [anchor=west] {$m_\sigma=\SI{800}{\mega\electronvolt}$, $\smash{B^\frac14} = \SI{27}{\mega\electronvolt}$};
\draw [Latex-] (11.5,2.00) -- (15.0,2.00) node [anchor=west] {$m_\sigma=\SI{700}{\mega\electronvolt}$, $\smash{B^\frac14} = \SI{68}{\mega\electronvolt}$};
\draw [Latex-] (11.5,1.98) -- (15.0,1.80) node [anchor=west] {$m_\sigma=\SI{600}{\mega\electronvolt}$, $\smash{B^\frac14} = \SI{111}{\mega\electronvolt}$};
\node[scale=0.75] at (11.223, 1.999) {\goldenstar};

\end{groupplot}
\node (title) at ($(group c1r1.north west)!0.5!(group c2r1.north east)$) [above, yshift=\pgfkeysvalueof{/pgfplots/every axis title shift}, text width=8cm] {\subcaption{\label{fig:hybrid:mass-radius-2}Two-flavor hybrid stars}};
\end{tikzpicture}

\bigskip

\tikzsetnextfilename{hybrid-3-flavor-mass-radius}
\begin{tikzpicture}
\begin{groupplot}[
	group style={group size={2 by 1}, vertical sep=0cm, horizontal sep=1.0cm},
	height=7cm,
	point meta=explicit, point meta min=33, point meta max=36,
	%colorbar horizontal, colormap name=plasmarev, colorbar style={xlabel=$\log_{10} (P_c \, / \, \si{\pascal})$, xtick distance=1, minor x tick num=9, at={(0.5,1.03)}, anchor=south, xticklabel pos=upper},
	/tikz/declare function={
		e0 = 4.266500881855304e+37;
	},
]
\tikzset{
	Bpin/.style={gray, sloped, allow upside down=true, rotate=180, yshift=+0.4cm, font=\small},
}
\nextgroupplot[
	width=10cm, 
	xlabel={$R \, / \, \si{\kilo\meter}$ },
	ylabel={$M \, / \, M_\odot$}, %title={Mass-radius diagram for 2-flavor quark stars }, title style={yshift=2.0cm},
	xmin=5, xmax=35, ymin=0.5, ymax=2.5, xtick distance=5, ytick distance=0.5, minor tick num=4, grid=major,
	legend columns=5, legend pos=south east, legend cell align=right, legend transposed,
	colorbar horizontal, colormap name=plasmarev, colorbar style={width=11cm, ylabel=$\log_{10} (P_c \, / \, \si{\pascal})$, ylabel style={rotate=-90}, xtick distance=1, minor x tick num=9, at={(0.8,-0.3)}, anchor=north, xticklabel pos=lower},
];
\addplot [name path=J0748lo, draw=none, forget plot, domain=0:40] {2.08 - 0.07};
\addplot [name path=J0748hi, draw=none, forget plot, domain=0:40] {2.08 + 0.07};
\addplot [red, opacity=0.4] fill between [of=J0748lo and J0748hi]; \addlegendentry{PSR J0748$+$6620};

\addplot [name path=J0348lo, draw=none, forget plot, domain=0:40] {2.01 - 0.04};
\addplot [name path=J0348hi, draw=none, forget plot, domain=0:40] {2.01 + 0.04};
\addplot [green, opacity=0.4] fill between [of=J0348lo and J0348hi]; \addlegendentry{PSR J0348$+$0432};

\addplot [name path=J1614lo, draw=none, forget plot, domain=0:40] {1.91 - 0.02};
\addplot [name path=J1614hi, draw=none, forget plot, domain=0:40] {1.91 + 0.02};
\addplot [yellow, opacity=0.4] fill between [of=J1614lo and J1614hi]; \addlegendentry{PSR J1614$-$2230};

\addplot+ [solid, gray] table [x=R, y=M, meta expr={log10(\thisrow{P}*e0)}] {../code/data/LSM3F_APR/stars_hadron.dat}; \addlegendentry{neutron stars};
\addplot+ [solid, mesh, mesh line legend] table [x=R, y=M, meta expr={log10(\thisrow{P}*e0)}] {../code/data/LSM3F_APR/stars_sigma_600_B14_111.dat}; \addlegendentry{hybrid stars};
\addplot+ [solid, mesh] table [x=R, y=M, meta expr={log10(\thisrow{P}*e0)}] {../code/data/LSM3F_APR/stars_sigma_700_B14_68.dat}; % node [Bpin, pos=0.920] {$B = (\SI{27}{\mega\electronvolt})^4$};
\addplot+ [solid, mesh] table [x=R, y=M, meta expr={log10(\thisrow{P}*e0)}] {../code/data/LSM3F_APR/stars_sigma_800_B14_27.dat}; % node [Bpin, pos=0.920] {$B = (\SI{27}{\mega\electronvolt})^4$};
\draw [Latex-] (11.45,2.08) -- (15.0,2.20) node [anchor=west] {$m_\sigma=\SI{800}{\mega\electronvolt}$, $\smash{B^\frac14} = \SI{27}{\mega\electronvolt}$};
\draw [Latex-] (11.55,1.97) -- (15.0,2.00) node [anchor=west] {$m_\sigma=\SI{700}{\mega\electronvolt}$, $\smash{B^\frac14} = \SI{68}{\mega\electronvolt}$};
\draw [Latex-] (11.65,1.90) -- (15.0,1.80) node [anchor=west] {$m_\sigma=\SI{600}{\mega\electronvolt}$, $\smash{B^\frac14} = \SI{111}{\mega\electronvolt}$};
\node[scale=0.75] at (11.467, 1.904) {\goldenstar};

\nextgroupplot[
	width=5cm, 
	xlabel={$R \, / \, \si{\kilo\meter}$},
	xmin=11, xmax=12, ymin=1.85, ymax=2.15, xtick distance=1, ytick distance=0.1, minor tick num=9, %grid=major,
	extra y ticks={2}, extra y tick labels={,,}, extra tick style={grid=major},
];
\addplot [name path=J0748lo, draw=none, forget plot, domain=0:40] {2.08 - 0.07};
\addplot [name path=J0748hi, draw=none, forget plot, domain=0:40] {2.08 + 0.07};
\addplot [red, opacity=0.4] fill between [of=J0748lo and J0748hi];

\addplot [name path=J0348lo, draw=none, forget plot, domain=0:40] {2.01 - 0.04};
\addplot [name path=J0348hi, draw=none, forget plot, domain=0:40] {2.01 + 0.04};
\addplot [green, opacity=0.4] fill between [of=J0348lo and J0348hi];

\addplot [name path=J1614lo, draw=none, forget plot, domain=0:40] {1.91 - 0.02};
\addplot [name path=J1614hi, draw=none, forget plot, domain=0:40] {1.91 + 0.02};
\addplot [yellow, opacity=0.4] fill between [of=J1614lo and J1614hi];

\addplot+ [solid, gray] table [x=R, y=M, meta expr={log10(\thisrow{P}*e0)}] {../code/data/LSM3F_APR/stars_hadron.dat}; % node [Bpin, pos=0.920] {$B = (\SI{27}{\mega\electronvolt})^4$};
\addplot+ [solid, mesh] table [x=R, y=M, meta expr={log10(\thisrow{P}*e0)}] {../code/data/LSM3F_APR/stars_sigma_600_B14_111.dat}; % node [Bpin, pos=0.920] {$B = (\SI{27}{\mega\electronvolt})^4$};
\addplot+ [solid, mesh] table [x=R, y=M, meta expr={log10(\thisrow{P}*e0)}] {../code/data/LSM3F_APR/stars_sigma_700_B14_68.dat}; % node [Bpin, pos=0.920] {$B = (\SI{27}{\mega\electronvolt})^4$};
\addplot+ [solid, mesh] table [x=R, y=M, meta expr={log10(\thisrow{P}*e0)}] {../code/data/LSM3F_APR/stars_sigma_800_B14_27.dat}; % node [Bpin, pos=0.920] {$B = (\SI{27}{\mega\electronvolt})^4$};
\draw [Latex-] (11.45,2.08) -- (15.0,2.20) node [anchor=west] {$m_\sigma=\SI{800}{\mega\electronvolt}$, $\smash{B^\frac14} = \SI{27}{\mega\electronvolt}$};
\draw [Latex-] (11.55,1.97) -- (15.0,2.00) node [anchor=west] {$m_\sigma=\SI{700}{\mega\electronvolt}$, $\smash{B^\frac14} = \SI{68}{\mega\electronvolt}$};
\draw [Latex-] (11.65,1.90) -- (15.0,1.80) node [anchor=west] {$m_\sigma=\SI{600}{\mega\electronvolt}$, $\smash{B^\frac14} = \SI{111}{\mega\electronvolt}$};
\node[scale=0.75] at (11.467, 1.904) {\goldenstar};

\end{groupplot}
\node (title) at ($(group c1r1.north west)!0.5!(group c2r1.north east)$) [above, yshift=\pgfkeysvalueof{/pgfplots/every axis title shift}, text width=8cm] {\phantomsubcaption\subcaption{\label{fig:hybrid:mass-radius-3}Three-flavor hybrid stars} };
\end{tikzpicture}

\caption{\label{fig:hybrid:mass-radius}%
	Mass-radius solutions of the Tolman-Oppenheimer-Volkoff equations \eqref{eq:master_intro:tov} parametrized by the central pressure $P_c$,
	using the hybrid equations of state in \cref{fig:hybrid:eos-all}.
	The colored bands show measured masses of three heavy pulsars from \cite{ref:antoniadis,ref:arzoumanian,ref:fonseca}.
}
\end{figure}

\begin{figure}
\centering
\tikzsetnextfilename{hybrid-2-flavor-extreme-star}
\begin{tikzpicture}
\begin{groupplot}[
	group style={group size={2 by 2}, horizontal sep=1.2cm, vertical sep=0.4cm},
	width=8cm, height=6cm,
	ylabel style={yshift=-0.2cm},
	enlargelimits=false, xtick distance=1.0, minor xtick={0,0.1,...,11.6},
	legend cell align=left,
]
\nextgroupplot[
	xticklabels={,,},
	ymax=1.5, ytick distance=0.5, minor y tick num=4,
	ylabel={$\{\epsilon,P\} \, / \, (\si{\giga\electronvolt\per\femto\meter\cubed})$ },
];
\addplot+ [blue] table [x=r, y=epsilon] {../code/data/LSM2F_APR/star_sigma_600_B14_111_Pc_0.0011807.dat}; \addlegendentry{$\epsilon$};
\addplot+ [red] table [x=r, y=P] {../code/data/LSM2F_APR/star_sigma_600_B14_111_Pc_0.0011807.dat}; \addlegendentry{$P$};
\nextgroupplot[
	ylabel=$m \, / \, M_\odot$,
	xticklabels={,,},
	ymax=2.1, ytick distance=0.5, minor y tick num=4,
];
\addplot+ [black] table [x=r, y=m] {../code/data/LSM2F_APR/star_sigma_600_B14_111_Pc_0.0011807.dat};
\nextgroupplot[
	xlabel=$r \, / \, \si{\kilo\meter}$,
	ylabel=$\mu_B \, / \, \si{\giga\electronvolt}$,
	ymin=0.9, ymax=1.5, ytick distance=0.1, minor y tick num=4, % restrict y to domain=0.8:1.6,
];
\addplot+ [black] table [x=r, y expr={\thisrow{muQ}*3.0/1000.0}] {../code/data/LSM2F_APR/star_sigma_600_B14_111_Pc_0.0011807.dat};
\nextgroupplot[
	xlabel=$r \, / \, \si{\kilo\meter}$,
	ylabel=$n_i \, / \, n_\text{sat}$,
	ymax=15, ytick distance=5, minor y tick num=4,
];
\addplot+ [red, x filter/.expression={\thisrow{r}<3.14 ? x : nan}] table [x=r, y=nu] {../code/data/LSM2F_APR/star_sigma_600_B14_111_Pc_0.0011807.dat}; \addlegendentry{$n_u$};
\addplot+ [darkgreen, x filter/.expression={\thisrow{r}<3.14 ? x : nan}] table [x=r, y=nd] {../code/data/LSM2F_APR/star_sigma_600_B14_111_Pc_0.0011807.dat}; \addlegendentry{$n_d$};
%\addplot+ [purple, x filter/.expression={\thisrow{r}<3.14 ? x : nan}] table [x=r, y=ns] {../code/data/LSM2F_APR/star_sigma_600_B14_111_Pc_0.0011807.dat}; \addlegendentry{$n_s$};
%\addplot+ [blue, x filter/.expression={\thisrow{r}<3.14}] table [x=r, y=ne] {../code/data/LSM2F_APR/star_sigma_600_B14_111_Pc_0.0011807.dat}; \addlegendentry{$n_e$};
\addplot+ [black, dashed, forget plot] table [x=r, y expr={(\thisrow{nu}+\thisrow{nd}+\thisrow{ns})/3}] {../code/data/LSM2F_APR/star_sigma_600_B14_111_Pc_0.0011807.dat};
\addplot+ [black, solid, forget plot, x filter/.expression={\thisrow{r}>=3.14 ? x : nan}] table [x=r, y expr={(\thisrow{nu}+\thisrow{nd}+\thisrow{ns})/3}] {../code/data/LSM2F_APR/star_sigma_600_B14_111_Pc_0.0011807.dat};
% cheat legend
\addplot+ [draw=none, black, solid] table [x=r, y expr={(\thisrow{nu}+\thisrow{nd}+\thisrow{ns})/3}] {../code/data/LSM2F_APR/star_sigma_600_B14_111_Pc_0.0011807.dat}; \addlegendentry{$n_B$};
\end{groupplot}
\node (title) at (group c2r1.north east) [above, anchor=south east, yshift=\pgfkeysvalueof{/pgfplots/every axis title shift}, text width=15cm] {\subcaption{\label{fig:hybrid:star-2}\goldenstar Two-flavor maximum mass star ($m_\sigma=\SI{600}{\mega\electronvolt}$, $B^\frac14 = \SI{111}{\mega\electronvolt}$, $P_c=10^{34.59} \, \si{\pascal}$)}};
\end{tikzpicture}

\bigskip

\tikzsetnextfilename{hybrid-3-flavor-extreme-star}
\begin{tikzpicture}
\begin{groupplot}[
	group style={group size={2 by 2}, horizontal sep=1.2cm, vertical sep=0.4cm},
	width=8cm, height=6cm,
	ylabel style={yshift=-0.2cm},
	enlargelimits=false, xtick distance=1.0, minor xtick={0,0.1,...,11.6},
	legend cell align=left,
]
\nextgroupplot[
	xticklabels={,,},
	ymax=1.5, ytick distance=0.5, minor y tick num=4,
	ylabel={$\{\epsilon,P\} \, / \, (\si{\giga\electronvolt\per\femto\meter\cubed})$ },
];
\addplot+ [blue] table [x=r, y=epsilon] {../code/data/LSM3F_APR/star_sigma_600_B14_111_Pc_0.0008161.dat}; \addlegendentry{$\epsilon$};
\addplot+ [red] table [x=r, y=P] {../code/data/LSM3F_APR/star_sigma_600_B14_111_Pc_0.0008161.dat}; \addlegendentry{$P$};
\nextgroupplot[
	ylabel=$m \, / \, M_\odot$,
	xticklabels={,,},
	ymax=2.1, ytick distance=0.5, minor y tick num=4,
];
\addplot+ [black] table [x=r, y=m] {../code/data/LSM3F_APR/star_sigma_600_B14_111_Pc_0.0008161.dat};
\nextgroupplot[
	xlabel=$r \, / \, \si{\kilo\meter}$,
	ylabel=$\mu_B \, / \, \si{\giga\electronvolt}$,
	ymin=0.9, ymax=1.5, ytick distance=0.1, minor y tick num=4, % restrict y to domain=0.8:1.6,
];
\addplot+ [black] table [x=r, y expr={\thisrow{muQ}*3.0/1000.0}] {../code/data/LSM3F_APR/star_sigma_600_B14_111_Pc_0.0008161.dat};
\nextgroupplot[
	xlabel=$r \, / \, \si{\kilo\meter}$,
	ylabel=$n_i \, / \, n_\text{sat}$,
	ymax=15, ytick distance=5, minor y tick num=4,
];
\addplot+ [red, x filter/.expression={\thisrow{r}<1.64 ? x : nan}] table [x=r, y=nu] {../code/data/LSM3F_APR/star_sigma_600_B14_111_Pc_0.0008161.dat}; \addlegendentry{$n_u$};
\addplot+ [darkgreen, x filter/.expression={\thisrow{r}<1.64 ? x : nan}] table [x=r, y=nd] {../code/data/LSM3F_APR/star_sigma_600_B14_111_Pc_0.0008161.dat}; \addlegendentry{$n_d$};
\addplot+ [purple, x filter/.expression={\thisrow{r}<1.64 ? x : nan}] table [x=r, y=ns] {../code/data/LSM3F_APR/star_sigma_600_B14_111_Pc_0.0008161.dat}; \addlegendentry{$n_s$};
%\addplot+ [blue, x filter/.expression={\thisrow{r}<1.64}] table [x=r, y=ne] {../code/data/LSM3F_APR/star_sigma_600_B14_111_Pc_0.0008161.dat}; \addlegendentry{$n_e$};
\addplot+ [black, dashed, forget plot] table [x=r, y expr={(\thisrow{nu}+\thisrow{nd}+\thisrow{ns})/3}] {../code/data/LSM3F_APR/star_sigma_600_B14_111_Pc_0.0008161.dat};
\addplot+ [black, solid, forget plot, x filter/.expression={\thisrow{r}>=1.64 ? x : nan}] table [x=r, y expr={(\thisrow{nu}+\thisrow{nd}+\thisrow{ns})/3}] {../code/data/LSM3F_APR/star_sigma_600_B14_111_Pc_0.0008161.dat};
% cheat legend
\addplot+ [draw=none, black, solid] table [x=r, y expr={(\thisrow{nu}+\thisrow{nd}+\thisrow{ns})/3}] {../code/data/LSM3F_APR/star_sigma_600_B14_111_Pc_0.0008161.dat}; \addlegendentry{$n_B$};
\end{groupplot}
\node (title) at (group c2r1.north east) [above, anchor=south east, yshift=\pgfkeysvalueof{/pgfplots/every axis title shift}, text width=15cm] {\phantomsubcaption\subcaption{\label{fig:hybrid:star-3}\goldenstar Three-flavor maximum mass star ($m_\sigma=\SI{600}{\mega\electronvolt}$, $B^\frac14 = \SI{111}{\mega\electronvolt}$, $P_c=10^{34.51} \, \si{\pascal}$)}};
\end{tikzpicture}
\caption{\label{fig:hybrid:star}%
	Radial profiles for the
	pressure $P$,
	energy density $\epsilon$,
	cumulative mass $m$,
	baryon chemical potential $\mu_B$
	and particle densities $n_i$
	for the maximum mass hybrid stars \goldenstar in \cref{fig:hybrid:mass-radius}.
}

\end{figure}

Solving the Tolman-Oppenheimer-Volkoff with the hybrid equations of state in \cref{fig:hybrid:eos-all},
we obtain the mass-radius curves in \cref{fig:hybrid:mass-radius}:
\begin{itemize}
\item After onset of the quark phase, the mass-radius curve develops new hybrid star branches below the maximum mass hadron-only neutron star
      with maximum masses $2.00 M_\odot \leq M \leq 2.09 M_\odot$ with two flavors in the core
      and $1.90 M_\odot \leq M \leq 2.07 M_\odot$ with three flavors.
\item At first glance, the hybrid branches in the left panel seem to plummet in mass immediately after diverging from the mass-increasing hadronic curve,
      indicating that all the hybrid stars are unstable.
      However, close inspection of the right panel shows that the dive is preceded by a small segment of mass-increasing and thus stable hybrid stars.
      This stable segment is longer in the two-flavor model and for smaller values of $m_\sigma$,
      and such a segment is visible on all six branches, if only a very tiny one.
      This supports the general claim in \cite{ref:quark_star_review} that quark cores are rare and very small, if they exist at all.
\item In fact, \cite[equation 15]{ref:hybrid_star_stability_criterion} and references within show that under general circumstances and no matter how small the quark core is,
      the energy density drop $\Delta \epsilon = \epsilon_Q^0 - \epsilon_H^0$ in the phase transition
      immediately destabilizes the hybrid branch \emph{if and only if}
      \begin{equation}
          \Delta \epsilon > \frac12 \epsilon_H^0 + \frac32 P^0.
      \label{eq:hybrid:stability_criterion}
      \end{equation}
      From \cref{fig:hybrid:eos-all} it is straightforward to verify that \emph{none} of the six equations of state satisfy this criterion,
      although the three-flavor equation of state with $m_\sigma=\SI{800}{\mega\electronvolt}$ comes close.
      Since the transition pressure $P^0$
      is almost equal with two and three flavors,
      the higher drop $\Delta \epsilon$ in the three-flavor model explains its shorter stable hybrid branches.
      This shows that criterion \eqref{eq:hybrid:stability_criterion} is consistent with our findings.
\item Like before, we focus on the lowest bag constants respecting the lower bounds \eqref{eq:lsm:bag_lower_bound}
      because they generate more interesting results with longer branches of stable hybrid stars.
      For a \emph{fixed} value of $m_\sigma$,
      a greater $B$ in the bag shift \eqref{eq:mit:bag_shift}
      lowers the quark pressure curve in the upper panels of \cref{fig:hybrid:eos},
      moving its intersection with the hadronic curve to a greater baryon chemical potential $\mu_B^0$ and pressure $P^0$.
      The hybrid mass-radius curve would then branch off the hadronic curve closer to the maximum mass,
      where the stable hybrid segment shortens.
      %Alternatively, this can be understood from the stability criterion \eqref{eq:hybrid:stability_criterion}.
      This effect is shown clearly in \cite[figure 14]{ref:lsm3f_hybrid_stars}.
\item The maximum mass stars live in the neighborhood of the heavy pulsars PSR J0348$+$0432, PSR J1614$-$2230 and PSR J0740$+$6620 and are certainly compatible with observations.
\end{itemize}

We inspect two maximum mass stars on the hybrid branches in \cref{fig:hybrid:star}:
\begin{itemize}
\item With both $N_f=\{2,3\}$, the quark core has a central baryon density $n_B = 6 n_\text{sat}$.
      It extends $R' = \{3.1,1.6\} \, \si{\kilo\meter}$ and has masses $M_\text{core} = \{0.12,0.02\} \, M_\odot$.
      Like the pure quark stars in \cref{chap:mit,chap:lsm2f,chap:lsm3f},
      two-flavor quark cores in hybrid stars are generally larger and heavier than three-flavor ones
      due to their softer equations of state.
%\item We again see that the drops $\Delta \epsilon$ and $\Delta n_B$ in energy density and baryon density 
      %are notably lower in the two-flavor hybrid stars.
\item The hadronic phase dominates with both $N_f=\{2,3\}$, 
      making up $\{98.0,99.7\} \, \si{\percent}$ of the stars' volume and $\{94,99\} \, \si{\percent}$ of their masses.
\end{itemize}

\begin{figure}[t!]
\centering
\tikzsetnextfilename{hybrid-flavor-eos-consistent}
\begin{tikzpicture}
\begin{groupplot}[
	group style={group size={1 by 1}, vertical sep=1cm},
	width=8cm, height=7cm,
	minor tick num=9,
	xlabel = {$P \, / \, (\si{\giga\electronvolt\per\femto\meter\cubed})$},
	xmin=0, ymin=0, xmax=0.5, ymax=3.5, restrict x to domain=-1:+4, restrict y to domain=-1:+4, xtick distance=0.1,
	legend columns=1,
	legend style={anchor=north, at={(0.5, 0.98)}},
	grid = both, minor grid style = {ultra thin, gray!10!white},
];
\nextgroupplot[
	title = {Consistently fit two-flavor equations of state}, title style={text width=12cm},
	ylabel = {$\epsilon \, / \, (\si{\giga\electronvolt\per\femto\meter\cubed})$},
];
\addplot [red, solid] table [x=P, y=epsilon] {../code/data/LSM2FC_APR/eos_sigma_400.dat};
\addlegendentry{$m_\sigma=\SI{400}{\mega\electronvolt}$, $\vphantom{\big)}\smash{B^\frac14} = \SI{107}{\mega\electronvolt}$};
\addplot [darkgreen, dashed] table [x=P, y=epsilon] {../code/data/LSM2FC_APR/eos_sigma_500.dat};
\addlegendentry{$m_\sigma=\SI{500}{\mega\electronvolt}$, $\vphantom{\big)}\smash{B^\frac14} = \phantom{0}\SI{84}{\mega\electronvolt}$};
\addplot [blue, dotted] table [x=P, y=epsilon] {../code/data/LSM2FC_APR/eos_sigma_600.dat};
\addlegendentry{$m_\sigma=\SI{600}{\mega\electronvolt}$, $\vphantom{\big)}\smash{B^\frac14} = \phantom{0}\SI{27}{\mega\electronvolt}$};
\end{groupplot}
\end{tikzpicture}
\caption{\label{fig:hybrid:eos-consistent}%
	Hybrid equations of state like in \cref{fig:hybrid:eos-all}
	with the consistently fit two-flavor quark-meson model from \cref{sec:lsm2f:refinement}.
}
\end{figure}

\begin{figure}
\centering
\tikzsetnextfilename{hybrid-2-flavor-mass-radius-consistent}
\begin{tikzpicture}
\begin{groupplot}[
	group style={group size={2 by 1}, vertical sep=0cm, horizontal sep=1.0cm},
	height=7cm,
	point meta=explicit, point meta min=33, point meta max=36,
	%colorbar horizontal, colormap name=plasmarev, colorbar style={xlabel=$\log_{10} (P_c \, / \, \si{\pascal})$, xtick distance=1, minor x tick num=9, at={(0.5,1.03)}, anchor=south, xticklabel pos=upper},
	/tikz/declare function={
		e0 = 4.266500881855304e+37;
	},
]
\tikzset{
	Bpin/.style={gray, sloped, allow upside down=true, rotate=180, yshift=+0.4cm, font=\small},
}
\nextgroupplot[
	width=10cm, 
	xlabel={$R \, / \, \si{\kilo\meter}$ },
	ylabel={$M \, / \, M_\odot$}, %title={Mass-radius diagram for 2-flavor quark stars }, title style={yshift=2.0cm},
	xmin=5, xmax=35, ymin=0.5, ymax=2.5, xtick distance=5, ytick distance=0.5, minor tick num=4, grid=major,
	legend columns=5, legend pos=south east, legend cell align=right, legend transposed,
	colorbar horizontal, colormap name=plasmarev, colorbar style={width=11cm, ylabel=$\log_{10} (P_c \, / \, \si{\pascal})$, ylabel style={rotate=-90}, xtick distance=1, minor x tick num=9, at={(0.8,-0.3)}, anchor=north, xticklabel pos=lower},
];
\addplot [name path=J0748lo, draw=none, forget plot, domain=0:40] {2.08 - 0.07};
\addplot [name path=J0748hi, draw=none, forget plot, domain=0:40] {2.08 + 0.07};
\addplot [red, opacity=0.4] fill between [of=J0748lo and J0748hi]; \addlegendentry{PSR J0748$+$6620};

\addplot [name path=J0348lo, draw=none, forget plot, domain=0:40] {2.01 - 0.04};
\addplot [name path=J0348hi, draw=none, forget plot, domain=0:40] {2.01 + 0.04};
\addplot [green, opacity=0.4] fill between [of=J0348lo and J0348hi]; \addlegendentry{PSR J0348$+$0432};

\addplot [name path=J1614lo, draw=none, forget plot, domain=0:40] {1.91 - 0.02};
\addplot [name path=J1614hi, draw=none, forget plot, domain=0:40] {1.91 + 0.02};
\addplot [yellow, opacity=0.4] fill between [of=J1614lo and J1614hi]; \addlegendentry{PSR J1614$-$2230};

\addplot+ [solid, gray] table [x=R, y=M, meta expr={log10(\thisrow{P}*e0)}] {../code/data/LSM3F_APR/stars_hadron.dat}; \addlegendentry{neutron stars};
\addplot+ [solid, mesh, mesh line legend] table [x=R, y=M, meta expr={log10(\thisrow{P}*e0)}] {../code/data/LSM2FC_APR/stars_sigma_400_B14_107.dat}; \addlegendentry{hybrid stars};
\addplot+ [solid, mesh] table [x=R, y=M, meta expr={log10(\thisrow{P}*e0)}] {../code/data/LSM2FC_APR/stars_sigma_500_B14_84.dat}; % node [Bpin, pos=0.920] {$B = (\SI{27}{\mega\electronvolt})^4$};
\addplot+ [solid, mesh] table [x=R, y=M, meta expr={log10(\thisrow{P}*e0)}] {../code/data/LSM2FC_APR/stars_sigma_600_B14_27.dat}; % node [Bpin, pos=0.920] {$B = (\SI{27}{\mega\electronvolt})^4$};
\draw [Latex-] (11.4,2.09) -- (15.0,2.20) node [anchor=west] {$m_\sigma=\SI{800}{\mega\electronvolt}$, $\smash{B^\frac14} = \SI{27}{\mega\electronvolt}$};
\draw [Latex-] (11.5,2.00) -- (15.0,2.00) node [anchor=west] {$m_\sigma=\SI{700}{\mega\electronvolt}$, $\smash{B^\frac14} = \SI{68}{\mega\electronvolt}$};
\draw [Latex-] (11.5,1.98) -- (15.0,1.80) node [anchor=west] {$m_\sigma=\SI{600}{\mega\electronvolt}$, $\smash{B^\frac14} = \SI{111}{\mega\electronvolt}$};
\node[scale=0.75] at (11.216, 1.996) {\goldenstar};

\nextgroupplot[
	width=5cm, 
	xlabel={$R \, / \, \si{\kilo\meter}$},
	xmin=11, xmax=12, ymin=1.85, ymax=2.15, xtick distance=1, ytick distance=0.1, minor tick num=9, %grid=major,
	extra y ticks={2}, extra y tick labels={,,}, extra tick style={grid=major},
];
\addplot [name path=J0748lo, draw=none, forget plot, domain=0:40] {2.08 - 0.07};
\addplot [name path=J0748hi, draw=none, forget plot, domain=0:40] {2.08 + 0.07};
\addplot [red, opacity=0.4] fill between [of=J0748lo and J0748hi];

\addplot [name path=J0348lo, draw=none, forget plot, domain=0:40] {2.01 - 0.04};
\addplot [name path=J0348hi, draw=none, forget plot, domain=0:40] {2.01 + 0.04};
\addplot [green, opacity=0.4] fill between [of=J0348lo and J0348hi];

\addplot [name path=J1614lo, draw=none, forget plot, domain=0:40] {1.91 - 0.02};
\addplot [name path=J1614hi, draw=none, forget plot, domain=0:40] {1.91 + 0.02};
\addplot [yellow, opacity=0.4] fill between [of=J1614lo and J1614hi];

\addplot+ [solid, gray] table [x=R, y=M, meta expr={log10(\thisrow{P}*e0)}] {../code/data/LSM3F_APR/stars_hadron.dat}; % node [Bpin, pos=0.920] {$B = (\SI{27}{\mega\electronvolt})^4$};
\addplot+ [solid, mesh] table [x=R, y=M, meta expr={log10(\thisrow{P}*e0)}] {../code/data/LSM2FC_APR/stars_sigma_400_B14_107.dat}; % node [Bpin, pos=0.920] {$B = (\SI{27}{\mega\electronvolt})^4$};
\addplot+ [solid, mesh] table [x=R, y=M, meta expr={log10(\thisrow{P}*e0)}] {../code/data/LSM2FC_APR/stars_sigma_500_B14_84.dat}; % node [Bpin, pos=0.920] {$B = (\SI{27}{\mega\electronvolt})^4$};
\addplot+ [solid, mesh] table [x=R, y=M, meta expr={log10(\thisrow{P}*e0)}] {../code/data/LSM2FC_APR/stars_sigma_600_B14_27.dat}; % node [Bpin, pos=0.920] {$B = (\SI{27}{\mega\electronvolt})^4$};
\draw [Latex-] (11.4,2.09) -- (15.0,2.20) node [anchor=west] {$m_\sigma=\SI{800}{\mega\electronvolt}$, $\smash{B^\frac14} = \SI{27}{\mega\electronvolt}$};
\draw [Latex-] (11.5,2.00) -- (15.0,2.00) node [anchor=west] {$m_\sigma=\SI{700}{\mega\electronvolt}$, $\smash{B^\frac14} = \SI{68}{\mega\electronvolt}$};
\draw [Latex-] (11.5,1.98) -- (15.0,1.80) node [anchor=west] {$m_\sigma=\SI{600}{\mega\electronvolt}$, $\smash{B^\frac14} = \SI{111}{\mega\electronvolt}$};
\node[scale=0.75] at (11.216, 1.996) {\goldenstar};

\end{groupplot}
\node (title) at ($(group c1r1.north west)!0.5!(group c2r1.north east)$) [above, yshift=\pgfkeysvalueof{/pgfplots/every axis title shift}, align=center] {Consistently fit two-flavor hybrid star mass-radius solutions};
\end{tikzpicture}
\caption{\label{fig:hybrid:mass-radius-consistent}%
	Mass-radius solutions like in \cref{fig:hybrid:mass-radius}
	with the consistently fit two-flavor hybrid equations of state from \cref{fig:hybrid:eos-consistent}.
}
\end{figure}

\begin{figure}
\centering
\tikzsetnextfilename{hybrid-flavor-extreme-star-consistent}
\begin{tikzpicture}
\begin{groupplot}[
	group style={group size={2 by 2}, horizontal sep=1.2cm, vertical sep=0.4cm},
	width=8cm, height=6cm,
	ylabel style={yshift=-0.2cm},
	enlargelimits=false, xtick distance=1.0, minor xtick={0,0.1,...,11.6},
	legend cell align=left,
]
\nextgroupplot[
	xticklabels={,,},
	ymax=1.5, ytick distance=0.5, minor y tick num=4,
	ylabel={$\{\epsilon,P\} \, / \, (\si{\giga\electronvolt\per\femto\meter\cubed})$ },
];
\addplot+ [blue] table [x=r, y=epsilon] {../code/data/LSM2FC_APR/star_sigma_400_B14_107_Pc_0.0011807.dat}; \addlegendentry{$\epsilon$};
\addplot+ [red] table [x=r, y=P] {../code/data/LSM2FC_APR/star_sigma_400_B14_107_Pc_0.0011807.dat}; \addlegendentry{$P$};
\nextgroupplot[
	ylabel=$m \, / \, M_\odot$,
	xticklabels={,,},
	ymax=2.1, ytick distance=0.5, minor y tick num=4,
];
\addplot+ [black] table [x=r, y=m] {../code/data/LSM2FC_APR/star_sigma_400_B14_107_Pc_0.0011807.dat};
\nextgroupplot[
	xlabel=$r \, / \, \si{\kilo\meter}$,
	ylabel=$\mu_B \, / \, \si{\giga\electronvolt}$,
	ymin=0.9, ymax=1.5, ytick distance=0.1, minor y tick num=4, % restrict y to domain=0.8:1.6,
];
\addplot+ [black] table [x=r, y expr={\thisrow{muQ}*3.0/1000.0}] {../code/data/LSM2FC_APR/star_sigma_400_B14_107_Pc_0.0011807.dat};
\nextgroupplot[
	xlabel=$r \, / \, \si{\kilo\meter}$,
	ylabel=$n_i \, / \, n_\text{sat}$,
	ymax=15, ytick distance=5, minor y tick num=4,
];
\addplot+ [red, x filter/.expression={\thisrow{r}<3.22 ? x : nan}] table [x=r, y=nu] {../code/data/LSM2FC_APR/star_sigma_400_B14_107_Pc_0.0011807.dat}; \addlegendentry{$n_u$};
\addplot+ [darkgreen, x filter/.expression={\thisrow{r}<3.22 ? x : nan}] table [x=r, y=nd] {../code/data/LSM2FC_APR/star_sigma_400_B14_107_Pc_0.0011807.dat}; \addlegendentry{$n_d$};
%\addplot+ [purple, x filter/.expression={\thisrow{r}<3.22 ? x : nan}] table [x=r, y=ns] {../code/data/LSM2FC_APR/star_sigma_400_B14_107_Pc_0.0011807.dat}; \addlegendentry{$n_s$};
%\addplot+ [blue, x filter/.expression={\thisrow{r}<3.22}] table [x=r, y=ne] {../code/data/LSM2FC_APR/star_sigma_400_B14_107_Pc_0.0011807.dat}; \addlegendentry{$n_e$};
\addplot+ [black, dashed, forget plot] table [x=r, y expr={(\thisrow{nu}+\thisrow{nd}+\thisrow{ns})/3}] {../code/data/LSM2FC_APR/star_sigma_400_B14_107_Pc_0.0011807.dat};
\addplot+ [black, solid, forget plot, x filter/.expression={\thisrow{r}>=3.22 ? x : nan}] table [x=r, y expr={(\thisrow{nu}+\thisrow{nd}+\thisrow{ns})/3}] {../code/data/LSM2FC_APR/star_sigma_400_B14_107_Pc_0.0011807.dat};
% cheat legend
\addplot+ [draw=none, black, solid] table [x=r, y expr={(\thisrow{nu}+\thisrow{nd}+\thisrow{ns})/3}] {../code/data/LSM2FC_APR/star_sigma_400_B14_107_Pc_0.0011807.dat}; \addlegendentry{$n_B$};
\end{groupplot}
\node (title) at ($(group c1r1.north west)!0.5!(group c2r1.north east)$) [above, yshift=\pgfkeysvalueof{/pgfplots/every axis title shift}, align=center] {\goldenstar Consistently fit two-flavor maximum mass star\\($m_\sigma=\SI{400}{\mega\electronvolt}$, $B^\frac14 = \SI{107}{\mega\electronvolt}$, $P_c=10^{34.59} \, \si{\pascal}$)};
\end{tikzpicture}
\caption{\label{fig:hybrid:star-consistent}%
	Radial profiles like in \cref{fig:hybrid:star} for the maximum mass star \goldenstar in \cref{fig:hybrid:mass-radius-consistent}.
}
\end{figure}

For completeness and to round everything off,
we pay one last visit to our consistently fit two-flavor quark-meson model in \cref{sec:lsm2f:refinement}.
Recall that the quark star results with inconsistently fit $\SI{600}{\mega\electronvolt} \leq m_\sigma \leq \SI{800}{\mega\electronvolt}$
effectively coincided with those from consistently fit $\SI{400}{\mega\electronvolt} \leq m_\sigma \leq \SI{600}{\mega\electronvolt}$,
so we should expect the same to hold when using this model for the quark core in a hybrid star.
Using the slightly modified lower bag constant bounds \eqref{eq:lsm:bag_lower_bound_ref} that we found with the consistently fit model,
we repeat our calculations and find the results in \cref{fig:hybrid:eos-consistent,fig:hybrid:mass-radius-consistent,fig:hybrid:star-consistent}.
Indeed, we see that they agree very well with the results using inconsistently fit $\SI{600}{\mega\electronvolt} \leq m_\sigma \leq \SI{800}{\mega\electronvolt}$,
and we hope the same is true in the three-flavor situation.

\section{Summary}

In this chapter we have constructed hybrid stars modeled with the two-flavor and three-flavor quark-meson model in the core,
and the hadronic Akmal-Pandharipande-Ravenhall equation of state in the outer region.
Using $\sigma$-meson masses $\SI{600}{\mega\electronvolt} \leq m_\sigma \leq \SI{800}{\mega\electronvolt}$ in the quark-meson model,
we saw that the hadronic mass-radius relation developed very short branches of hybrid stars
up to maximum masses $2.00 M_\odot \leq M \leq 2.09 M_\odot$ with two flavors in the core
and $1.90 M_\odot \leq M \leq 2.07 M_\odot$ with three flavors.
In general, the two-flavor cores are both more massive and stable than three-flavor cores
due to their softer equations of state.
These maximum masses overlap with recent mass measurements of the heavy pulsars PSR J0348$+$0432, PSR J1614$-$2230 and PSR J0740$+$6620 around and above the $2 M_\odot$-limit.
After the maximum mass star, a discontinuous transition between the two phases destabilizes the stars.
This transition is generally more dramatic with three flavors,
so the stable two-flavor segments are slightly longer.
