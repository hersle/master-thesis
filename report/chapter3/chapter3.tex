\chapter{Cold free fermion neutron stars}

\TODO{intro}

\TODO{small $p$ for momentum, big $P$ for pressure everywhere}

In \cref{chap:tft}, we found $Z = Z(T, \mu, V)$ for two sample systems, and in particular a gas of free Dirac fermions.
From $Z$, we can derive the pressure $P = P(T, \mu)$ and energy density $\epsilon = \thermalavg{E}/V = \epsilon(T, \mu)$ using \cref{eq:tft:average_number,eq:tft:average_energy,eq:tft:average_pressure}, where the volume $V$ has been eliminated by division, because $P$ and $\epsilon$ are intensive quantities.
At some fixed temperature $T$, we can therefore eliminate $\mu$ to express $\epsilon$ in terms of $P$.
This gives us an equation of state $\epsilon = \epsilon(P)$ with which we can solve the TOV system \TODO{ref}.

In this chapter, we will use the free Dirac fermion partition function \eqref{eq:tft:dirac_partition_function} and do exactly this to model a cold neutron star made of free neutrons.

\section{Equation of state}

For easy reference, the logarithm of the free Dirac fermion partition function \eqref{eq:tft:dirac_partition_function} is
\begin{equation}
	\log Z = 2 V \int \frac{\dif^3 p}{(2 \pi \hbar)^3} \left\{ \beta E(\vec{p}) + \log \left[ e^{-\beta (E(\vec{p}) - \mu)}+1 \right] + \log \left[ e^{-\beta (E(\vec{p}) + \mu)} + 1\right] \right\}.
\end{equation}
First, the particle number density $n = \thermalavg{N}/V$ follows from the derivative \eqref{eq:tft:average_number} and is
\begin{equation}
	n = 
	\frac{1}{\beta} \pdv{\log Z}{\mu} =
	2 \int \frac{\dif^3 p}{(2 \pi \hbar)^3} \Big\{ n\big[ E(\vec{p})-\mu \big] - n\big[ E(\vec{p})+\mu \big] \Big\} ,
\label{eq:nstars:density}
\end{equation}
where we defined the \textbf{Fermi-Dirac distribution}
\begin{equation}
	n(E) = \frac{1}{e^{-\beta E} + 1}.
\label{eq:nstars:fermi_dirac_distribution}
\end{equation}
Do not confuse the particle density $n$ on the left with the Fermi-Dirac distributions $n[E(\vec{p}) \mp \mu]$ on the right!
We will soon integrate away $n[E(\vec{p}) \mp \mu]$, anyway.
From the density \eqref{eq:nstars:density}, we see that $n = n(\mu, T)$ is a function of the chemical potential $\mu$ and temperature $T$, so that at some fixed temperature, the value of $\mu$ decides the particle density $n$.

Second, we calculate the energy density $\epsilon = \thermalavg{E} / V$ from \cref{eq:tft:average_energy}.
It comes out as
\begin{equation}
	\epsilon = 
	\mu n - \frac{1}{V} \pdv{\log Z}{\beta} =
	2 \int \frac{\dif^3 p}{(2 \pi \hbar)^3} \Big\{ -E(\vec{p}) + E(\vec{p}) \, n\big[ E(\vec{p})-\mu \big] + E(\vec{p}) \, n\big[ E(\vec{p})+\mu \big] \Big\}.
\label{eq:nstars:energy_density}
\end{equation}

Third, we find that the pressure \eqref{eq:tft:average_pressure} is
\begin{equation}
	P =
	\frac{\log Z}{\beta V} = 
	2 \int \frac{\dif^3 p}{(2 \pi \hbar)^3} \Big\{ E(\vec{p}) + \log \left[ e^{-\beta(E(\vec{p})-\mu)} + 1 \right] + \log \left[ e^{-\beta(E(\vec{p})+\mu)} + 1 \right] \Big\}.
\label{eq:nstars:pressure}
\end{equation}

The first term of the energy density \eqref{eq:nstars:energy_density} and the pressure \eqref{eq:nstars:pressure} is infinite, as the integrand never decays.
This can be interpreted as an infinite shift of the vacuum energy.
In contrast, the two last terms are finite as the integrand is suppressed for large $\abs{\vec{p}}$ by the Fermi-Dirac distribution.
It makes no sense to include a term that integrates over every possible value of the momentum $\vec{p}$ for physical particles whose momentum cannot exceed a certain value due to energy conservation.
Here, we will make the assumption that we can simply drop the infinite term.
We will return later to investigate this term by regularization and renormalization. \TODO{does this make sense?}

From the particle density \eqref{eq:nstars:density} at constant temperature $T$, we see that the sign of $n$ is determined by the sign of $\mu$.
The total density $n$ is expressed as a balance between \emph{particles} with energy $E(\vec{p}) > 0$ living relative to the chemical potential $\mu$ and \emph{antiparticles} with energy $E(\vec{p}) < 0$ living relative to the chemical potential $-\mu$.
Thus, the chemical potential $\mu$ determines the balance between particles and antiparticles in the system.
Similarly, the two last terms in the energy density \eqref{eq:nstars:energy_density} and pressure \eqref{eq:nstars:pressure} can be interpreted as contributions from particles and antiparticles.
We choose a large, positive value of $\mu > 0$, so that the particles dominate the system, while antiparticles are hardly present.
With this choice, $n \left[ E(\vec{p}) - \mu \right] \gg n \left[ E(\vec{p}) + \mu \right]$, and we drop the last term from the particle density \eqref{eq:nstars:density}, energy density \eqref{eq:nstars:energy_density} and pressure \eqref{eq:nstars:pressure}.

Dropping terms as described in the last paragraphs, we are left with the
\begin{equation}
\begin{aligned}
	& \text{particle density} & n        &=  2 \int \frac{\dif^3 p}{(2 \pi \hbar)^3} \, n \left[ E(\vec{p})-\mu \right] , \\
	& \text{energy density}   & \epsilon &=  2 \int \frac{\dif^3 p}{(2 \pi \hbar)^3} \, E(\vec{p}) \, n \left[ E(\vec{p})-\mu \right] \text{ and} \\
	& \text{pressure}         & P        &= -2 \int \frac{\dif^3 p}{(2 \pi \hbar)^3} \, \log \left[ e^{-\beta(E(\vec{p})-\mu)} + 1 \right] . \\
\end{aligned}
\label{eq:nstars:dropped_infinities}
\end{equation}
The integrals become nasty after plugging in the dispersion relation $E(\vec{p}) = \sqrt{\vec{p}^2 c^2 + m^2 c^4}$ \TODO{write in display form in TFT chapter and only refer back here} and the Fermi-Dirac distribution \eqref{eq:nstars:fermi_dirac_distribution}, and it is overly optimistic to expect that all of them can be evaluated analytically.
We will make one final approximation that will make all the integrals surmountable.

By human standards, it is very hot inside a neutron star.
In fact, studies place typical core temperatures around $T_0 \approx \SI{1e6}{\kelvin}$.
However, neutrons have mass $m \approx \SI{1.67e-27}{\kilogram}$, so everywhere inside a neutron star we have $\beta E(\vec{p}) = \sqrt{\vec{p}^2 c^2 + m^2 c^4} / k_B T > m c^2 / k_B T_0 \approx 10^7 \gg 1$.
Although the temperature is very large compared to everyday temperatures, the thermal energy $k_B T$ is in fact very low relative to the energy $E(\vec{p})$ of the nuclei.
It is therefore an excellent approximation to take the \textbf{zero temperature limit}
\begin{equation}
	\beta E(\vec{p}) \gg 1 .
\end{equation}
In the zero temperature limit,
\TODO{make two simple figures to show how the two functions ``converge''}
\begin{equation}
\begin{split}
	n(E)                                 \quad &\rightarrow \quad \theta(-E) , \\
	\log \left[ e^{-\beta E} + 1 \right] \quad &\rightarrow \quad -\beta E \theta(-E) . \\
\end{split}
\end{equation}
Thus, the zero temperature limit effectively limits the integrals \eqref{eq:nstars:dropped_infinities} to those momenta $\vec{p}$ with $E(\vec{p}) < \mu$.
We therefore call
\begin{equation}
	\mu = E_F = E(p_F) = \sqrt{p_F^2 c^2 + m^2 c^4}
\end{equation}
the Fermi energy and the corresponding momentum $p_F$ the Fermi momentum, representing the occupied state in momentum-space with greatest energy and momentum.

Now the particle density \TODO{ref} becomes simply
\begin{equation}
	n = 
	2 \int \frac{\dif p \, 4 \pi p^2}{(2 \pi \hbar)^3} \theta \left[ \mu - E(p) \right] =
	2 \int_0^{p_F} \frac{\dif p \, 4 \pi p^2}{(2 \pi \hbar)^3} = \frac{p_F^3}{3 \pi^2 \hbar^3} .
\label{eq:nstars:density_zeroT}
\end{equation}
Using integral \TODO{ref} and defining the dimensionless momentum $x = p / mc$, the energy density \TODO{ref} is
\begin{equation}
\begin{split}
	\epsilon &=  2 \int \frac{\dif p \, 4 \pi p^2}{(2 \pi \hbar)^3} \, E(p) \, \theta \left[ \mu - E(p) \right] \\
	         &=  2 \int_0^{p_F} \frac{\dif p \, 4 \pi p^2}{(2 \pi \hbar)^3} \, \sqrt{p^2 c^2 + m^2 c^4} \\
	         &= \frac{m^4 c^5}{\pi^2 \hbar^3} \int_0^{x_F} \dif x \, x^2 \sqrt{1 + x^2} \\
	         &= \frac{m^4 c^5}{8 \pi^2 \hbar^3} \left[ \left( 2 x_F^3 + x_F \right) \sqrt{1 + x_F^2} - \asinh x_F \right] . \\
\end{split}
\label{eq:nstars:energy_density_zeroT}
\end{equation}
Finally, using the same integral, the pressure \TODO{ref} is
\begin{equation}
\begin{split}
	P &= \frac{2}{\beta} \int \frac{\dif p \, 4 \pi p^2}{(2 \pi \hbar)^3} \, \beta \left[ \mu - E(p) \right] \, \theta \left[ \mu - E(p) \right] \\
	  &= 2 \int_0^{p_F} \frac{\dif p \, 4 \pi p^2}{(2 \pi \hbar)^3} \, \left[ \sqrt{p_F^2 c^2 + m^2 c^4} - \sqrt{p^2 c^2 + m^2 c^4} \right] \\
	  &= \frac{m^4 c^5}{\pi^2 \hbar^3} \int_0^{x_F} \dif x \, x^2 \left[ \sqrt{x_F^2+1} - \sqrt{x^2+1} \right] \\
	  &= \frac{m^4 c^5}{24 \pi^2 \hbar^3} \left[ \left( 2 x_F^3 - 3 x_F \right) \sqrt{x_F^2 + 1} + 3 \asinh x_F \right] . \\
\end{split}
\label{eq:nstars:pressure_zeroT}
\end{equation}

The equation of state $\epsilon = \epsilon(P)$ follows by eliminating $x_F$ from the energy density \eqref{eq:nstars:energy_density_zeroT} and pressure \eqref{eq:nstars:pressure_zeroT}.
Due to their complicated dependence on $x_F$, we will do so in three cases of increasing difficulty.

\subsection{Ultra-relativistic limit}
\label{sec:nstars:ur_limit}

First, consider the ultra-relativistic limit
\begin{equation}
	x_F \gg 1 , 
\label{eq:nstars:ur_limit}
\end{equation}
where the Fermi energy $E_F = \sqrt{p_F^2 c^2 + m^2 c^4} \taylor p_F c$ is dominated by the contribution from the Fermi momentum.
Since $\asinh x_F = \log \left[ x_F + \sqrt{x_F^2 + 1} \right] \taylor \log 2 x_F$ diverges logarithmically, we see that both the energy density \eqref{eq:nstars:energy_density_zeroT} and pressure \eqref{eq:nstars:pressure_zeroT} are dominated by their first term with $2 x_F^3 \sqrt{x_F^2 + 1} \taylor 2 x_F^4$.
In the ultra-relativistic limit, then,
\begin{equation}
	\epsilon \taylor \frac{m^4 c^5 x_F^4}{4 \pi^2 \hbar^3}
	\qquad \text{and} \qquad
	P        \taylor \frac{m^4 c^5 x_F^4}{12 \pi^2 \hbar^3},
\end{equation}
and $x_F$ is easily eliminated, yielding the very simple equation of state
\begin{equation}
	\epsilon = 3 P .
\label{eq:nstars:ur_eos}
\end{equation}

In this particular case, the TOV system \TODO{ref} can be solved analytically with the polynomial trial solution
\begin{equation}
	P(r) = A r^n .
\label{eq:nstars:ur_ansatz}
\end{equation}
Then the mass equation \TODO{ref} is
\begin{equation}
	\odv{m}{r} = \frac{12 \pi A}{c^2} r^{n+2},
	\qquad \text{so} \qquad
	m(r) = \frac{12 \pi A}{(n+3) c^2} r^{n+3}
	\quad (n \neq -3).
\label{eq:nstars:ur_mass}
\end{equation}
With the equation of state \eqref{eq:nstars:ur_eos}, mass \eqref{eq:nstars:ur_mass} and trial solution \eqref{eq:nstars:ur_ansatz}, the TOV equation \eqref{eq:tov} reads
\begin{equation}
	n A r^{n-1} =
	-\frac{48 \pi G A^2 r^{2n+1}}{(n+3) c^4} \left[ 2 + \frac{n}{3} \right] \left[ 1 - \frac{24 \pi G A r^{n+2}}{(n+3) c^4} \right]^{-1} .
\end{equation}
We can attain equality for all $r$ if we choose $n = -2$.
Then the rightmost factor no longer depends on $r$, and both sides have the $r^{-3}$-dependence
\begin{equation}
	- 2 A r^{-3} = - \frac{64 \pi G A^2 r^{-3}}{c^4} \left[ 1 - \frac{24 \pi G A}{c^4} \right]^{-1} .
\end{equation}
Equality is established if we match the prefactors by choosing $A = c^4 / 56 \pi G$.
Then the solutions for the pressure and mass are
\begin{equation}
	P(r) = \frac{c^4}{56 \pi G} \frac{1}{r^2}
	\qquad \text{and} \qquad
	m(r) = \frac{3 c^2}{14 G} r .
\end{equation}
This is a highly unphysical result.
The pressure diverges at the center, so nothing could ever hold such a star together.
In addition, $p(r) > 0$ for all $r$, so the star has no surface and hence infinite mass $M = m(\infty) = \infty$.

\TODO{make plot of $P(r)$ and $m(r)$ for a star}

\subsection{Non-relativistic limit}
\label{sec:nstars:nr_limit}

Next, let us consider the more difficult non-relativistic limit
\begin{equation}
	x_F \ll 1,
\label{eq:nstars:nr_limit}
\end{equation}
where the Fermi energy $E_F = \sqrt{p_F^2 c^2 + m^2 c^4} \taylor m c^2$ is dominated by the rest energy of the fermions.
Taylor expanding the energy density \eqref{eq:nstars:energy_density_zeroT} and presure \eqref{eq:nstars:pressure_zeroT} around $x_F = 0$ to lowest order, we find
\begin{equation}
	%\epsilon &\taylor \frac{m c^2 p_F^3}{3 \pi^2 \hbar^3} + \frac{p_F^5}{10 \pi^2 \hbar^3 m} = n m c^2 + \frac{p_F^5}{10 \pi^2 \hbar^3 m} \\
	%P        &\taylor \frac{p_F^5}{15 \pi^2 \hbar^3 m}
	\epsilon \taylor \frac{m c^2 p_F^3}{3 \pi^2 \hbar^3}
	\qquad \text{and} \qquad
	P        \taylor \frac{p_F^5}{15 \pi^2 \hbar^3 m} .
\end{equation}
Note that with the density \eqref{eq:nstars:density}, the energy density can be written $\epsilon = n m c^2$, so it is only due to the rest mass of the particles, as if all fermions have broken free from the Pauli exclusion principle and possess the same rest energy $m c^2$.
This is only a mathematical feature of the non-relativistic limit -- the fermions still occupy different states with different momentum, but the momenta are so small that the differences are negligible compared to the rest energy $mc^2$.
Again, it is straightforward to eliminate $x_F$ to find the equation of state, only this time there is some extra bookkeeping with all the exponents.
Carefully gathering all the prefactors under the same roof, we find
\begin{equation}
	%\epsilon = \frac{mc^2}{3\pi^2\hbar^3} \left( 15 \pi^2 \hbar^3 m P \right)^{3/5}
	\epsilon = \left( \frac{5^3 m^8 c^{10}}{3^2 \pi^4 \hbar^6} \right)^{\frac15}  P^{\frac35} .
\end{equation}
With this power dependence, it is not easy, if even possible, to solve the TOV equation analytically.
The trial solution \eqref{eq:nstars:ur_ansatz} we employed in \cref{sec:nstars:ur_limit} fails miserably, as we do not get the same fortunate cancellations of $r$.
We therefore resort to the numerical solution method described in \cref{sec:nstars:numtov}, parametrizing different stars by their center pressure $P_0$ and integrating the TOV equation until the pressure $p(R)$ vanishes, using the corresponding radius $R$ to establish the mass $M = m(R)$ of the star.
The results are shown in \cref{fig:nstars:massradius}.

\iffalse
Dimensionless equation of state
\begin{equation}
	\diml{\epsilon} = \left[ \frac{4^2 5^3}{3^4 \pi^2} \frac{m^8 c^6 r_0^6}{m_0^2 \hbar^6} \diml{P}^3 \right]^{\frac{1}{5}}
\end{equation}
\fi

\subsection{General Fermi momenta}
\label{sec:nstars:gr_limit}

How can we find the energy density
\begin{equation}
	\epsilon = \frac{m^4 c^5}{8 \pi^2 \hbar^3} \left[ \left( 2 x_F^3 + x_F \right) \sqrt{1 + x_F^2} - \asinh x_F \right]
\label{eq:nstars:gr_limit_energy_density}
\end{equation}
that corresponds to a given pressure
\begin{equation}
	P = \frac{m^4 c^5}{24 \pi^2 \hbar^3} \left[ \left( 2 x_F^3 - 3 x_F \right) \sqrt{x_F^2 + 1} + 3 \asinh x_F \right] 
\label{eq:nstars:gr_limit_pressure}
\end{equation}
for general $x_F$?
Since we are already solving the TOV equation on a computer, we can do so by numerical root finding.
At every step $r$ in the numerical integration algorithm, we know the current pressure $P = P(r)$.
Using a numerical root finding algorithm, we can find the root $x_F$ of the function
\begin{equation}
	f(x_F) = P(x_F) - P = 0,
\end{equation}
where $P(x_F)$ is the pressure \eqref{eq:nstars:gr_limit_pressure} as a function of $x_F = p_F / m c$.
Having found the root, we can simply calculate the corresponding energy density $\epsilon(x_F)$ from \cref{eq:nstars:gr_limit_energy_density}.
This whole procedure can be elegantly encapsulated into a function that implements an implicit equation of state $\epsilon = \epsilon(P)$, which in turn is straightforward to plug into our solver described in \cref{sec:nstars:numtov}.
The results are shown in \cref{fig:nstars:massradius}.

\iffalse
\begin{equation}
	\diml{P}(x_F) = \frac{m^4 c^3 r_0^3}{18 \pi m_0 \hbar^3} \left[ (2 x_F^3 - 3 x_F) \sqrt{x_F^2 + 1} + 3 \asinh x_F \right]
\end{equation}

At every integration step, we have a value of the pressure $P$.
Then find the root $x_F$ of
\begin{equation}
	P(x_F) - P = 0
\end{equation}
and then calculate
\begin{equation}
	\diml{ϵ} = \diml{ϵ}(x_F) = \diml{P}(x_F) = \frac{m^4 c^3 r_0^3}{6 \pi m_0 \hbar^3} \left[ (2 x_F^3 + x_F) \sqrt{x_F^2 + 1} - \asinh x_F \right]
\end{equation}
\fi

\tablemaximum{../code/data/nr.dat}{M}{\maxMnr}
\tablemaximum{../code/data/gr.dat}{M}{\maxMgr}

\begin{figure}
\centering
\begin{tikzpicture}
\begin{axis}[
	width=15cm, height=10cm,
	xlabel=$R / \si{\kilo\meter}$, ylabel=$M / \solarmass$, title={Mass-radius relation for cold free Fermi gas neutron star}, title style={yshift=1.8cm},
	grid=major,
	colorbar horizontal, point meta=explicit, colormap name=plasmarev, colorbar style={xlabel=$\log_{10} (P_0 / \epsilon_0)$, xtick distance=3, minor x tick num=2, at={(0.5,1.03)}, anchor=south, xticklabel pos=upper},
	extra y ticks/.expanded={\maxMnr, \maxMgr}, extra y tick style={dashed}, % https://tex.stackexchange.com/a/333974
]
\addplot [mark=none, mesh, semithick] table [x expr={10*\thisrow{R}}, y=M, meta expr={log10(\thisrow{P})}] {../code/tov_fermi_gas/data/nr.dat} node [pos=0.47, pin={[text=black]0:Non-relativistic limit $p_F \ll 1$}] {};
\addplot [mark=none, mesh, semithick] table [x expr={10*\thisrow{R}}, y=M, meta expr={log10(\thisrow{P})}] {../code/tov_fermi_gas/data/gr.dat} node [pos=0.55, pin={[text=black]-100:Arbitrary $p_F$}] {};

%\edef\doplot{\noexpand\addplot [domain=-10:10, dashed, update limits=false] {\maxM};} % see https://tex.stackexchange.com/a/73916, https://tex.stackexchange.com/a/519
%\doplot

\end{axis}
\end{tikzpicture}

\caption{\label{fig:nstars:massradius}%
	Mass-radius relation for cold neutron stars parametrized by their central pressures $P_0$, obtained by numerically integrating the TOV equation from the central pressure $P_0$ until it vanishes at the surface.
	The numerical integration is carried out using an explicit equation of state in the non-relativistic limit with Fermi momenta $p_F \ll m c$, and using a root-finding algorithm to calculate an implicit equation of state for general $p_F$.
	Non-relativistic stars are parametrized by central pressures $\SI{1e-6}{} \epsilon_0 \le P_0 \le \SI{1e21}{} \epsilon_0$, while stars with arbitrary Fermi momenta are parametrized by $\SI{1e-6}{} \epsilon_0 \le P_0 \le \SI{1e17}{} \epsilon_0$, where $\epsilon_0 = \solarmass c^2 / (4 \pi R_0^3 / 3) = \SI{4.27e34}{\pascal}$, $\solarmass = \SI{1.99e30}{\kilogram}$ is the mass of the sun, $c = \SI{299792458}{\meter\per\second}$ is the speed of light and $R_0 = \SI{10}{\kilo\meter}$.
}

\end{figure}

\section{Discussion / stability analysis}

\TODO{discuss mass bound}

\TODO{compare with earlier results, most massive neutron stars}

\TODO{stability analysis, which parts of the plot are meaningful?}
