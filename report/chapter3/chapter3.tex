\chapter{Neutron stars modelled by a free Fermi gas}

\TODO{intro}

\section{Numerical solution of the \texorpdfstring{\\}{}Tolman-Oppenheimer-Volkoff equation}

In \cref{chap:tov}, we derived the Tolman-Oppenheimer-Volkoff equation
\begin{equation}
	\odv{p}{r} = -\frac{G m(r) \epsilon(r)}{r^2 c^2} \left[ 1 + \frac{p(r)}{\epsilon(r)} \right] \left[ 1 + \frac{4 \pi r^3 p(r)}{m(r) c^2} \right] \left[ 1 - \frac{2 G m(r)}{r c^2} \right]^{-1} ,
\label{eq:nstars:tov}
\end{equation}
and
\begin{equation}
	\odv{m}{r} = \frac{4 \pi r^2 \epsilon(r)}{c^2}
\label{eq:nstars:dmdr}
\end{equation}
In \cref{chap:tft}, we found $Z = Z(T, \mu, V)$ for two sample systems.
From $Z$, we could derive $P = P(T, \mu)$ and $\epsilon = \epsilon(T, \mu)$, where the volume $V$ has been eliminated by division, because $P$ and $\epsilon$ are intensive quantities independent of $V$.
At some fixed temperature $T$, we can eliminate $\mu$ to express $\epsilon$ in terms of $P$.
Thus, we can obtain an equation of state
\begin{equation}
	\epsilon = \epsilon(P).
\label{eq:nstars:eos}
\end{equation}
Given the equation of state, we can eliminate $\epsilon$ in terms of $P$ in \cref{eq:nstars:tov,eq:nstars:dmdr} and write the system
\TODO{small $p$ for momentum, big $P$ for pressure everywhere}
\begin{equation}
\begin{split}
	\odv{p(r)}{r} &= -\frac{G m(r) \epsilon(p(r))}{r^2 c^2} \left[ 1 + \frac{p(r)}{\epsilon(p(r))} \right] \left[ 1 + \frac{4 \pi r^3 p(r)}{m(r) c^2} \right] \left[ 1 - \frac{2 G m(r)}{r c^2} \right]^{-1}, \\
	\odv{m(r)}{r} &= \frac{4 \pi r^2 \epsilon(p(r))}{c^2}. \\
\end{split}
\end{equation}
This is a system of two first-order differential equations for $P(r)$ and $m(r)$!
Given a boundary condition, it can be integrated analytically or numerically to obtain $P(r)$ and $m(r)$.
One natural choice of boundary conditions is that, at the center, $m(r) = 0$, while $P(0)$ can be anything.
So we choose the boundary condition
\begin{equation}
	P(0) = P_0
	\quad \text{and} \quad
	m(0) = 0
	\quad \text{at} \quad
	r = 0
\end{equation}
for some $P_0$.
With different values for $P_0$, we essentially parametrize different stars with different pressure and mass profiles $P(r)$ and $m(r)$ at some fixed temperature $T$.
In particular, if $p(R) = 0$ for some $R$, we define $R$ as the radius of the star, and $M = m(R)$ as its mass.

In a few special situations, the TOV system can be solved analytically, like in \cref{sec:incompressible_star}.
More often, however, we must resort to solving the system numerically.
This is straightforward from the system \TODO{ref system}, which is of the form $\odv{\vec{y}}/{t} = \vec{f}(t, \vec{y})$ with $\vec{y} = [P, m]$ and $t = r$ and thus suitable for numerical integration methods like Runge-Kutta methods.
One issue, however, is the scale of variables.
For this purpose, we define the dimensionless quantities
\begin{equation}
	\diml{\epsilon}(r) = \frac{\epsilon(r)}{\epsilon_0}, \quad
	\diml{P}(r) = \frac{P(r)}{\epsilon_0}, \quad
	\diml{m}(r) = \frac{m(r)}{m_0}, \quad
	\diml{r}(r) = \frac{r}{r_0},
\end{equation}
where $m_0$ and $r_0$ are two natural scales of stellar mass and radius, and
\begin{equation}
	\epsilon_0 = \frac{m_0 c^2}{4 \pi r_0^3 / 3}
	%m_0 = \text{solar mass ?}, \quad
	%r_0 = \text{10 km ?}.
\end{equation}
Upon substitution into the TOV system, we get the dimensionless TOV system
\begin{equation}
	\frac{\epsilon_0}{r_0} \odv{\diml{p}}{\diml{r}} = -\frac{G m_0 \epsilon_0}{r_0^2 c^2} \frac{\diml{m} \diml{\epsilon}}{\diml{r}^2} \left[ 1 + \frac{\diml{p}}{\diml{\epsilon}} \right] \left[ 1 + \frac{4 \pi r_0^3 \epsilon_0}{m_0 c^2} \frac{\diml{r}^3 \diml{p}}{\diml{m}} \right] \left[ 1 - \frac{2 G m_0}{r_0 c^2} \frac{\diml{m}}{\diml{r}} \right]^{-1}.
\end{equation}
This can be simplified further.
First, insert the definition of $\epsilon_0$ in the middle parenthesis.
Second, observe that $G_0 = r_0 c^2 / m_0$ has the same units as the gravitational constant $G$, so define the dimensionless gravitational constant $\hat{G} = G / G_0 = G / (r_0 c^2 / m_0)$.
\begin{equation}
	\odv{\diml{p}}{\diml{r}} = - \frac{\diml{G} \diml{m} \diml{\epsilon}}{\diml{r}^2} \left[ 1 + \frac{\diml{p}}{\diml{\epsilon}} \right] \left[ 1 + \frac{3 \diml{r}^3 \diml{p}}{\diml{m}} \right] \left[ 1 - \frac{2 \diml{G} \diml{m}}{\diml{r}} \right]^{-1}.
\end{equation}
Similarly, the dimensionless mass equation becomes
\begin{equation}
	\frac{m_0}{r_0} \odv{\diml{m}(\diml{r})}{\diml{r}} = \frac{4 \pi r_0^2 \epsilon_0}{c^2} \diml{r}^2\diml{\epsilon}(\diml{p}(\diml{r}))
	\quad \text{or} \quad
	\odv{\diml{m}}{\diml{r}} = 3 \diml{r}^2 \diml{\epsilon}
\end{equation}
To summarize, the dimensionless TOV system is
\begin{equation}
\begin{split}
	\odv{\diml{P}}{\diml{r}} &= - \frac{\diml{G} \diml{m} \diml{\epsilon}(\diml{P}(\diml{r}))}{\diml{r}^2} \left[ 1 + \frac{\diml{p}}{\diml{\epsilon}(\diml{P}(\diml{r}))} \right] \left[ 1 + \frac{3 \diml{r}^3 \diml{p}}{\diml{m}} \right] \left[ 1 - \frac{2 \diml{G} \diml{m}}{\diml{r}} \right]^{-1}, \\
	\odv{\diml{m}}{\diml{r}} &= 3 \diml{r}^2 \diml{\epsilon}(\diml{P}(\diml{r})), \\
	\text{subject to } (\diml{P}(0), \diml{m}_0) &= (\diml{P}_0, 0) \text{ at $\diml{r} = 0$}.
\end{split}
\end{equation}

\section{Solution for free Fermi gas neutron star}

Let us use the logarithm of the partition function \eqref{eq:tft:dirac_partition_function} for free Dirac fermions to derive an equation of state inside a neutron star composed of neutrons only.
Then we will solve the TOV equation \eqref{eq:tov} with this equation of state and investigate the mass-radius relation of such neutron stars.

For easy reference, let us repeat the logarithm of the partition function \eqref{eq:tft:dirac_partition_function}.
It is
\begin{equation}
	\log Z = 2 V \int \frac{\dif^3 p}{(2 \pi \hbar)^3} \left\{ \beta E(\vec{p}) + \log \left[ e^{-\beta (E(\vec{p}) - \mu)}+1 \right] + \log \left[ e^{-\beta (E(\vec{p}) + \mu)} + 1\right] \right\}.
\end{equation}
First, the particle number density $n = \thermalavg{N}/V$ follows from the derivative \eqref{eq:tft:average_number} and is
\begin{equation}
	n = 
	\frac{1}{\beta} \pdv{\log Z}{\mu} =
	2 \int \frac{\dif^3 p}{(2 \pi \hbar)^3} \Big\{ n\big[ E(\vec{p})-\mu \big] - n\big[ E(\vec{p})+\mu \big] \Big\} ,
\label{eq:nstars:density}
\end{equation}
where we defined the \textbf{Fermi-Dirac distribution}
\begin{equation}
	n(E) = \frac{1}{e^{-\beta E} + 1}.
\label{eq:nstars:fermi_dirac_distribution}
\end{equation}
The $n$ on the left side of \cref{eq:nstars:density} denotes the particle number density $n = \thermalavg{N}/V$, while $n \left[ E(\vec{p}) \pm \mu \right]$ on the right is the Fermi-Dirac distribution \eqref{eq:nstars:fermi_dirac_distribution} and will \emph{always} be written \emph{with} an argument.
We hope this clearly distinguishes the two and that the naming appears intuitive -- soon we will get rid of $n(E)$, anyway.
From the density \eqref{eq:nstars:density}, we see that $n = n(\mu, T)$ is a function of the chemical potential $\mu$ and temperature $T$, so that at some fixed temperature, the value of $\mu$ decides the particle density $n$.

Second, we calculate the energy density $\epsilon = \thermalavg{N} / V$ from \cref{eq:tft:average_energy}.
It comes out as
\begin{equation}
	\epsilon = 
	\mu n - \frac{1}{V} \pdv{\log Z}{\beta} =
	2 \int \frac{\dif^3 p}{(2 \pi \hbar)^3} \Big\{ -E(\vec{p}) + E(\vec{p}) \, n\big[ E(\vec{p})-\mu \big] + E(\vec{p}) \, n\big[ E(\vec{p})+\mu \big] \Big\}.
\label{eq:nstars:energy_density}
\end{equation}

Third, we find that the pressure \eqref{eq:tft:average_pressure} is
\begin{equation}
	P =
	\frac{\log Z}{\beta V} = 
	2 \int \frac{\dif^3 p}{(2 \pi \hbar)^3} \Big\{ E(\vec{p}) + \log \left[ e^{-\beta(E(\vec{p})-\mu)} + 1 \right] + \log \left[ e^{-\beta(E(\vec{p})+\mu)} + 1 \right] \Big\}.
\label{eq:nstars:pressure}
\end{equation}

The first term of the energy density \eqref{eq:nstars:energy_density} and the pressure \eqref{eq:nstars:pressure} is infinite, as the integrand never decays.
This can be interpreted as an infinite shift of the vacuum energy.
In contrast, the next two terms are finite as the integrand is suppressed for large $\abs{\vec{p}}$ by the Fermi-Dirac distribution.
It makes no sense to include a term that integrates over every possible value of the momentum $\vec{p}$ for physical particles whose momentum cannot exceed a certain value due to energy conservation.
Here, we will make the assumption that we can simply drop this term.
Later, we will return to investigate this term by regularization and renormalization. \TODO{does this make sense?}

From the particle density \eqref{eq:nstars:density} at constant temperature $T$, we see that the sign of $n$ is determined by the sign of $\mu$.
The total density $n$ is expressed as a balance between \emph{particles} with energy $E(\vec{p}) > 0$ living relative to the chemical potential $\mu$ and \emph{antiparticles} with energy $E(\vec{p}) < 0$ living relative to the chemical potential $-\mu$.
Thus, the chemical potential $\mu$ determines the balance between particles and antiparticles in the system.
Similarly, the two last terms in the energy density \eqref{eq:nstars:energy_density} and pressure \eqref{eq:nstars:pressure} can be interpreted as contributions from particles and antiparticles.
We choose a large, positive value of $\mu > 0$, so that the particles dominate the system, while antiparticles are hardly present.
With this choice, $n \left[ E(\vec{p}) - \mu \right] \gg n \left[ E(\vec{p}) + \mu \right]$, and we drop the last term from the particle density \eqref{eq:nstars:density}, energy density \eqref{eq:nstars:energy_density} and pressure \eqref{eq:nstars:pressure}.

We are now left with the 
\begin{equation}
\begin{aligned}
	& \text{particle density} & n        &=  2 \int \frac{\dif^3 p}{(2 \pi \hbar)^3} \, n \left[ E(\vec{p})-\mu \right] , \\
	& \text{energy density}   & \epsilon &=  2 \int \frac{\dif^3 p}{(2 \pi \hbar)^3} \, E(\vec{p}) \, n \left[ E(\vec{p})-\mu \right] \text{ and} \\
	& \text{pressure}         & P        &= -2 \int \frac{\dif^3 p}{(2 \pi \hbar)^3} \, \log \left[ e^{-\beta(E(\vec{p})-\mu)} + 1 \right] . \\
\end{aligned}
\end{equation}
The integrals become nasty after plugging in the dispersion relation $E(\vec{p}) = \sqrt{\vec{p}^2 c^2 + m^2 c^4}$ \TODO{write in display form and refer back to this} and the Fermi-Dirac distribution \eqref{eq:nstars:fermi_dirac_distribution}, and it is overly optimistic to expect that all of them can be evaluated analytically.
We will make one final approximation that will make all the integrals surmountable.

By human standards, it is very hot inside a neutron star.
In fact, studies place typical core temperatures around $T_0 \approx \SI{1e6}{\kelvin}$.
However, neutrons have mass $m \approx \SI{1.67e-27}{\kilogram}$, so everywhere inside a neutron star we have $\beta E(\vec{p}) = \sqrt{\vec{p}^2 c^2 + m^2 c^4} / k_B T > m c^2 / k_B T_0 \approx 10^7 \gg 1$.
Although the temperature is very large compared to everyday temperatures, the thermal energy $k_B T$ is in fact very low relative to the energy $E(\vec{p})$ of the nuclei.
It is therefore an excellent approximation to take the \textbf{zero temperature limit}
\begin{equation}
	\beta E(\vec{p}) \gg 1 .
\end{equation}
In the zero temperature limit,
\begin{equation}
\begin{split}
	n(E)                                 &\rightarrow \theta(-E) = 1 - \theta(E), \\
	\log \left[ e^{-\beta E} + 1 \right] &\rightarrow -\beta E \theta(-E) = -\beta E [1 - \theta(E)]. \\
\end{split}
\end{equation}
as shown in \TODO{make simple figure with converging FD distribution}.
Thus, the zero temperature limit effectively limits the integration to those momenta $\vec{p}$ such that $E(\vec{p}) < \mu$.
We therefore call
\begin{equation}
	\mu = E_F = E(p_F) = \sqrt{p_F^2 c^2 + m^2 c^4}
\end{equation}
the Fermi energy and the corresponding momentum $p_F$ the Fermi momentum, representing the occupied state with highest energy and momentum.

Now the particle density \TODO{ref?} becomes simply
\begin{equation}
	n = 
	2 \int \frac{\dif p \, 4 \pi p^2}{(2 \pi \hbar)^3} \theta \left[ \mu - E(p) \right] =
	2 \int_0^{p_F} \frac{\dif p \, 4 \pi p^2}{(2 \pi \hbar)^3} = \frac{p_F^3}{3 \pi^2 \hbar^3} .
\end{equation}
Using integral \TODO{ref} and defining the dimensionless momentum $x = p / mc$, the energy density is
\begin{equation}
\begin{split}
	\epsilon &=  2 \int \frac{\dif p \, 4 \pi p^2}{(2 \pi \hbar)^3} \, E(p) \, \theta \left[ \mu - E(p) \right] \\
	         &=  2 \int_0^{p_F} \frac{\dif p \, 4 \pi p^2}{(2 \pi \hbar)^3} \, \sqrt{p^2 c^2 + m^2 c^4} \\
	         &= \frac{m^4 c^5}{\pi^2 \hbar^3} \int_0^{x_F} \dif x \, x^2 \sqrt{1 + x^2} \\
	         &= \frac{m^4 c^5}{8 \pi^2 \hbar^3} \left[ \left( 2 x_F^3 + x_F \right) \sqrt{1 + x_F^2} - \asinh x_F \right] . \\
\end{split}
\end{equation}
Finally, using the same integral, the pressure is
\begin{equation}
\begin{split}
	P &= \frac{2}{\beta} \int \frac{\dif p \, 4 \pi p^2}{(2 \pi \hbar)^3} \, \beta \left[ \mu - E(p) \right] \, \theta \left[ \mu - E(p) \right] \\
	  &= 2 \int_0^{p_F} \frac{\dif p \, 4 \pi p^2}{(2 \pi \hbar)^3} \, \left[ \sqrt{p_F^2 c^2 + m^2 c^4} - \sqrt{p^2 c^2 + m^2 c^4} \right] \\
	  &= \frac{m^4 c^5}{\pi^2 \hbar^3} \int_0^{x_F} \dif x \, x^2 \left[ \sqrt{x_F^2+1} - \sqrt{x^2+1} \right] \\
	  &= \frac{m^4 c^5}{24 \pi^2 \hbar^3} \left[ \left( 2 x_F^3 - 3 x_F \right) \sqrt{x_F^2 + 1} + 3 \asinh x_F \right] . \\
\end{split}
\end{equation}

The equation of state $\epsilon = \epsilon(P)$ follows by eliminating $x_F$, or equivalently $\mu$.
Due to the complicated dependence of $\epsilon$ and $P$ on $x_F$, we will consider three cases of increasing difficulty.

\subsection{Ultra-relativistic limit}

Ultra-relativistic limit, $x_F \gg 1$.
\begin{equation}
\begin{split}
	P        &\taylor \frac{1}{24} \frac{m^4 c^5}{\pi^2 \hbar^3} \left[ 2 x^{4} - 2 x^{2} + 3 \log 2 + 3 \log x - \frac{7}{4} \right] \taylor \frac{1}{12} \frac{m^4 c^5 x_F^4}{\pi^2 \hbar^3} \\
	\epsilon &\taylor \frac{1}{8}  \frac{m^4 c^5}{\pi^2 \hbar^3} \left[ 2 x^{4} + 2 x^{2} - \log 2 - \log x + \frac{1}{4} \right] \taylor \frac{1}{4} \frac{m^4 c^5 x_F^4}{\pi^2 \hbar^3} \\
\end{split}
\end{equation}
Equation of state
\begin{equation}
	\epsilon = 3 P
\end{equation}

Can solve with trial solution
\begin{equation}
	P(r) = A r^n
\end{equation}
\begin{equation}
	\odv{m(r)}{r} = \frac{12 \pi A}{c^2} r^{n+2},
	\quad \text{so} \quad
	m(r) = \frac{12 \pi A}{(n+3) c^2} r^{n+3}
	\quad (n \neq -3)
\end{equation}
Insert into TOV
\begin{equation}
	n A r^{n-1} =
	-\frac{48 \pi G A^2 r^{2n+1}}{(n+3) c^4} \left[ 2 + \frac{n}{3} \right] \left[ 1 - \frac{24 \pi G A r^{n+2}}{(n+3) c^4} \right]^{-1}
\end{equation}
Must choose $n = -2$.
\begin{equation}
	\frac{2 A}{r^3} \left[ -1 + \frac{32 \pi G A}{c^4} \left( 1 - \frac{24 \pi G A}{c^4} \right)^{-1} \right] = 0
\end{equation}
or
\begin{equation}
	\frac{2 A}{r^3} \left[ -1 + \frac{56 \pi G A}{c^4} \right] = 0
\end{equation}
Then we must choose
\begin{equation}
	A = \frac{c^4}{56 \pi G}
\end{equation}
so
\begin{equation}
	P(r) = \frac{c^4}{56 \pi G} \frac{1}{r^2}
\end{equation}
and
\begin{equation}
	m(r) = \frac{3 c^2 r}{14 G}
\end{equation}

\subsection{Non-relativistic limit}

Non-relativistic limit, $x_F \ll 1$.
\begin{equation}
\begin{split}
	%\epsilon &\taylor \frac{m c^2 p_F^3}{3 \pi^2 \hbar^3} + \frac{p_F^5}{10 \pi^2 \hbar^3 m} = n m c^2 + \frac{p_F^5}{10 \pi^2 \hbar^3 m} \\
	%P        &\taylor \frac{p_F^5}{15 \pi^2 \hbar^3 m}
	\epsilon &\taylor \frac{m c^2 p_F^3}{3 \pi^2 \hbar^3} \\
	P        &\taylor \frac{p_F^5}{15 \pi^2 \hbar^3 m}
\end{split}
\end{equation}
Eliminate $p_F$, equation of state
\begin{equation}
	\epsilon = \frac{mc^2}{3\pi^2\hbar^3} \left( 15 \pi^2 \hbar^3 m P \right)^{3/5}
\end{equation}

Cannot solve TOV exactly.
Solve it numerically.

Dimensionless quantities
\begin{equation}
	\diml{\epsilon} = \frac{\epsilon}{\epsilon_0}, \quad
	\diml{P} = \frac{P}{\epsilon_0}, \quad
	\diml{r} = \frac{r}{r_0}, \quad
	\diml{m} = \frac{m}{m_0}.
\end{equation}
where
\begin{equation}
	\epsilon_0 = \frac{m_0 c^2}{4 \pi r_0^3 / 3}, \quad
	m_0 = \text{solar mass ?}, \quad
	r_0 = \text{10 km ?}.
\end{equation}
Dimensionless TOV
\begin{equation}
	\frac{\epsilon_0}{r_0} \odv{\diml{p}}{\diml{r}} = -\frac{G m_0 \epsilon_0}{r_0^2 c^2} \frac{\diml{m} \diml{\epsilon}}{\diml{r}^2} \left[ 1 + \frac{\diml{p}}{\diml{\epsilon}} \right] \left[ 1 + \frac{4 \pi r_0^3 \epsilon_0}{m_0 c^2} \frac{\diml{r}^3 \diml{p}}{\diml{m}} \right] \left[ 1 - \frac{2 G m_0}{r_0 c^2} \frac{\diml{m}}{\diml{r}} \right]^{-1}.
\end{equation}
Inserting $\epsilon_0 = \ldots$ and defining the dimensionless gravitational constant $\hat{G} = G / (r_0 c^2 / m_0)$, we get
\begin{equation}
	\odv{\diml{p}}{\diml{r}} = - \frac{\diml{G} \diml{m} \diml{\epsilon}}{\diml{r}^2} \left[ 1 + \frac{\diml{p}}{\diml{\epsilon}} \right] \left[ 1 + \frac{3 \diml{r}^3 \diml{p}}{\diml{m}} \right] \left[ 1 - \frac{2 \diml{G} \diml{m}}{\diml{r}} \right]^{-1}.
\end{equation}
Dimensionless mass equation
\begin{equation}
	\odv{\diml{m}}{\diml{r}} = 3 \diml{r}^2 \diml{\epsilon}
\end{equation}
Dimensionless equation of state
\begin{equation}
	\diml{\epsilon} = \left[ \frac{4^2 5^3}{3^4 \pi^2} \frac{m^8 c^6 r_0^6}{m_0^2 \hbar^6} \diml{P}^3 \right]^{\frac{1}{5}}
\end{equation}
Numerical equation to solve
\begin{equation}
	\odv{}{r} \begin{bmatrix} m(r) \\ P(r) \\ \end{bmatrix} = f \left( \begin{bmatrix} m(r) \\ P(r) \\ \end{bmatrix} \right)
	\quad \text{with boundary conditions} \quad
	\begin{bmatrix} m(0) \\ P(0) \\ \end{bmatrix} = \begin{bmatrix} 0 \\ P_0 \\ \end{bmatrix}.
\end{equation}
Each run is parametrized by some central pressure $P_0$.
At each integration step, the equation of state $\diml\epsilon = \diml\epsilon(\diml{P})$ is used.

\section{General}

How to eliminate $x_F$ in the general case?

\begin{equation}
	\diml{P}(x_F) = \frac{m^4 c^3 r_0^3}{18 \pi m_0 \hbar^3} \left[ (2 x_F^3 - 3 x_F) \sqrt{x_F^2 + 1} + 3 \asinh x_F \right]
\end{equation}

At every integration step, we have a value of the pressure $P$.
Then find the root $x_F$ of
\begin{equation}
	P(x_F) - P = 0
\end{equation}
and then calculate
\begin{equation}
	\diml{ϵ} = \diml{ϵ}(x_F) = \diml{P}(x_F) = \frac{m^4 c^3 r_0^3}{6 \pi m_0 \hbar^3} \left[ (2 x_F^3 + x_F) \sqrt{x_F^2 + 1} - \asinh x_F \right]
\end{equation}
