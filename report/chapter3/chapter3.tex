\chapter{Cold free fermion neutron stars}
\label{chap:nstars}

\TODO{better title}

\TODO{inspiration, references}

\TODO{intro}

\TODO{inspiration \cite{ref:stability_methods}, \cite{ref:glendenning}}

In \cref{chap:tft}, we found $Z = Z(T, \mu, V)$ for two sample systems, and in particular a gas of free Dirac fermions.
From $Z$, we can derive the pressure $P = P(T, \mu)$ and energy density $\epsilon = \thermalavg{E}/V = \epsilon(T, \mu)$ using \cref{eq:tft:average_number,eq:tft:average_energy,eq:tft:average_pressure}, where the volume $V$ has been eliminated by division, because $P$ and $\epsilon$ are intensive quantities.
At some fixed temperature $T$, we can therefore eliminate $\mu$ to express $\epsilon$ in terms of $P$.
This gives us an equation of state $\epsilon = \epsilon(P)$ with which we can solve the Tolman-Oppenheimer-Volkoff system \eqref{eq:tov:tovsys}.

In this chapter, we will use partition function \eqref{eq:tft:dirac_partition_function} for free Dirac fermions and do exactly this to model a neutron star with a cold ideal Fermi gas of neutrons.
\TODO{bad wording: use something like ``equation of state for free/ideal cold fermi gas?}

\section{Equation of state}

For easy reference, the logarithm of the free Dirac fermion partition function \eqref{eq:tft:dirac_partition_function} is
\begin{equation}
	\log Z = 2 V \int \frac{\dif^3 p}{(2 \pi \hbar)^3} \bigg\{ \beta E(\vec{p}) + \log \left[ e^{-\beta (E(\vec{p}) - \mu)}+1 \right] + \log \left[ e^{-\beta (E(\vec{p}) + \mu)} + 1\right] \bigg\}.
\end{equation}
First, the particle number density $n = \thermalavg{N}/V$ follows from the derivative \eqref{eq:tft:average_number} and is
\begin{equation}
	n = 
	\frac{1}{\beta} \pdv{\log Z}{\mu} =
	2 \int \frac{\dif^3 p}{(2 \pi \hbar)^3} \Big\{ n\big[ E(\vec{p})-\mu \big] - n\big[ E(\vec{p})+\mu \big] \Big\} ,
\label{eq:nstars:density}
\end{equation}
where we defined the \textbf{Fermi-Dirac distribution}
\begin{equation}
	n(E) = \frac{1}{e^{-\beta E} + 1}.
\label{eq:nstars:fermi_dirac_distribution}
\end{equation}
Do not confuse the particle density $n$ on the left with the Fermi-Dirac distributions $n[E(\vec{p}) \mp \mu]$ on the right!
We will soon perform the integral over $\vec{p}$ and get rid of $n[E(\vec{p}) \mp \mu]$, anyway.
From the density \eqref{eq:nstars:density}, we see that $n = n(\mu, T)$ is a function of the chemical potential $\mu$ and temperature $T$, so that at some fixed temperature, the value of $\mu$ determines the particle density $n$.

Second, we calculate the energy density $\epsilon = \thermalavg{E} / V$ from \cref{eq:tft:average_energy}.
It comes out as \TODO{different wording}
\begin{equation}
	\epsilon = 
	\mu n - \frac{1}{V} \pdv{\log Z}{\beta} =
	2 \int \frac{\dif^3 p}{(2 \pi \hbar)^3} \Big\{ -E(\vec{p}) + E(\vec{p}) \, n\big[ E(\vec{p})-\mu \big] + E(\vec{p}) \, n\big[ E(\vec{p})+\mu \big] \Big\}.
\label{eq:nstars:energy_density}
\end{equation}

Third, we find that the pressure \eqref{eq:tft:average_pressure} is
\begin{equation}
	P =
	\frac{\log Z}{\beta V} = 
	2 \int \frac{\dif^3 p}{(2 \pi \hbar)^3} \bigg\{ E(\vec{p}) + \log \left[ e^{-\beta(E(\vec{p})-\mu)} + 1 \right] + \log \left[ e^{-\beta(E(\vec{p})+\mu)} + 1 \right] \bigg\}.
\label{eq:nstars:pressure}
\end{equation}

The first term of the energy density \eqref{eq:nstars:energy_density} and the pressure \eqref{eq:nstars:pressure} is infinite, as the integrand never decays \TODO{decay usually associated with radiation}.
This can be interpreted as an infinite shift of the vacuum energy.
In contrast, the two last terms are finite as the integrand is suppressed \TODO{?} for large $\abs{\vec{p}}$ by the Fermi-Dirac distribution.
It makes no sense to include a term that integrates over every possible value of the momentum $\vec{p}$ for physical particles whose momentum cannot exceed a certain value due to energy conservation.
Here, we will make the assumption that we can simply drop the infinite term.
We will return later to investigate this term by regularization and renormalization. \TODO{does this make sense?} \TODO{improve convergence/divergence arguments, look at measure $p^2$, etc.}

From the particle density \eqref{eq:nstars:density} at constant temperature $T$, we see that the sign of $n$ is determined by the sign of $\mu$.
The total density $n$ is expressed as a balance between \emph{particles} with energy $E(\vec{p}) > 0$ living relative to the chemical potential $\mu$ and \emph{antiparticles} with energy $E(\vec{p}) < 0$ living relative to the chemical potential $-\mu$.
Thus, the chemical potential $\mu$ determines the balance between particles and antiparticles in the system.
Similarly, the two last terms in the energy density \eqref{eq:nstars:energy_density} and pressure \eqref{eq:nstars:pressure} can be interpreted as contributions from particles and antiparticles.
We choose a large, positive value of $\mu > 0$, so that the particles dominate the system, while antiparticles are hardly present.
With this choice, $n \left[ E(\vec{p}) - \mu \right] \gg n \left[ E(\vec{p}) + \mu \right]$, and we drop the last term from the particle density \eqref{eq:nstars:density}, energy density \eqref{eq:nstars:energy_density} and pressure \eqref{eq:nstars:pressure}.

Dropping terms as described in the last paragraphs, we are left with the%
\begin{subequations}%
\begin{align}%
	& \text{particle density} & n        &=  2 \int \frac{\dif^3 p}{(2 \pi \hbar)^3} \, n \left[ E(\vec{p})-\mu \right] ,                    \label{eq:nstars:dropped_infinities_density} \\
	& \text{energy density}   & \epsilon &=  2 \int \frac{\dif^3 p}{(2 \pi \hbar)^3} \, E(\vec{p}) \, n \left[ E(\vec{p})-\mu \right] ,      \label{eq:nstars:dropped_infinities_energy_density} \\
	& \text{pressure}         & P        &= -2 \int \frac{\dif^3 p}{(2 \pi \hbar)^3} \, \log \left[ e^{-\beta(E(\vec{p})-\mu)} + 1 \right] . \label{eq:nstars:dropped_infinities_pressure}
\end{align}%
\label{eq:nstars:dropped_infinities}%
\end{subequations}%
The integrals become nasty after plugging in the relativistic dispersion relation \eqref{eq:tft:dispersion} and the Fermi-Dirac distribution \eqref{eq:nstars:fermi_dirac_distribution}, and it is overly optimistic to expect that all of them can be evaluated analytically.
We will make one final approximation that will make all the integrals surmountable.

\TODO{what about $\beta \mu \gg 1$? assume/take this into account.}
By human standards, it is very hot inside a neutron star.
In fact, studies estimate typical core temperatures around $T_0 \approx \SI{1e6}{\kelvin}$. \TODO{reference}
However, neutrons have mass $m \approx \SI{1.67e-27}{\kilogram}$, so everywhere inside a neutron star we have $\beta E(\vec{p}) = \sqrt{\vec{p}^2 c^2 + m^2 c^4} / k_B T > m c^2 / k_B T_0 \approx 10^7 \gg 1$.
Although the temperature is very large compared to everyday temperatures, the thermal energy $k_B T$ is in fact very low relative to the energy $E(\vec{p})$ of the nuclei.
It is therefore an excellent approximation to take the \textbf{zero-temperature limit}
\begin{equation}
	\beta E(\vec{p}) \gg 1 .
\end{equation}

\begin{figure}
	\centering
	\begin{tikzpicture}
		\begin{groupplot}[group style={group size=2 by 1, horizontal sep=5pt}, height=6cm, width=7cm, /tikz/declare function={
			n(\b,\E) = 1/(exp(\b*\E)+1);
		}]
			\nextgroupplot[xlabel=$E$, ylabel=$1/(e^{\beta E}+1)$, 
				colorbar horizontal, colorbar sampled, colorbar style={xlabel=$\beta$, samples=7, xtick={-0.5, 0.5, 1.5, 2.5, 3.5, 4.5}, xticklabels={$1/2$, $1$, $2$, $4$, $8$, $16$, $32$}, grid=none, xticklabel pos=upper, tickwidth=0, at={(0.0,1.03)}, anchor=south west},
				every colorbar/.append style={width=2*\pgfkeysvalueof{/pgfplots/parent axis width}+\pgfkeysvalueof{/pgfplots/group/horizontal sep}},
				xtick distance=10, ytick distance=0.5, grid=none, minor tick num=1,
			]
			\pgfplotsinvokeforeach{0.5, 1, 2, 4, 8, 16, 32}{
				\addplot[domain=-10:+10, samples=200, mesh, point meta={log2(#1)}] {n(#1, x)};
			}

			\nextgroupplot[xlabel=$E$, ylabel=$\log (e^{-\beta E}+1)/\beta$, xtick distance=10, yticklabel pos=right, ytick distance=5, minor tick num=1, grid=none]
			\pgfplotsinvokeforeach{0.5, 1, 2, 4, 8, 16, 32}{
				\addplot[domain=-10:+10, samples=200, mesh, point meta={log2(#1)}] {ln(exp(-#1*x)+1)/#1};
			}
		\end{groupplot}
	\end{tikzpicture}
\caption{\label{fig:nstars:distribution_convergence}%
	In the zero-temperature limit $\beta E \gg 1$, the Fermi-Dirac distribution $1/(e^{\beta E}+1) \rightarrow \theta(-E)$, while $\log (e^{-\beta E} + 1) \rightarrow -\beta E \, \theta(-E)$, where $\theta(E)$ is the step function.
}
\end{figure}

In the zero-temperature limit, the Fermi-Dirac distribution can be replaced by
\begin{equation}
	n(E)       =                 \begin{cases} 1 / (e^{+\beta \abs{E}}+1) & (E>0) \\ 1 / (e^{-\beta \abs{E}}+1) & (E<0) \end{cases}
	     \quad \rightarrow \quad \begin{cases} 0 & (E>0) \\ 1 & (E<0) \end{cases}
	     \quad =                 \theta(-E) ,
\end{equation}
while the logarithm in the pressure behaves as
\begin{equation}
	\log (e^{-\beta E}+1)       =                 \begin{cases} \log (e^{-\beta \abs{E}}+1) & (E>0) \\ \log (e^{+\beta \abs{E}}+1) & (E<0) \end{cases}
	                      \quad \rightarrow \quad \begin{cases} \log 1 & (E>0) \\ \log e^{-\beta E} & (E<0) \end{cases}
	                      \quad =                 -\beta E \, \theta(-E) .
\end{equation}
This is confirmed by the plots in \cref{fig:nstars:distribution_convergence} for increasing values of $\beta$.
Thus, the zero-temperature limit effectively limits the integrals \eqref{eq:nstars:dropped_infinities} to those momenta $\vec{p}$ with $E(\vec{p}) < \mu$.
We therefore call
\begin{equation}
	\mu = E_F = E(p_F) = \sqrt{p_F^2 c^2 + m^2 c^4}
\end{equation}
the Fermi energy and the corresponding momentum $p_F$ the Fermi momentum, representing the occupied state in momentum-space with largest energy and momentum.

Now the particle density \eqref{eq:nstars:dropped_infinities_density} simply becomes
\begin{equation}
	n = 
	2 \int_0^\infty \frac{\dif p \, 4 \pi p^2}{(2 \pi \hbar)^3} \theta \left[ \mu - E(p) \right] =
	2 \int_0^{p_F} \frac{\dif p \, 4 \pi p^2}{(2 \pi \hbar)^3} = \frac{p_F^3}{3 \pi^2 \hbar^3} .
\label{eq:nstars:density_zeroT}
\end{equation}
Using integral \TODO{ref} and defining the dimensionless momentum $x = p / mc$, the energy density \eqref{eq:nstars:dropped_infinities_energy_density} is
\begin{equation}
\begin{split}
	\epsilon &=  2 \int_0^\infty \frac{\dif p \, 4 \pi p^2}{(2 \pi \hbar)^3} \, E(p) \, \theta \left[ \mu - E(p) \right] \\
	         &=  2 \int_0^{p_F} \frac{\dif p \, 4 \pi p^2}{(2 \pi \hbar)^3} \, \sqrt{p^2 c^2 + m^2 c^4} \\
	         &= \frac{m^4 c^5}{\pi^2 \hbar^3} \int_0^{x_F} \dif x \, x^2 \sqrt{1 + x^2} \\
	         &= \frac{m^4 c^5}{8 \pi^2 \hbar^3} \left[ \left( 2 x_F^3 + x_F \right) \sqrt{1 + x_F^2} - \asinh x_F \right] . \\
\end{split}
\label{eq:nstars:energy_density_zeroT}
\end{equation}
Finally, using the same integral, the pressure \eqref{eq:nstars:dropped_infinities_pressure} is
\begin{equation}
\begin{split}
	P &= \frac{2}{\beta} \int_0^\infty \frac{\dif p \, 4 \pi p^2}{(2 \pi \hbar)^3} \, \beta \left[ \mu - E(p) \right] \, \theta \left[ \mu - E(p) \right] \\
	  &= 2 \int_0^{p_F} \frac{\dif p \, 4 \pi p^2}{(2 \pi \hbar)^3} \, \left[ \sqrt{p_F^2 c^2 + m^2 c^4} - \sqrt{p^2 c^2 + m^2 c^4} \right] \\
	  &= \frac{m^4 c^5}{\pi^2 \hbar^3} \int_0^{x_F} \dif x \, x^2 \left[ \sqrt{x_F^2+1} - \sqrt{x^2+1} \right] \\
	  &= \frac{m^4 c^5}{24 \pi^2 \hbar^3} \left[ \left( 2 x_F^3 - 3 x_F \right) \sqrt{x_F^2 + 1} + 3 \asinh x_F \right] . \\
\end{split}
\label{eq:nstars:pressure_zeroT}
\end{equation}

The equation of state $\epsilon = \epsilon(P)$ follows by eliminating $x_F$ from the energy density \eqref{eq:nstars:energy_density_zeroT} and pressure \eqref{eq:nstars:pressure_zeroT}.
Due to their complicated dependence on $x_F$, we will do so in three cases of increasing difficulty.

\TODO{first present three cases, \emph{then} do calculations, to not overload the reader}

\subsection{Ultra-relativistic limit}
\label{sec:nstars:ur_limit}

First, consider the ultra-relativistic limit
\begin{equation}
	x_F \gg 1 , 
\label{eq:nstars:ur_limit}
\end{equation}
where the Fermi energy $E_F = \sqrt{p_F^2 c^2 + m^2 c^4} \taylor p_F c$ is dominated by the contribution from the Fermi momentum.
Since $\asinh x_F = \log \left[ x_F + \sqrt{x_F^2 + 1} \right] \taylor \log 2 x_F$ diverges logarithmically, we see that both the energy density \eqref{eq:nstars:energy_density_zeroT} and pressure \eqref{eq:nstars:pressure_zeroT} are dominated by their first term with $2 x_F^3 \sqrt{x_F^2 + 1} \taylor 2 x_F^4$.
In the ultra-relativistic limit, then,
\begin{equation}
	\epsilon \taylor \frac{m^4 c^5 x_F^4}{4 \pi^2 \hbar^3}
	\qquad \text{and} \qquad
	P        \taylor \frac{m^4 c^5 x_F^4}{12 \pi^2 \hbar^3},
\end{equation}
and $x_F$ is easily eliminated, yielding the very simple equation of state
\begin{equation}
	\epsilon = 3 P .
\label{eq:nstars:ur_eos}
\end{equation}

In this particular case, the TOV system \eqref{eq:tov:tovsys} can be solved analytically with the polynomial trial solution
\begin{equation}
	P(r) = A r^n .
\label{eq:nstars:ur_ansatz}
\end{equation}
Then the mass equation \eqref{eq:tov:tovsys_mass} is
\begin{equation}
	\odv{m}{r} = \frac{12 \pi A}{c^2} r^{n+2},
	\qquad \text{so} \qquad
	m(r) = \frac{12 \pi A}{(n+3) c^2} r^{n+3}
	\quad (n \neq -3).
\label{eq:nstars:ur_mass}
\end{equation}
With the equation of state \eqref{eq:nstars:ur_eos}, mass \eqref{eq:nstars:ur_mass} and trial solution \eqref{eq:nstars:ur_ansatz}, the TOV equation \eqref{eq:tov:tovsys_pressure} reads
\begin{equation}
	n A r^{n-1} =
	-\frac{48 \pi G A^2 r^{2n+1}}{(n+3) c^4} \left[ 2 + \frac{n}{3} \right] \left[ 1 - \frac{24 \pi G A r^{n+2}}{(n+3) c^4} \right]^{-1} .
\end{equation}
We can attain equality for all $r$ if we choose $n = -2$.
Then the rightmost factor no longer depends on $r$, and both sides have the same $r^{-3}$-dependence
\begin{equation}
	- 2 A r^{-3} = - \frac{64 \pi G A^2 r^{-3}}{c^4} \left[ 1 - \frac{24 \pi G A}{c^4} \right]^{-1} .
\end{equation}
Equality is established if we match the prefactors by choosing $A = c^4 / 56 \pi G$.
Then the solutions for the pressure and mass are
\begin{equation}
	P(r) = \frac{c^4}{56 \pi G} \frac{1}{r^2}
	\qquad \text{and} \qquad
	m(r) = \frac{3 c^2}{14 G} r .
\end{equation}
This is a highly unphysical result.
The pressure diverges at the center, so nothing could ever hold such a star together.
In addition, $p(r) > 0$ for all $r$, so the star has no surface and hence infinite mass $M = m(\infty) = \infty$.

\TODO{make plot of $P(r)$ and $m(r)$ for a star.}

\subsection{Non-relativistic limit}
\label{sec:nstars:nr_limit}

Next, let us consider the more difficult non-relativistic limit
\begin{equation}
	x_F \ll 1,
\label{eq:nstars:nr_limit}
\end{equation}
where the Fermi energy $E_F = \sqrt{p_F^2 c^2 + m^2 c^4} \taylor m c^2$ is dominated by the rest energy of the fermions.
Taylor expanding the energy density \eqref{eq:nstars:energy_density_zeroT} and presure \eqref{eq:nstars:pressure_zeroT} around $x_F = 0$ to lowest order, we find
\begin{equation}
	%\epsilon &\taylor \frac{m c^2 p_F^3}{3 \pi^2 \hbar^3} + \frac{p_F^5}{10 \pi^2 \hbar^3 m} = n m c^2 + \frac{p_F^5}{10 \pi^2 \hbar^3 m} \\
	%P        &\taylor \frac{p_F^5}{15 \pi^2 \hbar^3 m}
	\epsilon \taylor \frac{m c^2 p_F^3}{3 \pi^2 \hbar^3}
	\qquad \text{and} \qquad
	P        \taylor \frac{p_F^5}{15 \pi^2 \hbar^3 m} .
\end{equation}
Note that with the density \eqref{eq:nstars:density}, the energy density can be written $\epsilon = n m c^2$, so it is only due to the rest mass of the particles, as if all fermions have broken free from the Pauli exclusion principle and possess the same rest energy $m c^2$.
This is only a mathematical feature of the non-relativistic limit -- the fermions still occupy different states with different momentum, but the momenta are so small that the differences are negligible compared to the rest energy $mc^2$.
Again, it is straightforward to eliminate $x_F$ to find the equation of state, only this time there is some extra bookkeeping with all the exponents.
Carefully gathering all the prefactors under the same roof, we find
\begin{equation}
	%\epsilon = \frac{mc^2}{3\pi^2\hbar^3} \left( 15 \pi^2 \hbar^3 m P \right)^{3/5}
	\epsilon = \left( \frac{5^3 m^8 c^{10}}{3^2 \pi^4 \hbar^6} \right)^{\frac15}  P^{\frac35} .
\end{equation}
With this power dependence, it is not easy, if even possible, to solve the TOV equation analytically.
The trial solution \eqref{eq:nstars:ur_ansatz} we employed in \cref{sec:nstars:ur_limit} fails miserably, as we do not get the same fortunate cancellations of $r$.
We therefore resort to the numerical solution method described in \cref{sec:nstars:numtov}, parametrizing different stars by their center pressure $P_0$ and integrating the TOV equation until the pressure $p(R)$ vanishes, using the corresponding radius $R$ to establish the mass $M = m(R)$ of the star.
The results are shown in \cref{fig:nstars:massradius}.

\iffalse
Dimensionless equation of state
\begin{equation}
	\diml{\epsilon} = \left[ \frac{4^2 5^3}{3^4 \pi^2} \frac{m^8 c^6 r_0^6}{m_0^2 \hbar^6} \diml{P}^3 \right]^{\frac{1}{5}}
\end{equation}
\fi

\subsection{General Fermi momenta}
\label{sec:nstars:gr_limit}

How can we find the energy density
\begin{equation}
	\epsilon = \frac{m^4 c^5}{8 \pi^2 \hbar^3} \left[ \left( 2 x_F^3 + x_F \right) \sqrt{1 + x_F^2} - \asinh x_F \right]
\label{eq:nstars:gr_limit_energy_density}
\end{equation}
that corresponds to a given pressure
\begin{equation}
	P = \frac{m^4 c^5}{24 \pi^2 \hbar^3} \left[ \left( 2 x_F^3 - 3 x_F \right) \sqrt{x_F^2 + 1} + 3 \asinh x_F \right] 
\label{eq:nstars:gr_limit_pressure}
\end{equation}
for general $x_F$?
Since we are already solving the TOV equation on a computer, we can do so by numerical root finding.
At every step $r$ in the numerical integration algorithm, we know the current pressure $P = P(r)$.
Using a numerical root finding algorithm, we can find the root $x_F$ of the function
\begin{equation}
	f(x_F) = P(x_F) - P = 0,
\end{equation}
where $P(x_F)$ is the pressure \eqref{eq:nstars:gr_limit_pressure} as a function of $x_F = p_F / m c$.
Having found the root, we can simply calculate the corresponding energy density $\epsilon(x_F)$ from \cref{eq:nstars:gr_limit_energy_density}.
This whole procedure can be elegantly encapsulated into a function that implements an implicit equation of state $\epsilon = \epsilon(P)$, which in turn is straightforward to plug into our solver described in \cref{sec:nstars:numtov}.
The results are shown in \cref{fig:nstars:massradius}.

\iffalse
\begin{equation}
	\diml{P}(x_F) = \frac{m^4 c^3 r_0^3}{18 \pi m_0 \hbar^3} \left[ (2 x_F^3 - 3 x_F) \sqrt{x_F^2 + 1} + 3 \asinh x_F \right]
\end{equation}

At every integration step, we have a value of the pressure $P$.
Then find the root $x_F$ of
\begin{equation}
	P(x_F) - P = 0
\end{equation}
and then calculate
\begin{equation}
	\diml{ϵ} = \diml{ϵ}(x_F) = \diml{P}(x_F) = \frac{m^4 c^3 r_0^3}{6 \pi m_0 \hbar^3} \left[ (2 x_F^3 + x_F) \sqrt{x_F^2 + 1} - \asinh x_F \right]
\end{equation}
\fi

\tablemaximum{../code/data/nr.dat}{M}{\maxMnr}{R}{\maxRnr}
\tablemaximum{../code/data/gr.dat}{M}{\maxMgr}{R}{\maxRgr}

\begin{figure}
\centering
\begin{tikzpicture}
\begin{axis}[
	width=15cm, height=15cm,
	xlabel=$R / \si{\kilo\meter}$, ylabel=$M / \solarmass$, title={Mass-radius relation for cold free Fermi gas neutron star}, title style={yshift=2.3cm},
	xmin=0, xmax=50, xtick distance=5, minor x tick num=4,
	ymin=0, ymax=1.1, ytick distance=0.1, minor y tick num=9,
	grid=major,
	colorbar horizontal, point meta=explicit, colormap name=plasmarev, colorbar style={xlabel=$\log_{10} (P_0 / \epsilon_0)$, xtick distance=1, minor x tick num=0, at={(0.5,1.03)}, anchor=south, xticklabel pos=upper},
	%extra y ticks/.expanded={\maxMnr, \maxMgr}, extra y tick style={dashed}, % https://tex.stackexchange.com/a/333974
	%extra x ticks/.expanded={{10*\maxRnr}, {10*\maxRgr}}, extra x tick style={dashed, tick label style={yshift=-1ex}},
]
\addplot [mark=none, mesh, thick] table [x expr={10*\thisrow{R}}, y=M, meta expr={log10(\thisrow{P})}] {../code/data/nr.dat} node [pos=0.67, pin={[text=black]0:Non-relativistic limit $p_F \ll mc$}] {};
\addplot [mark=none, mesh, thick] table [x expr={10*\thisrow{R}}, y=M, meta expr={log10(\thisrow{P})}] {../code/data/gr.dat} node [pos=0.61, pin={[text=black]-100:Arbitrary $p_F$}] {};
\node [circle, fill, inner sep=1pt, label={90:$(\pgfmathprintnumber\maxRnr, \pgfmathprintnumber\maxMnr)$}] at ({10*\maxRnr}, \maxMnr) {};
\node [circle, fill, inner sep=1pt, label={90:$(\pgfmathprintnumber\maxRgr, \pgfmathprintnumber\maxMgr)$}] at ({10*\maxRgr}, \maxMgr) {};

%\edef\doplot{\noexpand\addplot [domain=-10:10, dashed, update limits=false] {\maxM};} % see https://tex.stackexchange.com/a/73916, https://tex.stackexchange.com/a/519
%\doplot

\end{axis}
\end{tikzpicture}

\caption{\label{fig:nstars:massradius}%
Mass-radius relation for cold neutron stars parametrized by their central pressures $P_0$, obtained by numerically integrating the TOV equation from the central pressure $P_0$ until it vanishes at the surface $R$.
The numerical integration is carried out using an explicit equation of state in the non-relativistic limit with Fermi momenta $p_F \ll m c$, and using a root-finding algorithm to calculate an implicit equation of state for general $p_F$.
Stars are parametrized by central pressures $\SI{1e-6}{} \epsilon_0 \le P_0 \le \SI{1e7}{} \epsilon_0$, where $\epsilon_0 = \solarmass c^2 / (4 \pi R_0^3 / 3) = \SI{4.27e34}{\pascal}$, $R_0 = \SI{10}{\kilo\meter}$, $\solarmass$ is the mass of the sun and $c$ is the speed of light.
}

\end{figure}

\begin{figure}
\centering
\begin{tikzpicture}
\begin{groupplot}[
	group style={group size=1 by 2, vertical sep=10pt},
	width=14cm, height=9cm,
	ylabel=$M / \solarmass$, 
	grid=major,
	%point meta=explicit, point meta min=-1.5, point meta max=+1.5,
	colormap={signnegpos}{samples of colormap=(4 of stability)}, colormap access=const, 
	point meta=explicit, point meta min=0, point meta max=4,
	every axis title/.style={at={(0.97,0.95)}, text width=3cm, anchor=north east, draw=black, fill=white},
]
\nextgroupplot[
	title={\subcaption{\label{fig:nstars:stability_computed}Computed}},
	colorbar horizontal, colorbar sampled, colorbar style={xlabel={Number of unstable modes with $\omega_n^2 < 0$}, xtick={0.5, 1.5, 2.5, 3.5}, xticklabels={$0$, $1$, $2$, $3$}, at={(0.5,1.05)}, anchor=south, xticklabel pos=upper, tickwidth=0},
	xticklabels=\empty,
]
\addplot [mark=none, mesh, thick] table [x expr={10*\thisrow{R}}, y=M, meta=nu] {../code/data/gr2.dat};

\nextgroupplot[
	title={\subcaption{\label{fig:nstars:stability_exact}Exact}},
	xlabel=$R / \si{\kilo\meter}$,
]
\addplot [mark=none, mesh, thick] table [x expr={10*\thisrow{R}}, y=M, meta expr={greater(\thisrow{P},0.8) + greater(\thisrow{P},600) + greater(\thisrow{P},19000)}] {../code/data/gr2.dat};
\end{groupplot}
\end{tikzpicture}

\caption{\label{fig:nstars:stability}%
Stability analysis of the neutron stars with arbitrary Fermi momenta in \cref{fig:nstars:massradius}.
The sign of the squared eigenfreuqency $\omega_0^2$ of the lowest normal vibration mode determines whether a star is stable and oscillates like $e^{i \omega_0 t}$, or if it is unstable and grows or decays like $e^{\pm \abs{\omega_0} t}$.
\subref{fig:nstars:stability_computed} Squared eigenfrequencies computed numerically by solving the Sturm-Liouville eigenvalue equation with the shooting method.
\subref{fig:nstars:stability_exact} Stability determined exactly by the rule that a star transitions from stable to unstable as the central pressure increases beyond the one corresponding to the maximum mass.
}

\end{figure}

\section{Stability analysis}

The mass-radius curves in \cref{fig:nstars:massradius} display some interesting behavior.
In particular, the curves spiral for central pressures greater than that corresponding to the maximum mass.
Since we have used statistical physics to obtain an equation of state, all stars on the mass-radius curve are in \emph{equilibrium}.
\TODO{is the assumption of equilibrium instead/also baked into using perfect fluid energy momentum \eqref{eq:tov:energy_momentum_perfect_fluid}? and $U = (U^0, \vec{0})$?}
However, just like a pendulum can be in stable or unstable equilibria, the equilibrium of a star can be either stable or unstable with respect to small perturbations.
Let us investigate the stability of the sequence of stars on the curve in \cref{fig:nstars:massradius}.

\TODO{figure with analogy between unstable pendulum and unstable star?}

\subsection{Necessary conditions for stability}

We will start simple by presenting a few necessary conditions for stability.
However, neither of the conditions are \emph{sufficient} or a star to be stable, so we can only use them to identify unstable stars.

However exotic life inside a star may be, it cannot break causality. 
In particular, the speed of sound $v = \sqrt{\odv{P}/{\rho}} = c \sqrt{\odv{P}/{\epsilon}}$ should not exceed the speed of light $c$.
The equation of state $\epsilon = \epsilon(P)$ must therefore satisfy
\TODO{ref?}
\begin{equation}
	\odv{P}{\epsilon} < 1 .
\end{equation}
How does the equation of state for a free Fermi gas hold up in this regard?
Recall that the equation of state for general Fermi momenta $p_F$ followed by eliminating $x_F$ from the energy density \eqref{eq:nstars:energy_density_zeroT} to express it in terms of the pressure \eqref{eq:nstars:pressure_zeroT}.
It is straightforward to calculate the derivatives
\begin{equation}
	\odv{P}{x_F} = \frac{m^4 c^5}{3 \pi^2 \hbar^3} \frac{x_F^4}{\sqrt{x_F^2 + 1}}
	\quad \text{and} \quad
	\odv{\epsilon}{x_F} = \frac{m^4 c^5}{\pi^2 \hbar^3} x_F^2 \, \sqrt{x_F^2 + 1} .
\end{equation}
Then we can apply the chain rule and the rule of inverse derivatives to obtain
\begin{equation}
	\odv{P}{\epsilon} = 
	\odv{P}{x_F} \odv{x_F}{\epsilon} =
	\frac{\odv{P}/{x_F}}{\odv{\epsilon}/{x_F}} =
	\frac13 \, \frac{1}{1+1/x_F^2}.
\end{equation}
We see that $\odv{P}/{\epsilon} < 1$ for all $x_F$ and approaches $1/3$ in the ultra-relativistic limit $x_F \rightarrow \infty$, as we should expect from the corresponding equation of state \eqref{eq:nstars:ur_eos}.
Hence, all the stars in \cref{fig:nstars:massradius} satisfy this condition and it does not rule out any stars.

\usetikzlibrary{arrows.meta}
\usetikzlibrary{shapes.symbols}
\begin{figure}
\begin{tikzpicture}
\newcommand\drawstarforcebalance[8]{
	\draw [very thick, inner color=yellow!#6!gray, outer color=red!#6!gray] (#1,#2) circle [radius=#3];
	\draw [thick] [-{Latex[width=#7mm,length=#7mm]}] ({#1-0.1},{#2+0.1}) -- ({#1-0.1+#4*cos(45)},{#2+0.1+#4*sin(45)});
	\draw [thick] [{Latex[width=#8mm,length=#8mm]}-] ({#1+0.1},{#2-0.1}) -- ({#1+0.1+#5*cos(45)},{#2-0.1+#5*sin(45)});
}
\drawstarforcebalance{0.0}{0.0}{1.0}{0.9}{0.9}{0}{4}{4};
\drawstarforcebalance{3.0}{0.0}{1.5}{1.4}{0.7}{33}{5}{3};
\drawstarforcebalance{7.0}{0.0}{2.0}{1.9}{0.5}{66}{6}{2};
\node[starburst, fill=yellow, draw=red, line width=2pt, anchor=west, minimum size=5cm] at (10,0) {\huge\textbf{BOOM}};
\end{tikzpicture}
\caption{\label{fig:nstars:star_explosion}%
A slight decrease $\dif \epsilon < 0$ in energy density weakens the gravitational force pulling a star in.
If the equation of state $\epsilon = \epsilon(P)$ in a star satisfies $\odv{P}/{\epsilon} < 0$, such a change in energy density would cause an increase $\dif P > 0$ in the pressure pushing the star out.
The greater pressure then causes the star to expand, which in turn causes another decrease in energy density.
Repeated application of the same argument shows that the star continues to expand while the pressure grows indefinitely, so the star ultimately explodes.
}
\end{figure}

From the expression $v = c \sqrt{\odv{P}/{\epsilon}}$ of the speed of sound, it also sounds reasonable to require that
\begin{equation}
	\odv{P}{\epsilon} > 0
\label{eq:nstars:stability_pressure_energy_density}
\end{equation}
for it to be a real quantity.
Violation of this condition would in fact have dramatic consequences.
First, note that an increase in energy density $\dif \epsilon > 0$ always increase the gravitational force attempting to pull the star in.
If such an increase implied a \emph{decrease} $\dif P < 0$ in the pressure that pushes the star out, then the gravitational force would automatically ``win'', causing the star to contract, and hence the energy density to increase further.
Repeating the argument, we understand that the star collapses.
This process is illustrated in \cref{fig:nstars:stability_mass_pressure}.
Likewise, if a decrease $\dif \epsilon < 0$ caused an increase $\dif P > 0$, the same argument shows that the pressure wins and the star explodes.
However, if the two always change by the same sign, then the two forces will at the very least counteract each other instead of being driven apart.
In this case the star \emph{can} be stable, but the balance between the forces would have to be investigated in detail to conclude if it \emph{is}.
This criterion does not let us rule out any of our stars, either, but it will be useful to assume \TODO{ref} in the following.

\begin{figure}
\centering
\begin{tikzpicture}
\begin{axis}[axis x line=bottom, axis y line=left, xlabel=$P_0$, ylabel=$M$, xtick=\empty, ytick=\empty]
	\addplot[domain=-1:+1, black] {-x^2};

	\node [draw=black, circle, fill, inner sep=1pt, label=left:$E_1$] at (axis cs:-0.7,-0.49) {};
	\node [draw=black, circle, fill, inner sep=1pt, label=left:$E_2$] at (axis cs:-0.5,-0.25) {};
	\node [draw=black, circle, fill, inner sep=1pt, label=right:$C$ ] at (axis cs:-0.5,-0.49) {};

	\node [draw=black, circle, fill, inner sep=1pt, label=left:$E_2$] at (axis cs:+0.7,-0.49) {};
	\node [draw=black, circle, fill, inner sep=1pt, label=left:$E_1$] at (axis cs:+0.5,-0.25) {};
	\node [draw=black, circle, fill, inner sep=1pt, label=right:$C$ ] at (axis cs:+0.7,-0.25) {};
\end{axis}
\end{tikzpicture}
\caption{\label{fig:nstars:stability_mass_pressure}}
\end{figure}

A third necessary condition is
\begin{equation}
	\odv{M(P_0)}{P_0} > 0
\label{eq:nstars:stability_mass_pressure}
\end{equation}
for stars parametrized by their central pressure $P_0$.

Consider a star $E_1$ in equilibrium on the increasing part of the curve in \cref{fig:nstars:stability_mass_pressure}, where condition \eqref{eq:nstars:stability_mass_pressure} is satisfied.
Now compress this star to the non-equilibrium star $C$ with the same mass, and let $E_2$ be the star in equilibrium with the same central pressure as $C$.
Then $C$ has less mass than $E_2$, and hence weaker gravitational forces than $E_2$, but the same central pressure as $P_2$.
As a result, the star will expand and decrease its central density and pressure, causing it to return towards the equilibrium star $E_1$.

Let us we repeat the argument on the decreasing part of the curve, where condition \eqref{eq:nstars:stability_mass_pressure} does not hold.
After the compression from $E_1$ to $C$, we see that $C$ has more mass and hence stronger gravitational forces than $E_2$, but again the same central pressure.
The star therefore contracts, and repeating the argument causes it to collapse completely.

The criterion \eqref{eq:nstars:stability_mass_pressure} shows that the stars located between the maximum mass and the bottom of the spiral in \cref{fig:nstars:massradius} are unstable!

\subsection{General analysis}

\TODO{where to put this?}
Then he expanded the field equations to first order in the deviations $\delta f$ and simplified them by subtracting the equilibrium equations \eqref{eq:nstars:field_equations_equilibrium}.
Chandrasekhar's original paper \cite{ref:chandrasekhar_stability} is very compact and does not include many details of this calculation.
A more verbose outline can be found in \cite[§ 26.4d]{ref:mtw}, which we will follow here.
\TODO{but we dont do all details}


The arguments presented above were necessary, but not sufficient for stellar stability.
Let us analyze the stability of the stars in a more rigorous way by finding a mathematical definition of stability.

In \cref{sec:einstein_to_tov}, we showed that the Einstein field equations \eqref{eq:einstein} reduced to
\TODO{start with metric}
\begin{subequations}%
\begin{align}%
	\frac{1}{r^2} e^{-2 \beta_0} \left( 2 r \beta_0' - 1 + e^{2 \beta_0} \right)  &= \frac{8 \pi G}{c^4} \epsilon_0   && \left( G_{00} = \frac{8 \pi G}{c^4} T_{00} \right) , \label{eq:nstars:field_equations_equilibrium_00} \\
	\frac{1}{r^2} e^{-2 \beta_0} \left( 2 r \alpha_0' + 1 - e^{2 \beta_0} \right) &= \frac{8 \pi G}{c^4} P_0          && \left( G_{11} = \frac{8 \pi G}{c^4} T_{11} \right) , \label{eq:nstars:field_equations_equilibrium_11} \\
	\alpha_0'                                                                     &= \frac{-1}{\epsilon_0 + P_0} P_0' && \left( \nabla_\mu T\indices{^\mu_1} = 0 \right)    . \label{eq:nstars:field_equations_equilibrium_T}
	%e^{-2 \beta_0} \left( \alpha_0'' + (\alpha_0')^2 - \alpha_0' \beta_0' + \frac{1}{r} (\alpha_0' - \beta_0') \right) &= \frac{8 \pi G}{c^4} P_0,
\end{align}%
\label{eq:nstars:field_equations_equilibrium}%
\end{subequations}%
for a perfect fluid with energy-momentum
\begin{equation}
	T_{\mu \nu} = \frac{1}{c^2} U_\mu U_\nu (\epsilon_0 + P_0) - g_{\mu \nu} P_0
	\quad \text{where} \quad
	P_0 = P_0(r), \epsilon_0 = \epsilon_0(r) ,
\label{eq:nstars:energy_momentum}
\end{equation}
in \emph{equilibrium} with four-velocity
\begin{equation}
	U^\mu = (U^0, 0,0,0) ,
\label{eq:nstars:velocity_equilibrium}
\end{equation}
in the spherically symmetric metric
\begin{equation}
	\dif s^2 = e^{2 \alpha_0(r)} \dif t^2 - e^{2 \beta_0(r)} \dif r^2 - r^2 \left( \dif \theta^2 + \sin^2 \theta \dif \phi^2 \right) .
\end{equation}

Now we suppose that the fluid is \emph{no longer in equilibrium}, \TODO{better word?} but rather has the four-velocity
\begin{equation}
	U^\mu = (U^0, U^1, 0, 0)
	\quad
	\text{with a non-zero radial component $U^1$.}
\label{eq:nstars:velocity_unstable}
\end{equation}
By the spherical symmetry, we still assume there is no angular velocity component.
The metric should still be spherically symmetric, but $\alpha_0(r) \rightarrow \alpha(r,t)$ and $\beta_0(r) \rightarrow \beta(r,t)$ are promoted to time-dependent functions, and therefore so are the pressure $P_0(r) \rightarrow P(r,t)$ and energy density $\epsilon_0(r) \rightarrow \epsilon(r,t)$.
We therefore have the new metric
\begin{equation}
	\dif s^2 = e^{2 \alpha(t, r)} \dif t^2 - e^{2 \beta(t, r)} \dif r^2 - r^2 \left( \dif \theta^2 + \sin^2 \theta \dif \phi^2 \right) .
\label{eq:nstars:metric_unstable}
\end{equation}
To make calculations manageable, we assume that the fluid is only slightly \emph{perturbed} from equilibrium.
Following \cite{ref:chandrasekhar_stability}, our strategy will be to employ perturbation theory by writing the new functions as the small perturbations 
\begin{equation}
\begin{aligned}
	\alpha   (r, t) &= \alpha_0  (r) + \delta \alpha  (r, t), & \qquad \qquad
	\beta    (r, t) &= \beta_0   (r) + \delta \beta   (r, t), \\
	P        (r, t) &= P_0       (r) + \delta P       (r, t), & \qquad \qquad
	\epsilon (r, t) &= \epsilon_0(r) + \delta \epsilon(r, t), \\
\end{aligned}
\label{eq:nstars:perturbation_expansion}
\end{equation}
around equilibrium, and calculating the perturbations to first order.

The fundamental quantity that we want to determine is the displacement $\xi(t, r)$ of fluid elements from the unperturbed to the perturbed system.
Evolution of this quantity should give us insight into how the stellar material responds to perturbations.
Like $\delta \alpha$, $\delta \beta$, $\delta P$ and $\delta \epsilon$, we assume $\xi$ be a small quantity, and we will calculate to first order in all these five quantites.
Let us define $\xi(t, r)$ it so that if we attach a tracker to some fixed fluid element in the star, then
\TODO{I should not mention $\xi$ before it is defined}
\begin{equation}
\begin{split}
	\text{      in the unperturbed star, } & \text{the fluid element is at $\big(ct,r,\theta,\phi\big)$,} \\
	\text{while in the   perturbed star, } & \text{the fluid element is at $\big(ct,r+\xi(r,t),\theta,\phi\big)$.} \\
\end{split}
\end{equation}
What is the relation between the fluid element displacement $\xi(r,t)$ and the fluid's four-velocity?
By our definition, $\pdv{\xi}/{t} = \odv{r}/{t}$, where the latter derivative is the derivative of the fluid element's radial coordinate taken \emph{along its world line} -- or \emph{stream line} -- $x(\tau)$.
The chain rule then gives
\begin{equation}
	\dot\xi = \pdv{\xi}{t} = \odv{r}{t} = \frac{\odv{r}/{\tau}}{\odv{t}/{\tau}} = c \, \frac{U^1}{U^0} .
\end{equation}
We can determine $U^0$ and $U^1$ by combining this equation the normalization condition $U_\mu U^\mu = e^{2 \alpha} \left( U^0 \right)^2 - e^{2 \beta} \left( U^1 \right)^2 = c^2$.
To first order in $\xi$ and $\delta \alpha$, we find the components
\begin{equation}
	U^0%=       c \, \frac{U^1}{\dot\xi}
	    =       c \, \frac{e^{-\alpha}}{\sqrt{1 - \left( \frac{e^\beta \dot\xi}{c e^\alpha} \right)^2 }}
	    \taylor c \, e^{-\alpha_0} \, (1 - \delta \alpha)
	\qquad \text{and} \qquad
	U^1 =        \frac{e^{-\alpha}}{\sqrt{1 - \left( \frac{e^\beta \dot\xi}{c e^\alpha} \right)^2 }} \, \dot\xi
	    \taylor  e^{-\alpha_0} \dot\xi .
\end{equation}

In the perturbed system, the field equations \eqref{eq:nstars:field_equations_equilibrium} also change and must be rederived from the Einstein equations \eqref{eq:einstein} in the new metric \eqref{eq:nstars:metric_unstable} subject to the energy-momentum \eqref{eq:nstars:energy_momentum} with the non-equilibrium velocity \eqref{eq:nstars:velocity_unstable}.
As always, the field equations follow from the machinery of \cref{eq:def_christoffel,eq:def_riemann_tensor,eq:def_ricci_tensor,eq:def_ricci_scalar}.
This time, we calculate the field equations to first order in the small quantities and subtract the equilibrium equations \eqref{eq:nstars:field_equations_equilibrium} to simplify them.
A tedious, but in principle straightforward calculation (although in practice perhaps not), reveals that to first order in the perturbations, two of the new field equations are
\begin{subequations}
\begin{align}
	%\delta \beta   &= - \frac{4 \pi G}{c^4} (\epsilon_0 + P_0) r e^{2 \beta_0} \xi                                                                                                && \left( G_{01} = \frac{8 \pi G}{c^4} T_{01} \right) , \\
	%\delta \alpha' &= \frac{r}{2 e^{-2 \beta_0}} \left[ \frac{8 \pi G}{c^4} \delta P + 2 e^{-2 \beta_0} \left( \frac{2}{r} \alpha_0' + \frac{1}{r^2} \right) \delta \beta \right] && \left( G_{11} = \frac{8 \pi G}{c^4} T_{11} \right) .
	%%\delta \alpha' &= 4 \pi \left\{ -\gamma P_0 \frac{e^{2 \beta_0}}{r} (r^2 e^{-\alpha_0} \xi)' + \left[ P_0' r - (\epsilon_0 + P_0) e^{2 \beta_0} \xi \right] \right\}
	%
	\frac{2}{r} e^{-(\beta_0 + \alpha_0)} \dot{\delta\beta}                                                                     &= - \frac{8 \pi G}{c^4} (\epsilon_0 + P_0) e^{\beta_0 - \alpha_0} \dot\xi                         && \left( G_{01} = \frac{8 \pi G}{c^4} T_{01} \right) , \label{eq:nstars:field_equations_unstable_01} \\
	\frac{2}{r} e^{-2\beta_0} \delta\alpha' - 2 e^{-2 \beta_0} \left( \frac{2}{r} \alpha_0' + \frac{1}{r^2} \right) \delta\beta &= \frac{8 \pi G}{c^4} \delta P                                                                    && \left( G_{11} = \frac{8 \pi G}{c^4} T_{11} \right) . \label{eq:nstars:field_equations_unstable_11}
\end{align}
\TODO{integrate first with time!}
\Cref{eq:nstars:field_equations_unstable_01,eq:nstars:field_equations_unstable_11} are analogous to \cref{eq:nstars:field_equations_equilibrium_00,eq:nstars:field_equations_equilibrium_11}.
What is the third equation corresponding to \cref{eq:nstars:field_equations_equilibrium_T}, which we found from conservation of energy-momentum $\nabla_\mu T^{\mu \nu} = 0$?
In \cref{chap:relfluid}, we study relativistic fluid mechanics and start from $\nabla_\mu T^{\mu \nu} = 0$ to derive the relativistic generalization of the Euler equation,
\begin{equation*}
	  \frac{1}{c^2} \Big( \epsilon + P \Big) U^\mu \nabla_\mu U^\alpha = \nabla^\alpha P - \frac{1}{c^2} U^\alpha U^\mu \nabla_\nu P
	  \qquad \left( \nabla_\mu T^{\mu \nu} = 0 \right) .
\end{equation*}
By inserting the new four-velocity \eqref{eq:nstars:velocity_unstable} and the expansions \eqref{eq:nstars:perturbation_expansion} into the relativistic Euler equation and performing calculations in the new metric \eqref{eq:nstars:metric_unstable} to first order, one finds
\TODO{ref program?}
\begin{equation}
	\left( \epsilon_0 + P_0 \right) e^{2 \beta_0 - 2 \alpha_0} \ddot \xi = -c^2 \left[ \delta P' - \left( \delta \epsilon + \delta P \right) \alpha_0' - \left( \epsilon_0 + P_0 \right) \delta \alpha' \right] .
\label{eq:nstars:field_equations_unstable_T}
\end{equation}
\label{eq:nstars:field_equations_unstable}%
\end{subequations}
The system \eqref{eq:nstars:field_equations_unstable} consists of \emph{three} equations for the \emph{five} unknowns $\xi$, $\delta P$, $\delta \epsilon$, $\delta \alpha$ and $\delta \beta$.
We will complete the set by finding two additional equations from our study of relativistic fluid mechanics in \cref{chap:relfluid}.

To make use of the results of \cref{chap:relfluid} in our context, we must first learn to distinguish between \emph{Eularian} and \emph{Lagrangian} changes in fluids.
The quantities
\begin{subequations}
\begin{equation}
	\delta f(r, t) = f(r, t) - f_0(r)
\label{eq:nstars:change_eularian}
\end{equation}
that we defined in \cref{eq:nstars:perturbation_expansion} are \textbf{Eularian changes} in the quantities $f$, measured by an observer who is sitting duck at some fixed position $x^\mu = (ct, r, \theta, \phi)$.
In contrast, we define the \textbf{Lagrangian changes}
\begin{equation}
	\Delta f(r,t) = f(r+\xi(r,t), t) - f_0(r)
\label{eq:nstars:change_lagrangian}
\end{equation}
\end{subequations}
as the changes in the same quantities, but measured by an observer who is moving \emph{with} the fluid element as it flows from $r$ to $r + \xi(r,t)$.
One can always be converted to the other through the Taylor expansion
\begin{equation}
\begin{split}
	\Delta f(r,t) &\taylor f(r,t) + \xi(r,t) \, f'(r,t) - f_0(r) \\
	              &= \delta f(r,t) + \xi(r,t) \, f'(r,t) .
\end{split}
\label{eq:nstars:delta_taylor_expansion}
\end{equation}
To summarize, $\delta$ measures changes at a fixed position of different fluid elements, while $\Delta$ measures changes of fixed fluid elements at different positions.
The first stems from perturbation theory, and the second is natural to use in fluid mechanics.

In our study of relativistic fluid mechanics in \cref{chap:relfluid}, we require that the number of baryons in a fluid element must be conserved in flow.
By eliminating the baryon number density with \cref{eq:relfluid:baryon_number_rate_change}, the equation of energy conservation \eqref{eq:relfluid:energy_conservation_rewritten} is equivalent to
\TODO{say that we eliminate baryon number density}
\begin{equation}
	\odv{\epsilon}{\tau} = -(\epsilon+P) \nabla_\mu U^\mu 
\label{eq:nstars:energy_conservation}
\end{equation}
First, note that because the equilibrium energy density $\epsilon_0(r)$ is independent of time, $\odv{\epsilon}/{\tau} = \odv{\Delta \epsilon}/{\tau}$.
At first sight, it may sound more straightforward to write $\odv{\delta \epsilon}/{\tau}$ instead of $\odv{\Delta \epsilon}/{\tau}$.
However, since the derivative $\odv{}/{\tau}$ is along the stream line of the fluid element, the Lagrangian changes are more accurate than the Eularian changes to first order in perturbation theory.
To first order in the small quantities, only the time component contributes when applying the chain rule to the left side
\begin{equation}
	\odv{\epsilon}{\tau} =
	\odv{\Delta \epsilon}{\tau} =
	U^\mu \nabla_\mu \Delta \epsilon \taylor
	U^0 \pdv{\Delta \epsilon}{t} = U^0 \dot{\Delta \epsilon} =
	e^{-\alpha_0} \dot{\Delta \epsilon} .
\end{equation}
On the right side, we can calculate the quantity $\nabla_\mu U^\mu$ with the identity
\begin{equation}
\begin{split}
	\nabla_\mu U^\mu &= \frac{1}{\sqrt{-\det{g}}} \partial_\mu \left( \sqrt{-g} \, U^\mu \right) \\
	                 &= e^{-\alpha_0} \left[ \dot{\delta\beta} + \frac{e^{-\beta_0}}{r^2} \left( e^{\beta_0} r^2 \dot\xi \right)' \right] .
\end{split}
\end{equation}
To first order in the small quantities, \cref{eq:nstars:energy_conservation} then says
\begin{equation}
	\dot{\Delta \epsilon} = - \left( \epsilon_0 + P_0 \right) \left[ \dot{\delta\beta} + \frac{e^{-\beta_0}}{r^2} \left( e^{\beta_0} r^2 \dot\xi \right)' \right] .
\end{equation}
Again, integrate with respect to time to get rid of the dots and set the integration constant to zero, so $\Delta \epsilon = 0$ when $\delta \beta = 0$ and $\xi = 0$.
We then find our fourth main equation
\begin{equation}
	\Delta \epsilon = - \left( \epsilon_0 + P_0 \right) \left[ \delta\beta + \frac{e^{-\beta_0}}{r^2} \left( e^{\beta_0} r^2 \xi \right)' \right] .
\label{eq:nstars:Delta_epsilon}
\end{equation}

The last result of \cref{chap:relfluid} that we make use of is adiabadicity of the flow.
In \cref{sec:relfluid:adiabadicity}, we show that the flow of a perfect fluid is adiabatic with the \textbf{adiabatic index}
\begin{equation}
	\gamma = \frac{\epsilon+P}{P} \odv{P}{\epsilon} .
\label{eq:nstars:adiabatic_index0}
\end{equation}
The above expression can be calculated entirely from the fluid's equation of state $\epsilon = \epsilon(P)$, although in its current form, it is the perturbed quantities that enter, so for now it must be treated as an unknown quantity.
However, by interpreting the derivative as the ratio of Lagrangian changes, the adiabatic index can be rewritten in the visually similar, but fundamentally distinct form
\begin{equation}
	\gamma = 
	\frac{\epsilon+P}{P} \frac{\Delta P}{\Delta \epsilon} ,
	\quad \text{or} \quad
	\Delta P = \frac{P}{\epsilon + P} \, \gamma \, \Delta \epsilon .
\label{eq:nstars:adiabatic_index}
\end{equation}
\TODO{what about $P$ vs $P_0$ here?}
The latter is a relation between two first order quantities, and so for the purpose of perturbation theory, \emph{the only relevant part of $\gamma$ is the zeroth order part}
\begin{equation}
	\gamma_0 = \frac{\epsilon_0+P_0}{P_0} \frac{\Delta P_0}{\Delta \epsilon_0} .
\end{equation}
This gives us our fifth and final equation
\begin{equation}
	\Delta P = \frac{P_0}{\epsilon_0 + P_0} \, \gamma_0 \, \Delta \epsilon .
\label{eq:nstars:Delta_P}
\end{equation}

We have made quite a big mess by now, but we finally have all the information we need, and it remains only to clean up after ourselves.
Together, \cref{eq:nstars:field_equations_unstable_01,eq:nstars:field_equations_unstable_11,eq:nstars:field_equations_unstable_T,eq:nstars:Delta_epsilon,eq:nstars:Delta_P} constitute the system of five equations
\begin{subequations}
\begin{align}
	\frac{2}{r} e^{-(\beta_0 + \alpha_0)} \delta\beta                                                                           &= - \frac{8 \pi G}{c^4} (\epsilon_0 + P_0) e^{\beta_0 - \alpha_0} \xi \\
	\frac{2}{r} e^{-2\beta_0} \delta\alpha' - 2 e^{-2 \beta_0} \left( \frac{2}{r} \alpha_0' + \frac{1}{r^2} \right) \delta\beta &= \frac{8 \pi G}{c^4} \delta P                                  \\
	\left( \epsilon_0 + P_0 \right) e^{2 \beta_0 - 2 \alpha_0} \ddot \xi                                                        &= -c^2 \left[ \delta P' - \left( \delta \epsilon + \delta P \right) \alpha_0' - \left( \epsilon_0 + P_0 \right) \delta \alpha' \right] \\
	\Delta \epsilon                                                                                                             &= - \left( \epsilon_0 + P_0 \right) \left[ \delta\beta + \frac{e^{-\beta_0}}{r^2} \left( e^{\beta_0} r^2 \xi \right)' \right]  \\
	\Delta P &= \frac{P_0}{\epsilon_0 + P_0} \, \gamma_0 \, \Delta \epsilon
\end{align}%
\label{eq:nstars:perturbation_system}%
\end{subequations}%
for the five unknowns $\xi$, $\delta \alpha$, $\delta \beta$, $\delta P$ and $\delta \epsilon$.
Remember that any occurence of $\Delta P$ and $\Delta \epsilon$ can be traded for $\delta P$ and $\delta \epsilon$ by the Taylor expansion \eqref{eq:nstars:delta_taylor_expansion}.
Except for the independent variables $r$ and $t$, the only other quantities are the equilibrium values $\alpha_0$, $\beta_0$, $P_0$ and $\epsilon_0$ and derivatives thereof.
Solving the Tolman-Oppenheimer-Volkoff equations \eqref{eq:tov:tovsys} yield $P_0$, $m_0$ and $\alpha_0$, from which one can calculate $\epsilon_0$ by the equation of state \eqref{eq:tov:tovsys_eos} and $\beta_0$ from definition \eqref{eq:einstein_to_tov:def_m}.

As we remarked earlier, it is $\xi$ we want to calculate, so let us reduce the system \eqref{eq:nstars:perturbation_system} to a differential equation involving $\xi(t,r)$ as the only dependent variable.
Along the way, it is convenient to define
\begin{equation}
	\zeta(t,r) = r^2 e^{\beta_0(r)} \xi(t,r),
\end{equation}
because the only spatial derivative of $\xi$ appears in the combination $\left( e^{\beta_0} r^2 \xi \right)$.
Never forget that we are doing perturbation theory, so any product of two small quantities can be neglected, and we can simplify expressions using any equilibrium equation from above.
Again, the work before us is easy in principle and hard in practice.
In the end, one finds that $\zeta$ obeys the differential equation
\begin{equation}
	W(r) \ddot{\zeta}(r,t) = (\Pi(r) \zeta'(r))' + Q(r) \zeta(r) ,
\label{eq:nstars:diffeq_zeta}
\end{equation}
where the coefficient functions are
\TODO{use big/small $u$/$U$ in ST-eq/4-vel}
\TODO{check units}
\begin{subequations}
\begin{align}
	\Pi &= \frac{1}{r^2} e^{\beta_0 + 3 \alpha_0} \gamma P_0 , \\
	Q   &= -\frac{4}{r^3} e^{\beta_0 + 3 \alpha_0} P_0' - \frac{8 \pi}{r^2} e^{3 \beta_0 + 3 \alpha_0} P (\epsilon_0 + P_0) + \frac{e^{\beta_0 + 3 \alpha_0}}{r^2(\epsilon_0 + P_0)} (P_0')^2 , \\
	W   &= \frac{1}{r^2} e^{3 \beta_0 + \alpha_0} (\epsilon_0 + P_0) .
\end{align}
\label{eq:nstars:sturm_liouville_coefficients}
\end{subequations}

\TODO{incorporate how to find $\alpha_0$ into TOV chapter and/or code appendix}
In addition, $\alpha$ can be obtained by integrating its derivative \eqref{eq:einstein_to_tov:dadr1} or \eqref{eq:einstein_to_tov:dadr2}, too.
Since it is only the derivative of $\alpha$ that enters the TOV equation, we can start its integration subject to a convenient boundary condition such as $\alpha(0) = 0$.
Then, at the end, we shift $\alpha(r) \rightarrow \alpha(r) - \alpha(R) + \frac12 \log (1 - 2 G M / R)$ to match it to the Schwarzschild solution.

We can find a general solution of this equation using separation of variables.
Let us write
\begin{equation}
	\zeta(t,r) = T(t) \, U(r) .
\end{equation}
Substitution into the differential equation \eqref{eq:nstars:diffeq_zeta} shows that
\begin{equation}
	\frac{\ddot{T}}{T} = \frac{\left( \Pi \, U' \right)' + Q \, U}{W} = -\omega^2
\end{equation}
must be a constant $-\omega^2$, so we find that $T(t) = e^{i \omega t}$ and that $U(r)$ and $\omega^2$ must solve
\begin{equation}
	\odv*{ \left[ \Pi(r) \odv{U(r)}{r} \right] }{r} + \left[ Q(r) + \omega^2 W(r) \right] U(r) = 0 .
\label{eq:nstars:sturm_liouville}
\end{equation}

What are the boundary equations for $U(r)$?
At the center $r = 0$, spherical symmetry implies that there can be no flow, so the physical boundary condition there is
\begin{equation}
	\xi(t, 0) = 0.
\label{eq:nstars:boundary_condition_center_physical}
\end{equation}
If there was flow at the center, it would necessarily be uniform in all directions, so an inward flow would drain all of the star's material into a sink at the center, while an outward flow would expel the material and create an ``eye of the storm'' in the middle.
The translation of this condition for the displacement \eqref{eq:nstars:displacement_general_solution} to vanish is
\TODO{why not $r^4$, $r^5$, etc? do we need $\xi' > 0$ at $r=0$ or something?}
\begin{equation}
	U_n(r) \propto r^3 .
\label{eq:nstars:boundary_condition_center_mathematical}
\end{equation}
At the surface $r = R$, there can be no change in pressure as one follows a fluid element, just like at the interface between water and air down on earth.
The boundary condition here is therefore
\begin{equation}
	\Delta P =
	\frac{P_0}{P_0 + \epsilon_0} \, \gamma_0 \Delta \epsilon =
	\odv{P}{\epsilon} \Delta \epsilon =
	0 .
\end{equation}
The surface is defined by $P = 0$, so the derivative $\odv{\epsilon}/{P}$ there is
\TODO{ref to equation of state, equations and/or plot}
\TODO{show in more detail}
\begin{equation}
	\odv{P}{\epsilon} = \frac{1}{\odv{\epsilon}{P}} = \frac{1}{\epsilon'(P)} = \frac{1}{\epsilon'(0)} = \frac{1}{\infty} = 0,
\label{eq:nstars:boundary_condition_surface_physical}
\end{equation}
and the boundary condition \eqref{eq:nstars:boundary_condition_center_physical} is automatically satisfied \emph{provided that}
\begin{equation}
	\Delta \epsilon = - \left( \epsilon_0 + P_0 \right) \left[ \delta\beta + \frac{e^{-\beta_0(R)}}{R^2} U_n'(R) e^{i \omega t} \right] < \infty \quad \text{is finite}.
\end{equation}
By assumption, $\delta\beta$ is a small and thus finite quantity, and so is $e^{-\beta_0(R)} = \left( 1 - 2 G M / R c^2 \right)^{-1/2}$.
The translation of the boundary condition \eqref{eq:nstars:boundary_condition_surface_physical} at the surface into a criterion on $U_n(r)$ is therefore only that
\begin{equation}
	U_n'(R) < \infty \quad \text{is finite} .
\label{eq:nstars:boundary_condition_surface_mathematical}
\end{equation}
The differential equation \eqref{eq:nstars:sturm_liouville} is a \textbf{Sturm-Liouville problem} \TODO{ref some mathematics text} for the eigenfunction $U(r)$ and its corresponding squared eigenfrequency $\omega^2$.
In fact, a characteristic feature of Sturm-Liouville problems is that there are multiple solutions $U(r) = U_n(r)$ with corresponding squared frequencies $\omega^2 = \omega_n^2$ forming an infinite, discrete sequence
\begin{equation}
	\omega_0^2 < \omega_1^2 < \omega_2^2 < \cdots .
\end{equation}
A more familiar example, perhaps, is the time independent Schrödinger equation, for which the energies $E_n$ associated with the wave functions $\psi_n(x)$ form such an increasing, discrete sequence.
Noting that our our first order analysis has yielded the \emph{linear} differential equation \eqref{eq:nstars:diffeq_zeta} for $\zeta$, and thus $\xi$, we can superpose all solutions of the Sturm-Liouville problem \eqref{eq:nstars:sturm_liouville} into a general solution
\TODO{$c_n U_n(r)$, some normalization?}
\begin{equation}
	\xi(t,r) = \frac{e^{-\beta_0(r)}}{r^2} \sum_n U_n(r) e^{i \omega_n t} .
\label{eq:nstars:displacement_general_solution}
\end{equation}
This superposition will still satisfy the physical boundary conditions \eqref{eq:nstars:boundary_condition_center_physical} and \eqref{eq:nstars:boundary_condition_surface_physical}, provided that each $U_n(r)$ separately satisfies the mathematical boundary conditions \eqref{eq:nstars:boundary_condition_center_mathematical} and \eqref{eq:nstars:boundary_condition_surface_mathematical}.

At long last, we are in an excellent position to make a \textbf{rigorous definition of stellar stability}.
Suppose a star is exposed to some external perturbation $\xi(t,r)$.
Whether it is you touching the star with your finger, some imperfection in the stellar material or the blast from a nearby supernova explosion, it is the initial functional form of the perturbation $\xi(t,r)$ that determines its decomposition into the sum over all modes \eqref{eq:nstars:displacement_general_solution}.
In the real world, we expect that any such perturbation will activate \emph{all} modes $U_n(r)$ to some degree -- for one would need remarkable finger precision to activate only a finite selection of modes.
\begin{itemize}
\item If $\omega_0^2 > 0$, then all vibration modes have real frequencies $\omega_n > 0$.
      Then all terms merely oscillate back and forth like $\xi \propto e^{i \omega_n t}$, so the star attempts to return to equilibrium, and we say the star is \textbf{stable}.
\item If $\omega_0^2 < 0$, then at least one vibration mode has an imaginary frequency $\omega_n = \pm i \abs{\omega_n}$.
      Then some term in $\xi \propto e^{\pm \abs{\omega_n} t}$ takes off exponentially, so the star either implodes or explodes, and we say the star is \textbf{unstable}.
\end{itemize}

\TODO{innvending 1: vi gjør perturbasjonsteori, så gjelder vel ikke eksponensiell tidsvekst for evig?}


\TODO{instability, connect to attractor/chaos stuff from nonlinear dynamics?}

\TODO{break down BCs into physical, then ``translated'' parts}
\TODO{require $\xi' = 0$ at center, otherwise there would be a singularity?}
\TODO{$\delta r$ instead of $\xi$?}


There is a particularly fun numerical technique called the \textbf{shooting method} that can be used to determine the squared frequencies $\omega_N^2$ and their corresponding solutions $U_N(r)$.
It relies on the additional property of Sturm-Liouville problems that
\TODO{cite mathematics text here and above}
\begin{equation}
	\text{the $N$-th eigenmode $U_N(r)$ has exactly $N$ internal zeros for $0 < r < R$} .
\end{equation}
From their definitions \eqref{eq:nstars:sturm_liouville_coefficients} and \eqref{eq:nstars:adiabatic_index} and criterion \eqref{eq:nstars:stability_pressure_energy_density}, $\gamma$, and hence $\Pi$ and $W$, are positive for all $r$.
By rewriting the Sturm-Liouville equation \eqref{eq:nstars:sturm_liouville} in the form
\begin{equation}
\frac{\left[ \Pi(r) U'(r) \right]'}{U(r)} = - \left[ Q(r) + \omega_n^2 W(r) \right] ,
\end{equation}
it is evident that increasing $\omega^2$ increases the oscillatory behavior of $U(r)$.
Crucially, then, the higher the eigenmode index $N$, the greater the value of $\omega^2$ must necessarily be.

\TODO{also property on blowup}

This understanding enables us to explain the shooting method.
The strategy is to
\begin{enumerate}
\item Guess any value of $\omega^2$.
\item \emph{Impose} the boundary condition $U(r) \alpha r^3$ at $r = 0$, then use the Sturm-Liouville equation \eqref{eq:nstars:sturm_liouville} to ``shoot out'' the corresponding solution $U(r)$.
\item Count the number of nodes $n$ of $U(r)$.
      \begin{enumerate}
      \item If the number of nodes $n > N$ is too high, then we should decrease our guess for $\omega^2$ to weaken the oscillatory behavior of $U(r)$ hence decrease the number of nodes.
      \item If the number of nodes $n < N$ is too low, then we should increase our guess for $\omega^2$ to strengthen the oscillatory behavior of $U(r)$ and hence increase the number of nodes.
      \TODO{why does (b) hold ofr $n=N$? is it because of Sturm's oscillation theorem? refer to this, or explain?}
      \item If the number of nodes $n = N$ is ``just right'', then we should still increase our guess for $\omega^2$.
            Suppose we were looking for the first eigenvalue $\omega_0^2$ with zero nodes.
            If the true eigenvalue had been below our guess $\omega^2$, then any lower guess for $\omega^2$ would also have zero nodes, and the argument becomes a loop, and we eventually reach the conclusion that the eigenvalue is $-\infty$, which it cannot be.
      \end{enumerate}
% catalogue paper, see paragraph across page 507-508
\end{enumerate}
The rule that we should increase our guess not only if $n < N$, but also if $n = N$, is a little technical.
Suppose we are looking for the zeroth mode $N = 0$ and we have guessed $\omega^2$ with $n = 0$ zeros.
If the true eigenvalue had been less than $\omega^2$, then any lower guess of $\omega^2$ would also give $n = 0$ zeros, for it is impossible to have less than zero zeros.
But then we could repeat the same argument to argue that it the true value is even lower than the new guess, too, ultimately reaching the conclusion that the $\omega_0^2 = -\infty$, which is neither physically or mathematically sound.
By contradiction, then, the eigenvalue must be greater than the initial guess, which shows that we should increase our guess if $n = N$ with $N = 0$.

For $N > 0$, the reason follows by induction.
If $N = 1$ and we guess $\omega^2$ with $n = 1$ zeros, then decreasing our guess would eventually yield the eigenvalue $\omega_0^2$ -- not $\omega_1^2$.
Thus, we still have to increase our guess, and this follows for all $N$ by induction.

We implement the shooting method in \TODO{ref app}.
Initially, we find two lower and upper bounds $\omega_-^2$ and $\omega_+^2$ with $n \leq N$ and $n > N$ zeros, then inspect $\omega^2 = (\omega_-^2 + \omega_+)^2) / 2$ and replace either the lower or upper bound depending on its number of zeros.
Continuing to split the interval $(\omega_-^2, \omega_+^2)$ in this manner, the bounds come closer and closer and we terminate the procedure when they are so close that we are satisfied with any value in the interval.
Perhaps \cref{fig:nstars:shooting_convergence} sheds new light on this quite technical discussion -- there we show how the squared frequencies $\omega^2$ and the corresponding solutions $U(r)$ converge towards the exact values $\omega_0^2$ and $U_0(r)$ of the fundamental vibration mode $N=0$.

With the shooting method in our toolbox, we can compute $\omega_0^2$ for every star along the curve in \cref{fig:nstars:massradius} and determine whether it is stable or not by checking its sign.
We indicate the stability determined in this way in \cref{fig:nstars:stability} \TODO{a}.
Starting from the lowest central pressure and mass, it seems the stars are stable up to the maximum mass.
However, the numerical procedure also indicates that stars slightly beyond this point are stable, which we know \emph{cannot} be the case by criterion \eqref{eq:nstars:stability_mass_pressure}.

This error is very likely to be related to numerical inaccuracy, as there are a number of aspects with our implementation of the shooting method in \TODO{ref app} that are quite crude.
\begin{itemize}
\item Since the TOV equation is solved numerically, $\alpha_0(r)$, $\beta_0(r)$, $P_0(r)$ and $\epsilon_0(r)$ are only determined at discrete points between $0$ and $R$.
      Thus, the same is true for $\Pi(r)$, $Q(r)$, and $W(r)$.
      In addition, we approximate the derivatives derivatives $\odv{P_0}{\epsilon_0}$ and $\odv{P_0}{r}$ that are needed in $\Pi(r)$ and $Q(r)$ with finite differences.
\item We integrate the Sturm-Liouville equation numerically using finite differences.
\item The boundary condition $U(r) \propto r^3$ is implemented numerically by setting $U(r) = r^3$ for all $r < 0.01 R$.
\item Since $r=R$ is a singular point, the numerical integration freaks out as $r \rightarrow R$.
      To fix this, we linearly interpolate the next value of $U(r)$ from its value at the previous point beyond $r > 0.99 R$.
\end{itemize}
We expect that these sources of error accumulate, explaining the violation of condition \eqref{eq:nstars:stability_mass_pressure}.

\TODO{check if there are more negative modes inside the spiral! it seems to be the case!!}

However, trying to look behind the numerical inaccuracy, our computational results do suggest a simple rule for stability.
At every point along the mass-radius curve where $\odv{M(P_0)}/{P_0} = 0$, stars seem to acquire one more unstable vibration mode, starting with zero unstable modes for $P_0 \rightarrow 0$.
In fact, \TODO{ref} showed that if one plots the mass-radius curve for a sequence of \emph{cold} stars in a diagram with $R$ along the $x$-axis and $M$ along the $y$-axis, then the following holds:
\begin{itemize}
\item At every local mass extremum $\odv{M(P_0)}/{P_0} = 0$ where the curve bends \emph{counterclockwise}, one \emph{stable} mode becomes \emph{unstable}.
\item At every local mass extremum $\odv{M(P_0)}/{P_0} = 0$ where the curve bends \emph{clockwise}, one \emph{unstable} mode becomes \emph{stable}.
\end{itemize}
The curve in \cref{fig:nstars:massradius} always bends clockwise after the maximum mass. 
Ultimately, the most precise statement about the stability of the sequence of cold neutron stars we have computed in this chapter is therefore that all stars before the maximum mass are stable, while all stars beyond are unstable.
As with the Buchdal bound \eqref{eq:incompressible_star:buchdal} that we looked at a long time ago, the maximum mass is not only a limit on \emph{massiveness}, but also on \emph{stability}.

\usetikzlibrary{calc}
\begin{figure}
\begin{tikzpicture}
\pgfplotstablegetrowsof{\nmodestable}
\pgfmathparse{int(\pgfmathresult-1)}
\pgfplotstablegetelem{\pgfmathresult}{r}\of{\nmodestable}
\pgfmathsetmacro{\maxR}{\pgfplotsretval}
\begin{groupplot}[
	group style={group size=2 by 1, horizontal sep=5pt},
	width=8cm, height=7cm,
	xlabel=$r / R$,
	xmin=-0.1, xmax=1.1, ymin=-1.1, ymax=+1.1, restrict y to domain=-100:+100,
	xtick={0,0.5,1}, ytick={-1,0,1},
	ymajorgrids=true,
	no markers, cycle list name=color list, % or exotic
]
\nextgroupplot[
	ylabel=$U_n / \norm{U_n}_\infty$,
	legend cell align=left, legend columns=5, legend style={at={(1.0, 1.03)}, anchor=south},
]
\pgfplotstableread{../code/data/nmodes_norm.dat}{\nmodestable};
\pgfplotsinvokeforeach{0,1,2,3,4} {
	\addplot+ [thick] table [x expr=\thisrow{r}/\maxR, y=U#1] {\nmodestable};
	\pgfplotstablegetelem{#1}{omega2}\of{\nmodestable}
	\addlegendentryexpanded{$\omega_{#1}^2 = \pgfmathprintnumber[fixed, fixed zerofill, precision=2]{\pgfplotsretval}$};
}

\nextgroupplot[
	ylabel=$U_n / \norm{U_0}_\infty$, yticklabel pos=right, ylabel near ticks,
	yticklabels={},
];
\tablemaximum{../code/data/nmodes.dat}{U0}{\maxU}{r}{\maxr}
\pgfplotsinvokeforeach{0,1,2,3,4} {
	\addplot+ [thick] table [x expr=\thisrow{r}/\maxR, y expr=\thisrow{U#1}/\maxU] {../code/data/nmodes.dat};
}
\addplot [dashed, domain=0:0.5, restrict y to domain=0:1.1, samples=50] {(x*\maxR)^3 / \maxU} node[pos=0.6, label={[xshift=+1ex]135:$\propto r^3$}] {};
\end{groupplot}
\end{tikzpicture}
\end{figure}

\begin{figure}
\begin{tikzpicture}
\pgfplotstableread{../code/data/shoot.dat}{\shoottable}
\pgfplotstablegetrowsof{\shoottable}
\pgfmathparse{int(\pgfplotsretval-1)}
\pgfplotstablegetelem{\pgfmathresult}{r}\of{\shoottable}
\pgfmathsetmacro{\maxR}{\pgfplotsretval}
\begin{axis}[
	width=11cm, height=8cm,
	xlabel=$r/R$, ylabel=$U(r) / \norm{U_0(r)}_\infty$,
	title={Convergence of $U(r; \omega^2) \rightarrow U_0(r; \omega_0^2)$ \\ with the shooting method},
	%every axis title/.style={at={(0.5,1)}, align=center, above=1ex, text width=0.8*\pgfkeysvalueof{/pgfplots/width}]},
	ymin=-2.1, ymax=+2.1,
	xtick distance=0.25, ytick distance=1,
	ymajorgrids, xmajorgrids,
	cycle list name=color list,
	legend style={anchor=south west, at={(1.02, 0)}},
	legend cell align=right,
	colormap name=blackred, cycle list={[samples of colormap=33]},
]
\addlegendimage{empty legend}
\addlegendentry{\hspace{-0.6cm}\textbf{Value of $\omega^2$}} % hack for legend title: https://tex.stackexchange.com/questions/2311/add-a-legend-title-to-the-legend-box-with-pgfplots
\pgfplotsinvokeforeach{0,...,32}{ % 32, it works but it is very slow
	\pgfplotstablegetelem{#1}{omega2}\of{../code/data/shoot.dat}
	\addplot+ [thick, restrict y to domain=-5:5] table [x expr=\thisrow{r}/\maxR, y expr={\thisrow{U#1} / 0.000312}] {../code/data/shoot.dat}; % faster to read directly from file for some reason
	\addlegendentryexpanded{$\pgfmathprintnumber[precision=10, fixed, fixed zerofill]{\pgfplotsretval}$};
}
\end{axis}
\end{tikzpicture}
\caption{\label{fig:nstars:shooting_convergence}%
	The shooting method is used to determine the squared frequency $\omega_N^2$ of the vibration mode $U_N(r)$ with $N=0$ of a neutron star with central pressure $P_0 = 10^3 \epsilon_0$.
	First, it finds two values $\omega_-^2 = -16$ and $\omega_+^2 = 0$ whose corresponding solutions $U(r)$ have $n_- \leq N$ and $n_+ > N$ nodes, respectively.
	The exact squared frequency is then guaranteed to lie in $\omega_-^2 < \omega^2 < \omega_+^2$.
	Next, the algorithm finds the solution $U(r)$ corresponding to $\omega^2 = (\omega_-^2 + \omega_+^2) / 2 = -8$ and counts its number of nodes.
	Here, it is found to have $n > N$ nodes, so $\omega_+^2 = -8$ gives a tighter upper bound on $\omega^2$ than $0$.
	In the opposite case $n \leq N$, it would have been a better lower bound $\omega_-^2 = -8$.
	Moving on, the algorithm continues to split intervals in this fashion until the lower and upper bounds are so close that any value in the interval $[\omega_-^2, \omega_+^2]$ is satisfactory.
}
\end{figure}

\subsection{Sufficient conditions for stability}

\TODO{compare with earlier results, most massive neutron stars}

\TODO{keywords: $dp/d\epsilon = 1/3$ = conformal limit? no mass in this EOS = scale invariance. $T^{\mu \nu}$ traceless?}
