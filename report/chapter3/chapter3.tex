\chapter{Chapter 3}

\section{Something}

Logarithm of partition function
\begin{equation}
	\log Z = 2 V \int \frac{\dif^3 p}{(2 \pi \hbar)^3} \left\{ \beta E(\vec{p}) + \log \left[ e^{-\beta (E(\vec{p}) - \mu)}+1 \right] + \log \left[ e^{-\beta (E(\vec{p}) + \mu)} + 1\right] \right\} .
\end{equation}

Average number
\begin{equation}
	\thermalavg{N} = 
	\frac{1}{\beta} \pdv{\log Z}{\mu} =
	2 V \int \frac{\dif^3 p}{(2 \pi \hbar)^3} \Big\{ n\big[ E(\vec{p})-\mu \big] - n\big[ E(\vec{p})+\mu \big] \Big\} .
\end{equation}
$\mu$ determines $\thermalavg{N}$

Energy
\begin{equation}
	\thermalavg{E} = 
	\mu \thermalavg{N} - \pdv{\log Z}{\beta} =
	2 V \int \frac{\dif^3 p}{(2 \pi \hbar)^3} \Big\{ -E(\vec{p}) + E(\vec{p}) n\big[ E(\vec{p})-\mu \big] - (-E(\vec{p})) n\big[ E(\vec{p})+\mu \big] \Big\}
\end{equation}

Pressure
\begin{equation}
	\thermalavg{P} = 
	\frac{\log Z}{\beta V} = 
	2 \int \frac{\dif^3 p}{(2 \pi \hbar)^3} \Big\{ -E(\vec{p}) - \log \big[ n(E(\vec{p})-\mu) \, n(E(\vec{p})+\mu) \big] \Big\}
\end{equation}

Assume $\mu > 0$.
Forget diverging vacuum contribution.
Then
\begin{equation}
\begin{split}
	\epsilon &=  2 \int \frac{\dif^3 p}{(2 \pi \hbar)^3} \, n(E(\vec{p})-\mu) E(\vec{p}) \\
	P        &= -2 \int \frac{\dif^3 p}{(2 \pi \hbar)^3} \, \log n(E(\vec{p})-\mu) \\
\end{split}
\end{equation}
Zero-temperature limit

Number density (use $n$ without argument for $N/V$)
\begin{equation}
	n = \frac{\thermalavg{N}}{V} = 2 \int_0^{p_F} \frac{\dif p \, 4 \pi p^2}{(2 \pi \hbar)^3} = \frac{p_F^3}{3 \pi^2 \hbar^3}
\end{equation}

Energy-density
\begin{equation}
\begin{split}
	\epsilon &=  2 \int \frac{\dif^3 p}{(2 \pi \hbar)^3} \, \theta(E(\vec{p})-\mu) E(\vec{p}) \\
	         &=  2 \int_0^{p_F} \frac{\dif p \, 4 \pi p^2}{(2 \pi \hbar)^3} \, \sqrt{p^2 c^2 + m^2 c^4} \\
	         &= \frac{m^4 c^5}{\pi^2 \hbar^3} \int_0^{x_F} \dif x \, x^2 \sqrt{1 + x^2} \\
	         &= \frac{1}{8} \frac{m^4 c^5}{\pi^2 \hbar^3} \left[ \left( 2 x_F^3 + x_F \right) \sqrt{1 + x_F^2} - \asinh x_F \right] \\
\end{split}
\end{equation}

Pressure, use $\log (1 + e^{-\beta E}) \taylor -\beta E \theta(-E)$,
\begin{equation}
\begin{split}
	P &= \frac{2}{\beta} \int \frac{\dif^3 p}{(2 \pi \hbar)^3} \, \log \left[ e^{-\beta (E - \mu)} + 1 \right] \\
	  &= \frac{2}{\beta} \int \frac{\dif^3 p}{(2 \pi \hbar)^3} \, \beta (\mu - E) \theta(\mu - E) \\
	  &= \frac{2}{\beta} \int \frac{\dif p \, 4 \pi p^2}{(2 \pi \hbar)^3} \, \beta (\mu - E) \theta(\mu - E) \\
	  &= \frac{m^4 c^5}{\pi^2 \hbar^3} \int_0^{x_F} \dif x \, x^2 \left( \sqrt{x_F^2+1} - \sqrt{x^2+1} \right) \\
	  &= \frac{1}{24} \frac{m^4 c^5}{\pi^2 \hbar^3} \left[ \left( 2 x_F^3 - 3 x_F \right) \sqrt{x_F^2 + 1} + 3 \asinh x_F \right] \\
\end{split}
\end{equation}

\subsection{Ultra-relativistic limit}

Ultra-relativistic limit, $x_F \gg 1$.
\begin{equation}
\begin{split}
	P        &\taylor \frac{1}{24} \frac{m^4 c^5}{\pi^2 \hbar^3} \left[ 2 x^{4} - 2 x^{2} + 3 \log 2 + 3 \log x - \frac{7}{4} \right] \taylor \frac{1}{12} \frac{m^4 c^5 x_F^4}{\pi^2 \hbar^3} \\
	\epsilon &\taylor \frac{1}{8}  \frac{m^4 c^5}{\pi^2 \hbar^3} \left[ 2 x^{4} + 2 x^{2} - \log 2 - \log x + \frac{1}{4} \right] \taylor \frac{1}{4} \frac{m^4 c^5 x_F^4}{\pi^2 \hbar^3} \\
\end{split}
\end{equation}
Equation of state
\begin{equation}
	\epsilon = 3 P
\end{equation}

Can solve with trial solution
\begin{equation}
	P(r) = A r^n
\end{equation}
\begin{equation}
	\odv{m(r)}{r} = \frac{12 \pi A}{c^2} r^{n+2},
	\quad \text{so} \quad
	m(r) = \frac{12 \pi A}{(n+3) c^2} r^{n+3}
	\quad (n \neq -3)
\end{equation}
Insert into TOV
\begin{equation}
	n A r^{n-1} =
	-\frac{48 \pi G A^2 r^{2n+1}}{(n+3) c^4} \left[ 2 + \frac{n}{3} \right] \left[ 1 - \frac{24 \pi G A r^{n+2}}{(n+3) c^4} \right]^{-1}
\end{equation}
Must choose $n = -2$.
\begin{equation}
	\frac{2 A}{r^3} \left[ -1 + \frac{32 \pi G A}{c^4} \left( 1 - \frac{24 \pi G A}{c^4} \right)^{-1} \right] = 0
\end{equation}
or
\begin{equation}
	\frac{2 A}{r^3} \left[ -1 + \frac{56 \pi G A}{c^4} \right] = 0
\end{equation}
Then we must choose
\begin{equation}
	A = \frac{c^4}{56 \pi G}
\end{equation}
so
\begin{equation}
	P(r) = \frac{c^4}{56 \pi G} \frac{1}{r^2}
\end{equation}
and
\begin{equation}
	m(r) = \frac{3 c^2 r}{14 G}
\end{equation}

\subsection{Non-relativistic limit}

Non-relativistic limit, $x_F \ll 1$.
\begin{equation}
\begin{split}
	%\epsilon &\taylor \frac{m c^2 p_F^3}{3 \pi^2 \hbar^3} + \frac{p_F^5}{10 \pi^2 \hbar^3 m} = n m c^2 + \frac{p_F^5}{10 \pi^2 \hbar^3 m} \\
	%P        &\taylor \frac{p_F^5}{15 \pi^2 \hbar^3 m}
	\epsilon &\taylor \frac{m c^2 p_F^3}{3 \pi^2 \hbar^3} \\
	P        &\taylor \frac{p_F^5}{15 \pi^2 \hbar^3 m}
\end{split}
\end{equation}
Eliminate $p_F$, equation of state
\begin{equation}
	\epsilon = \frac{mc^2}{3\pi^2\hbar^3} \left( 15 \pi^2 \hbar^3 m P \right)^{3/5}
\end{equation}

Cannot solve TOV exactly.
Solve it numerically.

Dimensionless quantities
\begin{equation}
	\diml{\epsilon} = \frac{\epsilon}{\epsilon_0}, \quad
	\diml{P} = \frac{P}{\epsilon_0}, \quad
	\diml{r} = \frac{r}{r_0}, \quad
	\diml{m} = \frac{m}{m_0}.
\end{equation}
where
\begin{equation}
	\epsilon_0 = \frac{m_0 c^2}{4 \pi r_0^3 / 3}, \quad
	m_0 = \text{solar mass ?}, \quad
	r_0 = \text{10 km ?}.
\end{equation}
Dimensionless TOV
\begin{equation}
	\frac{\epsilon_0}{r_0} \odv{\diml{p}}{\diml{r}} = -\frac{G m_0 \epsilon_0}{r_0^2 c^2} \frac{\diml{m} \diml{\epsilon}}{\diml{r}^2} \left[ 1 + \frac{\diml{p}}{\diml{\epsilon}} \right] \left[ 1 + \frac{4 \pi r_0^3 \epsilon_0}{m_0 c^2} \frac{\diml{r}^3 \diml{p}}{\diml{m}} \right] \left[ 1 - \frac{2 G m_0}{r_0 c^2} \frac{\diml{m}}{\diml{r}} \right]^{-1}.
\end{equation}
Inserting $\epsilon_0 = \ldots$ and defining the dimensionless gravitational constant $\hat{G} = G / (r_0 c^2 / m_0)$, we get
\begin{equation}
	\odv{\diml{p}}{\diml{r}} = - \frac{\diml{G} \diml{m} \diml{\epsilon}}{\diml{r}^2} \left[ 1 + \frac{\diml{p}}{\diml{\epsilon}} \right] \left[ 1 + \frac{3 \diml{r}^3 \diml{p}}{\diml{m}} \right] \left[ 1 - \frac{2 \diml{G} \diml{m}}{\diml{r}} \right]^{-1}.
\end{equation}
Dimensionless mass equation
\begin{equation}
	\odv{\diml{m}}{\diml{r}} = 3 \diml{r}^2 \diml{\epsilon}
\end{equation}
Dimensionless equation of state
\begin{equation}
	\diml{\epsilon} = \left[ \frac{4^2 5^3}{3^4 \pi^2} \frac{m^8 c^6 r_0^6}{m_0^2 \hbar^6} \diml{P}^3 \right]^{\frac{1}{5}}
\end{equation}
Numerical equation to solve
\begin{equation}
	\odv{}{r} \begin{bmatrix} m(r) \\ P(r) \\ \end{bmatrix} = f \left( \begin{bmatrix} m(r) \\ P(r) \\ \end{bmatrix} \right)
	\quad \text{with boundary conditions} \quad
	\begin{bmatrix} m(0) \\ P(0) \\ \end{bmatrix} = \begin{bmatrix} 0 \\ P_0 \\ \end{bmatrix}.
\end{equation}
Each run is parametrized by some central pressure $P_0$.
At each integration step, the equation of state $\diml\epsilon = \diml\epsilon(\diml{P})$ is used.

\section{General}

How to eliminate $x_F$ in the general case?

\begin{equation}
	\diml{P}(x_F) = \frac{m^4 c^3 r_0^3}{18 \pi m_0 \hbar^3} \left[ (2 x_F^3 - 3 x_F) \sqrt{x_F^2 + 1} + 3 \asinh x_F \right]
\end{equation}

At every integration step, we have a value of the pressure $P$.
Then find the root $x_F$ of
\begin{equation}
	P(x_F) - P = 0
\end{equation}
and then calculate
\begin{equation}
	\diml{ϵ} = \diml{ϵ}(x_F) = \diml{P}(x_F) = \frac{m^4 c^3 r_0^3}{6 \pi m_0 \hbar^3} \left[ (2 x_F^3 + x_F) \sqrt{x_F^2 + 1} - \asinh x_F \right]
\end{equation}
