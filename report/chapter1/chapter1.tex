\chapter{Tolman-Oppenheimer-Volkoff equation}

TODO: write intro after I know everything we should do in this section

TODO: add page numbers to citations

\section{Derivation from the Einstein field equations}
\label{sec:tov}

To analyze astrophysical objects like stars, it is of considerable interest to relate the pressure $p(x)$ and energy density $\epsilon(x)$ at every position $x$ inside the object.
We will derive the relativistic relation between these quantities from the \textbf{Einstein field equations} \cite[equation 4.44]{ref:carroll}
\begin{equation}
	G\indices{_\mu_\nu} = R_{\mu \nu} - \frac{1}{2} R g_{\mu \nu} = \frac{8 \pi G}{c^4} T_{\mu \nu} ,
	\label{eq:einstein}
\end{equation}
It describes how the geometry of spacetime, described by the Ricci tensor $R\indices{_\mu_\nu}$ and Ricci scalar $R$ that are ultimately built from the metric $g\indices{_\mu_\nu}$ and encapsulated in the Einstein tensor $G\indices{_\mu_\nu}$ (see \cref{chap:gr_summary} for a summary), responds to the presence of energy-momentum in the energy-momentum tensor $T\indices{_\mu_\nu}$.
Here, $G$ is the gravitational constant and $c$ is the speed of light.

Unless rotating very fast, stars are well approximated by spheres.
For our purposes, we therefore consider the most general line element that exhibits spherical symmetry, namely \cite[§ 94-95]{ref:tolman}
(TODO: do more general with $\gamma(r)$, as done in Carroll?)
\begin{equation}
	% coordinates x = (ct, r, θ, ϕ)
	\dif s^2 = -e^{2 \alpha(r)} c^2 \dif t^2 + e^{2 \beta(r)} \dif r^2 + r^2 \left( \dif \theta^2 + \sin^2 \theta \dif \phi^2 \right) .
\end{equation}

We model the interior of the star as a perfect fluid with energy-momentum \cite[equation 1.114]{ref:carroll}
\begin{equation}
	T\indices{_\mu_\nu} = \frac{1}{c^2} (\epsilon+p) U_\mu U_\nu + p g\indices{_\mu_\nu}.
\end{equation}
For a static star whose fluid is at rest, $U_\mu = (U_0, \textbf{0})$ and the normalization condition $U_\mu U^\mu = -c^2$ requires $U_0 = \pm e^\alpha c$.
We choose the positive sign so the four-velocity lies in the future light cone, as we are interested in the evolution of the star.
Then the energy-momentum tensor takes the diagonal form
\begin{equation}
T\indices{_\mu_\nu} =
\begin{bmatrix}
	\epsilon e^{2\alpha} & 0            & 0     & 0                   \\
	0                    & p e^{2\beta} & 0     & 0                   \\
	0                    & 0            & p r^2 & 0                   \\
	0                    & 0            & 0     & p r^2 \sin^2 \theta \\
\end{bmatrix}
\qquad \text{or} \qquad
T\indices{_\mu^\nu} =
\begin{bmatrix}
	-\epsilon & 0 & 0 & 0 \\
	0         & p & 0 & 0 \\
	0         & 0 & p & 0 \\
	0         & 0 & 0 & p \\
\end{bmatrix}
.
\label{eq:einstein_to_tov:T}
\end{equation}

Starting with the metric, it is now straightforward, although tedious, to compute the left side of \cref{eq:einstein} from \cref{eq:def_christoffel,eq:def_riemann_tensor,eq:def_ricci_tensor,eq:def_ricci_scalar}.
For the details, refer to \cite[equation 5.11-5.15]{ref:carroll}.
After inserting the energy-momentum tensor on the right and simplifying, we get the three independent equations
(the fourth turns out proportional to the third)
\begin{subequations}
\begin{align}
	\frac{1}{r^2} e^{-2 \beta} \left( 2 r \beta' - 1 + e^{2 \beta} \right)                               &= \frac{8 \pi G}{c^4} \epsilon
	&& \left( G\indices{_t_t} = \frac{8 \pi G}{c^4} T\indices{_t_t} \right)                     , \label{eq:einstein_to_tov:tt} \\
	\frac{1}{r^2} e^{-2 \beta} \left( 2 r \alpha' + 1 - e^{2 \beta} \right)                              &= \frac{8 \pi G}{c^4} p
	&& \left( G\indices{_r_r} = \frac{8 \pi G}{c^4} T\indices{_r_r} \right)                     , \label{eq:einstein_to_tov:rr} \\
	e^{-2 \beta} \left( \alpha'' + (\alpha')^2 - \alpha' \beta' + \frac{1}{r} (\alpha' - \beta') \right) &= \frac{8 \pi G}{c^4} p
	&& \left( G\indices{_\theta_\theta} = \frac{8 \pi G}{c^4} T\indices{_\theta_\theta} \right) . \label{eq:einstein_to_tov:thetatheta}
\end{align}
\end{subequations}

Next, let us introduce the mass of the star.
Define the function $m(r)$ by
\begin{equation}
	e^{2 \beta} = \left( 1 - \frac{2 G m(r)}{r c^2} \right)^{-1} ,
	\label{eq:einstein_to_tov:def_m}
\end{equation}
so $g\indices{_r_r}$ resembles the Schwarzschild metric element.
Then \cref{eq:einstein_to_tov:tt} becomes
\begin{equation}
	\diff{(m c^2)}{r} = 4 \pi r^2 \epsilon(r) ,
	\label{eq:einstein_to_tov:m_rho}
\end{equation}
directly relating $m(r)$ and $\epsilon(r)$.
If we set $m(0) = 0$, we can integrate to get
\begin{equation}
	m(r) c^2 = \integral{\epsilon(r') 4 \pi r'^2}{r'}{0}{r} .
	\label{eq:einstein_to_tov:m_integral}
\end{equation}
\cite[page 602]{ref:mtw} shows that setting $m(0) \neq 0$ creates a singularity at the origin, which is not physically acceptable.
Outside a star that extends to $r = R$, there is vacuum with $\epsilon = 0$ and our metric should match the Schwarzschild metric with $g\indices{_r_r} = (1-2GM/rc^2)^{-1}$ and Schwarzschild mass $M$.
By comparison with \cref{eq:einstein_to_tov:def_m}, the Schwarzschild mass of the star must be given by
\begin{equation}
	M c^2 = m(R) c^2 = \integral{\epsilon(r) 4 \pi r^2}{r}{0}{R} .
	\label{eq:einstein_to_tov:schwarzschild_mass}
\end{equation}
It is tempting to interpret \cref{eq:einstein_to_tov:schwarzschild_mass} as the Newtonian mass of the star and \eqref{eq:einstein_to_tov:m_integral} as the volume integral of the energy density $\epsilon(r)$.
But $4 \pi r^2$ is not a proper volume element, as it does not involve the full spatial metric determinant.
We return to this question in \cref{sec:weak_field_limit} after studying incompressible stars in \cref{sec:incompressible_star}, where we will see that the first interpretation is correct, while the latter is more subtle and differs by the binding energy of the star.

Meanwhile, definition \eqref{eq:einstein_to_tov:def_m} turns \cref{eq:einstein_to_tov:rr} into
\begin{equation}
	\diff{\alpha}{r} = \frac{G}{r^2 c^4} \frac{m(r) c^2 + 4 \pi r^3 p}{1 - 2 G m(r) / r c^2} .
	\label{eq:einstein_to_tov:dadr1}
\end{equation}
To finally eliminate $\alpha$, we can replace all occurences of $\alpha'$ and $\beta$ in the remaining \cref{eq:einstein_to_tov:thetatheta} with the expressions \eqref{eq:einstein_to_tov:dadr1} and \eqref{eq:einstein_to_tov:def_m}.
Doing so is straightforward, but cumbersome and most easily done by a computer algebra system.
We show how to do this in \cref{sec:tov_cas_derivation}.
An elegant, but less straightforward argument is to use local energy-momentum conservation $\nabla_\mu T\indices{^\mu^\nu} = 0$, which is both physically reasonable and in fact possible to prove directly from the Einstein field equations \eqref{eq:einstein}.
For two different proofs, see \cite{ref:einstein_conservation_energy_momentum} and \cite[section 8.3.2]{ref:mika_gr_notes}.
Using \cref{eq:def_cov_deriv}, the $\nu=r$-component gives
\begin{equation*}
	0
	= \nabla_\mu T\indices{^\mu_r}
	= \partial_r T\indices{^r_r} + \Gamma^\sigma_{r \sigma} T\indices{^r_r} - \Gamma^\sigma_{r \mu} T\indices{^\mu_\sigma}
	= \partial_r T\indices{^r_r} + \Gamma^0_{r0} T\indices{^r_r} + \sum_{i=1}^3 \Gamma^i_{ri} T\indices{^r_r} - \Gamma^0_{r0} T\indices{^0_0} - \sum_{i=1}^3 \Gamma^i_{ri} T\indices{^i_i}
\end{equation*}
(TODO: replace $r,t$ indices with $0,1$ etc (after I have restored $c \neq 1$ units, it's impossible to use $ct$ as index, etc.))
Using $T\indices{^0_0} = -\epsilon$ and $T\indices{^1_1} = T\indices{^2_2} = T\indices{^3_3} = p$ from \cref{eq:einstein_to_tov:T}, the sums cancel, leaving
\begin{equation}
	\diff{\alpha}{r} = \frac{-1}{\epsilon+p} \diff{p}{r} .
	\label{eq:einstein_to_tov:dadr2}
\end{equation}
Now $\alpha$ is easily eliminated by equating \eqref{eq:einstein_to_tov:dadr1} and \eqref{eq:einstein_to_tov:dadr2}. 
Whichever approach we follow, we end up with the \textbf{Tolman-Oppenheimer-Volkow (TOV) equation}
\begin{equation}
	\diff{p}{r} = -\frac{G m(r) \epsilon(r)}{r^2 c^2} \left( 1 + \frac{p(r)}{\epsilon(r)} \right) \left( 1 + \frac{4 \pi r^3 p(r)}{m(r) c^2} \right) \left( 1 - \frac{2 G m(r)}{r c^2} \right)^{-1} .
	\label{eq:tov}
\end{equation}
It relates the pressure gradient $\diff{p}{r}$ and energy density $\epsilon$ at radius $r$ from the center of a spherical static star composed of a perfect fluid.
\Cref{eq:einstein_to_tov:m_rho,eq:tov} constitute two equations for the three unknowns $p$, $\epsilon$ and $m$.
To determine them, an additional equation of state like
\begin{equation}
	p = p(\epsilon)
	\quad \text{or} \quad
	\epsilon = \epsilon(p)
\end{equation}
is required, typically obtained from the domain of thermodynamics and statistical physics.
Given all three equations and the central pressure $p(0)$, we can integrate to find the pressure everywhere inside the star.
We define the radius of the star to be the radius $R$ at which $p(R) = 0$.
Carrying out this procedure for different values of $p(0)$, we can find a mass-radius relation $M(R)$ for stars parametrized by their central pressure $p(0)$.

The TOV equation was originally derived by \cite{ref:tov} using multiple results from \cite{ref:tolman}.

\section{Solution for an incompressible star}
\label{sec:incompressible_star}

% TODO: an incompressible star is unphysical

Although it may sound unphysical from the outset, we can make a somewhat realistic model of a star by assuming that the fluid is incompressible, meaning the energy density
\begin{equation}
	\epsilon(r) = \epsilon_0
\end{equation}
is constant inside the star.
For example, this results in a completely unrealistic speed of sound $v = c \sqrt{\difft{p}{\epsilon}} = c \sqrt{1/(\difft{\epsilon}{p})} = c \sqrt{1/0} = \infty$. \cite{ref:speed_of_sound}
Anyway, integrating \cref{eq:einstein_to_tov:m_integral,eq:einstein_to_tov:schwarzschild_mass} yield the masses
\begin{equation}
	m(r) c^2 = \frac{4}{3} \pi r^3 \epsilon_0 
	\quad \text{and} \quad
	M c^2 = m(R) c^2 = \frac{4}{3} \pi R^3 \epsilon_0 
	.
\end{equation}
Inserting the energy density $\epsilon(r)$ and mass $m(r)$ into \cref{eq:tov}, $p$ and $r$ separate to
\begin{equation*}
	\int \frac{\dif p}{(\epsilon_0+p)(\epsilon_0+3p)} = - \int \frac{4 \pi G r \dif r}{3 c^4 - 8\pi G r^2 \epsilon_0} .
\end{equation*}
The left side can now be split by the partial fraction decomposition
\begin{equation*}
	\frac{1}{(\epsilon_0+p)(\epsilon_0+3p)} = \frac{1}{2p} \left( \frac{1}{\epsilon_0+p} + \frac{1}{\epsilon_0+3p} \right) .
\end{equation*}
Performing all three integrals using the general antiderivatives
\begin{equation*}
	\int \frac{\dif x}{a x^2 + b x} = -\frac{1}{b} \log \left( a + \frac{b}{x} \right) + C
	\quad \text{and} \quad
	\int \frac{x \dif x}{a x^2 + b} = \frac{1}{2 a} \log \left( a x^2 + b \right) + C
\end{equation*}
and applying the boundary condition $p(R) = 0$ to determine the integration constant, we eventually find the radial pressure
\begin{equation}
	% p(r) in terms of M, R, 
	p(r) = \epsilon_0 \, \frac{\sqrt{1-\frac{2GMr^2}{R^3c^2}} - \sqrt{1-\frac{2GM}{Rc^2}}}{3 \sqrt{1-\frac{2GM}{Rc^2}} - \sqrt{1-\frac{2GMr^2}{R^3c^2}}} .
	% p(r) in terms of p0, M, R
	%p(r) = -\frac{M}{\frac{4}{3} \pi R^3} \frac{4 \pi R^{3} p_0 \left( \frac{1}{3} + \sqrt{1-\frac{2GMr^2}{R^3}} \right) + M \left( 1 + \sqrt{1-\frac{2GMr^2}{R^3}} \right)}
	%                                           {4 \pi R^{3} p_0 \left( 1           + \sqrt{1-\frac{2GMr^2}{R^3}} \right) + M \left( 3 + \sqrt{1-\frac{2GMr^2}{R^3}} \right)}
	\label{eq:incompressible_star:pressure}
\end{equation}
In particular, the central pressure is
\begin{equation}
	p(0) = \epsilon_0 \frac{1 - \sqrt{1 - \frac{2GM}{Rc^2}}}{3 \sqrt{1-\frac{2GM}{Rc^2}} - 1} .
	\label{eq:incompressible_star:central_pressure}
\end{equation}
It is interesting to note that the pressure is positive for $GM/Rc^2 < 4/9$, but explodes at $GM/Rc^2 = 4/9$ and becomes negative for $GM/Rc^2 > 4/9$.
Physically, this means that once a star with fixed radius becomes massive enough, it implodes and turns into a black hole.
This is an example of a more general result -- \textbf{Buchdal's theorem} states that all static spherical stars composed of perfect fluids must have
\begin{equation}
	M(R) < M_\text{max}(R) = \frac{4c^2}{9G} R
	\label{eq:incompressible_star:buchdal}
\end{equation}
for any energy density distribution $\epsilon(r)$ that does not decrease outwards. \cite{ref:buchdal}
The proof requires careful work, but we can still understand the result intuitively.
An object with a mass-energy distribution that somehow satures the limit set by nature itself should have the same density everywhere.
If it does not, there will be a place with lower density than its surroundings, which in turn will generate a pressure gradient force that seeks to establish equilibrium by eliminating the difference in density.
Thus, the bound we have found in our computation with constant energy density should be the most extreme bound.

In \cref{sec:weak_field_limit} we will see that Buchdal's theorem is a relativistic result and that no such bound arises from Newtonian gravity.

\begin{figure}[hb!]
\centering
\begin{tikzpicture}
\begin{groupplot}[group style={group size=2 by 1, horizontal sep=0.1\textwidth}, width=0.49\textwidth]
	\nextgroupplot[xlabel=$R$, ylabel=$M(R)$, xtick distance=1, legend pos=north west, legend cell align=left, declare function={
		G = 1;
		M(\R,\E) = 4/3*pi*(\R)^3*\E;
		Mmax(\R) = 4*\R/(9*G);
	}]
	\pgfplotsinvokeforeach{0.009, 0.006, 0.003} {
		\addplot [domain=0:5, solid ] {M(x,#1)} node[pos=0.90, sloped, yshift=+6pt] {\scriptsize $\epsilon_0 = #1$};
	}
	\addplot [domain=0:5, dashed] {Mmax(x)} node[pos=0.2, pin=above:{$M_\text{max}(R)$}] {};
	%\legend{$M(R) = 4 \pi R^3 \epsilon_0 / 3$, $M_\text{max}(R) = 4R/9G$};
	\nextgroupplot[xlabel=$r/R$, ylabel=$p(r)/p(0)$, xtick distance=0.25, declare function={
		G=1;
		R=2;
		p(\r,\M) = \M / (4/3*pi*R^3) * (sqrt(1-2*G*\M*(\r)^2/R^3) - sqrt(1-2*G*\M/R)) / (3*sqrt(1-2*G*\M/R) - sqrt(1-2*G*\M*(\r)^2/R^3));
		Mmax = 4*R/(9*G);
	}]
	\pgfplotsinvokeforeach{0.9, 0.99, 0.999} {
		\addplot [domain=0:1,samples=200] {p(x*R,#1*Mmax)/p(0,#1*Mmax)} node[pos=0.54, sloped, yshift=+16pt] {\scriptsize $M = #1 M_\text{max}$};
	}
\end{groupplot}
\end{tikzpicture}
\caption{Mass-radius relation $M(R)$ for stars of constant densities $\epsilon_0$ compared to the maximum supported mass $M_\text{max}(R)$. Pressure distribution $p(r)$ for particular stars with radius $R=2$ and different masses $M$ approaching $M_\text{max}(R)$.}
\end{figure}

\section{Weak-field limit and physical interpretation of mass}
\label{sec:weak_field_limit}

%TODO: move everything about interpretation of mass, energy density integral etc. to one section which has one thing in common: we look at the weak-field limit
%TODO: do weak-field limit of TOV equation, too

As we mentioned in \cref{sec:tov}, it is not apparent how to interpret the ``mass'' \eqref{eq:einstein_to_tov:m_integral}.
Here we will compare the results we have found so far to those of Newtonian gravity in the limit where particles move \emph{slowly} and the gravitational field is \emph{static} and \emph{weak}.

In Newtonian gravity, a particle is accelerated by the gravitational field
\begin{equation}
	\mathbf{g} = - \nabla V ,
	\label{eq:interpretation_m:newton2}
\end{equation}
where the gravitational potential $V$ is the solution to the Poisson equation
\begin{equation}
	\nabla^2 V = 4 \pi G \epsilon/c^2 .
	\label{eq:interpretation_m:poisson}
\end{equation}

In general relativity, freely falling particles move along geodesics $x(\tau)$ that satisfy the \textbf{geodesic equation}
\begin{equation}
	\diff[2]{x^\mu}{\tau} + \Gamma^\mu_{\rho \sigma} \diff{x^\rho}{\tau} \diff{x^\sigma}{\tau} = 0 .
	\label{eq:geodesic}
\end{equation}
A \emph{slowly} moving particle has velocity $\diff{x^\mu}{\tau}$ with spatial component $\diff{x^i}{t} \ll c$ and is thus dominated by the spatial component $\diff{t}{\tau} \approx 1$.
Then $\diff{x^i}{\tau} \ll c \diff{t}{\tau}$ and the geodesic equation can be approximated by
\begin{equation*}
	\diff[2]{x^\mu}{\tau} + \Gamma^\mu_{tt} \, c^2 \left( \diff{t}{\tau} \right)^2 = 0 .
\end{equation*}
A \emph{static} field has $\partial_t g\indices{_\mu_\nu} = 0$, so the Christoffel symbols \eqref{eq:def_christoffel} simplify to
\begin{equation*}
	\Gamma^\mu_{tt} = \frac{1}{2} g\indices{^\mu^\lambda} (\partial_t g\indices{_\lambda_t} + \partial_t g\indices{_t_\lambda} - \partial_\lambda g\indices{_t_t}) = -\frac{1}{2} g\indices{^\mu^\lambda} \partial_\lambda g\indices{_t_t} .
\end{equation*}
A \emph{weak} gravitational field can be written as a perturbation 
\begin{equation}
	g\indices{_\mu_\nu} = \eta\indices{_\mu_\nu} + h\indices{_\mu_\nu}
	\quad \text{with} \quad
	\abs{h\indices{^\mu^\nu}} \ll 1
	\label{eq:weak_field_limit:metric_perturbation}
\end{equation}
on top of flat Minkowski space $\eta\indices{^\mu^\nu} = \text{diag}(-1, +1, +1, +1)$.
The inverse metric $g\indices{^\mu^\nu}$ should satisfy $g\indices{^\mu^\nu} g\indices{_\nu_\sigma} = \delta^\mu_\sigma$, so to first order in $h$ we must have
\begin{equation*}
	g\indices{^\mu^\nu} = \eta\indices{^\mu^\nu} - h\indices{^\mu^\nu} ,
\end{equation*}
where $h\indices{^\mu^\nu} = \eta\indices{^\mu^\rho} \eta\indices{^\nu^\sigma} h\indices{_\rho_\sigma}$ is raised with the Minkowski metric.
Calculating the relevant Christoffel symbols to first order in $h$, we find $\Gamma^\mu_{tt} = -\frac{1}{2} \eta\indices{^\mu^\lambda} \partial_\lambda h\indices{_t_t}$, so the geodesic equation becomes
\begin{equation*}
	\diff[2]{x^\mu}{\tau} = \frac{1}{2} c^2 \eta\indices{^\mu^\lambda} \partial_\lambda h\indices{_t_t} \left( \diff{t}{\tau} \right)^2
	\quad \text{with spatial components} \quad
	\diff[2]{x^i}{t} = \frac{1}{2} c^2 \partial_i h\indices{_t_t} .
\end{equation*}
This is precisely \cref{eq:interpretation_m:newton2} if we identify $h\indices{_t_t} = -2V/c^2$, or $g\indices{_t_t} = -(1+2V/c^2)$, so a weak relativistic gravitational field $g\indices{_\mu_\nu} = \eta\indices{_\mu_\nu} + h\indices{_\mu_\nu}$ in fact describes Newtonian motion in the gravitational potential $V = -c^2 h\indices{_t_t} / 2$!
Moreover, the Schwarzschild metric has the element
\begin{equation}
	g\indices{_t_t} = - \left( 1 - \frac{2 G M}{r c^2} \right)
	\quad \text{with} \quad
	h\indices{_t_t} = 2GM/rc^2 ,
	\label{eq:weak_field_limit:schwarzschild_metric_tt}
\end{equation}
and $V = -c^2 h\indices{_t_t} / 2 = -G M / r$ is precisely the solution to Poisson's equation \eqref{eq:interpretation_m:poisson} for a spherical mass distribution, so general relativity does indeed describe Newtonian gravity in the weak-field limit!
In conclusion, this shows that it does make sense to interpret the Schwarzschild mass $M$ as the Newtonian mass of a star, as masses of distant stars are typically measured with Kepler's third law that follows from Newtonian gravity. \cite[box 23.1]{ref:mtw}

From the looks of \cref{eq:einstein_to_tov:schwarzschild_mass}, it is very tempting to also interpret $M c^2$ as the volume integral of the energy density over the star.
But $4 \pi r^2 \dif r$ is not a proper volume element.
In a proper spatial integral, the volume element should be $\sqrt{\abs{\gamma}} \dif^3 x = e^\beta r^2 \sin \theta \dif r \dif \theta \dif \phi$, where $\gamma\indices{_i_j} = g\indices{_i_j}$ is the spatial part of the metric and $\abs{\gamma}$ its determinant.
So the true volume integral of the energy density is really
\begin{equation*}
	\bar{M} c^2 = \integral{\epsilon(r) e^{\beta(r)} 4 \pi r^2}{r}{0}{R}.
\end{equation*}
The difference
\begin{equation*}
	\bar{M} c^2 - M c^2 = \integral{\epsilon(r) \left( \left( 1-\frac{2Gm}{r} \right)^{-1/2} - 1 \right) 4 \pi r^2}{r}{0}{R} > 0
\end{equation*}
is in fact the binding energy that arises due to the gravitational attraction between the individual fluid elements in the star.
To see this, consider again the weak-field limit.
Comparing \cref{eq:weak_field_limit:metric_perturbation,eq:weak_field_limit:schwarzschild_metric_tt}, we see that
\begin{equation}
	\frac{G m(r)}{rc^2} \ll 1
	\label{eq:weak_field_limit:small_gmr}
\end{equation}
Using the Taylor expansion $(1 - x)^{-1/2} = 1 + x/2$, we get
\begin{equation}
	\bar{M} c^2 - M c^2 \approx \integral{\epsilon(r) \frac{GM}{r} 4 \pi r^2}{r}{0}{R} .
\end{equation}
As shown in \cite[exercise 23.7]{ref:mtw} this is precisely the energy required to construct the star by sequentially placing thin shells with mass $\dif m = (\epsilon(r)/c^2) 4 \pi r^2 \dif r$ on top of each other, each subject to the gravitational attraction of the shells already placed below it.
This explains that $\bar{M} c^2 - M c^2$ is indeed the binding energy that would be required to disperse all the matter in the star to infinity.
% confusion about M, \bar{M}, resources:
% https://physics.stackexchange.com/q/196280/299916
% see MTW p. 602-, box at p. 603, p. 453
% see Schwarz p. 126

(TODO: in the Newtonian case, the density is mass density and not energy density -- clarify this!)

Finally, let us compare the TOV equation \eqref{eq:tov} and findings for relativistic incompressible stars from \cref{sec:incompressible_star} to those of Newtonian gravity.
To do so, we should first establish the Newtonian pressure gradient analogous to the relativistic one.

Consider the element $\dif m = (\epsilon(r)/c^2) \dif A \dif r$ at distance $r$ from the center of a Newtonian star.
By Gauss' law it is attracted to the mass $m(r)$ inside the radius $r$ as if it were concentrated at the center, but experiences no attraction whatsoever from the remaining mass outside $r$ due to symmetry.
By Newton's law of gravity it is therefore pulled upon by the force
\begin{equation}
	\dif \mathbf{F}_1 = -\frac{G m(r) \dif m}{r^2} \hat{\textbf{r}} .
	\label{eq:weak_field_limit:force_newton}
\end{equation}
If the star is in hydrostatic equilibrium, this force must be exactly cancelled by the force
\begin{equation}
	\dif \textbf{F}_2 = -(p(r + \dif r) - p(r)) \dif A \, \hat{\textbf{r}} = -\dif p \dif A \, \hat{\textbf{r}}
	\label{eq:weak_field_limit:force_pressure}
\end{equation}
that arises from the pressure difference above and below the element.
Setting $\dif \textbf{F}_1 + \dif \textbf{F}_2 = 0$ then gives the \textbf{Newtonian pressure gradient}
\begin{equation}
	\diff{p}{r} = -\frac{G m(r) \epsilon(r)}{r^2c^2} .
	\label{eq:weak_field_limit:newtonian_pressure_gradient}
\end{equation}

Solving this differential equation for a star of constant mass density $\epsilon(r)/c^2 = \epsilon_0/c^2$ like we solved the TOV equation \eqref{eq:tov} in \cref{sec:incompressible_star}, we get the pressures
\begin{equation}
	p(r) = \frac{\epsilon_0}{2} \frac{G M}{R c^2} \left( 1 + \frac{r}{R} \right) \left( 1 - \frac{r}{R} \right)
	\quad \text{and} \quad
	p(0) = \frac{\epsilon_0}{2} \frac{GM}{R c^2} .
	\label{eq:weak_field_limit:newtonian_pressure}
\end{equation}
In this case, the pressure is well-behaved for all $r$.
This shows that Buchdal's theorem \eqref{eq:incompressible_star:buchdal} is a purely \emph{relativistic} result, and that no such limitation arises in Newtonian gravity!

% TODO: makes us suspect that Newton is taylor expansion of TOV
% TODO: check limit

Furthermore, we suspect that the relativistic pressure gradient \eqref{eq:tov} reduces to the Newtonian pressure gradient \eqref{eq:weak_field_limit:newtonian_pressure_gradient} in the Newtonian limit.
For example, the Newtonian pressures \eqref{eq:weak_field_limit:newtonian_pressure} are precisely the Taylor expansions of the relativistic pressures \eqref{eq:incompressible_star:pressure} and \eqref{eq:incompressible_star:central_pressure} for small $GM/Rc^2 \ll 1$.
Comparing \cref{eq:tov} to \cref{eq:weak_field_limit:newtonian_pressure_gradient}, we see that the relativistic equation indeed reduces to the Newtonian one if all the corrections to $1$ in the three parentheses vanish.
Let us see that this is actually the case.

First, note that by \cref{eq:weak_field_limit:small_gmr}, we have the small quantities
\begin{equation}
	\frac{Gm(r)}{rc^2} \ll 1
	\quad \text{and} \quad
	\frac{GM}{Rc^2} \ll 1 ,
	\label{eq:weak_field_limit:small3}
\end{equation}
so the rightmost correction in \cref{eq:tov} vanishes.

%By Taylor expanding the pressures \eqref{eq:incompressible_star:central_pressure} around the small parameter $GM/R \ll 1$, we get precisely the pressures in \cref{eq:weak_field_limit:newtonian_pressure}.
Second, we argued in \cref{sec:incompressible_star} that a star with constant energy density $\epsilon_0$ is the one that can withstand the most extreme pressure.
We also mentioned above that the pressure approaches the one in \cref{eq:weak_field_limit:newtonian_pressure} in the limit \eqref{eq:weak_field_limit:small3}.
Using the central pressure $p(0)$ for this \emph{specific} case of a star of constant energy density $\epsilon_0$, we therefore expect that for \emph{any} kind of star, the pressure $p(r)$ and energy density $\epsilon(r)$ satisfy
\begin{equation}
	\frac{p(r)}{\epsilon(r)} \leq \frac{p(0)}{\epsilon_0} = \frac{1}{2} \frac{GM}{Rc^2} \ll 1 .
	\label{eq:weak_field_limit:small1}
\end{equation}
This causes the leftmost correction in \cref{eq:tov} to vanish, too.

Third, for a star with non-increasing energy density $\epsilon(r)$ away from the center $r=0$ -- which we deemed to be the only physical type of stars in \cref{sec:incompressible_star} -- we can pull the minimum density $\epsilon(r)$ outside the integral \eqref{eq:einstein_to_tov:m_integral} to get $m(r) c^2 \geq \frac{4}{3} \pi r^3 \epsilon(r)$, so
\begin{equation*}
	\frac{4 \pi r^3 p(r)}{m(r) c^2} \leq \frac{4 \pi r^3 p(r)}{\frac{4}{3} \pi r^3 \epsilon(r)}
	                                =    \frac{3 p(r)}{\epsilon(r)}
						            \ll  1
\end{equation*}
by \cref{eq:weak_field_limit:small1}, and the middle correction in \cref{eq:tov} also vanishes.

Thus, we do indeed recover the Newtonian pressure gradient \eqref{eq:weak_field_limit:newtonian_pressure_gradient} from the relativistic pressure gradient \eqref{eq:tov} in the Newtonian limit!
In fact, there is an alternative way to see it.
If we write \cref{eq:tov} with mass density $\rho = \epsilon/c^2$ insead of energy density $\epsilon$, it becomes
\begin{equation}
	\diff{p}{r} = -\frac{G m(r) \rho(r)}{r^2} \left( 1 + \frac{p(r)}{\rho(r) c^2} \right) \left( 1 + \frac{4 \pi r^3 p(r)}{m(r) c^2} \right) \left( 1 - \frac{2 G m(r)}{r c^2} \right)^{-1} .
	\label{eq:tov_units}
\end{equation}
The Newtonian limit corresponds to sending $c \rightarrow \infty$, as this reduces the Lorentz transformations of relativity to the Galilei transformations of Newtonian physics.
But sending $c \rightarrow \infty$ kills all corrections in the three parentheses of \cref{eq:tov_units}, which restores the Newtonian pressure gradient \eqref{eq:weak_field_limit:newtonian_pressure_gradient}!
