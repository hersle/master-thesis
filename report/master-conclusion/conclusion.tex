\chapter{Conclusions and Outlook}

\section{Conclusions}

\begin{figure}[p]
\centering
\tikzsetnextfilename{summary-mass-radius}
\begin{tikzpicture}
\newcommand\ploteverything{
\addplot [name path=J0748lo, draw=none, forget plot, domain=0:40] {2.08 - 0.07};
\addplot [name path=J0748hi, draw=none, forget plot, domain=0:40] {2.08 + 0.07};
\addplot [red, opacity=0.3, forget plot] fill between [of=J0748lo and J0748hi]; % \addlegendentry{PSR J0748$+$6620};
\node [anchor=west, red, font=\scriptsize] at (7.2, 2.08) {PSR J0748$+$6620};

\addplot [name path=J0348lo, draw=none, forget plot, domain=0:40] {2.01 - 0.04};
\addplot [name path=J0348hi, draw=none, forget plot, domain=0:40] {2.01 + 0.04};
\addplot [green, opacity=0.3, forget plot] fill between [of=J0348lo and J0348hi]; % \addlegendentry{PSR J0348$+$0432};
\node [anchor=west, green, font=\scriptsize] at (7.2, 2.01) {PSR J0348$+$0432};

\addplot [name path=J1614lo, draw=none, forget plot, domain=0:40] {1.91 - 0.02};
\addplot [name path=J1614hi, draw=none, forget plot, domain=0:40] {1.91 + 0.02};
\addplot [yellow, opacity=0.3, forget plot] fill between [of=J1614lo and J1614hi]; % \addlegendentry{PSR J1614$-$2230};
\node [anchor=west, yellow, font=\scriptsize] at (7.2, 1.91) {PSR J1614$-$2230};

\addplot+ [solid, very thick, index of colormap={8 of Set1-9}, draw=none] table [x=R, y=M] {../code/data/LSM3F_APR/stars_hadron.dat}; \addlegendentry{APR}; % hack: put BEFORE hybrids in legend
\addplot+ [white] {2.6}; \addlegendentry{};
\addplot+ [white] {2.6}; \addlegendentry{};
\addplot+ [white] {2.6}; \addlegendentry{};
\addplot+ [white] {2.6}; \addlegendentry{};
\addplot+ [white] {2.6}; \addlegendentry{};

\addplot+ [densely dashed, very thick, index of colormap={3 of Set1-9}, color=.!60!white] table [x=R, y=M] {../code/data/MIT2F/stars_sigma_800_B14_155.dat}; \addlegendentry{$\text{MIT}_2 \, (\smash{B^{\frac14}}\!\!=\!\SI{155}{\mega\electronvolt})$};
\addplot+ [densely dashed, very thick, index of colormap={3 of Set1-9}, color=.!80!white] table [x=R, y=M] {../code/data/MIT2F/stars_sigma_800_B14_150.dat}; \addlegendentry{$\text{MIT}_2 \, (\smash{B^{\frac14}}\!\!=\!\SI{150}{\mega\electronvolt})$};
\addplot+ [densely dashed, very thick, index of colormap={3 of Set1-9}, color=.!100!white] table [x=R, y=M] {../code/data/MIT2F/stars_sigma_800_B14_145.dat}; \addlegendentry{$\text{MIT}_2 \, (\smash{B^{\frac14}}\!\!=\!\SI{145}{\mega\electronvolt})$};
\addplot+ [solid,          very thick, index of colormap={3 of Set1-9}, color=.!60!white] table [x=R, y=M] {../code/data/MIT3F/stars_sigma_800_B14_155.dat}; \addlegendentry{$\text{MIT}_3 \, (\smash{B^{\frac14}}\!\!=\!\SI{155}{\mega\electronvolt})$};
\addplot+ [solid,          very thick, index of colormap={3 of Set1-9}, color=.!80!white] table [x=R, y=M] {../code/data/MIT3F/stars_sigma_800_B14_150.dat}; \addlegendentry{$\text{MIT}_3 \, (\smash{B^{\frac14}}\!\!=\!\SI{150}{\mega\electronvolt})$};
\addplot+ [solid,          very thick, index of colormap={3 of Set1-9}, color=.!100!white] table [x=R, y=M] {../code/data/MIT3F/stars_sigma_800_B14_145.dat}; \addlegendentry{$\text{MIT}_3 \, (\smash{B^{\frac14}}\!\!=\!\SI{145}{\mega\electronvolt})$};

\addplot+ [densely dashed, very thick, index of colormap={4 of Set1-9}, color=.!60!white] table [x=R, y=M] {../code/data/LSM2F/stars_sigma_600_B14_111.dat}; \addlegendentry{$\text{QM}_2 \, (m_\sigma\!\!=\!\SI{600}{\mega\electronvolt})\!$};
\addplot+ [densely dashed, very thick, index of colormap={4 of Set1-9}, color=.!80!white] table [x=R, y=M] {../code/data/LSM2F/stars_sigma_700_B14_68.dat};  \addlegendentry{$\text{QM}_2 \, (m_\sigma\!\!=\!\SI{700}{\mega\electronvolt})\!$};
\addplot+ [densely dashed, very thick, index of colormap={4 of Set1-9}, color=.!100!white] table [x=R, y=M] {../code/data/LSM2F/stars_sigma_800_B14_27.dat};  \addlegendentry{$\text{QM}_2 \, (m_\sigma\!\!=\!\SI{800}{\mega\electronvolt})\!$};
\addplot+ [solid,          very thick, index of colormap={4 of Set1-9}, color=.!60!white] table [x=R, y=M] {../code/data/LSM3F/stars_sigma_600_B14_111.dat}; \addlegendentry{$\text{QM}_3 \, (m_\sigma\!\!=\!\SI{600}{\mega\electronvolt})\!$};
\addplot+ [solid,          very thick, index of colormap={4 of Set1-9}, color=.!80!white] table [x=R, y=M] {../code/data/LSM3F/stars_sigma_700_B14_68.dat};  \addlegendentry{$\text{QM}_3 \, (m_\sigma\!\!=\!\SI{700}{\mega\electronvolt})\!$};
\addplot+ [solid,          very thick, index of colormap={4 of Set1-9}, color=.!100!white] table [x=R, y=M] {../code/data/LSM3F/stars_sigma_800_B14_27.dat};  \addlegendentry{$\text{QM}_3 \, (m_\sigma\!\!=\!\SI{800}{\mega\electronvolt})\!$};

\addplot+ [densely dashed, very thick, index of colormap={1 of Set1-9}, color=.!60!white] table [x=R, y=M] {../code/data/LSM2F_APR/stars_sigma_600_B14_111.dat}; \addlegendentry{$\text{QM}_2\!\!+\!\!\text{APR} \, (m_\sigma\!\!=\!\SI{600}{\mega\electronvolt})$};
\addplot+ [densely dashed, very thick, index of colormap={1 of Set1-9}, color=.!80!white] table [x=R, y=M] {../code/data/LSM2F_APR/stars_sigma_700_B14_68.dat};  \addlegendentry{$\text{QM}_2\!\!+\!\!\text{APR} \, (m_\sigma\!\!=\!\SI{700}{\mega\electronvolt})$};
\addplot+ [densely dashed, very thick, index of colormap={1 of Set1-9}, color=.!100!white] table [x=R, y=M] {../code/data/LSM2F_APR/stars_sigma_800_B14_27.dat};  \addlegendentry{$\text{QM}_2\!\!+\!\!\text{APR} \, (m_\sigma\!\!=\!\SI{800}{\mega\electronvolt})$};
\addplot+ [solid,          very thick, index of colormap={1 of Set1-9}, color=.!60!white] table [x=R, y=M] {../code/data/LSM3F_APR/stars_sigma_600_B14_111.dat}; \addlegendentry{$\text{QM}_3\!\!+\!\!\text{APR} \, (m_\sigma\!\!=\!\SI{600}{\mega\electronvolt})$};
\addplot+ [solid,          very thick, index of colormap={1 of Set1-9}, color=.!80!white] table [x=R, y=M] {../code/data/LSM3F_APR/stars_sigma_700_B14_68.dat};  \addlegendentry{$\text{QM}_3\!\!+\!\!\text{APR} \, (m_\sigma\!\!=\!\SI{700}{\mega\electronvolt})$};
\addplot+ [solid,          very thick, index of colormap={1 of Set1-9}, color=.!100!white] table [x=R, y=M] {../code/data/LSM3F_APR/stars_sigma_800_B14_27.dat};  \addlegendentry{$\text{QM}_3\!\!+\!\!\text{APR} \, (m_\sigma\!\!=\!\SI{800}{\mega\electronvolt})$};

\addplot+ [solid, very thick, index of colormap={8 of Set1-9}, forget plot] table [x=R, y=M] {../code/data/LSM3F_APR/stars_hadron.dat}; % hack: draw OVER hybrids
}
\begin{axis}[
	width=15.5cm,
	height=10.5cm,
	xlabel={$R \, / \, \si{\kilo\meter}$ },
	ylabel={$M \, / \, M_\odot$}, %title={Mass-radius diagram for 2-flavor quark stars }, title style={yshift=2.0cm},
	xmin=7, xmax=16, ymin=0.0, ymax=2.5, xtick distance=1, ytick distance=0.5, minor x tick num=9, minor y tick num=4,
	grid=both, minor grid style={ultra thin, gray!10!white},
	legend transposed, legend columns=6, legend cell align = left, legend style={font=\small, anchor=north, at={(0.5, -0.15)}},
	title = {Summary of all modeled quark and hybrid stars },
	restrict y to domain=0.2:3.0,
];
\ploteverything
\coordinate (zoomplot) at (15.9, 2.4);
\end{axis}
\node [anchor=north east] at (zoomplot) {
	\begin{tikzpicture}[trim axis left, trim axis right, baseline]
	\begin{axis}[
		axis background/.style={fill=white},
		width=5cm, height=6cm,
		xmin=11, xmax=12, ymin=1.85, ymax=2.15, xtick distance=1.0, ytick distance=0.1, minor x tick num=9, minor y tick num=9, %extra y ticks={1.9, 2.1},
		grid=both, minor grid style={ultra thin, gray!10!white},
	]
	\ploteverything
	\legend{};
	\end{axis}
	\end{tikzpicture}
};
\end{tikzpicture}
\caption{\label{fig:conclusion:mass-radius}%
	Summary of all mass-radius relations obtained in this thesis:
	neutron stars with the hadronic Akmal-Pandharipande-Ravenhall equation of state (APR);
	quark stars with the two-flavor and three-flavor MIT bag models ($\text{MIT}_2$ and $\text{MIT}_3$)
	and quark-meson models ($\text{QM}_2$ and $\text{QM}_3$);
	and hybrid stars joining $\text{QM}_2+\text{APR}$ and $\text{QM}_3+\text{APR}$.
	In the $\text{MIT}_2$ and $\text{MIT}_3$ models, we have used bag constants covering the bag window \eqref{eq:mit:bag_window}.
	In the $\text{QM}_2$ and $\text{QM}_3$ models, we vary $\SI{600}{\mega\electronvolt} \leq m_\sigma \leq \SI{800}{\mega\electronvolt}$
	and use the corresponding lowest bag constants at the lower bounds \eqref{eq:lsm:bag_lower_bound},
	as greater bag constants only generate less massive stars.
	The colored bands show measured masses
	of the heavy pulsars \textcolor{green}{PSR J0348$+$0432}, \textcolor{yellow}{PSR J1614$-$2230} and \textcolor{red}{PSR J0740$+$6620}
	from \cite{ref:antoniadis,ref:arzoumanian,ref:fonseca}.
	For clearer presentations of the results with each model,
	please consult \cref{fig:mit:mass_radius,fig:lsm:2-flavor-mass-radius,fig:lsm:3-flavor-mass-radius,fig:hybrid:mass-radius}.
}
\end{figure}

\begin{table}[p]
\centering
\caption{\label{tab:master_conclusion:results}%
Maximum masses and corresponding radii of the stellar sequences in \cref{fig:conclusion:mass-radius}.
}
{\setlength{\tabcolsep}{4pt} % affect only this table
\begin{tabular}{ c l l c c c }
	\toprule
	Flavors & Chapter & Model & Maximum masses & Corresponding radii                            \\
	\midrule
	%\Cref{chap:mit} & MIT bag model ($N_f\!=\!2$) & $1.76 \, M_\odot \leq M \leq 2.00 \, M_\odot$ & $\hphantom{1}\SI{9.70}{\kilo\meter} \leq R \leq \SI{10.83}{\kilo\meter}$ \\
	%\Cref{chap:mit} & MIT bag model ($N_f\!=\!3$) & $1.63 \, M_\odot \leq M \leq 1.85 \, M_\odot$ & $\hphantom{1}\SI{9.21}{\kilo\meter} \leq R \leq \SI{10.25}{\kilo\meter}$ \\
	%\Cref{chap:lsm2f} & Quark-meson model ($N_f\!=\!2$) & $1.77 \, M_\odot \leq M \leq 2.02 \, M_\odot$ & $\SI{10.91}{\kilo\meter} \leq R \leq \SI{10.98}{\kilo\meter}$ \\
	%\Cref{chap:lsm3f} & Quark-meson model ($N_f\!=\!3$) & $1.63 \, M_\odot \leq M \leq 1.81 \, M_\odot$ & $\SI{10.82}{\kilo\meter} \leq R \leq \SI{11.57}{\kilo\meter}$ \\
	%\Cref{chap:hybrid} & Hybrid model & $1.89 \, M_\odot \leq M \leq 2.06 \, M_\odot$ & $\SI{11.23}{\kilo\meter} \leq R \leq \SI{11.48}{\kilo\meter}$ \\
	$N_f=2$ & \Cref{chap:mit} & MIT bag model & $1.7 \, M_\odot \leq M \leq 2.0 \, M_\odot$ & $\hphantom{1}\SI{9.6}{\kilo\meter} \leq R \leq \SI{11.0}{\kilo\meter}$ \\
	$N_f=2$ & \Cref{chap:lsm2f} & Quark-meson model & $1.8 \, M_\odot \leq M \leq 2.0 \, M_\odot$ & $\SI{10.9}{\kilo\meter} \leq R \leq \SI{11.2}{\kilo\meter}$ \\
	$N_f=2$ & \Cref{chap:hybrid} & Hybrid model & $2.0 \, M_\odot \leq M \leq 2.1 \, M_\odot$ & $\SI{11.2}{\kilo\meter} \leq R \leq \SI{11.2}{\kilo\meter}$ \\
	\midrule
	$N_f=3$ & \Cref{chap:mit} & MIT bag model & $1.6 \, M_\odot \leq M \leq 1.9 \, M_\odot$ & $\hphantom{1}\SI{9.0}{\kilo\meter} \leq R \leq \SI{10.3}{\kilo\meter}$ \\
	$N_f=3$ & \Cref{chap:lsm3f} & Quark-meson model & $1.6 \, M_\odot \leq M \leq 1.8 \, M_\odot$ & $\SI{11.0}{\kilo\meter} \leq R \leq \SI{11.6}{\kilo\meter}$ \\
	$N_f=3$ & \Cref{chap:hybrid} & Hybrid model & $1.9 \, M_\odot \leq M \leq 2.1 \, M_\odot$ & $\SI{11.2}{\kilo\meter} \leq R \leq \SI{11.5}{\kilo\meter}$ \\
	\bottomrule
\end{tabular}}
\end{table}

In this thesis we have modeled quark stars using the MIT bag model and the quark-meson model with two and three flavors,
and finally hybrid stars by joining the quark-meson models with the hadronic Akmal-Pandharipande-Ravenhall equation of state from \cite{ref:apr,ref:apr_data}.
All resulting mass-radius relations, their maximum masses and corresponding radii
are summarized in \cref{fig:conclusion:mass-radius} and \cref{tab:master_conclusion:results}.

With the quark-meson model we could create pure quark stars with two flavors up to $M \leq 2.0 \, M_\odot$ and with three flavors up to $M \leq 1.8 \, M_\odot$.
Due to their softer equations of state, we saw that three-flavor stars are generally less massive than two-flavor stars.
As two-flavor quark matter is unstable compared to hadronic matter,
pure two-flavor quark stars are unlikely to be found in nature.
However, \emph{if} the strange matter hypothesis of \cite{ref:strange_hypothesis_bodmer,ref:strange_hypothesis_witten} is true,
strange quark stars consisting of three-flavor quark matter out to the surface would be stable and could exist.
We fail to model strange quark stars self-consistently with the three-flavor quark-meson model,
as the bag constant must exceed its upper bound calculated precisely by \emph{assuming} the strange matter hypothesis if the strange quark is to exist out to surface.

We calculated the grand potential of the quark-meson models to tree-level in bosons and one-loop in fermions.
This is \emph{inconsistent} in terms of the number of loops, but can be regarded \emph{consistent} in the one-loop large-$N_c$ limit.
The equations of state were then found at zero temperature, in $\beta$-equilibrium and subject to local electrical charge neutrality.

In particular, the parameter space of the quark-meson model is plagued by the ``ad-hoc'' $\sigma$ meson,
whose mass is only known to lie in $\SI{400}{\mega\electronvolt} \leq m_\sigma \leq \SI{550}{\mega\electronvolt}$. \cite{ref:pdg_review_2021}
Depending on its value, we saw that chiral symmetry restoration occurred in a rapid crossover for $m_\sigma \geq \SI{800}{\mega\electronvolt}$,
but a discontinuous phase transition for $m_\sigma < \SI{800}{\mega\electronvolt}$.
Moreover, we could only fit masses $m_\sigma \geq \SI{600}{\mega\electronvolt}$ that exceed the measurements
to prevent the minimum of the grand potential from disappearing in the vacuum phase.
With two flavors,
we nailed this problem down to the inconsistency of fitting parameters at tree-level to a potential that is calculated in the one-loop large-$N_c$ limit.
Repeating our calculation with a consistently fit potential in this limit found by \cite{ref:jo_lsm_consistent_chiral,ref:jo_lsm_consistent_physical},
we learned that we could still trust our original results,
as if inconsistently fit $\SI{600}{\mega\electronvolt} \leq m_\sigma \leq \SI{800}{\mega\electronvolt}$
correspond to consistently fit $\SI{400}{\mega\electronvolt} \leq m_\sigma \leq \SI{600}{\mega\electronvolt}$.

Using the quark-meson models in the core, we could also form short ranges of stable hybrid stars up to $M \leq 2.1 \, M_\odot$
with both two and three flavors, where the two-flavor stars are generally somewhat heavier due to their stiffer equations of state.
The hybrid stars contained very small quark cores no more massive than $0.12 M_\odot$ and $0.02 M_\odot$ with two and three flavors, respectively.
This resonates with the claim in \cite{ref:quark_star_review} that hybrid star quark cores are very small, if they exist at all.
Beyond the maximum mass star,
the discontinuous phase transition between the quark and hadronic phases becomes so severe
that it destabilizes stars with larger quark cores against radial perturbations.
We succeed in modeling hybrid stars with small quark cores
around the recent mass observations \cite{ref:antoniadis,ref:arzoumanian,ref:fonseca}
of the heavy pulsars PSR J1614$-$2230, PSR J0348$+$0432 and PSR J0748$+$6620 around and above $2 M_\odot$.

The indication that hybrid stars can have only small quark cores
suggests that they are hard to observe decisively in nature.
With future advances in our theoretical understanding and observational techniques,
such as the \textbf{N}eutron \textbf{S}tar \textbf{I}nterior \textbf{C}omposition \textbf{E}xplore\textbf{r} (\textbf{NICER}) mission \cite{ref:nicer},
heavy observed neutron stars could one day turn out to have quark cores.


In earlier work, \cite{ref:lsm3f_compact_stars} have modeled pure quark stars with a vector meson-extended three-flavor quark-meson model,
and this was incorporated into hybrid stars in \cite{ref:lsm3f_hybrid_stars}.
With the vector meson interaction turned off and using parameters comparable to ours,
they find hybrid stars with similar maximum masses $M \leq 1.9 M_\odot$, % only B differs?
and greater masses can be reached if this interaction is enabled.
Quark stars have also been modeled with the Nambu-Jona-Lasinio model in \cite{ref:quark_star_njl} and generalized to hybrid stars in \cite{ref:hybrid_stars_njl}, for example.
They also find maximum hybrid star masses $1.9 M_\odot \leq M \leq 2.1 M_\odot$ with reasonable parameter choices.
Even \cite{ref:quark_hybrid_additional_ref} find hybrid stars in the band $2.0 M_\odot \leq M \leq 2.1 M_\odot$
using a non-perturbative functional renormalization group approach with the quark-meson model.
Despite using different models for both the quark and hadronic phases,
our results are in good agreement with these works.

\section{Outlook}

A natural extension of our work would be to
explore the parameter space of the quark-meson model more thoroughly.
Our philosophy has been to keep almost all parameters fixed,
but vary the most uncertain parameter $m_\sigma$ and focus on the corresponding lowest instability-respecting bag constants $B$
because it generates stiffer equations of state and thus greater maximum masses.
A more detailed treatment could also vary the quark masses,
other meson masses and use larger bag constants.
For example, \cite{ref:lsm3f_compact_stars}
find that the maximum mass increases with the quark masses $m_u=m_d$
in the variation of our model mentioned above.
One can also include more parameters by adding terms describing the axial anomaly of quantum chromodynamics
and studying more complex symmetry breaking patterns, as described at length in \cite{ref:lsm3f_details}.

A consistent parameter fitting method in the one-loop large-$N_c$ limit is
only available for the two-flavor quark-meson model.
It would be interesting to generalize it to the three-flavor model
and see if the ability to fit experimental values of $m_\sigma$ carries over from the two-flavor case.

It is natural to expect that our greatest source of error
lies neither in the calculation of the grand potential nor its parameter fitting,
but in the quark-meson model \emph{itself}.
First-principle approaches of studying quark matter from quantum chromodynamics remains difficult,
as the sign problem plagues lattice calculations at finite baryon chemical potentials 
and perturbation theory is applicable only at high energies.
Of course, other and more sophisticated effective models could be examined.
For example, confinement could be incorporated with Polyakov-loop extended quark-meson models (PQM models, see for example \cite{ref:pqm_2f,ref:pqm_3f,ref:master_folkestad}) and NJL models (PNJL models, see for example \cite{ref:pnjl_2f,ref:pnjl_3f,ref:pnjl_3f_zeroT}).
One could also examine the effects of modeling the color-superconducting phase of the phase diagram in \cref{fig:qcd:phase_diagram}.

More effort could also be put into handling the phase transition in hybrid stars more carefully.
For example, the interpolation techniques described in \cite{ref:quark_star_review}
can be used near the intersection point of the transition,
where both the quark and hadronic equations of state are unreliable.
This method constructs a unified equation of state
that bridges the gap between the two phases,
constrained by physical requirements like retaining a causal speed of sound.

One can also refine calculations to nonzero temperature and beyond mean fields,
investigate effects of non-local charge neutrality,
and solve the axially symmetric generalization of the spherically symmetric Tolman-Oppenheimer-Volkoff
to model realistic spinning pulsars.
