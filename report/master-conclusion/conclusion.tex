\chapter{Conclusions and outlook}

\section{Conclusions}

\begin{table}[b!]
\centering
\caption{\label{tab:master_conclusion:results}%
Summary of maximum masses and corresponding radii obtained using all models in this thesis.
In the MIT bag model, all parameters are fixed except $\SI{145}{\mega\electronvolt} \leq B^\frac14 \leq \SI{155}{\mega\electronvolt}$.
The quark-meson model varies 
$\SI{600}{\mega\electronvolt} \leq m_\sigma \leq \SI{800}{\mega\electronvolt}$
and
$\SI{27}{\mega\electronvolt} \leq B^\frac14 \leq \SI{111}{\mega\electronvolt}$,
where the lower and upper $B$ is used only with the respective lower and upper $m_\sigma$.
The hybrid model joins the quark-meson model with the same parameters and the hadronic Akmal-Pandharipande-Ravenhall equation of state.
The bag constants are the lowest ones that respect instability of two-flavor quark matter compared to hadronic matter.
Greater bag constants violate the strange matter hypothesis and generate only lower maximum masses.
}
{\setlength{\tabcolsep}{4pt} % affect only this table
\begin{tabular}{ c l l c c c }
	\toprule
	Flavors & Chapter & Model & Maximum masses & Corresponding radii                            \\
	\midrule
	%\Cref{chap:mit} & MIT bag model ($N_f\!=\!2$) & $1.76 \, M_\odot \leq M \leq 2.00 \, M_\odot$ & $\hphantom{1}\SI{9.70}{\kilo\meter} \leq R \leq \SI{10.83}{\kilo\meter}$ \\
	%\Cref{chap:mit} & MIT bag model ($N_f\!=\!3$) & $1.63 \, M_\odot \leq M \leq 1.85 \, M_\odot$ & $\hphantom{1}\SI{9.21}{\kilo\meter} \leq R \leq \SI{10.25}{\kilo\meter}$ \\
	%\Cref{chap:lsm2f} & Quark-meson model ($N_f\!=\!2$) & $1.77 \, M_\odot \leq M \leq 2.02 \, M_\odot$ & $\SI{10.91}{\kilo\meter} \leq R \leq \SI{10.98}{\kilo\meter}$ \\
	%\Cref{chap:lsm3f} & Quark-meson model ($N_f\!=\!3$) & $1.63 \, M_\odot \leq M \leq 1.81 \, M_\odot$ & $\SI{10.82}{\kilo\meter} \leq R \leq \SI{11.57}{\kilo\meter}$ \\
	%\Cref{chap:hybrid} & Hybrid model & $1.89 \, M_\odot \leq M \leq 2.06 \, M_\odot$ & $\SI{11.23}{\kilo\meter} \leq R \leq \SI{11.48}{\kilo\meter}$ \\
	$N_f=2$ & \Cref{chap:mit} & MIT bag model & $1.7 \, M_\odot \leq M \leq 2.0 \, M_\odot$ & $\hphantom{1}\SI{9.6}{\kilo\meter} \leq R \leq \SI{11.0}{\kilo\meter}$ \\
	$N_f=2$ & \Cref{chap:lsm2f} & Quark-meson model & $1.8 \, M_\odot \leq M \leq 2.0 \, M_\odot$ & $\SI{10.9}{\kilo\meter} \leq R \leq \SI{11.2}{\kilo\meter}$ \\
	$N_f=2$ & \Cref{chap:hybrid} & Hybrid model & $2.0 \, M_\odot \leq M \leq 2.1 \, M_\odot$ & $\SI{11.2}{\kilo\meter} \leq R \leq \SI{11.2}{\kilo\meter}$ \\
	\midrule
	$N_f=3$ & \Cref{chap:mit} & MIT bag model & $1.6 \, M_\odot \leq M \leq 1.9 \, M_\odot$ & $\hphantom{1}\SI{9.0}{\kilo\meter} \leq R \leq \SI{10.3}{\kilo\meter}$ \\
	$N_f=3$ & \Cref{chap:lsm3f} & Quark-meson model & $1.6 \, M_\odot \leq M \leq 1.8 \, M_\odot$ & $\SI{11.0}{\kilo\meter} \leq R \leq \SI{11.6}{\kilo\meter}$ \\
	$N_f=3$ & \Cref{chap:hybrid} & Hybrid model & $1.9 \, M_\odot \leq M \leq 2.1 \, M_\odot$ & $\SI{11.2}{\kilo\meter} \leq R \leq \SI{11.5}{\kilo\meter}$ \\
	\bottomrule
\end{tabular}}
\end{table}

In this thesis we have modeled quark stars using the MIT bag model and quark-meson model with two and three flavors,
and finally hybrid stars by combining the quark-meson model and hadronic Akmal-Pandharipande-Ravenhall equation of state.
The resulting maximum masses and corresponding radii are summarized in \cref{tab:master_conclusion:results}.
In particular, with the quark-meson model we managed to create pure two-flavor quark stars with masses up to $M \leq 2.0 \, M_\odot$ and three-flavor quark stars up to $M \leq 1.8 \, M_\odot$.
Using $m_\sigma = \SI{600}{\mega\electronvolt}$ in the three-flavor quark-meson model,
we could also form a short range of stable hybrid stars up to $M \leq 1.9 \, M_\odot$ containing very small quark cores with mass $M_\text{core} \leq 0.02 M_\odot$,
but denser hybrid stars were unstable against radial perturbations due to a discontinuous phase transition between the quark and hadron phases.
With two-flavor quark cores,
we obtained more massive stable hybrid stars up towards $M \leq 2.1 \, M_\odot$ using larger masses $\SI{600}{\mega\electronvolt} \leq m_\sigma \leq \SI{800}{\mega\electronvolt}$.
In short, we find it possible to model hybrid stars with small quark cores around the recent mass observations of the pulsars PSR J1614$-$2230, PSR J0348$+$0432 and PSR J0748$+$6620 around and above $2 M_\odot$,
and pure quark stars of slightly lower mass.

The fact that hybrid stars seem to exhibit very small quark cores
make it very easy to mistake neutron stars for hybrid stars.
They could even be mistaken for pure quark stars, as they allow similar maximum masses.
With future advances in theory and observational techniques,
many observed neutron stars could one day really turn out to be hybrid stars, if not quark stars.

The parameter space of the quark-meson model is particularly plagued by the ``ad-hoc'' $\sigma$-meson,
whose mass is known only within the large range $\SI{400}{\mega\electronvolt} \leq m_\sigma \leq \SI{550}{\mega\electronvolt}$.
Moreover, we could only fit greater masses $m_\sigma \geq \SI{600}{\mega\electronvolt}$ to prevent the ground state of the grand potential from vanishing in vacuum.
We nailed this problem down to the inconsistency of fitting parameters at tree-level to a potential that is calculated to one loop,
and repeated our calculation with a consistently fit potential to learn that we could still trust our results,
as if the inconsistently fit values of $m_\sigma$ corresponded to $\SI{200}{\mega\electronvolt}$ lower values.
Such a consistently fit grand potential is currently only found for the two-flavor model, and it would be useful to find one also for the three-flavor model.
In addition, we saw that chiral symmetry restoration occurred in a rapid crossover for $m_\sigma \geq \SI{800}{\mega\electronvolt}$,
but a discontinuous phase transition for $m_\sigma < \SI{800}{\mega\electronvolt}$.

In earlier work, \cite{ref:lsm3f_compact_stars} have modeled pure quark stars with a vector meson-extended three-flavor quark meson model,
and this work has since been incorporated into hybrid stars in \cite{ref:lsm3f_hybrid_stars}.
With $m_u = m_d = \SI{300}{\mega\electronvolt}$, $m_\sigma = \SI{600}{\mega\electronvolt}$, $B = (\SI{140}{\mega\electronvolt})^4$ and the vector meson interaction turned off,
they find hybrid stars a maximum mass $M \lesssim 1.9 M_\odot$ in \cite[figure 8]{ref:lsm3f_hybrid_stars}, % (that is in figure 8)
and we find the same maximum mass using similar parameters except $B = (\SI{111}{\mega\electronvolt})^4$.
Quark stars have also been modeled with the Nambu-Jona-Lasinio model in \cite{ref:quark_star_njl} and generalized to hybrid stars in \cite{ref:hybrid_stars_njl}, for example.
In \cite[figure 3]{ref:hybrid_stars_njl}, they also find maximum hybrid star masses $1.9 M_\odot \leq M \leq 2.1 M_\odot$ with reasonable parameter choices.
Finally, the hybrid stars in \cite{ref:quark_hybrid_additional_ref} also lie in the band $2.0 M_\odot \leq M \leq 2.1 M_\odot$.
Although these works use different models and hadronic equations of state,
our results are in relatively good agreement.

\section{Outlook}

There are many possible directions for future work.

Our treatment of bosons to tree-level and fermions to one loop was inconsistent in terms of loops,
but consistent in the large-$N_c$ approximation scheme.
It would therefore be useful to refine calculations beyond mean bosonic fields.
We also saw that fitting parameters at tree-level to a one-loop grand potential was inconsistent
and performed a consistent one-loop fit in the two-flavor model using the results of \cite{ref:jo_lsm_consistent_chiral,ref:jo_lsm_consistent_physical}.
Such a consistently fit grand potential has not been found for the three-flavor model,
so it would be interesting to find one and see if the ability to fit experimental values of $m_\sigma$ carries over from the two-flavor model.

Different and more sophisticated effective models could also be examined.
For example, confinement could be incorporated with Polyakov-loop extended quark-meson models (PQM models, see for example \cite{ref:pqm_2f,ref:pqm_3f,ref:master_folkestad}) and NJL models (PNJL models, see for example \cite{ref:pnjl_2f,ref:pnjl_3f,ref:pnjl_3f_zeroT}).
It would also be interesting to see the effects of modeling the color superconducting phase of the phase diagram in \cref{fig:qcd:phase_diagram}.

More effort could also be put into studying hybrid stars by handling the phase transition in a more careful manner,
for example using interpolation techniques like those explained in \cite{ref:quark_star_review}.

One could also make more accurate calculations at nonzero temperature,
explore the parameter space of the bag constant, quark and meson masses in a detailed manner,
investigate effects of non-local charge neutrality in the phase diagram and the star,
and consider rotating stars that are more representative of observed pulsars than the static Tolman-Oppenheimer-Volkoff equation.
