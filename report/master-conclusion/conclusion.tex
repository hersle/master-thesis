\chapter{Conclusions and outlook}

\section{Conclusions}

\begin{table}
\centering
{\setlength{\tabcolsep}{4pt} % affect only this table
\begin{tabular}{ l l c c c }
	\toprule
	Chapter & Model & Maximum masses & Corresponding radii                            \\
	\midrule
	%\Cref{chap:mit} & MIT bag model ($N_f\!=\!2$) & $1.76 \, M_\odot \leq M \leq 2.00 \, M_\odot$ & $\hphantom{1}\SI{9.70}{\kilo\meter} \leq R \leq \SI{10.83}{\kilo\meter}$ \\
	%\Cref{chap:mit} & MIT bag model ($N_f\!=\!3$) & $1.63 \, M_\odot \leq M \leq 1.85 \, M_\odot$ & $\hphantom{1}\SI{9.21}{\kilo\meter} \leq R \leq \SI{10.25}{\kilo\meter}$ \\
	%\Cref{chap:lsm2f} & Quark-meson model ($N_f\!=\!2$) & $1.77 \, M_\odot \leq M \leq 2.02 \, M_\odot$ & $\SI{10.91}{\kilo\meter} \leq R \leq \SI{10.98}{\kilo\meter}$ \\
	%\Cref{chap:lsm3f} & Quark-meson model ($N_f\!=\!3$) & $1.63 \, M_\odot \leq M \leq 1.81 \, M_\odot$ & $\SI{10.82}{\kilo\meter} \leq R \leq \SI{11.57}{\kilo\meter}$ \\
	%\Cref{chap:hybrid} & Hybrid model & $1.89 \, M_\odot \leq M \leq 2.06 \, M_\odot$ & $\SI{11.23}{\kilo\meter} \leq R \leq \SI{11.48}{\kilo\meter}$ \\
	\Cref{chap:mit} & MIT bag model ($N_f=2$) & $1.7 \, M_\odot \leq M \leq 2.0 \, M_\odot$ & $\hphantom{1}\SI{9.6}{\kilo\meter} \leq R \leq \SI{11.0}{\kilo\meter}$ \\
	\Cref{chap:mit} & MIT bag model ($N_f=3$) & $1.6 \, M_\odot \leq M \leq 1.9 \, M_\odot$ & $\hphantom{1}\SI{9.0}{\kilo\meter} \leq R \leq \SI{10.3}{\kilo\meter}$ \\
	\Cref{chap:lsm2f} & Quark-meson model ($N_f=2$) & $1.8 \, M_\odot \leq M \leq 2.0 \, M_\odot$ & $\SI{10.9}{\kilo\meter} \leq R \leq \SI{11.2}{\kilo\meter}$ \\
	\Cref{chap:lsm3f} & Quark-meson model ($N_f=3$) & $1.6 \, M_\odot \leq M \leq 1.8 \, M_\odot$ & $\SI{11.0}{\kilo\meter} \leq R \leq \SI{11.6}{\kilo\meter}$ \\
	\Cref{chap:hybrid} & Hybrid model & $1.9 \, M_\odot \leq M \leq 2.1 \, M_\odot$ & $\SI{11.2}{\kilo\meter} \leq R \leq \SI{11.5}{\kilo\meter}$ \\
	\bottomrule
\end{tabular}}
\caption{\label{tab:master_conclusion:results}%
Summary of the maximum masses and corresponding radii obtained using all different models in this thesis.
For the MIT bag model, the specified mass and radius ranges correspond to those generated by bag constants within the stability window \eqref{eq:mit:bag_window}
that respect both instability of two-flavor quark matter and stability of three-flavor quark matter.
For the quark-meson model, the ranges are those obtained with varying $\sigma$-meson masses, but only the corresponding lowermost bag constants that respect instability of two-flavor quark matter.
\TODO{update hybrid with new results?}
}
\end{table}

In this thesis we have modeled quark stars using the MIT bag model and quark-meson model with two and three flavors,
and finally hybrid stars by combining the three-flavor quark-meson model and Akmal-Pandharipande-Ravenhall hadronic equation of state.
The resulting maximum masses and corresponding radii are summarized in \cref{tab:master_conclusion:results}.
With the quark-meson model we managed to create pure two-flavor quark stars with masses up to $M \leq 2.0 \, M_\odot$ and three-flavor quark stars up to $M \leq 1.8 \, M_\odot$.
Using $m_\sigma = \SI{600}{\mega\electronvolt}$ in the three-flavor quark-meson model,
we could also form a short range of hybrid stars up to $M \leq 1.9 \, M_\odot$ containing very small quark cores with mass $M_\text{core} \leq 0.02 M_\odot$,
but denser hybrid stars were unstable against radial perturbations due to discontinuous phase transition between the quark and hadron phases.
In short, we find it possible to model quark stars and hybrid stars right below the lower $2 \, M_\odot$-bound set by the pulsars PSR J1614$-$2230 and PSR J0348$+$0432.

The parameter space of the quark-meson model is particularly plagued by the ``ad-hoc'' $\sigma$-meson,
whose mass is known only within the large range $\SI{400}{\mega\electronvolt} \leq m_\sigma \leq \SI{550}{\mega\electronvolt}$.
Moreover, we could only fit greater masses $m_\sigma \geq \SI{600}{\mega\electronvolt}$ to prevent the ground state of the grand potential from vanishing in vacuum.
We nailed this problem down to the inconsistency of fitting parameters at tree-level to a potential that is calculated to one loop,
and repeated our calculation with a consistently fit potential to learn that we could still trust our results,
as if the inconsistently fit values of $m_\sigma$ corresponded to $\SI{200}{\mega\electronvolt}$ lower values.
Such a consistently fit grand potential is currently only found for the two-flavor model, and it would be useful to find one also for the three-flavor model.
In addition, we saw that chiral symmetry restoration occurred in a rapid crossover for $m_\sigma \geq \SI{800}{\mega\electronvolt}$,
but a discontinuous phase transition for $m_\sigma < \SI{800}{\mega\electronvolt}$.

In earlier work, \cite{ref:lsm3f_compact_stars} have modeled pure quark stars with a vector meson-extended three-flavor quark meson model,
and this work has since been incorporated into hybrid stars in \cite{ref:lsm3f_hybrid_stars}.
Quark stars have also been modeled with the Nambu-Jona-Lasinio model in \cite{ref:quark_star_njl} and generalized to hybrid stars in \cite{ref:hybrid_stars_njl}, for example.
Our results for both quark stars and hybrid stars are very comparable to theirs.

\TODO{compare quantitatively. also compare with \cite{ref:quark_hybrid_additional_ref}}

\section{Outlook}

In future work,
more effort should be put into studying hybrid stars by handling the phase transition in a more careful manner,
for example using interpolation techniques like those explained in \cite{ref:quark_star_review}.
More advanced effective models that describe confinement in a more sophisticated manner than the phenomenological bag constant,
such as Polyakov-loop extended quark-meson models (PQM models, see for example \cite{ref:pqm_2f,ref:pqm_3f,ref:master_folkestad}) and NJL models (PNJL models, see for example \cite{ref:pnjl_2f,ref:pnjl_3f,ref:pnjl_3f_zeroT}) could be used.
It would also be interesting to see the effects of modeling the color superconducting phase of the phase diagram in \cref{fig:qcd:phase_diagram}.
Finally, it would be useful to refine calculations beyond mean-field theory and
look for consistently fit grand potentials in three-flavor models similar to the one found for two flavors by \cite{ref:jo_lsm_consistent_chiral,ref:jo_lsm_consistent_physical}.
