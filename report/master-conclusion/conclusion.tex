\chapter{Conclusions and outlook}

\section{Conclusions}

\begin{table}[b!]
\centering
\caption{\label{tab:master_conclusion:results}%
Maximum masses and corresponding radii obtained using all models in this thesis summarized in \cref{fig:conclusion:mass-radius}.
}
{\setlength{\tabcolsep}{4pt} % affect only this table
\begin{tabular}{ c l l c c c }
	\toprule
	Flavors & Chapter & Model & Maximum masses & Corresponding radii                            \\
	\midrule
	%\Cref{chap:mit} & MIT bag model ($N_f\!=\!2$) & $1.76 \, M_\odot \leq M \leq 2.00 \, M_\odot$ & $\hphantom{1}\SI{9.70}{\kilo\meter} \leq R \leq \SI{10.83}{\kilo\meter}$ \\
	%\Cref{chap:mit} & MIT bag model ($N_f\!=\!3$) & $1.63 \, M_\odot \leq M \leq 1.85 \, M_\odot$ & $\hphantom{1}\SI{9.21}{\kilo\meter} \leq R \leq \SI{10.25}{\kilo\meter}$ \\
	%\Cref{chap:lsm2f} & Quark-meson model ($N_f\!=\!2$) & $1.77 \, M_\odot \leq M \leq 2.02 \, M_\odot$ & $\SI{10.91}{\kilo\meter} \leq R \leq \SI{10.98}{\kilo\meter}$ \\
	%\Cref{chap:lsm3f} & Quark-meson model ($N_f\!=\!3$) & $1.63 \, M_\odot \leq M \leq 1.81 \, M_\odot$ & $\SI{10.82}{\kilo\meter} \leq R \leq \SI{11.57}{\kilo\meter}$ \\
	%\Cref{chap:hybrid} & Hybrid model & $1.89 \, M_\odot \leq M \leq 2.06 \, M_\odot$ & $\SI{11.23}{\kilo\meter} \leq R \leq \SI{11.48}{\kilo\meter}$ \\
	$N_f=2$ & \Cref{chap:mit} & MIT bag model & $1.7 \, M_\odot \leq M \leq 2.0 \, M_\odot$ & $\hphantom{1}\SI{9.6}{\kilo\meter} \leq R \leq \SI{11.0}{\kilo\meter}$ \\
	$N_f=2$ & \Cref{chap:lsm2f} & Quark-meson model & $1.8 \, M_\odot \leq M \leq 2.0 \, M_\odot$ & $\SI{10.9}{\kilo\meter} \leq R \leq \SI{11.2}{\kilo\meter}$ \\
	$N_f=2$ & \Cref{chap:hybrid} & Hybrid model & $2.0 \, M_\odot \leq M \leq 2.1 \, M_\odot$ & $\SI{11.2}{\kilo\meter} \leq R \leq \SI{11.2}{\kilo\meter}$ \\
	\midrule
	$N_f=3$ & \Cref{chap:mit} & MIT bag model & $1.6 \, M_\odot \leq M \leq 1.9 \, M_\odot$ & $\hphantom{1}\SI{9.0}{\kilo\meter} \leq R \leq \SI{10.3}{\kilo\meter}$ \\
	$N_f=3$ & \Cref{chap:lsm3f} & Quark-meson model & $1.6 \, M_\odot \leq M \leq 1.8 \, M_\odot$ & $\SI{11.0}{\kilo\meter} \leq R \leq \SI{11.6}{\kilo\meter}$ \\
	$N_f=3$ & \Cref{chap:hybrid} & Hybrid model & $1.9 \, M_\odot \leq M \leq 2.1 \, M_\odot$ & $\SI{11.2}{\kilo\meter} \leq R \leq \SI{11.5}{\kilo\meter}$ \\
	\bottomrule
\end{tabular}}
\end{table}

\begin{figure}[t!]
\centering
\tikzsetnextfilename{summary-mass-radius}
\pgfplotsset{cycle list/Set1-9}
\begin{tikzpicture}
\newcommand\ploteverything{
\addplot [name path=J0748lo, draw=none, forget plot, domain=0:40] {2.08 - 0.07};
\addplot [name path=J0748hi, draw=none, forget plot, domain=0:40] {2.08 + 0.07};
\addplot [red, opacity=0.3, forget plot] fill between [of=J0748lo and J0748hi]; % \addlegendentry{PSR J0748$+$6620};
\node [anchor=west, red, font=\scriptsize] at (7.2, 2.08) {PSR J0748$+$6620};

\addplot [name path=J0348lo, draw=none, forget plot, domain=0:40] {2.01 - 0.04};
\addplot [name path=J0348hi, draw=none, forget plot, domain=0:40] {2.01 + 0.04};
\addplot [green, opacity=0.3, forget plot] fill between [of=J0348lo and J0348hi]; % \addlegendentry{PSR J0348$+$0432};
\node [anchor=west, green, font=\scriptsize] at (7.2, 2.01) {PSR J0348$+$0432};

\addplot [name path=J1614lo, draw=none, forget plot, domain=0:40] {1.91 - 0.02};
\addplot [name path=J1614hi, draw=none, forget plot, domain=0:40] {1.91 + 0.02};
\addplot [yellow, opacity=0.3, forget plot] fill between [of=J1614lo and J1614hi]; % \addlegendentry{PSR J1614$-$2230};
\node [anchor=west, yellow, font=\scriptsize] at (7.2, 1.91) {PSR J1614$-$2230};

\addplot+ [solid, index of colormap={8 of Set1-9}, draw=none] table [x=R, y=M] {../code/data/LSM3F_APR/stars_hadron.dat}; \addlegendentry{APR}; % hack: put BEFORE hybrids in legend
\addplot+ [white] {2.6}; \addlegendentry{};
\addplot+ [white] {2.6}; \addlegendentry{};
\addplot+ [white] {2.6}; \addlegendentry{};
\addplot+ [white] {2.6}; \addlegendentry{};
\addplot+ [white] {2.6}; \addlegendentry{};

\addplot+ [densely dashed, opacity=0.6, index of colormap={3 of Set1-9}] table [x=R, y=M] {../code/data/MIT2F/stars_sigma_800_B14_155.dat}; \addlegendentry{$\text{MIT}_2 \, (\smash{B^{\frac14}}\!\!=\!\SI{155}{\mega\electronvolt})$};
\addplot+ [densely dashed, opacity=0.8, index of colormap={3 of Set1-9}] table [x=R, y=M] {../code/data/MIT2F/stars_sigma_800_B14_150.dat}; \addlegendentry{$\text{MIT}_2 \, (\smash{B^{\frac14}}\!\!=\!\SI{150}{\mega\electronvolt})$};
\addplot+ [densely dashed, opacity=1.0, index of colormap={3 of Set1-9}] table [x=R, y=M] {../code/data/MIT2F/stars_sigma_800_B14_145.dat}; \addlegendentry{$\text{MIT}_2 \, (\smash{B^{\frac14}}\!\!=\!\SI{145}{\mega\electronvolt})$};
\addplot+ [solid,          opacity=0.6, index of colormap={3 of Set1-9}] table [x=R, y=M] {../code/data/MIT3F/stars_sigma_800_B14_155.dat}; \addlegendentry{$\text{MIT}_3 \, (\smash{B^{\frac14}}\!\!=\!\SI{155}{\mega\electronvolt})$};
\addplot+ [solid,          opacity=0.8, index of colormap={3 of Set1-9}] table [x=R, y=M] {../code/data/MIT3F/stars_sigma_800_B14_150.dat}; \addlegendentry{$\text{MIT}_3 \, (\smash{B^{\frac14}}\!\!=\!\SI{150}{\mega\electronvolt})$};
\addplot+ [solid,          opacity=1.0, index of colormap={3 of Set1-9}] table [x=R, y=M] {../code/data/MIT3F/stars_sigma_800_B14_145.dat}; \addlegendentry{$\text{MIT}_3 \, (\smash{B^{\frac14}}\!\!=\!\SI{145}{\mega\electronvolt})$};

\addplot+ [densely dashed, opacity=0.6, index of colormap={4 of Set1-9}] table [x=R, y=M] {../code/data/LSM2F/stars_sigma_600_B14_111.dat}; \addlegendentry{$\text{QM}_2 \, (m_\sigma\!\!=\!\SI{600}{\mega\electronvolt})\!$};
\addplot+ [densely dashed, opacity=0.8, index of colormap={4 of Set1-9}] table [x=R, y=M] {../code/data/LSM2F/stars_sigma_700_B14_68.dat};  \addlegendentry{$\text{QM}_2 \, (m_\sigma\!\!=\!\SI{700}{\mega\electronvolt})\!$};
\addplot+ [densely dashed, opacity=1.0, index of colormap={4 of Set1-9}] table [x=R, y=M] {../code/data/LSM2F/stars_sigma_800_B14_27.dat};  \addlegendentry{$\text{QM}_2 \, (m_\sigma\!\!=\!\SI{800}{\mega\electronvolt})\!$};
\addplot+ [solid,          opacity=0.6, index of colormap={4 of Set1-9}] table [x=R, y=M] {../code/data/LSM3F/stars_sigma_600_B14_111.dat}; \addlegendentry{$\text{QM}_3 \, (m_\sigma\!\!=\!\SI{600}{\mega\electronvolt})\!$};
\addplot+ [solid,          opacity=0.8, index of colormap={4 of Set1-9}] table [x=R, y=M] {../code/data/LSM3F/stars_sigma_700_B14_68.dat};  \addlegendentry{$\text{QM}_3 \, (m_\sigma\!\!=\!\SI{700}{\mega\electronvolt})\!$};
\addplot+ [solid,          opacity=1.0, index of colormap={4 of Set1-9}] table [x=R, y=M] {../code/data/LSM3F/stars_sigma_800_B14_27.dat};  \addlegendentry{$\text{QM}_3 \, (m_\sigma\!\!=\!\SI{800}{\mega\electronvolt})\!$};

\addplot+ [densely dashed, opacity=0.6, index of colormap={1 of Set1-9}] table [x=R, y=M] {../code/data/LSM2F_APR/stars_sigma_600_B14_111.dat}; \addlegendentry{$\text{QM}_2\!\!+\!\!\text{APR} \, (m_\sigma\!\!=\!\SI{600}{\mega\electronvolt})$};
\addplot+ [densely dashed, opacity=0.8, index of colormap={1 of Set1-9}] table [x=R, y=M] {../code/data/LSM2F_APR/stars_sigma_700_B14_68.dat};  \addlegendentry{$\text{QM}_2\!\!+\!\!\text{APR} \, (m_\sigma\!\!=\!\SI{700}{\mega\electronvolt})$};
\addplot+ [densely dashed, opacity=1.0, index of colormap={1 of Set1-9}] table [x=R, y=M] {../code/data/LSM2F_APR/stars_sigma_800_B14_27.dat};  \addlegendentry{$\text{QM}_2\!\!+\!\!\text{APR} \, (m_\sigma\!\!=\!\SI{800}{\mega\electronvolt})$};
\addplot+ [solid,          opacity=0.6, index of colormap={1 of Set1-9}] table [x=R, y=M] {../code/data/LSM3F_APR/stars_sigma_600_B14_111.dat}; \addlegendentry{$\text{QM}_3\!\!+\!\!\text{APR} \, (m_\sigma\!\!=\!\SI{600}{\mega\electronvolt})$};
\addplot+ [solid,          opacity=0.8, index of colormap={1 of Set1-9}] table [x=R, y=M] {../code/data/LSM3F_APR/stars_sigma_700_B14_68.dat};  \addlegendentry{$\text{QM}_3\!\!+\!\!\text{APR} \, (m_\sigma\!\!=\!\SI{700}{\mega\electronvolt})$};
\addplot+ [solid,          opacity=1.0, index of colormap={1 of Set1-9}] table [x=R, y=M] {../code/data/LSM3F_APR/stars_sigma_800_B14_27.dat};  \addlegendentry{$\text{QM}_3\!\!+\!\!\text{APR} \, (m_\sigma\!\!=\!\SI{800}{\mega\electronvolt})$};

\addplot+ [solid, index of colormap={8 of Set1-9}, forget plot] table [x=R, y=M] {../code/data/LSM3F_APR/stars_hadron.dat}; % hack: draw OVER hybrids
}
\begin{axis}[
	width=15.5cm,
	height=10.5cm,
	xlabel={$R \, / \, \si{\kilo\meter}$ },
	ylabel={$M \, / \, M_\odot$}, %title={Mass-radius diagram for 2-flavor quark stars }, title style={yshift=2.0cm},
	xmin=7, xmax=16, ymin=0.0, ymax=2.5, xtick distance=1, ytick distance=0.5, minor x tick num=9, minor y tick num=4,
	grid=both, minor grid style={ultra thin, gray!10!white},
	legend transposed, legend columns=6, legend cell align = left, legend style={font=\small, anchor=north, at={(0.5, -0.15)}},
	title = {Quark and hybrid star mass-radius relations for all models},
	restrict y to domain=0.2:3.0,
];
\ploteverything
\coordinate (zoomplot) at (15.9, 2.4);
\end{axis}
\node [anchor=north east] at (zoomplot) {
	\begin{tikzpicture}[trim axis left, trim axis right, baseline]
	\begin{axis}[
		axis background/.style={fill=white},
		width=5cm, height=6cm,
		xmin=11, xmax=12, ymin=1.85, ymax=2.15, xtick distance=1.0, ytick distance=0.1, minor x tick num=9, minor y tick num=9, %extra y ticks={1.9, 2.1},
		grid=both, minor grid style={ultra thin, gray!10!white},
	]
	\ploteverything
	\legend{};
	\end{axis}
	\end{tikzpicture}
};
\end{tikzpicture}
\caption{\label{fig:conclusion:mass-radius}%
	Summary of all mass-radius relations obtained in this thesis:
	neutron stars with the hadronic Akmal-Pandharipande-Ravenhall equation of state (APR);
	quark stars with the two-flavor and three-flavor MIT bag models ($\text{MIT}_2$ and $\text{MIT}_3$)
	and quark-meson models ($\text{QM}_2$ and $\text{QM}_3$);
	and hybrid stars joining $\text{QM}_2+\text{APR}$ and $\text{QM}_3+\text{APR}$.
	In the $\text{MIT}_2$ and $\text{MIT}_3$ models, we have used bag constants covering the bag window \eqref{eq:mit:bag_window}.
	In the $\text{QM}_2$ and $\text{QM}_3$ models, we vary $\SI{600}{\mega\electronvolt} \leq m_\sigma \leq \SI{800}{\mega\electronvolt}$
	and use the corresponding lowest bag constants above the lower bounds \eqref{eq:lsm:bag_lower_bound},
	as greater bag constants only generate less massive stars.
	For comparison, the colored bands show measured masses
	of the heavy pulsars \textcolor{green}{PSR J0348$+$0432}, \textcolor{yellow}{PSR J1614$-$2230} and \textcolor{red}{PSR J0740$+$6620}
	from \cite{ref:antoniadis,ref:arzoumanian,ref:fonseca}.
	For clearer presentations of the results with each model,
	please look at \cref{fig:mit:mass_radius,fig:lsm:2-flavor-mass-radius,fig:lsm:3-flavor-mass-radius,fig:hybrid:mass-radius}.
	\TODO{JO se}
}
\end{figure}

In this thesis we have modeled quark stars using the MIT bag model and the quark-meson model with two and three flavors,
and finally hybrid stars by combining the quark-meson model and hadronic Akmal-Pandharipande-Ravenhall equation of state.
The resulting mass-radius relations are summarized in \cref{fig:conclusion:mass-radius},
and their maximum masses and corresponding radii are gathered in \cref{tab:master_conclusion:results}.
In particular, with the quark-meson model we could create pure quark stars with two-flavor up to $M \leq 2.0 \, M_\odot$ and with three-flavors up to $M \leq 1.8 \, M_\odot$.
We could also form short ranges of stable hybrid stars up to $M \leq 2.1 \, M_\odot$.
They contained very small quark cores no more massive than $0.12 M_\odot$ and $0.02 M_\odot$ with two and three flavors, respectively.
This resonates with the general view in \cite{ref:quark_star_review} that hybrid star quark cores are generally small.
We saw that three-flavor quark and hybrid stars are generally less massive than two-flavor stars due to their softer equations of state.
Denser hybrid stars beyond the maximum mass are unstable against radial perturbations due to the discontinuous phase transition between the quark and hadron phases.
We find it possible to model hybrid stars with small quark cores around the recent mass observations of the heavy pulsars PSR J1614$-$2230, PSR J0348$+$0432 and PSR J0748$+$6620 around and above $2 M_\odot$,
and pure quark stars of slightly lower mass.

The fact that hybrid stars seem to exhibit very small quark cores
make it very easy to mistake them for neutron stars.
It is difficult to decide whether a pulsar has a tiny quark core based on measurements made lightyears away from it.
With future advances in our theoretical understanding and observational techniques,
many observed neutron stars could really turn out to be hybrid stars.

The parameter space of the quark-meson model is particularly plagued by the ``ad-hoc'' $\sigma$-meson,
whose mass is only known to lie within $\SI{400}{\mega\electronvolt} \leq m_\sigma \leq \SI{550}{\mega\electronvolt}$.
Depending on its value, we saw that chiral symmetry restoration occurred in a rapid crossover for $m_\sigma \geq \SI{800}{\mega\electronvolt}$,
but a discontinuous phase transition for $m_\sigma < \SI{800}{\mega\electronvolt}$.
Moreover, we could only fit masses $m_\sigma \geq \SI{600}{\mega\electronvolt}$ that exceed the measurements
to prevent the ground state of the grand potential from vanishing in vacuum.
We nailed this problem down to the inconsistency of fitting parameters at tree-level to a potential that is calculated to one loop.
Repeating our calculation with a consistently fit potential in the large-$N_c$ limit found by Adhikari and others,
we learned that we could still trust our results,
as if inconsistently fit masses $\SI{600}{\mega\electronvolt} \leq m_\sigma \leq \SI{800}{\mega\electronvolt}$
corresponds to consistently fit masses $\SI{400}{\mega\electronvolt} \leq m_\sigma \leq \SI{600}{\mega\electronvolt}$.
Such a consistently fit grand potential has only been found for the two-flavor model.

In earlier work, \cite{ref:lsm3f_compact_stars} have modeled pure quark stars with a vector meson-extended three-flavor quark-meson model,
and this was incorporated into hybrid stars in \cite{ref:lsm3f_hybrid_stars}.
With the vector meson interaction turned off and using parameters comparable to ours,
they find hybrid stars with similar maximum masses $M \lesssim 1.9 M_\odot$ in \cite[figure 8]{ref:lsm3f_hybrid_stars}. % only B differs?
Quark stars have also been modeled with the Nambu-Jona-Lasinio model in \cite{ref:quark_star_njl} and generalized to hybrid stars in \cite{ref:hybrid_stars_njl}, for example.
In \cite[figure 3]{ref:hybrid_stars_njl}, they also find maximum hybrid star masses $1.9 M_\odot \leq M \leq 2.1 M_\odot$ with reasonable parameter choices.
Even \cite{ref:quark_hybrid_additional_ref} find hybrid stars in the band $2.0 M_\odot \leq M \leq 2.1 M_\odot$
using a non-perturbative functional renormalization group approach with the quark-meson model.
Despite using different models for both the quark and hadronic phases,
our results are in good agreement with these works.

\section{Outlook}

A natural extension of our work would be to
explore the parameter space more thoroughly.
Our philosophy has been to keep almost all parameters fixed,
but vary the most uncertain parameter $m_\sigma$ and focus on the corresponding lowest stability-respecting $B$
because it generates greater maximum masses.
A more detailed treatment could also vary the quark masses,
other meson masses and use larger bag constants.
For example, \cite{ref:lsm3f_compact_stars}
find that the maximum mass increases with the quark masses $m_u=m_d$
with the variation of our model mentioned above.

Our treatment of bosons to tree-level and fermions to one loop was inconsistent in terms of loops,
but consistent in the large-$N_c$ approximation scheme.
It would be useful to refine calculations beyond bosonic mean fields.
We also saw that fitting parameters at tree-level to a one-loop grand potential was inconsistent
and performed a consistent one-loop fit in the two-flavor model using the results of \cite{ref:jo_lsm_consistent_chiral,ref:jo_lsm_consistent_physical}.
Such a consistently fit grand potential has not been found for the three-flavor model,
so it would be interesting to find one and see if the ability to fit experimental values of $m_\sigma$ carries over from the two-flavor model.

Different and more sophisticated effective models could also be examined.
For example, confinement could be incorporated with Polyakov-loop extended quark-meson models (PQM models, see for example \cite{ref:pqm_2f,ref:pqm_3f,ref:master_folkestad}) and NJL models (PNJL models, see for example \cite{ref:pnjl_2f,ref:pnjl_3f,ref:pnjl_3f_zeroT}).
It would also be interesting to see the effects of modeling the color-superconducting phase of the phase diagram in \cref{fig:qcd:phase_diagram}.

More effort could also be put into handling the phase transition in the hybrid stars more carefully.
For example, one could use interpolation techniques described in \cite{ref:quark_star_review}
near the intersection point of the transition,
where both the quark and hadronic equations of state are unreliable.
The interpolation is severely constrained by several physical requirements,
such as the speed of sound $v = c \sqrt{\odv{P}/{\epsilon}}$ not exceeding that of light.

One could also make more accurate calculations at nonzero temperature,
investigate effects of non-local charge neutrality,
and consider rotating stars that are more representative of observed pulsars than the static Tolman-Oppenheimer-Volkoff equation by using a different spacetime metric.
