\chapter{\texorpdfstring{\textls[-22]{The Three-Flavor Quark-Meson Model}}{The Three-Flavor Quark-Meson Model}}
\label{chap:lsm3f}

With three flavors, the effective low-energy degrees of freedom of quantum chromodynamics are the scalar and pseudoscalar mesons. %\cite{ref:lsm3f_details}
In this chapter we will study the generalization of the quark-meson model in \cref{chap:lsm2f} to three flavors,
thereby throwing the strange quark into the mix.
Most features of this model behave as natural and somewhat more complex generalizations of those in the two-flavor model.
For example, we will handle the strange quark condensate in parallel with the common up and down quark condensate.
Whereas there is a unique way of fixing the parameters in the two-flavor model, however,
we will see that the three-flavor model presents us with a multitude of ways of doing it,
so that some experimental values must be predicted rather than fitted.

Three-flavor models are particularly interesting for quark stars.
While two-flavor quark matter is unstable with respect to hadronic matter,
pure quark stars consisting only of strange quark matter would be stable and more likely to exist
\emph{if} the strange matter hypothesis is true.
Due to the heavier mass of the strange quark,
we expect its effects to become important in the mass-radius diagram
only for stars that have a high central pressure and thus large chemical potentials.

\textit{This chapter is inspired by references \cite{ref:lsm3f} and \cite{ref:lsm3f_details}.}

\section{Lagrangian, vacuum and symmetries}

The Lagrangian density of the three-flavor quark-meson model is \cite{ref:lsm3f,ref:lsm3f_details}
\begin{equation}
	\lagr = \bar{q} \Big[ i \slashed\partial + \mu \gamma^0 - g \big(\sigma_a + i \gamma^5 \pi_a\big) T_a \Big] q + \trace\Big[(\partial_\mu \phi)^\dagger (\partial^\mu \phi)\Big] - \pot(\sigma, \pi)
\label{eq:lsm:lagrangian3f}
\end{equation}
with the meson potential
\begin{equation}
	\pot(\sigma, \pi) = m^2 \trace\Big[\phi^\dagger \phi\Big] + \lambda_1 \Big[\trace(\phi^\dagger \phi)\Big]^2 + \lambda_2 \trace\Big[(\phi^\dagger \phi)^2\Big] - \trace\Big[H(\phi+\phi^\dagger)\Big].
\label{eq:lsm:potential3f}
\end{equation}
The quark fields $q$ and chemical potential matrix $\mu = \diag(\mu_u,\mu_d,\mu_s)$ now have the $N_f=3$ flavors $\{u,d,s\}$,
while $\sigma_a$ and $\pi_a$ are members of the scalar ($J^P=0^+$) and pseudoscalar ($J^P=0^-$) meson nonets,
packed into the $N_f \times N_f$ meson matrix $\phi = \phi_a T_a = (\sigma_a + i \pi_a) T_a$.
The eight $SU(3)$ generators $T_a = \lambda_a/2$ in flavor space
are extended with the identity in the common normalization $\trace[T_a T_b] = \delta_{ab}/2$ of the Gell-Mann matrices
\begin{equation}
\begin{NiceMatrixBlock}[auto-columns-width]
\setlength{\arraycolsep}{0pt}
\NiceMatrixOptions{cell-space-limits = 5pt}
\begin{aligned}
	\lambda_0 &= \begin{bNiceMatrix} \smash{\sqrt{\frac{2}{3}}} &  0 &  0 \\ 0 &  \smash{\sqrt{\frac{2}{3}}} &  0 \\ 0 & 0 &  \smash{\sqrt{\frac{2}{3}}} \end{bNiceMatrix}, &\quad&
	\lambda_1 &= \begin{bNiceMatrix}                          0 &  1 &  0 \\ 1 &                           0 &  0 \\ 0 & 0 &                           0 \end{bNiceMatrix}, &\quad&
	\lambda_2 &= \begin{bNiceMatrix}                          0 & -i &  0 \\ i &                           0 &  0 \\ 0 & 0 &                           0 \end{bNiceMatrix}, \\
	\lambda_3 &= \begin{bNiceMatrix}                          1 &  0 &  0 \\ 0 &                          -1 &  0 \\ 0 & 0 &                           0 \end{bNiceMatrix}, &\quad&
	\lambda_4 &= \begin{bNiceMatrix}                          0 &  0 &  1 \\ 0 &                           0 &  0 \\ 1 & 0 &                           0 \end{bNiceMatrix}, &\quad&
	\lambda_5 &= \begin{bNiceMatrix}                          0 &  0 & -i \\ 0 &                           0 &  0 \\ i & 0 &                           0 \end{bNiceMatrix}, \\
	\lambda_6 &= \begin{bNiceMatrix}                          0 &  0 &  0 \\ 0 &                           0 &  1 \\ 0 & 1 &                           0 \end{bNiceMatrix}, &\quad&
	\lambda_7 &= \begin{bNiceMatrix}                          0 &  0 &  0 \\ 0 &                           0 & -i \\ 0 & i &                           0 \end{bNiceMatrix}, &\quad&
	\lambda_8 &= \begin{bNiceMatrix} \smash{\frac{1}{\sqrt{3}}} &  0 &  0 \\ 0 &  \smash{\frac{1}{\sqrt{3}}} &  0 \\ 0 & 0 & -\smash{\frac{2}{\sqrt{3}}} \end{bNiceMatrix}.
\end{aligned}
\end{NiceMatrixBlock}
\label{eq:lsm:gell_mann_matrices}
\end{equation}
%which are normalized to $\trace[T_a T_b] = \delta_{ab}/2$ and $\trace[\lambda_a \lamda_b] = 2 \delta_{ab}$.
Like before, the Lagrangian has $U(1)_V \times U(1)_A \times SU(N_f)_L \times SU(N_f)_R$ symmetry
in the absence of the explicit symmetry breakers $h_a$ in the matrix $H = h_a T_a$.
Moreover, there are now \emph{two} quartic couplings $\lambda_1$ and $\lambda_2$,
not to be confused with the Gell-Mann matrices \eqref{eq:lsm:gell_mann_matrices}.
This can be understood from the \emph{Hamilton-Caley theorem} \cite[equation (1) and (2)]{ref:hamilton_caley},
which links the $N$ traces $\trace A, \ldots, \trace A^N$ of any $N \times N$ matrix $A$.
With three flavors we can therefore form \emph{two} independent coupling terms
$\lambda_1 \trace[(\phi^\dagger \phi)]^2$ and $\lambda_2 \trace[(\phi^\dagger \phi)^2]$ up to quadratic order%
\footnote{That precisely fourth order is the target implies that the theory is renormalizable,
in the sense that divergences can be removed by shifting the couplings already in the Lagrangian with counterterms.
Divergences from, say, sixth-order terms in the Lagrangian could only have been removed by counterterms in higher-order terms
and would make the theory non-renormalizable.}
in the fields, while only one can be constructed with two flavors.
Every other feature of the Lagrangian is identical to or a natural generalization of that in the corresponding two-flavor Lagrangian \eqref{eq:lsm:lagrangian}.

This general form of the model has thirteen undetermined parameters $g$, $m^2$, $\lambda_1$, $\lambda_2$ and $\{h_0,\,\ldots,\,h_8\}$.
We continue to neglect pion condensation by setting $\avg{\pi_a}=0$,
so a nonzero symmetry breaker $h_a$ creates a non-vanishing vacuum expectation value of the corresponding scalar field $\avg{\sigma_a}$,
which in turn creates a nonzero quark condensate $\avg{\bar{q} \, T_a q}$ in vacuum through the Yukawa interaction in the Lagrangian \eqref{eq:lsm:lagrangian3f}.
In particular, this vacuum expectation value should have the same vanishing electrical charge as the vacuum.
We therefore only allow for flavor-like charge-neutral condensates such as $\avg{\bar{u} u}$ in the vacuum,
and no mixed-flavor charged condensates like $\avg{\bar{u} d}$.
To accomplish this, we set all symmetry breakers to zero \emph{except} those that correspond to diagonal flavor-space matrices \eqref{eq:lsm:gell_mann_matrices},
or $T_a$,
namely $\{h_0,h_3,h_8\} \neq 0$.

Using different combinations of the three remaining nonzero symmetry breakers $\{h_0,h_3,h_8\}$,
one can study different symmetry breaking patterns among the $u$, $d$ and $s$ quarks.
By setting only $h_0 \neq 0$, we see that the first Gell-Mann matrix $\lambda_0$ weighs all three quarks equally with common mass $m_u=m_d=m_s$.
If we also unlock $h_8 \neq 0$, the last matrix $\lambda_8$ separates the strange and non-strange quarks with $m_u = m_d \neq m_s$.
Including all three symmetry breakers with $h_3 \neq 0$, too, the fourth matrix $\lambda_3$ also distinguishes the non-strange quarks and treats all flavors with separate masses $m_u \neq m_d \neq m_s$.
These different symmetry breaking patterns are discussed in more detail in \cite[section III]{ref:lsm3f_details}.
Since we treated up and down quarks with degenerate mass in \cref{chap:lsm2f},
we will set $h_3 = 0$ and keep $\{h_0,h_8\} \neq 0$ to account for the heavier strange quark separately from the non-strange quarks.
This leaves the six unknown parameters $g$, $m^2$, $\lambda_1$, $\lambda_2$, $h_0$ and $h_8$ that will later be fit to as many experimental values.

The explicit symmetry breakers $\{h_0,h_8\}\neq 0$ dig a global minimum for the vacuum at
\begin{equation}
	\sigma_a = \avg{\sigma_a} \quad \text{and} \quad \pi_a = \avg{\pi_a} = 0, \quad \text{where only $\avg{\sigma_0} \neq 0$ and $\avg{\sigma_8} \neq 0$ are nonzero.}
\label{eq:lsm:ground_state_3f}
\end{equation}
Like before, we jump down into the hole \eqref{eq:lsm:ground_state_3f} and study the quantum fluctuations $\tilde{\sigma}_a$ and $\tilde{\pi}_a$ of the meson fields by writing
\begin{equation}
	\sigma_a = \avg{\sigma_a} + \tilde{\sigma}_a
	\qquad \text{and} \qquad
	\pi_a = \avg{\pi_a} + \tilde{\pi}_a.
\end{equation}

We will shortly come back to the precise location of the vacuum, but first examine how it couples to the quarks.
Coupled to the nonzero expectation values \eqref{eq:lsm:ground_state_3f}, the Yukawa term in the Lagrangian \eqref{eq:lsm:lagrangian3f} reads
\begin{equation}
\begin{split}
	  & -g \bar{q} \big( \avg{\sigma_0} T_0 + \avg{\sigma_8} T_8 \big) q \\
	= & -\frac{g}{2} \begin{bNiceMatrix} \vphantom{\sqrt{\frac13}} u^\dagger \gamma^0 \\ \vphantom{\sqrt{\frac13}} d^\dagger \gamma^0 \\ \vphantom{\sqrt{\frac13}} s^\dagger \gamma^0 \end{bNiceMatrix}^\mathsf{T} \begin{bNiceMatrix} \sqrt{\frac23} \avg{\sigma_0} + \frac{1}{\sqrt{3}} \avg{\sigma_8} & 0 & 0 \\ 0 & \sqrt{\frac23} \avg{\sigma_0} + \frac{1}{\sqrt{3}} \avg{\sigma_8} & 0 \\ 0 & 0 & \sqrt{\frac23} \avg{\sigma_0} - \frac{2}{\sqrt{3}} \avg{\sigma_8} \end{bNiceMatrix} \begin{bNiceMatrix} \vphantom{\sqrt{\frac13}} u \\ \vphantom{\sqrt{\frac13}} d \\ \vphantom{\sqrt{\frac13}} s \end{bNiceMatrix} . \\
\end{split}
\label{eq:lsm:yukawa3f}
\end{equation}
In particular, it mixes $\sigma_0$ and $\sigma_8$ interactions with all three quarks.
For conceptual cleanliness,
we seek a transformation from the mixed $\sigma_0\text{-}\sigma_8$-basis to a strangeness-separated $\sigma_x\text{-}\sigma_y$-basis
in which the up and down quarks couple only to $\sigma_x$ and the strange quark only to $\sigma_y$.
This is accomplished by
\begin{equation}
	\begin{bmatrix} \sigma_x \\ \sigma_y \end{bmatrix} = M \begin{bmatrix} \sigma_0 \\ \sigma_8 \end{bmatrix},
	%\quad \text{or} \quad
	%\begin{bmatrix} \sigma_0 \\ \sigma_8 \end{bmatrix} = M \begin{bmatrix} \sigma_x \\ \sigma_y \end{bmatrix},
	\qquad \text{where} \qquad
	M = M^{-1} = M^\mathsf{T} = \frac{1}{\sqrt{3}} \begin{bmatrix} \sqrt{2} & 1 \\ 1 & -\sqrt{2} \end{bmatrix}.
\label{eq:lsm:strange_basis}
\end{equation}
We also define the transformed symmetry breakers 
$[h_x,h_y]^\mathsf{T} = M [h_0,h_8]^\mathsf{T}$
%$h_x$ and $h_y$ from $h_0$ and $h_8$ 
using the same transformation.
In the new basis \eqref{eq:lsm:strange_basis}, the Yukawa coupling \eqref{eq:lsm:yukawa3f} takes our sought-after form
\begin{equation}
	-\smashoperator{\sum_{f=\{u,d,s\}}} m_f \bar{q}_f q_f
	\quad \text{with quark masses} \quad
	m_u = m_d = m_x = \frac{g \avg{\sigma_x}}{2}
	\quad \text{and} \quad
	m_s = m_y = \frac{g \avg{\sigma_y}}{\sqrt{2}}.
\label{eq:lsm:quark_masses_3f}
\end{equation}

Finally, we examine how the meson potential \eqref{eq:lsm:potential3f} behaves in and around the minimum \eqref{eq:lsm:ground_state_3f}
by looking for its mass-generating expansion
\begin{equation}
	\pot(\sigma,\pi) \taylor \pot(\avg{\sigma},\avg{\pi}) + \frac12 \big(m^2_{\sigma\sigma}\big)_{ab} \tilde{\sigma}_a \tilde{\sigma}_b + \frac12 \big(m^2_{\pi\pi}\big)_{ab} \tilde{\pi}_a \tilde{\pi}_b.
\label{eq:lsm3f:potential_before_diagonalization}
\end{equation}
Like in \cref{chap:lsm2f},
we will use $\pot(\avg{\sigma},\avg{\pi})$ in the grand potential,
relate $\pdv{\pot}/{\sigma_{x}}=\pdv{\pot}/{\sigma_{y}}=0$
to the minima $\{\avg{\sigma_x},\avg{\sigma_y}\}$ and $\{h_x,h_y\}$,
and fit the (squared) curvature masses
$\pdv{\pot}/{\sigma_a, \sigma_b} = \smash{\big(m^2_{\sigma\sigma}\big)_{ab}}$
and
$\pdv{\pot}/{\pi_a, \pi_b} = \smash{\big(m^2_{\pi\pi}\big)_{ab}}$
to particle masses.
To do so, we need the flavor-space traces
\begin{subequations}
\begin{align}
	\trace\big[\phi^\dagger \phi\big]     &= (\sigma_a - i \pi_a) (\sigma_b + i \pi_b) \trace \big[ T_a T_b \big] = \frac12 \big(\sigma_a^2+\pi_a^2\big) , \\
	\trace\big[(\phi^\dagger \phi)^2\big] &= (\sigma_a - i \pi_a) (\sigma_b + i \pi_b) (\sigma_a - i \pi_c) (\sigma_b + i \pi_d) \trace \big[ T_a T_b T_c T_d \big], \\
	\trace\big[H(\phi+\phi^\dagger)\big]  &= 2 h_a \sigma_b \trace \big[ T_a T_b \big] = \vphantom{\frac12} h_a \sigma_a . \label{eq:lsm3f:traces_hphi}
\end{align}%
\label{eq:lsm3f:traces}%
\end{subequations}%
The quadruple product traces $\trace[T_a T_b T_c T_d]$ are more challenging than the double product traces $\trace[T_a T_b] = \delta_{ab}/2$.
If one figures out the structure constants $f_{abc}$ and $d_{abc}$
defined by the commutators $[T_a,T_b]=i f_{abc} T_c$ and anti-commutators $\{T_a,T_b\}=d_{abc} T_c$,
these traces can be done with pen, paper and painkillers
by bootstrapping traces of the Gell-Mann matrices like
\begin{subequations}
\begin{align}
	\trace\Big[T_a T_b\Big] &= \frac{1}{2} \delta_{ab} \label{eq:lsm3f:trace2}, \\
	\trace\Big[T_a T_b T_c\Big] &= \frac{1}{2} \trace\Big[T_a\Big(\overbrace{[T_b,T_c]}^{\smash{i f_{bcd} T_d}}+\overbrace{\{T_b,T_c\}}^{\smash{d_{bcd} T_d}}\Big)\Big]
	                             %= \frac{1}{2} \Big( i f_{bcd} + d_{bcd} \Big) \trace\Big[T_a T_d\Big]
	                             \equalexplabove{\smash[t]{use \eqref{eq:lsm3f:trace2}}} \frac{1}{2^2} \Big( i f_{bca} + d_{bca} \Big), \label{eq:lsm3f:trace3} \\
	\trace\Big[T_a T_b T_c T_d\Big] &= \frac{1}{2} \trace\Big[T_a T_b \Big( \underbrace{[T_c,T_d]}_{\smash{i f_{cde} T_e}} + \underbrace{\{T_c,T_d\}}_{\smash{d_{cde} T_e}} \Big) \Big]
	                                 %= \frac{1}{2} \Big( i f_{cde} + d_{cde} \Big) \trace\Big[T_a T_b T_e\Big]
	                                 \equalexplbelow{\smash[b]{use \eqref{eq:lsm3f:trace3}}} \frac{1}{2^3} \Big( i f_{cde} + d_{cde} \Big) \Big( i f_{bea} + d_{bea} \Big). \label{eq:lsm3f:trace4}
\end{align}
\end{subequations}
However, with our explicit knowledge \eqref{eq:lsm:gell_mann_matrices} of the matrices,
it is easier and less error-prone to simply calculate all the traces by brute force with a symbolic computer program.
Using the program in \cref{chap:lsm3fpotential},
we create a long explicit representation of $\pot(\sigma,\pi)$.
Evaluated in the minimum \eqref{eq:lsm:ground_state_3f}, its value is
\begin{equation}
	\pot(\avg{\sigma},\avg{\pi}) = \frac{m^2}{2} \Big[ \avg{\sigma_x}^2 + \avg{\sigma_y}^2 \Big] + \frac{\lambda_1}{4} \Big[ \avg{\sigma_x}^2 + \avg{\sigma_y}^2 \Big]^2 + \frac{\lambda_2}{8} \Big[ \avg{\sigma_x}^4 + 2 \avg{\sigma_y}^4 \Big] - h_x \avg{\sigma_x} - h_y \avg{\sigma_y}.
\label{eq:lsm:potential_tree_3f}
\end{equation}
The locations of the nonzero minima $\avg{\sigma_x}$ and $\avg{\sigma_y}$ are determined
by taking first derivatives of the explicit representation before evaluating it in the minimum \eqref{eq:lsm:ground_state_3f}.
We find that all $\pdv{\pot}/{\pi_a}=0$
and $\pdv{\pot}/{\sigma_a} = -h_a = 0$ for $a \neq \{0,8\}$ vanish by themselves,
while we demand
\begin{subequations}
\begin{align}
	0 &= \pdv{\pot}{\sigma_x} = \avg{\sigma_x} \Big[ m^2 + \lambda_1 \Big( \avg{\sigma_x}^2 + \avg{\sigma_y}^2 \Big) + \frac{\lambda_2}{2} \avg{\sigma_x}^2 \Big] - h_x, \\
	0 &= \pdv{\pot}{\sigma_y} = \avg{\sigma_y} \Big[ m^2 + \lambda_1 \Big( \avg{\sigma_x}^2 + \avg{\sigma_y}^2 \Big) + \lambda_2 \avg{\sigma_y}^2 \Big] - h_y.
\end{align}
\label{eq:lsm3f:symmetry_breakers}%
\end{subequations}

Things get messier and our computational approach really comes in handy when evaluating the mass-generating second derivatives
\begin{equation}
	\big(m^2_{\sigma\sigma}\big)_{ab} = \pdv{\pot}{\sigma_a, \sigma_b}
	\qquad \text{and} \qquad
	\big(m^2_{\pi\pi}\big)_{ab}       = \pdv{\pot}{\pi_a, \pi_b}
\end{equation}
in the minimum \eqref{eq:lsm:ground_state_3f}.
All mixed derivatives $\pdv{\pot}/{\sigma_a,\pi_b} = 0$ automatically vanish.
Grouped by common values, the nonzero entries of the scalar mass matrix are
\begin{subequations}
\begin{equation}
\begin{NiceMatrixBlock}[]
\NiceMatrixOptions{cell-space-limits=1pt}
% smash all fracs to make consistent spacing between row entries
\begin{array}{l @{\,} >{\displaystyle}l}
	\begin{NiceArray}{c}
	\big(m^2_{\sigma\sigma}\big)_{00} \\
	\end{NiceArray} \hspace{0.2mm} \custombracketr{\}}{0.39cm}
	& = m^2 + \smash{\frac{\lambda_1}{3}} \Big(4 \sqrt{2} \avg{\sigma_x} \avg{\sigma_y} + 7 \avg{\sigma_x}^2 + 5 \avg{\sigma_y}^2\Big) + \lambda_2 \Big(\avg{\sigma_x}^2 + \avg{\sigma_y}^2\Big), \\
	\begin{NiceArray}{c}
	\big(m^2_{\sigma\sigma}\big)_{11} \\
	\big(m^2_{\sigma\sigma}\big)_{22} \\
	\big(m^2_{\sigma\sigma}\big)_{33} \\
	\end{NiceArray} \custombracketr{\}}{0.9cm}
	& = m^2 + \lambda_1 \Big(\avg{\sigma_x}^2 + \avg{\sigma_y}^2\Big) + \smash{\frac32} \lambda_2 \avg{\sigma_x}^2, \\
	\begin{NiceArray}{c}
	\big(m^2_{\sigma\sigma}\big)_{44} \\
	\big(m^2_{\sigma\sigma}\big)_{55} \\
	\big(m^2_{\sigma\sigma}\big)_{66} \\
	\big(m^2_{\sigma\sigma}\big)_{77} \\
	\end{NiceArray} \custombracketr{\}}{1.2cm}
	& = m^2 + \lambda_1 \Big(\avg{\sigma_x}^2 + \avg{\sigma_y}^2\Big) + \smash{\frac{\lambda_2}{2}} \Big(\sqrt{2} \avg{\sigma_x} \avg{\sigma_y} + \avg{\sigma_x}^2 + 2 \avg{\sigma_y}^2\Big), \\
	\begin{NiceArray}{c}
	\big(m^2_{\sigma\sigma}\big)_{88} \\
	\end{NiceArray} \hspace{0.2mm} \custombracketr{\}}{0.39cm}
	&= m^2 - \smash{\frac{\lambda_1}{3}} \Big(4 \sqrt{2} \avg{\sigma_x} \avg{\sigma_y} - 5 \avg{\sigma_x}^2 - 7 \avg{\sigma_y}^2\Big) + \smash{\frac{\lambda_2}{2}} \Big(\avg{\sigma_x}^2 + 4 \avg{\sigma_y}^2\Big), \\
	\begin{NiceArray}{c}
	\big(m^2_{\sigma\sigma}\big)_{08} \\
	\big(m^2_{\sigma\sigma}\big)_{80} \\
	\end{NiceArray} \custombracketr{\}}{0.7cm}
	& = \smash{\frac23} \lambda_1 \Big(\sqrt{2} \avg{\sigma_x}^2 - \sqrt{2} \avg{\sigma_y}^2 - \avg{\sigma_x} \avg{\sigma_y}\Big) + \smash{\frac{\lambda_2}{\sqrt{2}}} \Big(\avg{\sigma_x}^2 - 2 \avg{\sigma_y}^2\Big).\\[0.1cm]
\end{array}
\end{NiceMatrixBlock}
\label{eq:lsm3f:mass_sigma_sigma}
\end{equation}
The nonzero entries of the pseudoscalar mass matrix are
\begin{equation}
\begin{NiceMatrixBlock}[]
\NiceMatrixOptions{cell-space-limits=1pt}
% smash all fracs to make consistent spacing between row entries
\begin{array}{l @{\,} >{\displaystyle}l}
	\begin{NiceArray}{c}
	\big(m^2_{\pi\pi}\big)_{00} \\
	\end{NiceArray} \hspace{0.2mm} \custombracketr{\}}{0.39cm}
	&= m^2 + \lambda_1 \Big(\avg{\sigma_x}^2 + \avg{\sigma_y}^2\Big) + \smash{\frac{\lambda_2}{3}} \Big(\avg{\sigma_x}^2 + \avg{\sigma_y}^2\Big), \\
	\begin{NiceArray}{c}
	\big(m^2_{\pi\pi}\big)_{11} \\
	\big(m^2_{\pi\pi}\big)_{22} \\
	\big(m^2_{\pi\pi}\big)_{33} \\
	\end{NiceArray} \custombracketr{\}}{0.9cm}
	&= m^2 + \lambda_1 \Big(\avg{\sigma_x}^2 + \avg{\sigma_y}^2\Big) + \smash{\frac{\lambda_2}{2}} \avg{\sigma_x}^2, \\
	\begin{NiceArray}{c}
	\big(m^2_{\pi\pi}\big)_{44} \\
	\big(m^2_{\pi\pi}\big)_{55} \\
	\big(m^2_{\pi\pi}\big)_{66} \\
	\big(m^2_{\pi\pi}\big)_{77} \\
	\end{NiceArray} \custombracketr{\}}{1.2cm}
	&= m^2 + \lambda_1 \Big(\avg{\sigma_x}^2 + \avg{\sigma_y}^2\Big) - \smash{\frac{\lambda_2}{2}} \Big(\sqrt{2} \avg{\sigma_x} \avg{\sigma_y} - \avg{\sigma_x}^2 - 2 \avg{\sigma_y}^2\Big), \\
	\begin{NiceArray}{c}
	\big(m^2_{\pi\pi}\big)_{88} \\
	\end{NiceArray} \hspace{0.2mm} \custombracketr{\}}{0.39cm}
	&= m^2 + \lambda_1 \Big(\avg{\sigma_x}^2 + \avg{\sigma_y}^2\Big) + \smash{\frac{\lambda_2}{6}} \Big(\avg{\sigma_x}^2 + 4 \avg{\sigma_y}^2\Big), \\ 
	\begin{NiceArray}{c}
	\big(m^2_{\pi\pi}\big)_{08} \\
	\big(m^2_{\pi\pi}\big)_{80} \\
	\end{NiceArray} \custombracketr{\}}{0.7cm}
	&= \smash{\frac{\lambda_2}{6}} \Big(\sqrt{2} \avg{\sigma_x}^2 - 2 \sqrt{2} \avg{\sigma_y}^2\Big).
\end{array}
\end{NiceMatrixBlock}
\label{eq:lsm3f:mass_pi_pi}
\end{equation}%
\label{eq:lsm3f:mass_matrices}%
\end{subequations}%
Notice that both matrices are non-diagonal due to the nonzero $\{08,80\}$ corner entries.
However, it is only fields corresponding to diagonal masses that represent true mass eigenstates of physically propagating particles: \cite{ref:lsm3f_details}
%As we summarizing the identification of different scalar and pseudoscalar meson species made in this reference,
%we will therefore rotate the $0$-$8$ field pairs (or $x$-$y$ pairs) to new bases that diagonalizes the $0$-$8$-sector of the mass matrices.
\begin{itemize}
\item The three diagonal masses $\smash{\big(m^2_{\sigma\sigma}\big)_{11} = \big(m^2_{\sigma\sigma}\big)_{22} = \big(m^2_{\sigma\sigma}\big)_{33} = m^2_{a_0}}$ correspond to
      one degenerate mass of the charged $a_0^\pm = (\sigma_1 \pm i \sigma_2) / \sqrt{2}$ and neutral $a_0^0 = \sigma_3$ mesons.
\item The three diagonal masses $\smash{\big(m^2_{\pi\pi}\big)_{11} = \big(m^2_{\pi\pi}\big)_{22} = \big(m^2_{\pi\pi}\big)_{33} = m^2_{\pi}}$ correspond to 
      one degenerate mass of the charged $\pi^\pm = (\pi_1 \pm i \pi_2) / \sqrt{2}$ and neutral $\pi^0 = \pi_3$ mesons.
\item The four diagonal masses $\smash{\big(m^2_{\sigma\sigma}\big)_{44} = \big(m^2_{\sigma\sigma}\big)_{55} = \big(m^2_{\sigma\sigma}\big)_{66} = \big(m^2_{\sigma\sigma}\big)_{77} = m^2_\kappa}$
      correspond to one degenerate mass
      of the charged $\kappa^\pm = (\sigma_4 \pm i \sigma_5) / \sqrt{2}$ and neutral $\kappa^0 = (\sigma_6 + i \sigma_7) / \sqrt{2}$ and $\bar{\kappa}^0 = (\sigma_6 - i \sigma_7) / \sqrt{2}$ mesons.
\item The four diagonal masses $\smash{\big(m^2_{\pi\pi}\big)_{44} = \big(m^2_{\pi\pi}\big)_{55} = \big(m^2_{\pi\pi}\big)_{66} = \big(m^2_{\pi\pi}\big)_{77} = m^2_K}$
      correspond to one degenerate mass
      of the charged $K^\pm = (\pi_4 \pm i \pi_5) / \sqrt{2}$ and neutral $K^0 = (\pi_6 + i \pi_7) / \sqrt{2}$ and $\bar{K}^0 = (\pi_6 - i \pi_7) / \sqrt{2}$ mesons.
\item The \emph{diagonalization}
      $\big\{m_\sigma^2, m_{f_0}^2\big\}$
      of the non-diagonal matrix sector 
      $\big\{\smash{\big(m^2_{\sigma\sigma}\big)_{00}}, \smash{\big(m^2_{\sigma\sigma}\big)_{88}}, \allowbreak \smash{\big(m^2_{\sigma\sigma}\big)_{08}}, \smash{\big(m^2_{\sigma\sigma}\big)_{80}}\big\}$
      corresponds to two different masses of the $f_0$ and $\sigma$ mesons.
\item The \emph{diagonalization}
      $\big\{m_\eta^2, m_{\eta'}^2\big\}$
      of the non-diagonal matrix sector 
      $\big\{\smash{\big(m^2_{\pi\pi}\big)_{00}}, \smash{\big(m^2_{\pi\pi}\big)_{88}}, \allowbreak \smash{\big(m^2_{\pi\pi}\big)_{08}}, \smash{\big(m^2_{\pi\pi}\big)_{80}}\big\}$
      corresponds to two different masses of the $\eta$ and $\eta'$ mesons.
\end{itemize}

The two diagonalizations are achieved by two rotations
\begin{equation}
	\begin{bmatrix} f_0 \\ \sigma \\ \end{bmatrix} = \begin{bmatrix} \cos \theta_\sigma & -\sin \theta_\sigma \\ \sin \theta_\sigma & \phantom{-} \cos \theta_\sigma \\ \end{bmatrix} \begin{bmatrix} \sigma_8 \\ \sigma_0 \\ \end{bmatrix}
	\qquad \text{and} \qquad
	\begin{bmatrix} \eta \\ \eta' \\ \end{bmatrix} = \begin{bmatrix} \cos \theta_\pi & -\sin \theta_\pi \\ \sin \theta_\pi & \phantom{-} \cos \theta_\pi \\ \end{bmatrix} \begin{bmatrix} \pi_8 \\ \pi_0 \\ \end{bmatrix}
	\label{eq:lsm3f:diagonalization_transformation}
\end{equation}
of the $\{\sigma_0, \sigma_8\}$ and $\{\pi_0, \pi_8\}$ fields parametrized by two mixing angles $\{\theta_\sigma, \theta_\pi\}$.
To find the pseudoscalar angle $\theta_\pi$, for example,
we invert transformation \eqref{eq:lsm3f:diagonalization_transformation} and use the trigonometric identities $\cos^2 \theta_\pi - \sin^2 \theta_\pi = \cos 2 \theta_\pi$ and $2 \sin\theta_\pi \cos\theta_\pi = \sin 2 \theta_\pi$ to expand
\begin{equation}
\begin{split}
	\smash{\smashoperator{\sum_{\substack{a=\{0,8\}\\b=\{0,8\}}}}} \big(m^2_{\pi\pi}\big)_{ab} \tilde{\pi}_a \tilde{\pi}_b &= \Big\{ \big(m^2_{\pi\pi}\big)_{00} \sin^2 \theta_\pi + \big(m^2_{\pi\pi}\big)_{88} \cos^2 \theta_\pi + \big(m^2_{\pi\pi}\big)_{08} \sin 2 \theta_\pi \Big\} \, \tilde{\eta}^2 \\
	                                                                                                                       &+ \Big\{ \big(m^2_{\pi\pi}\big)_{00} \cos^2 \theta_\pi + \big(m^2_{\pi\pi}\big)_{88} \sin^2 \theta_\pi - \big(m^2_{\pi\pi}\big)_{08} \sin 2 \theta_\pi \Big\} \, \tilde{\eta}'^2 \\
	                                                                                                                       &+ \underbrace{\Big\{ 2 \Big[ \big(m^2_{\pi\pi}\big)_{00} - \big(m^2_{\pi\pi}\big)_{88} \Big] \sin 2 \theta_\pi + 2 \big(m^2_{\pi\pi}\big)_{08} \cos 2 \theta_\pi \Big\}}_{\text{$0$ by demand!}} \, \tilde{\eta} \tilde{\eta}' ,
\end{split}
\label{eq:lsm3f:diagonalization_sum}
\end{equation}
and then require the indicated coefficient of $\tilde{\eta} \tilde{\eta}'$ to vanish.
The scalar angle $\theta_\sigma$ is found in the same way, only with
$\{m^2_{\pi\pi}, \eta, \eta'\} \rightarrow \{m^2_{\sigma\sigma}, f_0, \sigma\}$.
Thus, the third line gives the mixing angles
\begin{equation}
	\theta_\sigma = \frac12 \arctan \Bigg[ \frac{2\big(m^2_{\sigma\sigma}\big)_{08}}{\big(m^2_{\sigma\sigma}\big)_{88} - \big(m^2_{\sigma\sigma}\big)_{00}} \Bigg]
	\quad \text{and} \quad
	\theta_\pi = \frac12 \arctan \Bigg[ \frac{2\big(m^2_{\pi\pi}\big)_{08}}{\big(m^2_{\pi\pi}\big)_{88} - \big(m^2_{\pi\pi}\big)_{00}} \Bigg].
\label{eq:lsm3f:mixing_angles}
\end{equation}
This in turn fixes the curvature masses of the pseudoscalar $\{\eta,\eta'\}$ mesons
as their coefficients on the two first lines in the sum \eqref{eq:lsm3f:diagonalization_sum},
and analogously for the scalar $\{f_0,\sigma\}$ mesons.
To be very clear,
the diagonalized mass-generating meson potential \eqref{eq:lsm3f:potential_before_diagonalization} can now be explicitly written
\begin{equation}
\begin{split}
	\hspace{-2cm} \pot(\sigma,\pi) \taylor \pot(\avg{\sigma},\avg{\pi}) &+ \frac12 m^2_{f_0} \tilde{f_0}^2  + \frac12 m^2_{\sigma} \tilde{\sigma}^2 \,\, + \frac12 m^2_{a_0} \sum_{\mathrlap{\!\!\!\!a_0 = \{a_0^+,a_0^-,a_0^0\}}} \tilde{a}_0^2               \,\, + \, \frac12 m^2_{\kappa} \sum_{\mathrlap{\!\!\!\!\kappa = \{\kappa^+,\kappa^-,\kappa^0,\bar{\kappa}^0\}}} \tilde{\kappa}^2 \\
	                                                                    &+ \frac12 m^2_{\eta} \tilde{\eta}^2 \,\,\,\,\, + \frac12 m^2_{\eta'} \tilde{\eta}'^2 + \frac12 m^2_{\pi_{\phantom{0}}} \sum_{\mathrlap{\!\!\!\!\pi = \{\pi^+,\pi^-,\pi^0\}}} \tilde{\pi}^2 \, + \, \frac12 m^2_{K} \sum_{\mathrlap{\!\!\!\!K = \{K^+,K^-,K^0,\bar{K}^0\}}} \tilde{K}^2. \\
\end{split}
\label{eq:lsm3f:potential_after_diagonalization}
\end{equation}
This shows very elaborately how the three couplings $\{m^2, \lambda_1, \lambda_2\}$,
through the mass matrices \eqref{eq:lsm3f:mass_matrices} and the mixing angles \eqref{eq:lsm3f:mixing_angles},
generate the eight scalar and pseudoscalar particle masses
\begin{subequations}
\begin{align}
	& m^2_{f_0}   && \hspace{-2.9cm} = \big(m^2_{\sigma\sigma}\big)_{00} \sin^2 \theta_\sigma + \big(m^2_{\sigma\sigma}\big)_{88} \cos^2 \theta_\sigma + \big(m^2_{\sigma\sigma}\big)_{08} \sin 2 \theta_\sigma, \\ % \quad \text{(with $\theta_\sigma$ from \eqref{eq:lsm3f:mixing_angles})} \\
	& m^2_\sigma  && \hspace{-2.9cm} = \big(m^2_{\sigma\sigma}\big)_{00} \cos^2 \theta_\sigma + \big(m^2_{\sigma\sigma}\big)_{88} \sin^2 \theta_\sigma - \big(m^2_{\sigma\sigma}\big)_{08} \sin 2 \theta_\sigma, \\ % \quad \text{(with $\theta_\sigma$ from \eqref{eq:lsm3f:mixing_angles})} \\
	& m^2_{a_0}   && \hspace{-2.9cm} = \big(m^2_{\sigma\sigma}\big)_{11} = \big(m^2_{\sigma\sigma}\big)_{22} = \big(m^2_{\sigma\sigma}\big)_{33}, \\
	& m^2_\kappa  && \hspace{-2.9cm} = \big(m^2_{\sigma\sigma}\big)_{44} = \big(m^2_{\sigma\sigma}\big)_{55} = \big(m^2_{\sigma\sigma}\big)_{66} = \big(m^2_{\sigma\sigma}\big)_{77}, \\
	& m^2_{\eta}  && \hspace{-2.9cm} = \big(m^2_{\pi\pi}\big)_{00} \sin^2 \theta_\pi + \big(m^2_{\pi\pi}\big)_{88} \cos^2 \theta_\pi + \big(m^2_{\pi\pi}\big)_{08} \sin 2 \theta_\pi, \\ % \quad \text{($\theta_\pi$ from \eqref{eq:lsm3f:mixing_angles})} \\
	& m^2_{\eta'} && \hspace{-2.9cm} = \big(m^2_{\pi\pi}\big)_{00} \cos^2 \theta_\pi + \big(m^2_{\pi\pi}\big)_{88} \sin^2 \theta_\pi - \big(m^2_{\pi\pi}\big)_{08} \sin 2 \theta_\pi, \\ % \quad \text{($\theta_\pi$ from \eqref{eq:lsm3f:mixing_angles})} \\
	& m^2_\pi     && \hspace{-2.9cm} = \big(m^2_{\pi\pi}\big)_{11} = \big(m^2_{\pi\pi}\big)_{22} = \big(m^2_{\pi\pi}\big)_{33}, \\
	& m^2_K       && \hspace{-2.9cm} = \big(m^2_{\pi\pi}\big)_{44} = \big(m^2_{\pi\pi}\big)_{55} = \big(m^2_{\pi\pi}\big)_{66} = \big(m^2_{\pi\pi}\big)_{77}.
\end{align}%
\label{eq:lsm:mass_system_3f}%
\end{subequations}%
These relations will come in handy when we fit the parameters of the model in \cref{sec:lsm3f:parameter_fit}.


\section{Grand potential}

Let us calculate the grand potential $\Omega$ in the same way as we did with two flavors in \cref{sec:lsm:grand_potential},
integrating over one fermion loop and
neglecting pion condensation by setting $\avg{\pi}=0$
while using the mean-field approximation for $\avg{\sigma_x}$ and $\avg{\sigma_y}$,
determining them retrospectively according to the self-consistency equations
\begin{equation}
	\pdv{\Omega}{\avg{\sigma_x}} =
	\pdv{\Omega}{\avg{\sigma_y}} = 0.
\end{equation}
As explained in \cref{chap:lsm2f},
this is inconsistent in terms of the number of loops considered for the fermionic and bosonic fields,
but consistent in the \textbf{one-loop large-$N_c$ limit}.

The calculation proceeds more or less identically up to the non-renormalized grand potential \eqref{eq:lsm:grand_potential_nonrenormalized},
only with a different vacuum potential $\pot(\avg{\sigma},0)$, $N_f=3$ flavors and $m_u = m_d = m_x \neq m_y = m_s$.
In addition, instead of using one common renormalization scale $\Lambda$,
we operate with two separate scales $\Lambda_x$ and $\Lambda_y$ for the non-strange and strange quarks.
We comment on this in the next section.
This gives the divergent grand potential
\begin{equation}
\begin{split}
	\Omega(\avg{\sigma},\vec{\mu}) &= \pot(\avg{\sigma},0) + \frac{N_c m_x^4}{8 \pi^2} \Bigg[ \textcolor{red}{\frac{1}{\epsilon}} + \frac{3}{2} + \log\bigg(\frac{\Lambda_x^2}{m_x^2}\bigg) \Bigg] + \frac{N_c m_y^4}{16 \pi^2} \Bigg[ \textcolor{red}{\frac{1}{\epsilon}} + \frac{3}{2} + \log\bigg(\frac{\Lambda_y^2}{m_y^2}\bigg) \Bigg] \\ % + N_c \smashoperator{\sum_{\vphantom{\big|}f=\{u,d,s\}}} \frac{m_f^4}{16 \pi^2} \left[ \frac{1}{\epsilon} + \frac{3}{2} + \log\left(\frac{{\Lambda}^2}{m_f^2}\right) \right] \\
	                               &-\smashoperator{\sum_{f=1}^{N_c}} \frac{N_c}{24 \pi^2} \left[ \left( 2 \mu_f^2 - 5 m_f^2 \right) \mu_f \sqrt{\mu_f^2 - m_f^2} + 3 m_f^4 \asinh \left( \sqrt{\frac{\mu_{\smash{f}}^2}{m_f^2}-1} \right) \right]. \\
\end{split}
\end{equation}
The divergence $\textcolor{red}{N_c (2 m_x^4 + m_y^2) / 16 \pi^2 \epsilon} = \textcolor{red}{N_c g^4 (\avg{\sigma_x}^4 + 2\avg{\sigma_y}^4) / 128 \epsilon}$
from the pole in $\epsilon$
%\begin{equation}
%	\textcolor{red}{
%	N_c \frac{2 m_x^4 + m_y^4}{16 \pi^2 \epsilon} = \frac{N_c g^4}{16 \pi^2} \frac{\avg{\sigma_x}^4 + 2 \avg{\sigma_y}^4}{8 \epsilon}
%	}
%\end{equation}
can again be absorbed by $\textcolor{blue}{\lambda_2 (\avg{\sigma_x}^4+2\avg{\sigma_y}^4)/8}$ 
in the meson potential \eqref{eq:lsm:potential_tree_3f} if we shift
\begin{equation}
	\lambda_2 \rightarrow \lambda_2 + \textcolor{blue}{\delta\lambda_2} \qquad \text{with the counterterm} \qquad \textcolor{blue}{\delta\lambda_2 = -\frac{N_c g^4}{16 \pi^2 \epsilon}}.
\end{equation}
Then $\textcolor{red}{N_c (2 m_x^4 + m_y^2) / 16 \pi^2 \epsilon} + \textcolor{blue}{\delta\lambda_2 (\avg{\sigma_x}^4+2\avg{\sigma_y}^4)} = 0$,
so this theory is also renormalizable.
Adding electrons, we obtain the natural three-flavor generalization of the renormalized and finite grand potential \eqref{eq:lsm:grand_potential},
\begin{equation}
\begin{split}
	\Omega(\avg{\sigma},\vec{\mu}) &= \pot(\avg{\sigma},0) + N_c \frac{m_x^4}{8 \pi^2} \Bigg[ \frac{3}{2} + \log\bigg(\frac{\Lambda_x^2}{m_x^2}\bigg) \Bigg] + N_c \frac{m_y^4}{16 \pi^2} \Bigg[ \frac{3}{2} + \log\bigg(\frac{\Lambda_y^2}{m_y^2}\bigg) \Bigg] \\ % + N_c \smashoperator{\sum_{\vphantom{\big|}f=\{u,d,s\}}} \frac{m_f^4}{16 \pi^2} \left[ \frac{1}{\epsilon} + \frac{3}{2} + \log\left(\frac{{\Lambda}^2}{m_f^2}\right) \right] \\
	                               &-\smashoperator{\sum_{f=1}^{N_c}} \frac{N_c}{24 \pi^2} \left[ \left( 2 \mu_f^2 - 5 m_f^2 \right) \mu_f \sqrt{\mu_f^2 - m_f^2} + 3 m_f^4 \asinh \left( \sqrt{\frac{\mu_{\smash{f}}^2}{m_f^2}-1} \right) \right] \\
	                               &-\phantom{\sum} \, \frac{1}{24 \pi^2} \left[ \left( 2 \mu_e^2 - 5 m_e^2 \right) \mu_e \sqrt{\mu_e^2 - m_e^2} \, + 3 m_e^4 \asinh \left( \sqrt{\frac{\mu_{\smash{e}}^2}{m_e^2}-1} \right) \right].
\end{split}
\label{eq:lsm3f:grand_potential}
\end{equation}

\section{Parameter fit at tree-level}
\label{sec:lsm3f:parameter_fit}

We are now in position to fit the six parameters $g$, $m^2$, $\lambda_1$, $\lambda_2$, $h_x$ and $h_y$.
In vacuum, the pion and kaon decay constants $f_\pi$ and $f_K$ are related to the nonzero expectation values
\cite{ref:lsm3f_details}
\begin{equation}
	%f_\pi = \sqrt{\frac23} \avg{\sigma_0} + \frac{\avg{\sigma_8}}{\sqrt{3}}
	%\quad \text{and} \quad
	%f_K = \sqrt{\frac23} \avg{\sigma_0} - \frac{\avg{\sigma_8}}{\sqrt{12}},
	%\quad \text{or} \quad
	\avg{\sigma_x} = f_\pi
	\qquad \text{and} \qquad
	\avg{\sigma_y} = \sqrt{2} f_K - \frac{f_\pi}{\sqrt{2}}.
\label{eq:lsm3f:decay_constants}
\end{equation}
With vacuum measurements of the two decay constants and one of the quark masses \eqref{eq:lsm:quark_masses_3f},
we determine the parameter $g$, then use it to predict the other quark masses.
Likewise, we use vacuum measurements of three of the eight meson masses \eqref{eq:lsm:mass_system_3f} to determine the three parameters $\{m^2, \lambda_1, \lambda_2\}$,
then use them to predict the five remaining meson masses.
With $\avg{\sigma_x}$, $\avg{\sigma_y}$, $m^2$, $\lambda_1$ and $\lambda_2$ in hand,
the two symmetry breakers $h_x$ and $h_y$ follow from equation \eqref{eq:lsm3f:symmetry_breakers}.

\Cref{tab:lsm3f:parameters} shows values for the fitted and predicted particle masses.
In general there are $8!/3!5! = 56$ ways of selecting three of the eight meson masses \eqref{eq:lsm:mass_system_3f},
each returning different values for $\{m^2,\lambda_1,\lambda_2\}$.
We keep using $m_\sigma$ and $m_\pi$ for continuity from \cref{chap:lsm2f}
and include $m_K$ as the third mass, thereby obtaining the unique selection of mesons with lowest mass and energy.
This is the same selection used by \cite{ref:lsm3f} and \cite{ref:lsm3f_details}, for example.

Note the particularly large discrepancy between the modeled and experimental $\eta'$ mass.
This can be improved by including a term $c \, (\tdet{\phi} + \tdet{\phi^\dagger})$ in the Lagrangian
that models the anomalous axial $U(1)_A$ current of quantum chromodynamics discussed in \cref{sec:master_intro:qcd}.
How to do so is also shown in \cite{ref:lsm3f} and \cite{ref:lsm3f_details},
but we leave it out in order to keep the two-flavor and three-flavor models as similar and comparable as possible.
In particular, they find that it drives the chiral transition closer to a discontinuous phase transition than a crossover.

Like in \cref{sec:lsm:parameter_fit},
we keep all parameters fixed except $m_\sigma$, due to its large experimental uncertainty.
As shown in the lasagna of different vacuum potentials in \cref{fig:lsm3f:potential_sigma_mass},
they still admit minima only for $m_\sigma \geq \SI{600}{\mega\electronvolt}$,
so we continue to use $m_\sigma=\{600,700,800\} \, \si{\mega\electronvolt}$.

With two flavors we determined the renormalization scale \eqref{eq:lsm:potential_vacuum_minimum} by requiring the minimum to remain at $\avg{\sigma}=f_\pi$ in vacuum.
The natural generalization of this procedure to three flavors is to determine $\Lambda_x$ and $\Lambda_y$ so the minimum remains at \eqref{eq:lsm3f:decay_constants} in vacuum.
This is accomplished by
\begin{equation}
	\pdv{\Omega}{\avg{\sigma_x}} = \pdv{\Omega}{\avg{\sigma_y}} = 0,
	\quad \text{so}
	\quad \Lambda_x = \frac{m_x}{\sqrt{e}} = \SI{182.0}{\mega\electronvolt}
	\quad \text{and} \quad
	\Lambda_y = \frac{m_y}{\sqrt{e}} = \SI{260.2}{\mega\electronvolt}.
\end{equation}
This is not the only way to do it.
Like \cite{ref:master_berge},
we could rather operate with one common scale $\Lambda = \Lambda_x = \Lambda_y$ in the grand potential,
for example chosen as the flavor-weighted average $\Lambda = (2 \cdot \SI{182.0}{\mega\electronvolt} + \SI{260.2}{\mega\electronvolt}) / 3 = \SI{208.0}{\mega\electronvolt}$ of the two we have found.
In particular, this means that the minimum of $\Omega$ in vacuum is no longer at the minimum of $\pot$.
This is unsatisfying and has surprising consequences.
For example, if $\Omega$ is fit to some quark mass $m_x = g f_\pi /2$,
its minimum $\avg{\sigma_x} \neq f_\pi$ moves \emph{away} from $f_\pi$ because $\pdv{\Omega}/{\avg{\sigma_x}} \neq 0$ at $f_\pi$,
in turn generating a \emph{different} vacuum quark mass $g \avg{\sigma_x} / 2 \neq g f_\pi / 2$ than the one that was fitted!
Of course, our approach also has its caveats, as it is not apparent how to interpret the presence of two different renormalization scales, let alone one.
But it does not behave in this unpredictable way, and it permits using the same masses $m_\sigma \geq \SI{600}{\mega\electronvolt}$ as in \cref{chap:lsm2f} without breaking the potential.
Ideally, a more consistent calculation such as the one in \cref{sec:lsm2f:refinement} should be done anyway,
where one preferable renormalization scale appears, and we even saw that it dropped out in that specific case.
It is hard to say whether our or \cite{ref:master_berge}'s approach is better,
but this is only another argument to at the very least take the unexplored path over an already paved way.

\begin{table}
\centering
\caption{\label{tab:lsm3f:parameters}%
Variables in the left table are used as \textbf{input} to determine the six model parameters in the right table
from equations \eqref{eq:lsm:quark_masses_3f}, \eqref{eq:lsm3f:symmetry_breakers} and \eqref{eq:lsm:mass_system_3f},
which in turn are used to predict the remaining non-bold masses in the left table
with the same equations.
Three different values are used for $m_\sigma$,
generating three parameter sets with different $m^2$, $\lambda_1$ and $m^2_{f_0}$.
Experimental values are taken from \cite{ref:pdg_review_2021}.
}
\begin{tabular}{ l r r r r c l r r r}
\toprule
\multicolumn{5}{c}{Physical variables} & & \multicolumn{4}{c}{Model parameters} \\
\cmidrule(lr){1-5} \cmidrule(lr){7-10}
\multirow{2}[1]{*}{Variable} & \multicolumn{3}{c}{Modeled} & \multirow{2}[1]{*}{Measured} & & \multirow{2}[1]{*}{Parameter} & \multicolumn{3}{c}{Modeled} \\
\cmidrule(lr){2-4} \cmidrule(lr){8-10}
& Set 1 & Set 2 & Set 3 & & & & Set 1 & Set 2 & Set 3 \\
\midrule
%\cmidrule(lr){1-5} \cmidrule(lr){6-9}
%& $m_\sigma=\SI{600}{\mega\electronvolt} & $m_\sigma=\SI{700}{\mega\electronvolt} & $m_\sigma=\SI{800}{\mega\electronvolt} & & & $m_\sigma=\SI{600}{\mega\electronvolt} & $m_\sigma=\SI{700}{\mega\electronvolt} & $m_\sigma=\SI{800}{\mega\electronvolt} \\
$\mathmakebox[\widthof{$m_{f_0}$}][l]{f_\pi}     \,/\, \si{\mega\electronvolt}$ & $\textbf{\SI{93}{}}$ & $\textbf{\SI{93}{}}$ & $\textbf{\SI{93}{}}$ & \SI{92}{}-\SI{93}{} & & $g$ & $\SI{6.45}{}^{\phantom{1}}$ & $\SI{6.45}{}^{\phantom{1}}$ & $\SI{6.45}{}^{\phantom{1}}$ \\
$\mathmakebox[\widthof{$m_{f_0}$}][l]{f_K}       \,/\, \si{\mega\electronvolt}$ & $\textbf{\SI{113}{}}$ & $\textbf{\SI{113}{}}$ & $\textbf{\SI{113}{}}$ & \SI{113}{} & & $m^2 \,/\, (\si{\mega\electronvolt})^2$ & $-\SI{178}{}^2$ & $-\SI{351}{}^2$ & $-\SI{492}{}^2$ \\
$\mathmakebox[\widthof{$m_{f_0}$}][l]{m_u}       \,/\, \si{\mega\electronvolt}$ & $\textbf{\SI{300}{}}$ & $\textbf{\SI{300}{}}$ & $\textbf{\SI{300}{}}$ & \approx \, \SI{300}{} & & $\lambda_1$ & $\SI{-18.2}{}^{\phantom{1}}$ & $\SI{-13.0}{}^{\phantom{1}}$ & $\SI{-6.2}{}^{\phantom{1}}$ \\
$\mathmakebox[\widthof{$m_{f_0}$}][l]{m_d}       \,/\, \si{\mega\electronvolt}$ & $\textbf{\SI{300}{}}$ & $\textbf{\SI{300}{}}$ & $\textbf{\SI{300}{}}$ & \approx \, \SI{300}{} & & $\lambda_2$ & $\SI{85.3}{}^{\phantom{1}}$ & $\SI{85.3}{}^{\phantom{1}}$ & $\SI{85.3}{}^{\phantom{1}}$ \\
$\mathmakebox[\widthof{$m_{f_0}$}][l]{m_s}       \,/\, \si{\mega\electronvolt}$ & $\SI{429}{}$ & $\SI{429}{}$ & $\SI{429}{}$ & \approx \, \SI{500}{} & & $\mathmakebox[\widthof{$m^2$}][l]{h_x} \,/\, (\si{\mega\electronvolt})^3$ & $\SI{121}{}^3$ & $\SI{121}{}^3$ & $\SI{121}{}^3$ \\
$\mathmakebox[\widthof{$m_{f_0}$}][l]{m_{f_0}}   \,/\, \si{\mega\electronvolt}$ & $\SI{1294}{}$ & $\SI{1315}{}$ & $\SI{1347}{}$ & \SI{1200}{}-\SI{1500}{} & & $\mathmakebox[\widthof{$m^2$}][l]{h_y} \,/\, (\si{\mega\electronvolt})^3$ & $\SI{336}{}^3$ & $\SI{336}{}^3$ & $\SI{336}{}^3$ \\
$\mathmakebox[\widthof{$m_{f_0}$}][l]{m_\sigma}  \,/\, \si{\mega\electronvolt}$ & $\textbf{\SI{600}{}}$ & $\textbf{\SI{700}{}}$ & $\textbf{\SI{800}{}}$ & \SI{400}{}-\SI{550}{} \\
$\mathmakebox[\widthof{$m_{f_0}$}][l]{m_{a_0}}   \,/\, \si{\mega\electronvolt}$ & $\SI{870}{}$ & $\SI{870}{}$ & $\SI{870}{}$ & \SI{980}{} \\
$\mathmakebox[\widthof{$m_{f_0}$}][l]{m_\kappa}  \,/\, \si{\mega\electronvolt}$ & $\SI{1141}{}$ & $\SI{1141}{}$ & $\SI{1141}{}$ & \SI{1414}{} \\
$\mathmakebox[\widthof{$m_{f_0}$}][l]{m_\eta}    \,/\, \si{\mega\electronvolt}$ & $\SI{636}{}$ & $\SI{636}{}$ & $\SI{636}{}$ & \SI{548}{} \\
$\mathmakebox[\widthof{$m_{f_0}$}][l]{m_{\eta'}} \,/\, \si{\mega\electronvolt}$ & $\SI{138}{}$ & $\SI{138}{}$ & $\SI{138}{}$ & \SI{958}{} \\
$\mathmakebox[\widthof{$m_{f_0}$}][l]{m_\pi}     \,/\, \si{\mega\electronvolt}$ & $\textbf{\SI{138}{}}$ & $\textbf{\SI{138}{}}$ & $\textbf{\SI{138}{}}$ & \SI{138}{} \\
$\mathmakebox[\widthof{$m_{f_0}$}][l]{m_K}       \,/\, \si{\mega\electronvolt}$ & $\textbf{\SI{496}{}}$ & $\textbf{\SI{496}{}}$ & $\textbf{\SI{496}{}}$ & \SI{496}{} \\
\bottomrule
\end{tabular}
\end{table}

\begin{figure}
\centering
\tikzsetnextfilename{3-flavor-potential-vacuum}
\begin{tikzpicture}
\begin{axis}[
	width = 15cm, height = 10cm,
	restrict x to domain=-510:+510, xmin=-500, xmax=+500, xtick distance = 150, minor x tick num=14,
	restrict y to domain=-610:+610, ymin=-600, ymax=+600, ytick distance = 150, minor y tick num=14,
	%ymin=-11, ymax=+6, ytick distance=5, minor y tick num=4,
	xlabel = {$m_x \, / \, \si{\mega\electronvolt}$ }, ylabel = {$m_y \, / \, \si{\mega\electronvolt}$}, zlabel = {$\Omega(\vec{\mu}=0) \, / \, f_\pi^4$},
	%legend style = {at={(0.5,1.03)}, anchor=south}, transpose legend, legend columns=2,
	%cycle list/YlOrRd-9,
	grid = major,
	view = {127}{14}, % 127 7 before
	colormap/Greys, colormap/YlOrRd, % load
	colormap/Greys, mesh/interior colormap name=YlOrRd, % set with name (https://tex.stackexchange.com/a/359491)
]

%\pgfplotsset{cycle list shift=+1} % skip weakest line
\addplot3 [surf, thin, point meta=explicit] table [x=Deltax, y=Deltay, z expr={\thisrow{Omega} +  0}, meta expr={\thisrow{Omega}}] {../code/data/LSM3F/potential_vacuum_sigma500.dat};% \addlegendentry{$m_\sigma = \SI{600}{\mega\electronvolt}$};
\addplot3 [surf, thin, point meta=explicit] table [x=Deltax, y=Deltay, z expr={\thisrow{Omega} + 40}, meta expr={\thisrow{Omega}}] {../code/data/LSM3F/potential_vacuum_sigma600.dat};% \addlegendentry{$m_\sigma = \SI{600}{\mega\electronvolt}$};
\addplot3 [surf, thin, point meta=explicit] table [x=Deltax, y=Deltay, z expr={\thisrow{Omega} + 80}, meta expr={\thisrow{Omega}}] {../code/data/LSM3F/potential_vacuum_sigma700.dat};% \addlegendentry{$m_\sigma = \SI{700}{\mega\electronvolt}$};
\addplot3 [surf, thin, point meta=explicit] table [x=Deltax, y=Deltay, z expr={\thisrow{Omega} +120}, meta expr={\thisrow{Omega}}] {../code/data/LSM3F/potential_vacuum_sigma800.dat};% \addlegendentry{$m_\sigma = \SI{800}{\mega\electronvolt}$};

\addplot3 coordinates {(300,429,-33.0)} node [above, xshift=-0.0cm, yshift=-0.0cm, black, font=\footnotesize] {no minimum}; % 500
\addplot3 [every axis plot post/.append style={only marks, mark=*}, mark size=0.5pt] coordinates {(300,429,+7.0)} node [above, font=\footnotesize] {minimum}; % 600
\addplot3 [every axis plot post/.append style={only marks, mark=*}, mark size=0.5pt] coordinates {(300,429,+41.6)} node [above, font=\footnotesize] {minimum}; % 700
\addplot3 [every axis plot post/.append style={only marks, mark=*}, mark size=0.5pt] coordinates {(300,429,+74.7)} node [above, font=\footnotesize] {minimum}; % 800

\addplot3 coordinates {(450,0,0)} node [black, rotate=-30, yshift=+0.5cm] {$m_\sigma = \SI{500}{\mega\electronvolt}$};
\addplot3 coordinates {(450,0,40)} node [black, rotate=-30, yshift=+0.5cm] {$m_\sigma = \SI{600}{\mega\electronvolt}$};
\addplot3 coordinates {(450,0,80)} node [black, rotate=-30, yshift=+0.5cm] {$m_\sigma = \SI{700}{\mega\electronvolt}$};
\addplot3 coordinates {(450,0,120)} node [black, rotate=-30, yshift=+0.5cm] {$m_\sigma = \SI{800}{\mega\electronvolt}$};
\end{axis}
\end{tikzpicture}
\caption{\label{fig:lsm3f:potential_sigma_mass}%
The three-flavor grand potential \eqref{eq:lsm3f:grand_potential},
here offset by constants into a human-digestible lasagna,
admits minima for the quark masses $m_x$ and $m_y$ only for $m_\sigma \geq \SI{600}{\mega\electronvolt}$
in vacuum, where $\vec{\mu}=0$.
}
\end{figure}

Recall from the consistently fit two-flavor model in \cref{sec:lsm2f:refinement}
that our way of fitting parameters at tree-level to a grand potential calculated in the one-loop large-$N_c$ limit is really inconsistent.
However, we know of no consistent calculation with three flavors like the one presented there.
The lesson we learned back there was that inconsistently fit $\SI{600}{\mega\electronvolt} \leq m_\sigma \leq \SI{800}{\mega\electronvolt}$
is representative of physically measured masses $\SI{400}{\mega\electronvolt} \leq m_\sigma \leq \SI{600}{\mega\electronvolt}$,
so we can only hope that the same is true with three flavors.

\section{Equation of state}

\begin{figure}
\centering
\tikzsetnextfilename{3-flavor-eos}
\begin{tikzpicture}
\tikzset{declare function={
	nq3(\mu)=3/(3*pi^2)*(\mu)^3;
	ne3(\mu)=1/(3*pi^2)*(\mu)^3;
	nconv=1.29619e-7;
}};
\begin{groupplot}[
	group style={group size={1 by 3}, vertical sep=1.9cm},
	width=13cm, height=7cm,
	extra tick style={grid=major, grid style={dashed}},
	minor tick num=9,
	legend style={xshift=-0.2cm, fill=none},
]
\nextgroupplot[
	xlabel={$\mu \, / \, \si{\mega\electronvolt}$},
	ylabel={$\{m_i,\mu_i\} \, / \, \si{\mega\electronvolt}$},
	%xmin=0, xmax=600, ymax=500, xtick distance=100, ytick distance=100, minor x tick num=9,
	xmin=0, xmax=700, xtick distance=100, minor x tick num=9,
	ymin=-20, ymax=850, ytick distance=100, 
	%ymax=600, 
	title={\subcaption{\label{fig:lsm:3-flavor-eos-parametrization}Parametrization of solutions}},
	legend cell align=left, legend pos=north west, legend columns=5, legend transposed,
];
\addplot+ [blue, solid, very thick, opacity=0.4, forget plot, domain=0:800] {0};
\addplot+ [orange, solid, very thick, opacity=0.4, forget plot, domain=0:800] {0};
\addplot+ [yellow, dashed, very thick, opacity=0.4, forget plot, domain=0:800] {0};
\addplot+ [red, solid, very thick, opacity=0.4, forget plot, domain=0:800] {x};
\addplot+ [darkgreen, dashed, very thick, opacity=0.4, forget plot, domain=0:800] {x};

\addplot+ [orange,    solid,          opacity=0.7,            ] table [x expr={(\thisrow{muu}+\thisrow{mud})/2}, y=mu]  {../code/data/LSM3F/eos_sigma_800.dat}; \addlegendentry{$m_x$};
\addplot+ [orange,    densely dashed, opacity=0.7, forget plot] table [x expr={(\thisrow{muu}+\thisrow{mud})/2}, y=mu]  {../code/data/LSM3F/eos_sigma_700.dat};
\addplot+ [orange,    densely dotted, opacity=0.7, forget plot] table [x expr={(\thisrow{muu}+\thisrow{mud})/2}, y=mu]  {../code/data/LSM3F/eos_sigma_600.dat};
\addplot+ [yellow,    solid,          opacity=0.7,            ] table [x expr={(\thisrow{muu}+\thisrow{mud})/2}, y=ms]  {../code/data/LSM3F/eos_sigma_800.dat}; \addlegendentry{$m_y$};
\addplot+ [yellow,    densely dashed, opacity=0.7, forget plot] table [x expr={(\thisrow{muu}+\thisrow{mud})/2}, y=ms]  {../code/data/LSM3F/eos_sigma_700.dat};
\addplot+ [yellow,    densely dotted, opacity=0.7, forget plot] table [x expr={(\thisrow{muu}+\thisrow{mud})/2}, y=ms]  {../code/data/LSM3F/eos_sigma_600.dat};
\addplot+ [red,       solid,          opacity=0.7,            ] table [x expr={(\thisrow{muu}+\thisrow{mud})/2}, y=muu] {../code/data/LSM3F/eos_sigma_800.dat}; \addlegendentry{$\mu_u$};
\addplot+ [red,       densely dashed, opacity=0.7, forget plot] table [x expr={(\thisrow{muu}+\thisrow{mud})/2}, y=muu] {../code/data/LSM3F/eos_sigma_700.dat};
\addplot+ [red,       densely dotted, opacity=0.7, forget plot] table [x expr={(\thisrow{muu}+\thisrow{mud})/2}, y=muu] {../code/data/LSM3F/eos_sigma_600.dat};
\addplot+ [darkgreen, solid,          opacity=0.7,            ] table [x expr={(\thisrow{muu}+\thisrow{mud})/2}, y=mud] {../code/data/LSM3F/eos_sigma_800.dat}; \addlegendentry{$\mu_d=\mu_s$};
\addplot+ [darkgreen, densely dashed, opacity=0.7, forget plot] table [x expr={(\thisrow{muu}+\thisrow{mud})/2}, y=mud] {../code/data/LSM3F/eos_sigma_700.dat};
\addplot+ [darkgreen, densely dotted, opacity=0.7, forget plot] table [x expr={(\thisrow{muu}+\thisrow{mud})/2}, y=mud] {../code/data/LSM3F/eos_sigma_600.dat};
\addplot+ [darkgreen, solid,          opacity=0.7, forget plot] table [x expr={(\thisrow{muu}+\thisrow{mud})/2}, y=mus] {../code/data/LSM3F/eos_sigma_800.dat}; %\addlegendentry{$\mu_s$};
\addplot+ [darkgreen, densely dashed, opacity=0.7, forget plot] table [x expr={(\thisrow{muu}+\thisrow{mud})/2}, y=mus] {../code/data/LSM3F/eos_sigma_700.dat};
\addplot+ [darkgreen, densely dotted, opacity=0.7, forget plot] table [x expr={(\thisrow{muu}+\thisrow{mud})/2}, y=mus] {../code/data/LSM3F/eos_sigma_600.dat};
\addplot+ [blue,      solid,          opacity=0.7,            ] table [x expr={(\thisrow{muu}+\thisrow{mud})/2}, y=mue] {../code/data/LSM3F/eos_sigma_800.dat}; \addlegendentry{$\mu_e$};
\addplot+ [blue,      densely dashed, opacity=0.7, forget plot] table [x expr={(\thisrow{muu}+\thisrow{mud})/2}, y=mue] {../code/data/LSM3F/eos_sigma_700.dat};
\addplot+ [blue,      densely dotted, opacity=0.7, forget plot] table [x expr={(\thisrow{muu}+\thisrow{mud})/2}, y=mue] {../code/data/LSM3F/eos_sigma_600.dat};

\addplot+ [black, solid] {-100}; \addlegendentry{$m_\sigma=\SI{800}{\mega\electronvolt}$};
\addplot+ [black, densely dashed] {-100}; \addlegendentry{$m_\sigma=\SI{700}{\mega\electronvolt}$};
\addplot+ [black, densely dotted] {-100}; \addlegendentry{$m_\sigma=\SI{600}{\mega\electronvolt}$};
\addplot+ [black, solid, very thick, opacity=0.4] {-100}; \addlegendentry{$m_{\phantom{\sigma}}=0$};
\addlegendimage{empty legend} \addlegendentry{}

\nextgroupplot[
	xlabel={$\mu \, / \, \si{\mega\electronvolt}$}, ylabel={$n_i \, / \, (1/\si{\femto\meter\cubed})$},
	xmin=0, xmax=700, xtick distance=100, minor x tick num=9,
	ymin=-0.01, ymax=4.0, ytick distance=1.0, minor y tick num=9, restrict y to domain=-10:10,
	title={\subcaption{\label{fig:lsm:3-flavor-eos-density}Particle number densities}},
	legend cell align=left, legend pos=north west, legend columns=4, legend transposed,
];
%\coordinate (zoomplot) at (232, 1.32);
%\draw [draw=none, fill=gray!50] (250, -0.005) rectangle (350, 0.02);
%\draw [-Latex, gray] (300,0.03) to [out=90, in=0] (232, 0.75);
\addplot+ [blue,      solid, very thick, opacity=0.4, forget plot, domain=0:700] {0};
\addplot+ [red,       solid, very thick, opacity=0.4, forget plot, domain=0:700] {nq3(x)*nconv};
\addplot+ [purple,    dashed, very thick, opacity=0.4, forget plot, domain=0:700] {nq3(x)*nconv};
\addplot+ [darkgreen, dotted, very thick, opacity=0.4, forget plot, domain=0:700] {nq3(x)*nconv};

\addplot+ [red,       solid,          opacity=0.7,            ] table [x expr={(\thisrow{muu}+\thisrow{mud})/2}, y=nu] {../code/data/LSM3F/eos_sigma_800.dat}; \addlegendentry{$n_u$};
\addplot+ [red,       densely dashed, opacity=0.7, forget plot] table [x expr={(\thisrow{muu}+\thisrow{mud})/2}, y=nu] {../code/data/LSM3F/eos_sigma_700.dat};
\addplot+ [red,       densely dotted, opacity=0.7, forget plot] table [x expr={(\thisrow{muu}+\thisrow{mud})/2}, y=nu] {../code/data/LSM3F/eos_sigma_600.dat};
\addplot+ [darkgreen, solid,          opacity=0.7,            ] table [x expr={(\thisrow{muu}+\thisrow{mud})/2}, y=nd] {../code/data/LSM3F/eos_sigma_800.dat}; \addlegendentry{$n_d$};
\addplot+ [darkgreen, densely dashed, opacity=0.7, forget plot] table [x expr={(\thisrow{muu}+\thisrow{mud})/2}, y=nd] {../code/data/LSM3F/eos_sigma_700.dat};
\addplot+ [darkgreen, densely dotted, opacity=0.7, forget plot] table [x expr={(\thisrow{muu}+\thisrow{mud})/2}, y=nd] {../code/data/LSM3F/eos_sigma_600.dat};
\addplot+ [purple,    solid,          opacity=0.7,            ] table [x expr={(\thisrow{muu}+\thisrow{mud})/2}, y=ns] {../code/data/LSM3F/eos_sigma_800.dat}; \addlegendentry{$n_s$};
\addplot+ [purple,    densely dashed, opacity=0.7, forget plot] table [x expr={(\thisrow{muu}+\thisrow{mud})/2}, y=ns] {../code/data/LSM3F/eos_sigma_700.dat};
\addplot+ [purple,    densely dotted, opacity=0.7, forget plot] table [x expr={(\thisrow{muu}+\thisrow{mud})/2}, y=ns] {../code/data/LSM3F/eos_sigma_600.dat};
\addplot+ [blue,      solid,          opacity=0.7,            ] table [x expr={(\thisrow{muu}+\thisrow{mud})/2}, y=ne] {../code/data/LSM3F/eos_sigma_800.dat}; \addlegendentry{$n_e$};
\addplot+ [blue,      densely dashed, opacity=0.7, forget plot] table [x expr={(\thisrow{muu}+\thisrow{mud})/2}, y=ne] {../code/data/LSM3F/eos_sigma_700.dat};
\addplot+ [blue,      densely dotted, opacity=0.7, forget plot] table [x expr={(\thisrow{muu}+\thisrow{mud})/2}, y=ne] {../code/data/LSM3F/eos_sigma_600.dat};

\addplot+ [black, solid] {-1}; \addlegendentry{$m_\sigma=\SI{800}{\mega\electronvolt}$};
\addplot+ [black, densely dashed] {-1}; \addlegendentry{$m_\sigma=\SI{700}{\mega\electronvolt}$};
\addplot+ [black, densely dotted] {-1}; \addlegendentry{$m_\sigma=\SI{600}{\mega\electronvolt}$};
\addplot+ [black, solid, very thick, opacity=0.4] {-1}; \addlegendentry{$m_{\phantom{\sigma}}=0$};

\nextgroupplot[
	xlabel={$P        \, / \, (\si{\giga\electronvolt\per\femto\meter\cubed})$},
	ylabel={$\epsilon \, / \, (\si{\giga\electronvolt\per\femto\meter\cubed})$},
	xmin=-0.005, xmax=0.8, ymin=0, ymax=3.5, xtick distance=0.1, ytick distance=1.0, minor y tick num=9, restrict y to domain=-10:+10,
	title={\subcaption{\label{fig:lsm:3-flavor-eos-eos}Equation of state} },
	legend cell align=left, legend pos=north west, legend columns=4, legend transposed,
];
%\addplot+ [black!50!white, densely dashed, semithick, opacity=0.7, forget plot] table [x=P, y expr={3*\thisrow{P}+4*0.07774628475613433}] {../code/data/LSM2F/eos.dat}; % +4*P0, since both ϵ and P modified by P0
%\addplot+ [black, densely dashed, opacity=0.4, domain=0:0.1, forget plot] {3*x}; % +4*P0, since both ϵ and P modified by P0
\addplot+ [black, solid, opacity=0.7] table [x=P,y=epsilon] {../code/data/LSM3F/eos_sigma_800.dat}; \addlegendentry{$m_\sigma=\SI{800}{\mega\electronvolt}$};
\addplot+ [gray, densely dashed, opacity=0.7, forget plot] table [x=Porg,y=epsilonorg] {../code/data/LSM3F/eos_sigma_700.dat};
\addplot+ [black, densely dashed, opacity=0.7] table [x=P,y=epsilon] {../code/data/LSM3F/eos_sigma_700.dat}; \addlegendentry{$m_\sigma=\SI{700}{\mega\electronvolt}$};
\addplot+ [gray, densely dotted, opacity=0.7, forget plot] table [x=Porg,y=epsilonorg] {../code/data/LSM3F/eos_sigma_600.dat};
\addplot+ [black, densely dotted, opacity=0.7] table [x=P,y=epsilon] {../code/data/LSM3F/eos_sigma_600.dat}; \addlegendentry{$m_\sigma=\SI{600}{\mega\electronvolt}$};
\addplot+ [black, solid, very thick, opacity=0.4, domain=0:3.0] {3*x}; \addlegendentry{$m_{\phantom{\sigma}}=0$};
\addplot+ [gray, solid, domain=0:0.01] {-0.01}; \addlegendentry{uncorrected}
\addplot+ [black, solid, domain=0:0.01] {-0.01}; \addlegendentry{\phantom{un}corrected}
\addlegendimage{empty legend} \addlegendentry{}
\coordinate (zoomplot) at (0.55, 0.0);
\end{groupplot}
\node [anchor=south west] at (zoomplot) {
	\begin{tikzpicture}[trim axis left, trim axis right] % use axes as bounding box
	\begin{axis}[
		width=4cm, height=4cm,
		axis background/.style={fill=white},
		xmin=-0.004, xmax=+0.01, xtick distance=0.01, restrict x to domain=-0.03:+0.03,
		ymin=0, ymax=0.3, restrict y to domain=-1:+1,
	]
	\addplot+ [black, solid, opacity=0.7] table [x=P,y=epsilon] {../code/data/LSM3F/eos_sigma_800.dat};
	\addplot+ [gray, densely dashed, opacity=0.7] table [x=Porg,y=epsilonorg] {../code/data/LSM3F/eos_sigma_700.dat};
	\addplot+ [black, densely dashed, opacity=0.7] table [x=P,y=epsilon] {../code/data/LSM3F/eos_sigma_700.dat};
	\addplot+ [gray, densely dotted, opacity=0.7] table [x=Porg,y=epsilonorg] {../code/data/LSM3F/eos_sigma_600.dat};
	\addplot+ [black, densely dotted, opacity=0.7] table [x=P,y=epsilon] {../code/data/LSM3F/eos_sigma_600.dat};
	\addplot+ [black, dotted, opacity=0.7, domain=0:0.05] {3*x};
	\end{axis}
	\end{tikzpicture}
};
\end{tikzpicture}
\caption{\label{fig:lsm:3-flavor-eos}%
Properties of electrically charge neutral quark matter in $\beta$-equilibrium in the three-flavor quark meson model.
Upper panel \subref{fig:lsm:3-flavor-eos-parametrization} shows solutions to equation \eqref{eq:lsm3f:minsys},
middle panel \subref{fig:lsm:3-flavor-eos-density} the corresponding particle number densities \eqref{eq:lsm3f:particle_densities} and
lower panel \subref{fig:lsm:3-flavor-eos-eos} the corresponding equation of state $\epsilon(P)$ \textbf{\textcolor{gray}{before}} and \textbf{after} the Maxwell construction.
Three different values of $m_\sigma$ are used,
and the thickest lines show the massless solutions \eqref{eq:mit:densities_massless}, \eqref{eq:mit:eos_ur} and \eqref{eq:mit:chemical_potentials_massless_3f}.
\TODO{fix}
}
\end{figure}

\begin{figure}
\centering
\tikzsetnextfilename{3-flavor-density-ratios}
\begin{tikzpicture}
\begin{axis}[
	width=12cm, height=5cm,
	xlabel={$n_B \, / \, n_\text{sat}$}, ylabel={$n_i \, / \, n_B$},
	xmin=0, xmax=50, xtick distance=5, minor x tick num=4, restrict x to domain=0.01:100,
	ymin=-0.05, ymax=2.05, ytick distance=0.5, minor y tick num=4,
	declare function={sat=0.165;},
	legend cell align=left, legend columns=4, legend transposed,
]
\addplot [blue] table [x expr={(\thisrow{nu}+\thisrow{nd}+\thisrow{ns})/sat/3}, y expr={3*\thisrow{ne}/(\thisrow{nu}+\thisrow{nd}+\thisrow{ns})}] {../code/data/LSM3F/eos.dat};
\addlegendentry{$n_e$};
\addplot [red] table [x expr={(\thisrow{nu}+\thisrow{nd}+\thisrow{ns})/sat/3}, y expr={3*\thisrow{nu}/(\thisrow{nu}+\thisrow{nd}+\thisrow{ns})}] {../code/data/LSM3F/eos.dat};
\addlegendentry{$n_u$};
\addplot [darkgreen] table [x expr={(\thisrow{nu}+\thisrow{nd}+\thisrow{ns})/sat/3}, y expr={3*\thisrow{nd}/(\thisrow{nu}+\thisrow{nd}+\thisrow{ns})}] {../code/data/LSM3F/eos.dat};
\addlegendentry{$n_d$};
\addplot [purple] table [x expr={(\thisrow{nu}+\thisrow{nd}+\thisrow{ns})/sat/3}, y expr={3*\thisrow{ns}/(\thisrow{nu}+\thisrow{nd}+\thisrow{ns})}] {../code/data/LSM3F/eos.dat};
\addlegendentry{$n_s$};
\end{axis}
\end{tikzpicture}
\caption{\label{fig:lsm3f:3-flavor-density-ratios}%
Fractions of each particle density $n_i$ to the baryon density $n_B = (n_u+n_d+n_s)/3$
as a function of $n_B$ for the charge-neutral three-flavor quark matter in \cref{fig:lsm:3-flavor-eos}.
The nuclear saturation density is $n_\text{sat} = \SI{0.165}{\per\femto\meter\cubed}$.
}
\end{figure}

To find the equation of state we will follow the same procedure as in \cref{chap:lsm2f}.
The generalization of the system of equations \eqref{eq:lsm:minsys} with strange quarks that we need to solve,
subject to the constraints of chemical equilibrium \eqref{eq:lsm:chemical_equilibrium} and charge neutrality \eqref{eq:lsm:charge_neutrality},
is
\begin{subequations}
\begin{align}
	0 &= \pdv{\Omega}{\avg{\sigma_x}} , \label{eq:lsm3f:minsys_minx} \\
	0 &= \pdv{\Omega}{\avg{\sigma_y}} , \label{eq:lsm3f:minsys_miny} \\
	0 &= 2 \Big(\mu_u^2-m_u^2\Big)^\frac32 - \Big(\mu_d^2-m_d^2\Big)^\frac32 - \Big(\mu_s^2-m_s^2\Big)^\frac32 - \Big(\mu_e^2-m_e^2\Big)^\frac32 \label{eq:lsm:minsys3f_charge}, \\
	\mu_d &= \mu_u + \mu_e, \\
	\mu_s &= \mu_d .
\end{align}%
\label{eq:lsm3f:minsys}%
\end{subequations}%
This is a system of five equations for the six unknowns
$\avg{\sigma_x}$, $\avg{\sigma_y}$, $\mu_u$, $\mu_d$, $\mu_s$ and $\mu_e$.
Like in \cref{chap:lsm2f},
we parametrize solutions with $\avg{\sigma_x}$,
evaluate the particle densities \eqref{eq:master_intro:densities} as
\begin{equation}
	n_f = -\pdv{\Omega}{\mu_f} = \frac{N_c}{3 \pi^2} \Big( \mu_f^2 - m_f^2 \Big)^{\frac32}
	\qquad \text{and} \qquad
	n_e = -\pdv{\Omega}{\mu_e} = \frac{  1}{3 \pi^2} \Big( \mu_e^2 - m_e^2 \Big)^{\frac32},
\label{eq:lsm3f:particle_densities}
\end{equation}%
then calculate them, the pressure \eqref{eq:master_intro:pressure} and the energy density \eqref{eq:master_intro:energy_density},
and finally eliminate the free variable to get the equation of state $\epsilon(P)$.
The numerical implementation in \cref{sec:num:qstars2f} yields
the solutions, particle densities and equation of state shown in \cref{fig:lsm:3-flavor-eos}:
\begin{itemize}
\item There is a non-strange crossover (for $m_\sigma \geq \SI{800}{\mega\electronvolt}$)
      or discontinuous phase transition (for $m_\sigma < \SI{800}{\mega\electronvolt}$)
      after $\mu=\SI{300}{\mega\electronvolt}$, as in the two-flavor case in \cref{fig:lsm:2-flavor-eos}.
      Even the strange minimum $\avg{\sigma_y}$ moves a bit during this transition
      due to the cross term $\lambda_1 \avg{\sigma_x}^2 \avg{\sigma_y}^2 / 2$ in the meson potential \eqref{eq:lsm:potential_tree_3f} changing.
      Accordingly, the equations of state for the two-flavor and three-flavors models are quite similar up to $P \leq \SI{0.05}{\giga\electronvolt\per\femto\meter\cubed}$.
\item There is a strange crossover after the non-strange crossover or phase transition.
      For example, for $m_\sigma=\SI{800}{\mega\electronvolt}$, it starts at $\mu \approx \SI{370}{\mega\electronvolt}$ 
      or $\mu_s = m_y \approx \SI{410}{\mega\electronvolt}$.
      This is \emph{below} the fitted vacuum mass $m_s=\SI{429}{\mega\electronvolt}$
      because \emph{both} $m_x$ and $m_y$ drop during the non-strange transition, as explained above.
      Moreover, it is \emph{always} a crossover and \emph{never} a phase transition,
      is significantly slower than the non-strange transition,
      and the strange quark retains a noticeable mass when the non-strange quarks have lost theirs.
      During this crossover, the equation of state gently ramps up from the two-flavor level towards a second plateau
      and softens compared to the two-flavor equation of state.
\item Before the strange crossover, the up and down quark chemical potentials $\mu_u$ and $\mu_d$
      spread in order to ensure charge neutrality together with the electrons.
      After the strange crossover, however, the strange quark kicks out the electron
      and provides charge neutrality together with the up and down quarks,
      as the quark chemical potentials find back to each other in the ultra-relativistic limit $\mu_u \rightarrow \mu_d = \mu_s \rightarrow \mu$.
      In contrast to the two-flavor case,
      the isospin chemical potential $\mu_I=(\mu_u-\mu_d)/2$ never exceeds half the pion mass,
      so neglecting pion condensation is a consistent assumption for all $\mu$ with three flavors.
\item As shown in \cref{fig:lsm3f:3-flavor-density-ratios},
      each quark density converges towards a third of the total quark density $n = 3 n_B$ as it increases.
      At low density the strange quark is absent and there are twice as many down quarks as up quarks,
      like in \cref{chap:lsm2f},
      and the strange quark enters for $n_B \geq 4 n_\text{sat}$.
      The fraction of up quarks is the same at all densities.
\item The ultra-relativistic solution in \cref{sec:mit:eos} is recovered
      as $\avg{\sigma_x} \rightarrow \avg{\sigma_y} \rightarrow 0$, 
\item The equation of state satisfies the criteria of causality \eqref{eq:nstars:stability_speed_of_sound}
      and stability \eqref{eq:nstars:stability_pressure_energy_density}.
\item Like in \cref{chap:lsm2f}, we use the Maxwell construction to correct the ambiguous equations of state with phase transitions as described in \cite[equation (4.69)]{ref:master_francesco}.
\end{itemize}

We again modify the equation of state with bag constants $B$ with the shift \eqref{eq:mit:bag_shift}
and determine their upper bounds by solving the three-flavor inequality \eqref{eq:mit:bag_stability}, owing to the strange matter hypothesis.
This yields the upper bag constant bounds
\begin{subequations}
\begin{align}
	%\text{no solution!} \Big(\text{with } m_\sigma = \SI{700}{\mega\electronvolt}\Big), \label{eq:lsm3f:bag_upper_bound_700} \\
	%B(f_\pi)^\frac14 < \SI{46.5}{\mega\electronvolt}, \quad \text{or} \quad B(0)^\frac14 > \SI{156.5}{\mega\electronvolt} \qquad \Big(\text{with } m_\sigma = \SI{800}{\mega\electronvolt}\Big). \label{eq:lsm3f:bag_upper_bound_800}
	B \leq (\SI{112.0}{\mega\electronvolt})^4           \quad \Big(\text{or } B-\pot_0 \leq (\SI{226.4}{\mega\electronvolt})^4\Big) \qquad \Big(m_\sigma = \SI{600}{\mega\electronvolt}\Big), \label{eq:lsm3f:bag_upper_bound_600} \\
	B \leq \phantom{0}(\SI{68.2}{\mega\electronvolt})^4 \quad \Big(\text{or } B-\pot_0 \leq (\SI{231.4}{\mega\electronvolt})^4\Big) \qquad \Big(m_\sigma = \SI{700}{\mega\electronvolt}\Big), \label{eq:lsm3f:bag_upper_bound_700} \\
	B \leq \phantom{0}(\SI{27.0}{\mega\electronvolt})^4 \quad \Big(\text{or } B-\pot_0 \leq (\SI{241.3}{\mega\electronvolt})^4\Big) \qquad \Big(m_\sigma = \SI{800}{\mega\electronvolt}\Big). \label{eq:lsm3f:bag_upper_bound_800}
\end{align}%
\label{eq:lsm3f:bag_upper_bound}%
\end{subequations}%
These are so close to the lower bounds \eqref{eq:lsm:bag_lower_bound} that the inequalities practically become equalities,
and the bag constant is more or less fixed for a given $m_\sigma$ \emph{if} the strange matter hypothesis holds, but can take greater values if it does not.
Like before, we focus on the lowest bag constants since they generate stiffer equations of state and greater maximum masses.

Note that if one uses a single renormalization scale $\Lambda$ and let the vacuum quark masses move from their fit values,
like \cite{ref:master_berge} and as discussed in \cref{sec:lsm3f:parameter_fit},
one obtains the rather different bounds $B^\frac14 \leq \{6.3,48.2\} \, \si{\mega\electronvolt}$ for $m_\sigma=\{700,800\}\,\si{\mega\electronvolt}$,
whereas the potential is broken for $m_\sigma=\SI{600}{\mega\electronvolt}$.
These results are impossible to reconcile with the lower bag bounds \eqref{eq:lsm:bag_lower_bound}
and is one of the reasons for why we choose to go with two separate renormalization scales here.

\pgfplotsset{
	mesh line legend/.style={legend image code/.code=\meshlinelegend#1},
}
%% Code for the coloured line legend
%% adapted from https://tex.stackexchange.com/a/59075 with pgfplots manual "line legend" size (0.6cm, 0.1cm)
\makeatletter
\long\def\meshlinelegend#1{%
    \scope[%
        #1,
        /pgfplots/mesh/rows=1,
        /pgfplots/mesh/cols=4,
        /pgfplots/mesh/num points=,
        /tikz/x={(0.6cm,0cm)}, %/tikz/x={(0.44237cm,0cm)},
        /tikz/y={(0cm,0.1cm)}, %/tikz/y={(0cm,0.23932cm)},
        /tikz/z={(0.0cm,0cm)},
        scale=1.0, %scale=0.4,
    ]
    \let\pgfplots@metamax=\pgfutil@empty
    \pgfplots@curplot@threedimtrue

    \pgfplotsplothandlermesh
    \pgfplotstreamstart

    \def\simplecoordinate(##1,##2,##3){%
        \pgfmathparse{1000*(##3)}%
        \pgfmathfloatparsenumber\pgfmathresult
        \let\pgfplots@current@point@meta=\pgfmathresult
        \pgfplotstreampoint{\pgfqpointxyz@orig{##1}{##2}{##3}}%
    }%

    \simplecoordinate(0,0,0)
    \simplecoordinate(0.125,0,0.125)
    \simplecoordinate(0.25,0,0.25)
    \simplecoordinate(0.375,0,0.375)
    \simplecoordinate(0.5,0,0.5)
    \simplecoordinate(0.625,0,0.625)
    \simplecoordinate(0.75,0,0.75)
    \simplecoordinate(0.875,0,0.875)
    \simplecoordinate(1,0,1)

    \pgfplotstreamend
    \pgfusepath{stroke}
    \endscope
}%
\makeatother
%% End code for the coloured line legend

\begin{figure}[p]
\centering
\tikzsetnextfilename{3-flavor-mass-radius}
\begin{tikzpicture}
\begin{groupplot}[
	group style={group size={3 by 1}, vertical sep=0cm, horizontal sep=0.3cm},
	width=6cm, height=6cm,
	xmin=5, xmax=20, ymin=0.5, ymax=2.5, xtick distance=5, ytick distance=0.5, minor tick num=4, grid=major,
	point meta=explicit, point meta min=33, point meta max=36,
	%colorbar horizontal, colormap name=plasmarev, colorbar style={xlabel=$\log_{10} (P_c \, / \, \si{\pascal})$, xtick distance=1, minor x tick num=9, at={(0.5,1.03)}, anchor=south, xticklabel pos=upper},
	/tikz/declare function={
		e0 = 4.266500881855304e+37;
	},
	legend columns=2, legend style={anchor=north, at={(0.5, 0.96)}},
]
\tikzset{
	Bpin/.style={gray, sloped, allow upside down=true, rotate=180, yshift=+0.4cm, font=\small},
}
\nextgroupplot[
	xlabel={$R \, / \, \si{\kilo\meter}$ },
	ylabel={$M \, / \, M_\odot$}, %title={Mass-radius diagram for 2-flavor quark stars }, title style={yshift=2.0cm},
	title = {\subcaption{\label{fig:lsm:3-flavor-mass-radius-600}$m_\sigma = \SI{600}{\mega\electronvolt}$} $B^\frac14 = \{111,131,151\} \, \si{\mega\electronvolt}$},
];
\addplot+ [solid, gray, opacity=0.5] table [x=R, y=M, meta expr={log10(\thisrow{P}*e0)}] {../code/data/LSM2F/stars_sigma_600_B14_111.dat}; % node [Bpin, pos=0.920] {$B = (\SI{27}{\mega\electronvolt})^4$};
\addplot+ [solid, gray, opacity=0.5, forget plot] table [x=R, y=M, meta expr={log10(\thisrow{P}*e0)}] {../code/data/LSM2F/stars_sigma_600_B14_131.dat}; % node [Bpin, pos=0.920] {$B = (\SI{27}{\mega\electronvolt})^4$};
\addplot+ [solid, gray, opacity=0.5, forget plot] table [x=R, y=M, meta expr={log10(\thisrow{P}*e0)}] {../code/data/LSM2F/stars_sigma_600_B14_151.dat}; % node [Bpin, pos=0.920] {$B = (\SI{27}{\mega\electronvolt})^4$};
\addlegendentry{$N_f \!=\! 2$};
\addplot+ [solid, mesh, mesh line legend] table [x=R, y=M, meta expr={log10(\thisrow{P}*e0)}] {../code/data/LSM3F/stars_sigma_600_B14_111.dat}; % node [Bpin, pos=0.920] {$B = (\SI{27}{\mega\electronvolt})^4$};
\addplot+ [solid, mesh, forget plot] table [x=R, y=M, meta expr={log10(\thisrow{P}*e0)}] {../code/data/LSM3F/stars_sigma_600_B14_131.dat}; % node [Bpin, pos=0.920] {$B = (\SI{27}{\mega\electronvolt})^4$};
\addplot+ [solid, mesh, forget plot] table [x=R, y=M, meta expr={log10(\thisrow{P}*e0)}] {../code/data/LSM3F/stars_sigma_600_B14_151.dat}; % node [Bpin, pos=0.920] {$B = (\SI{27}{\mega\electronvolt})^4$};
\addlegendentry{$N_f \!=\! 3$};

\nextgroupplot[
	xlabel={$R \, / \, \si{\kilo\meter}$},
	yticklabels={,,},
	title = {\subcaption{\label{fig:lsm:3-flavor-mass-radius-700}$m_\sigma = \SI{700}{\mega\electronvolt}$} $B^\frac14 = \{68,88,108\} \, \si{\mega\electronvolt}$},
	colorbar horizontal, colormap name=plasmarev, colorbar style={width=11cm, ylabel=$\log_{10} (P_c \, / \, \si{\pascal})$, ylabel style={rotate=-90}, xtick distance=1, minor x tick num=9, at={(0.5,-0.3)}, anchor=north, xticklabel pos=lower},
];
\addplot+ [solid, gray, opacity=0.5] table [x=R, y=M, meta expr={log10(\thisrow{P}*e0)}] {../code/data/LSM2F/stars_sigma_700_B14_68.dat}; % node [Bpin, pos=0.920] {$B = (\SI{27}{\mega\electronvolt})^4$};
\addplot+ [solid, gray, opacity=0.5, forget plot] table [x=R, y=M, meta expr={log10(\thisrow{P}*e0)}] {../code/data/LSM2F/stars_sigma_700_B14_88.dat}; % node [Bpin, pos=0.920] {$B = (\SI{27}{\mega\electronvolt})^4$};
\addplot+ [solid, gray, opacity=0.5, forget plot] table [x=R, y=M, meta expr={log10(\thisrow{P}*e0)}] {../code/data/LSM2F/stars_sigma_700_B14_108.dat}; % node [Bpin, pos=0.920] {$B = (\SI{27}{\mega\electronvolt})^4$};
\addlegendentry{$N_f=2$};
\addplot+ [solid, mesh, mesh line legend] table [x=R, y=M, meta expr={log10(\thisrow{P}*e0)}] {../code/data/LSM3F/stars_sigma_700_B14_68.dat};  % node [Bpin, pos=0.920] {$B = (\SI{27}{\mega\electronvolt})^4$};
\addplot+ [solid, mesh, forget plot] table [x=R, y=M, meta expr={log10(\thisrow{P}*e0)}] {../code/data/LSM3F/stars_sigma_700_B14_88.dat};  % node [Bpin, pos=0.920] {$B = (\SI{27}{\mega\electronvolt})^4$};
\addplot+ [solid, mesh, forget plot] table [x=R, y=M, meta expr={log10(\thisrow{P}*e0)}] {../code/data/LSM3F/stars_sigma_700_B14_108.dat}; % node [Bpin, pos=0.920] {$B = (\SI{27}{\mega\electronvolt})^4$};
\addlegendentry{$N_f=3$};

\nextgroupplot[
	xlabel={$R \, / \, \si{\kilo\meter}$},
	yticklabels={,,},
	title = {\subcaption{\label{fig:lsm:3-flavor-mass-radius-800}$m_\sigma = \SI{800}{\mega\electronvolt}$} $B^\frac14 = \{27,47,67\} \, \si{\mega\electronvolt}$},
];
\addplot+ [solid, gray, opacity=0.5] table [x=R, y=M, meta expr={log10(\thisrow{P}*e0)}] {../code/data/LSM2F/stars_sigma_800_B14_27.dat}; % node [Bpin, pos=0.920] {$B = (\SI{27}{\mega\electronvolt})^4$};
\addplot+ [solid, gray, opacity=0.5, forget plot] table [x=R, y=M, meta expr={log10(\thisrow{P}*e0)}] {../code/data/LSM2F/stars_sigma_800_B14_47.dat}; % node [Bpin, pos=0.920] {$B = (\SI{27}{\mega\electronvolt})^4$};
\addplot+ [solid, gray, opacity=0.5, forget plot] table [x=R, y=M, meta expr={log10(\thisrow{P}*e0)}] {../code/data/LSM2F/stars_sigma_800_B14_67.dat}; % node [Bpin, pos=0.920] {$B = (\SI{27}{\mega\electronvolt})^4$};
\addlegendentry{$N_f=2$};
\addplot+ [solid, mesh, mesh line legend] table [x=R, y=M, meta expr={log10(\thisrow{P}*e0)}] {../code/data/LSM3F/stars_sigma_800_B14_27.dat}; % node [Bpin, pos=0.920] {$B = (\SI{27}{\mega\electronvolt})^4$};
\addplot+ [solid, mesh, forget plot] table [x=R, y=M, meta expr={log10(\thisrow{P}*e0)}] {../code/data/LSM3F/stars_sigma_800_B14_47.dat}; % node [Bpin, pos=0.920] {$B = (\SI{27}{\mega\electronvolt})^4$};
\addplot+ [solid, mesh, forget plot] table [x=R, y=M, meta expr={log10(\thisrow{P}*e0)}] {../code/data/LSM3F/stars_sigma_800_B14_67.dat}; % node [Bpin, pos=0.920] {$B = (\SI{27}{\mega\electronvolt})^4$};
\addlegendentry{$N_f=3$};
\node[scale=0.75] at (11.565, 1.630) {\goldenstar};

\end{groupplot}
\node [anchor=south, yshift=+1.5cm] at (group c2r1.north) {Three-flavor quark-meson model quark stars};
\end{tikzpicture}
\caption{\label{fig:lsm:3-flavor-mass-radius}%
	Mass-radius solutions of the Tolman-Oppenheimer-Volkoff equation \eqref{eq:master_intro:tov} parametrized by the central pressure $P_c$,
	using equations of state for three-flavor quark matter in \cref{fig:lsm:3-flavor-eos-eos} modified by the bag shift \eqref{eq:mit:bag_shift} with bag constants at and above the bounds \eqref{eq:lsm:bag_lower_bound}.
}
\end{figure}

\begin{figure}
\tikzsetnextfilename{3-flavor-extreme-star}
\begin{tikzpicture}
\begin{groupplot}[
	group style={group size={2 by 2}, horizontal sep=1.2cm, vertical sep=0.4cm},
	width=8cm, height=6cm,
	ylabel style={yshift=-0.2cm},
	enlargelimits=false, xtick distance=1.0, minor xtick={0,0.1,...,11.6},
	legend cell align=left,
]
\nextgroupplot[
	xticklabels={,,},
	ymax=1.5, ytick distance=0.5, minor y tick num=4,
	ylabel={$\{\epsilon,P\} \, / \, (\si{\giga\electronvolt\per\femto\meter\cubed})$},
];
\addplot+ [blue] table [x=r, y=epsilon] {../code/data/LSM3F/star_sigma_800_B14_27_Pc_0.0009376.dat}; \addlegendentry{$\epsilon$};
\addplot+ [red] table [x=r, y=P] {../code/data/LSM3F/star_sigma_800_B14_27_Pc_0.0009376.dat}; \addlegendentry{$P$};
\nextgroupplot[
	ylabel=$m \, / \, M_\odot$,
	xticklabels={,,},
	ymax=2, ytick distance=0.5, minor y tick num=4,
];
\addplot+ [black] table [x=r, y=m] {../code/data/LSM3F/star_sigma_800_B14_27_Pc_0.0009376.dat};
\nextgroupplot[
	xlabel=$r \, / \, \si{\kilo\meter}$,
	ylabel=$\mu \, / \, \si{\mega\electronvolt}$,
	ymin=300, ymax=500, ytick distance=50, minor y tick num=4,
];
\addplot+ [black] table [x=r, y=muQ] {../code/data/LSM3F/star_sigma_800_B14_27_Pc_0.0009376.dat};
\nextgroupplot[
	xlabel=$r \, / \, \si{\kilo\meter}$,
	ylabel=$n_i \, / \, n_\text{sat}$,
	ymax=15, ytick distance=5, minor y tick num=4,
];
\addplot+ [red] table [x=r, y=nu] {../code/data/LSM3F/star_sigma_800_B14_27_Pc_0.0009376.dat}; \addlegendentry{$n_u$};
\addplot+ [darkgreen] table [x=r, y=nd] {../code/data/LSM3F/star_sigma_800_B14_27_Pc_0.0009376.dat}; \addlegendentry{$n_d$};
\addplot+ [purple] table [x=r, y=ns] {../code/data/LSM3F/star_sigma_800_B14_27_Pc_0.0009376.dat}; \addlegendentry{$n_s$};
\addplot+ [blue] table [x=r, y=ne] {../code/data/LSM3F/star_sigma_800_B14_27_Pc_0.0009376.dat}; \addlegendentry{$n_e$};
\addplot+ [black, dashed, forget plot] table [x=r, y expr={(\thisrow{nu}+\thisrow{nd}+\thisrow{ns})/3}] {../code/data/LSM3F/star_sigma_800_B14_27_Pc_0.0009376.dat};
% cheat legend
\addplot+ [draw=none, black, solid] table [x=r, y expr={(\thisrow{nu}+\thisrow{nd}+\thisrow{ns})/3}] {../code/data/LSM3F/star_sigma_800_B14_27_Pc_0.0009376.dat}; \addlegendentry{$n_B$};
\end{groupplot}
\node (title) at ($(group c1r1.north)!0.5!(group c2r1.north)$) [above, yshift=\pgfkeysvalueof{/pgfplots/every axis title shift}] {\goldenstar Maximum mass star ($m_\sigma=\SI{800}{\mega\electronvolt}$, $B^\frac14 = \SI{27}{\mega\electronvolt}$, $P_c=10^{34.60} \, \si{\pascal}$)};
\end{tikzpicture}
\caption{\label{fig:lsm:3-flavor-star}%
	Radial profiles for the
	pressure $P$,
	energy density $\epsilon$,
	cumulative mass $m$,
	quark chemical potential $\mu$,
	particle densities $n_i$
	and baryon density $n_B = (n_u+n_d+n_s)/3$
	for the maximum mass three-flavor quark star \goldenstar in \cref{fig:lsm:3-flavor-mass-radius}.
	The nuclear saturation density is $n_\text{sat} = \SI{0.165}{\per\femto\meter\cubed}$.
	\TODO{\goldenstar color}
}

\end{figure}

\section{Quark star solutions}

With the numerical implementation in \cref{sec:num:qstars2f},
we now modify the equations of state in \cref{fig:lsm:3-flavor-eos-eos}
with the bag shift \eqref{eq:mit:bag_shift}
using bag constants above the lower bounds \eqref{eq:lsm:bag_lower_bound}
and solve the corresponding Tolman-Oppenheimer-Volkoff equations \eqref{eq:master_intro:tov}.
This produces the mass-radius solutions in \cref{fig:lsm:3-flavor-mass-radius}:
\begin{itemize}
\item For stars with low central pressure,
      the quark chemical potential $\mu$ does not reach large enough levels to activate the strange quark.
      Then the three-flavor equation of state and mass-radius curve track their two-flavor siblings almost perfectly.
      As the two models use different vacuum potentials, it is somewhat surprising that the correspondence is so good.
\item For stars with greater central pressure, $\mu$ reaches sufficient levels and the strange quark is present in the core.
      Exactly at the threshold, we see that the mass-radius curve grows apart from the two-flavor curve into a new branch.
      This branch has lower masses because of the softening of the equation of state as it ramps up during the strange crossover.
\item The masses and radii are comparable to the two-flavor results in \cref{sec:lsm:stars}, except that the mass is lowered for stars with strange quarks.
      For stars that just satisfy the lower bag bounds \eqref{eq:lsm:bag_lower_bound} and hence respect instability of two-flavor quark matter with respect to hadronic matter,
      we find maximum masses $1.6 \, M_\odot\leq M \leq 1.8 \, M_\odot$ and corresponding radii $\SI{11}{\kilo\meter} \leq R \leq \SI{12}{\kilo\meter}$ depending on the precise mass $m_\sigma$.
      Greater bag constants violate the strange matter hypothesis and result in only lighter and smaller stars.
\end{itemize}

In \cref{fig:lsm:3-flavor-star} we take a more detailed look at the maximum mass star that has no phase transition and respects the strange matter hypothesis:
\begin{itemize}
\item Apart from the presence of strange quarks, many features are similar to that of the two-flavor star we discussed on page \pageref{list:lsm:2-flavor-star-discussion}.
\item The star is composed of three-flavor quark matter including strange quarks for $r \leq \SI{6.5}{\kilo\meter}$
      and two-flavor quark matter for $r \geq \SI{6.5}{\kilo\meter}$ out to the surface $R=\SI{11.6}{\kilo\meter}$.
\item With a maximum chemical potential $\mu=\SI{465}{\mega\electronvolt}$, the strange quark is much heavier than the up and down quarks at the center, as seen from \cref{fig:lsm:3-flavor-eos-parametrization}.
      Thus, the star is far from realizing ultra-relativistic three-flavor quark matter.
\item The maximum central chemical potential is comparable in the two-flavor and three-flavor models.
      If we assert that this carries over to a four-flavor model that also includes the charm quark with constituent mass $m_c > \SI{1}{\giga\electronvolt}$ \cite{ref:pdg_review_2021},
      it would only be present for extreme central pressures far beyond that of the maximum mass star.
      This is why we only need to include the three lightest $\{u,d,s\}$ quarks when modeling stable quark stars.
%\item If the core of this star is used to model the core of a hybrid neutron star with a quark model for $n_B > 4 n_\text{sat}$,
      %this core would extend for $R'=\SI{5}{\kilo\meter}$, weigh in at $M' = m(R') = 0.4 \, M_\odot$, have a central pressure $P_c'=p(R')=\SI{0.12}{\giga\electronvolt\per\femto\meter\cubed}$ and be entirely composed of three-flavor quark matter.
\end{itemize}
Since these pure quark stars consist of three-flavor quark matter only in the core
and two-flavor quark matter out to the surface,
and the latter is unstable with respect to hadronic matter,
they are unlikely to be found in nature.
The only hope of realizing pure three-flavor quark stars is
if the strange matter hypothesis is true and they consist of strange quark matter \emph{everywhere}.
Looking back at \cref{fig:lsm:3-flavor-eos-density}, the strange quark is present for all $m_\sigma$ only when $\mu \geq \SI{350}{\mega\electronvolt}$.
The only way of creating a strange quark star with our model is to increase $B$
so much that after the bag shift \eqref{eq:mit:bag_shift},
we have $\mu \geq \SI{350}{\mega\electronvolt}$ at the surface defined by $P=0$.
Now look at $P(\mu)$ in \cref{fig:lsm:3-flavor-pressure};
in order for $P(\mu \geq \SI{350}{\mega\electronvolt}) \leq 0$, we need to shift $P(\mu) \rightarrow P(\mu) - B$
with $\smash{B^{1/4}} \geq \SI{120}{\mega\electronvolt}$.
This is the minimum required bag constant for the strange quark to appear out to the surface for all $m_\sigma$.
Unfortunately, this \emph{exceeds} even the greatest upper bound \eqref{eq:lsm3f:bag_upper_bound_600},
\emph{which was calculated precisely by assuming the strange matter hypothesis}.
This means that \emph{it is not possible to self-consistently describe pure strange quark stars with this model!}

\begin{figure}[b!]
\centering
\tikzsetnextfilename{3-flavor-pressure}
\begin{tikzpicture}
\begin{axis}[
	width=13cm, height=7cm,
	xmin=250, xmax=500, ymin=-75, ymax=250,
	xtick distance=25, ytick distance=50, minor tick num=4, grid=major,
	xlabel = {$\mu \, / \, \si{\mega\electronvolt}$},
	ylabel = {$\text{sign}(P) \cdot \abs{P}^\frac14 \, / \, \si{\mega\electronvolt}$},
	declare function = {conv = 1.3061920575081208e-10;},
	legend pos = south east,
]
\addplot+ [black, densely dotted] table [x expr={(\thisrow{muu}+\thisrow{mud}) / 2}, y expr={sign(\thisrow{Porg})*abs(\thisrow{Porg}/conv)^0.25}] {../code/data/LSM3F/eos_sigma_600.dat}; \addlegendentry{$m_\sigma=\SI{600}{\mega\electronvolt}$};
\addplot+ [black, densely dashed] table [x expr={(\thisrow{muu}+\thisrow{mud}) / 2}, y expr={sign(\thisrow{Porg})*abs(\thisrow{Porg}/conv)^0.25}] {../code/data/LSM3F/eos_sigma_700.dat}; \addlegendentry{$m_\sigma=\SI{700}{\mega\electronvolt}$};
\addplot+ [black, solid]          table [x expr={(\thisrow{muu}+\thisrow{mud}) / 2}, y expr={sign(\thisrow{Porg})*abs(\thisrow{Porg}/conv)^0.25}] {../code/data/LSM3F/eos_sigma_800.dat}; \addlegendentry{$m_\sigma=\SI{800}{\mega\electronvolt}$};
\end{axis}
\end{tikzpicture}
\caption{\label{fig:lsm:3-flavor-pressure}%
	Signed fourth root of the pressure $P(\mu)$ as a function of the quark chemical potential $\mu$
	corresponding to the equations of state in \cref{fig:lsm:3-flavor-eos-eos},
	before the Maxwell construction and the bag shift \eqref{eq:mit:bag_shift}.
	The funny behavior in the lower left is caused by the phase transition,
	and is corrected with a boring straight line after the Maxwell construction.
}
\end{figure}

\section{Summary}

In this chapter we have modeled quark stars with the three-flavor quark-meson model.
We found that the addition of the strange quark lowered the maximum masses of pure quark stars to $1.6 \, M_\odot \leq M \leq 1.8 \, M_\odot$
with corresponding radii $\SI{11}{\kilo\meter} \leq R \leq \SI{12}{\kilo\meter}$,
using $\SI{600}{\mega\electronvolt} \leq m_\sigma \leq \SI{800}{\mega\electronvolt}$ and the lowest instability-respecting bag constants \eqref{eq:lsm:bag_lower_bound}.
If the strange matter hypothesis is violated, one can use greater bag constants and produce smaller and less massive stars.
The three-flavor stars are less massive than their two-flavor counterparts
due to their softer equations of state.
Only close to the maximum mass star does the pressure and density become sufficiently high
to activate the presence of the strange quark.

As with the two-flavor model, chiral symmetry restoration happens in a crossover for $m_\sigma \geq \SI{800}{\mega\electronvolt}$,
but a first-order phase transition for $m_\sigma < \SI{800}{\mega\electronvolt}$.
The non-strange transition is followed by a much slower strange crossover for all $m_\sigma$.

At tree-level, it was again impossible to fit measured masses $\SI{400}{\mega\electronvolt} \leq m_\sigma \leq \SI{550}{\mega\electronvolt}$ without breaking the grand potential.
Ideally, a fully consistent calculation in the one-loop large-$N_c$ limit like the one presented in \cref{sec:lsm2f:refinement}
should be carried out also for the three-flavor model,
where loop effects are taken into account when fitting the parameters.
Unfortunately, no such calculation has been done yet,
and we rely on the lesson learned back there:
that inconsistently fit $\SI{600}{\mega\electronvolt} \leq m_\sigma \leq \SI{800}{\mega\electronvolt}$
are representative of consistently fit masses $\SI{400}{\mega\electronvolt} \leq m_\sigma \leq \SI{600}{\mega\electronvolt}$
that are compatible with measurements.

Unfortunately, we fail to model pure strange quark stars consistently with this model.
The stars we found consisted of unstable two-flavor quark matter out to the surface.
For them to be stable, the strange matter hypothesis must be true and the strange quark must exist out to the surface.
This could only be achieved by
increasing the bag constant beyond its upper bound \eqref{eq:lsm3f:bag_upper_bound} that was calculated precisely by \emph{assuming the strange matter hypothesis}.
