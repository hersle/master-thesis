\chapter{Three-flavor quark-meson model}
\label{chap:lsm3f}

With three flavors, the effective low-energy degrees of freedom of quantum chromodynamics are the scalar and pseudoscalar mesons. %\cite{ref:lsm3f_details}
In this chapter we will study the generalization of the quark-meson model in \cref{chap:lsm2f} to three flavors,
thereby throwing the strange quark into the mix.
Most features of this model behave as natural generalizations of the two-flavor model.
For example, we will handle a strange quark condensate in parallel with the common up and down quark condensates.
Whereas there is a unique way of fixing the parameters in the two-flavor model, however,
we will see that the three-flavor model presents us with a multitude of ways of doing this,
so that some experimental values must be predicted rather than fit.

Three-flavor models are particularly interesting for quark stars.
If the strange matter hypothesis is true,
there is a greater possibility of finding stars that consist of stable three-flavor quark matter
as opposed to unstable two-flavor quark matter, both with respect to hadronic matter.
Due to the heavier mass of the strange quark,
we expect its effects to become apparent in the mass-radius diagram
only for stars that have a large central pressure and thus large quark chemical potential.

\textit{This chapter is inspired by references \cite{ref:lsm3f} and \cite{ref:lsm3f_details}.}

\section{Lagrangian, vacuum and symmetry breaking}

The Lagrangian density of the three-flavor model is \cite{ref:lsm3f,ref:lsm3f_details}
\begin{equation}
	\lagr = \bar{q} \Big[ i \slashed\partial + \mu \gamma^0 - g \big(\sigma_a + i \gamma^5 \pi_a\big) T_a \Big] q + \trace\Big[(\partial_\mu \phi)^\dagger (\partial_\mu \phi)\Big] - \pot(\sigma, \pi)
\label{eq:lsm:lagrangian3f}
\end{equation}
with the meson potential
\begin{equation}
	\pot(\sigma, \pi) = m^2 \trace\Big[\phi^\dagger \phi\Big] + \lambda_1 \Big[\trace(\phi^\dagger \phi)\Big]^2 + \lambda_2 \trace\Big[(\phi^\dagger \phi)^2\Big] - \trace\Big[H(\phi+\phi^\dagger)\Big].
\label{eq:lsm:potential3f}
\end{equation}
The quark fields $q$ and chemical potentials $\mu$ now have $N_f=3$ flavors $\{u,d,s\}$,
while $\sigma_a$ and $\pi_a$ are members of the scalar $J^P=0^+$ and pseudoscalar $J^P=0^-$ meson nonets,
packed into the $N_f \times N_f$ matrix $\phi = \phi_a T_a = (\sigma_a + i \pi_a) T_a$.
The eight $SU(3)$ generators $T_a = \lambda_a/2$ are extended with the identity with the common normalization $\trace[T_a T_b] = \delta_{ab}/2$ of the Gell-Mann matrices
\begin{equation}
\begin{NiceMatrixBlock}[auto-columns-width]
\setlength{\arraycolsep}{0pt}
\NiceMatrixOptions{cell-space-limits = 5pt}
\begin{aligned}
	\lambda_0 &= \begin{bNiceMatrix} \smash{\sqrt{\frac{2}{3}}} &  0 &  0 \\ 0 &  \smash{\sqrt{\frac{2}{3}}} &  0 \\ 0 & 0 &  \smash{\sqrt{\frac{2}{3}}} \end{bNiceMatrix}, &\quad&
	\lambda_1 &= \begin{bNiceMatrix}                          0 &  1 &  0 \\ 1 &                           0 &  0 \\ 0 & 0 &                           0 \end{bNiceMatrix}, &\quad&
	\lambda_2 &= \begin{bNiceMatrix}                          0 & -i &  0 \\ i &                           0 &  0 \\ 0 & 0 &                           0 \end{bNiceMatrix}, \\
	\lambda_3 &= \begin{bNiceMatrix}                          1 &  0 &  0 \\ 0 &                          -1 &  0 \\ 0 & 0 &                           0 \end{bNiceMatrix}, &\quad&
	\lambda_4 &= \begin{bNiceMatrix}                          0 &  0 &  1 \\ 0 &                           0 &  0 \\ 1 & 0 &                           0 \end{bNiceMatrix}, &\quad&
	\lambda_5 &= \begin{bNiceMatrix}                          0 &  0 & -i \\ 0 &                           0 &  0 \\ i & 0 &                           0 \end{bNiceMatrix}, \\
	\lambda_6 &= \begin{bNiceMatrix}                          0 &  0 &  0 \\ 0 &                           0 &  1 \\ 0 & 1 &                           0 \end{bNiceMatrix}, &\quad&
	\lambda_7 &= \begin{bNiceMatrix}                          0 &  0 &  0 \\ 0 &                           0 & -i \\ 0 & i &                           0 \end{bNiceMatrix}, &\quad&
	\lambda_8 &= \begin{bNiceMatrix} \smash{\frac{1}{\sqrt{3}}} &  0 &  0 \\ 0 &  \smash{\frac{1}{\sqrt{3}}} &  0 \\ 0 & 0 & -\smash{\frac{2}{\sqrt{3}}} \end{bNiceMatrix}.
\end{aligned}
\end{NiceMatrixBlock}
\label{eq:lsm:gell_mann_matrices}
\end{equation}
%which are normalized to $\trace[T_a T_b] = \delta_{ab}/2$ and $\trace[\lambda_a \lamda_b] = 2 \delta_{ab}$.
Like before, the Lagrangian has $U(1)_V \times U(1)_A \times SU(N_f)_L \times SU(N_f)_R$ symmetry
in the absence of the explicit symmetry breakers $h_a$ in the matrix $H = h_a T_a$.
Moreover, there are now \emph{two} quartic couplings $\lambda_1$ and $\lambda_2$,
not to be confused with the Gell-Mann matrices \eqref{eq:lsm:gell_mann_matrices}.
This can be understood from the \emph{Hamilton-Caley theorem} \cite[equation 1 and 2]{ref:hamilton_caley},
which links the $N$ traces $\trace A, \ldots, \trace A^N$ for any $N \times N$ matrix $A$.
With three flavors we can therefore form \emph{two} independent coupling terms
$\lambda_1 \trace[(\phi^\dagger \phi)]^2$ and $\lambda_2 \trace[(\phi^\dagger \phi)^2]$ up to quadratic order%
\footnote{That precisely fourth order is the target implies that the theory is renormalizable,
in the sense that divergences can be removed by shifting the couplings already in the Lagrangian with counterterms.
Divergences from, say, sixth-order terms in the Lagrangian could only have been removed by counterterms in higher-order terms
and would make the theory non-renormalizable.}
in the fields, while only one can be constructed with two flavors.
Every other feature of the Lagrangian is identical to or a natural generalization of that in the corresponding two-flavor Lagrangian \eqref{eq:lsm:lagrangian}.

This general form of the model has thirteen undetermined parameters $g$, $m^2$, $\lambda_1$, $\lambda_2$ and $h_0,\ldots,h_8$.
We continue to neglect pion condensation by setting $\avg{\pi_a}=0$,
so a nonzero symmetry breaker $h_a$ creates a non-vanishing vacuum expectation value of the corresponding field $\avg{\sigma_a}$,
which in turn creates a nonzero quark condensate $\avg{\bar{q} \, T_a q}$ in vacuum through the Yukawa interaction in the Lagrangian \eqref{eq:lsm:lagrangian3f}.
In particular, this vacuum expectation value should have the same vanishing electrical charge as the vacuum.
We therefore only allow for flavor-like charge-neutral condensates such as $\avg{\bar{u} u}$,
and no mixed-flavor charged condensates like $\avg{\bar{u} d}$.
To accomplish this, we set all symmetry breakers to zero \emph{except} those that correspond to diagonal flavor-space matrices \eqref{eq:lsm:gell_mann_matrices},
or $T_a$,
namely $\{h_0,h_3,h_8\} \neq 0$.

Using different combinations of the three remaining nonzero symmetry breakers $\{h_0,h_3,h_8\}$,
one can study different symmetry breaking patterns among the $u$, $d$ and $s$ quarks.
By setting only $h_0 \neq 0$, we see that the first Gell-Mann matrix $\lambda_0$ weighs all three quarks equally with common mass $m_u=m_d=m_s$.
If we also unlock $h_8 \neq 0$, the last matrix $\lambda_8$ separates the strange and non-strange quarks with $m_u = m_d \neq m_s$.
Including all three symmetry breakers with $h_3 \neq 0$, too, the fourth matrix $\lambda_3$ also distinguishes the non-strange quarks and treats all flavors separately with $m_u \neq m_d \neq m_s$.
These different symmetry breaking patterns are discussed in more detail in \cite[section III]{ref:lsm3f_details}.
Since we treated up and down quarks with degenerate mass in \cref{chap:lsm2f},
we will set $h_3 = 0$ and keep $\{h_0,h_8\} \neq 0$ to account for the heavier strange quark separately from the non-strange quarks.
This leaves six unknown parameters $g$, $m^2$, $\lambda_1$, $\lambda_2$, $h_0$, $h_8$ that must be fit to as many experimental values.

The explicit symmetry breakers $\{h_0,h_8\}\neq 0$ dig a global minimum for the vacuum and the classical ground state at
\begin{equation}
	\sigma_a = \avg{\sigma_a} \quad \text{and} \quad \pi_a = \avg{\pi_a} = 0, \quad \text{where only $\avg{\sigma_0} \neq 0$ and $\avg{\sigma_8} \neq 0$ are nonzero.}
\label{eq:lsm:ground_state_3f}
\end{equation}
Once again, we jump down into the hole \eqref{eq:lsm:ground_state_3f} and study the quantum fluctuations $\tilde{\sigma}_a$ and $\tilde{\pi}_a$ of the meson fields by writing
\begin{equation}
	\sigma_a = \avg{\sigma_a} + \tilde{\sigma}_a
	\qquad \text{and} \qquad
	\pi_a = \avg{\pi_a} + \tilde{\pi}_a.
\end{equation}

We will shortly come back to the precise location of the vacuum, but first examine how it couples to the quarks.
Coupled to the nonzero expectation values, the Yukawa term in the Lagrangian \eqref{eq:lsm:lagrangian3f} reads
\begin{equation}
\begin{split}
	  & -g \bar{q} \big( \avg{\sigma_0} T_0 + \avg{\sigma_8} T_8 \big) q \\
	= & -g \begin{bNiceMatrix} \vphantom{\sqrt{\frac13}} u^\dagger \gamma_0 \\ \vphantom{\sqrt{\frac13}} d^\dagger \gamma_0 \\ \vphantom{\sqrt{\frac13}} s^\dagger \gamma_0 \end{bNiceMatrix}^T \begin{bNiceMatrix} \sqrt{\frac23} \avg{\sigma_0} + \frac{1}{\sqrt{3}} \avg{\sigma_8} & 0 & 0 \\ 0 & \sqrt{\frac23} \avg{\sigma_0} + \frac{1}{\sqrt{3}} \avg{\sigma_8} & 0 \\ 0 & 0 & \sqrt{\frac23} \avg{\sigma_0} - \frac{2}{\sqrt{3}} \avg{\sigma_8} \end{bNiceMatrix} \begin{bNiceMatrix} \vphantom{\sqrt{\frac13}} u \\ \vphantom{\sqrt{\frac13}} d \\ \vphantom{\sqrt{\frac13}} s \end{bNiceMatrix} , \\
\end{split}
\label{eq:lsm:yukawa3f}
\end{equation}
mixing $\sigma_0$ and $\sigma_8$ interactions with each quark.
For conceptual cleanliness,
we seek a unitary transformation from the mixed $\sigma_0\text{-}\sigma_8$-basis to a segregated $\sigma_x\text{-}\sigma_y$-basis
in which the up and down quarks couple only to $\sigma_x$ and the strange quark only to $\sigma_y$.
This is accomplished by
\begin{equation}
	\begin{bmatrix} \sigma_x \\ \sigma_y \end{bmatrix} = M \begin{bmatrix} \sigma_0 \\ \sigma_8 \end{bmatrix},
	%\quad \text{or} \quad
	%\begin{bmatrix} \sigma_0 \\ \sigma_8 \end{bmatrix} = M \begin{bmatrix} \sigma_x \\ \sigma_y \end{bmatrix},
	\qquad \text{where} \qquad
	M = M^{-1} = \frac{1}{\sqrt{3}} \begin{bmatrix} \sqrt{2} & 1 \\ 1 & -\sqrt{2} \end{bmatrix}.
\label{eq:lsm:strange_basis}
\end{equation}
We also define the transformed symmetry breakers $h_x$ and $h_y$ from $h_0$ and $h_8$ using the same transformation.
In the new basis \eqref{eq:lsm:strange_basis}, the Yukawa coupling \eqref{eq:lsm:yukawa3f} takes our sought-after form
\begin{equation}
	- \smashoperator{\sum_{f=\{u,d,s\}}} m_f \bar{q}_f q_f
	\quad \text{with quark masses} \quad
	m_u = m_d = m_x = \frac{g \avg{\sigma_x}}{2}
	\quad \text{and} \quad
	m_s = m_y = \frac{g \avg{\sigma_y}}{\sqrt{2}}.
\label{eq:lsm:quark_masses_3f}
\end{equation}

Finally, we examine how the meson potential \eqref{eq:lsm:potential3f} behaves in and around the vacuum
by looking for its mass-generating expansion
\begin{equation}
	\pot(\sigma,\pi) \taylor \pot(\avg{\sigma},\avg{\pi}) + \frac12 \big(m^2_{\sigma\sigma}\big)_{ab} \tilde{\sigma}_a \tilde{\sigma}_b + \frac12 \big(m^2_{\pi\pi}\big)_{ab} \tilde{\pi}_a \tilde{\pi}_b.
\label{eq:lsm3f:potential_before_diagonalization}
\end{equation}
To do so, we need to evaluate the traces
\begin{subequations}
\begin{align}
	\trace\big[\phi^\dagger \phi\big]     &= (\sigma_a - i \pi_a) (\sigma_b + i \pi_b) \trace \big[ T_a T_b \big] = \frac12 \big(\sigma_a^2+\pi_a^2\big) , \\
	\trace\big[(\phi^\dagger \phi)^2\big] &= (\sigma_a - i \pi_a) (\sigma_b + i \pi_b) (\sigma_a - i \pi_c) (\sigma_b + i \pi_d) \trace \big[ T_a T_b T_c T_d \big], \\
	\trace\big[H(\phi+\phi^\dagger)\big]  &= 2 h_a \sigma_b \trace \big[ T_a T_b \big] = \vphantom{\frac12} h_a \sigma_a . \label{eq:lsm3f:traces_hphi}
\end{align}%
\label{eq:lsm3f:traces}%
\end{subequations}%
The quadruple product traces $\trace[T_a T_b T_c T_d]$ are more challenging than the double product traces $\trace[T_a T_b] = \delta_{ab}/2$.
If one figures out the structure constants $f_{abc}$ and $d_{abc}$
defined by the (anti-)commutators $[T_a,T_b]=i f_{abc} T_c$ and $\{T_a,T_b\}=d_{abc} T_c$,
these traces can be done with pen, paper and painkillers
by bootstrapping traces of the Gell-Mann matrices like
\begin{subequations}
\begin{align}
	\trace\Big[T_a T_b\Big] &= \frac{1}{2} \delta_{ab} \label{eq:lsm3f:trace2}, \\
	\trace\Big[T_a T_b T_c\Big] &= \frac{1}{2} \trace\Big[T_a\Big(\overbrace{[T_b,T_c]}^{\smash{i f_{bcd} T_d}}+\overbrace{\{T_b,T_c\}}^{\smash{d_{bcd} T_d}}\Big)\Big]
	                             %= \frac{1}{2} \Big( i f_{bcd} + d_{bcd} \Big) \trace\Big[T_a T_d\Big]
	                             \equalexplabove{\smash[t]{use \eqref{eq:lsm3f:trace2}}} \frac{1}{2^2} \Big( i f_{bca} + d_{bca} \Big), \label{eq:lsm3f:trace3} \\
	\trace\Big[T_a T_b T_c T_d\Big] &= \frac{1}{2} \trace\Big[T_a T_b \Big( \underbrace{[T_c,T_d]}_{\smash{i f_{cde} T_e}} + \underbrace{\{T_c,T_d\}}_{\smash{d_{cde} T_e}} \Big) \Big]
	                                 %= \frac{1}{2} \Big( i f_{cde} + d_{cde} \Big) \trace\Big[T_a T_b T_e\Big]
	                                 \equalexplbelow{\smash[b]{use \eqref{eq:lsm3f:trace3}}} \frac{1}{2^3} \Big( i f_{cde} + d_{cde} \Big) \Big( i f_{bea} + d_{bea} \Big). \label{eq:lsm3f:trace4}
\end{align}
\end{subequations}
However, with our explicit knowledge \eqref{eq:lsm:gell_mann_matrices} of the matrices
it is easier and less error-prone to simply calculate all the traces by brute force with a symbolic computer program.
Using the program in \cref{chap:lsm3fpotential},
we create a 559-term-long explicit representation of $\pot(\sigma,\pi)$ for arbitrary $h_a$.
Evaluated in the minimum \eqref{eq:lsm:ground_state_3f}, its value is
\begin{equation}
	\pot(\avg{\sigma},\avg{\pi}) = \frac{m^2}{2} \Big[ \avg{\sigma_x}^2 + \avg{\sigma_y}^2 \Big] + \frac{\lambda_1}{4} \Big[ \avg{\sigma_x}^2 + \avg{\sigma_y}^2 \Big]^2 + \frac{\lambda_2}{8} \Big[ \avg{\sigma_x}^4 + 2 \avg{\sigma_y}^4 \Big] - h_x \avg{\sigma_x} - h_y \avg{\sigma_y}.
\label{eq:lsm:potential_tree_3f}
\end{equation}
The location of the nonzero minima fields $\avg{\sigma_x}$ and $\avg{\sigma_y}$ is determined
by taking first derivatives of the 559 terms and evaluating them in the minimum \eqref{eq:lsm:ground_state_3f}:
\begin{subequations}
\begin{align}
	0 &= \pdv{\pot}{\sigma_x} = \avg{\sigma_x} \Big[ m^2 + \lambda_1 \big( \avg{\sigma_x}^2 + \avg{\sigma_y}^2 \big) + \frac{\lambda_2}{2} \avg{\sigma_x}^2 \Big] - h_x, \\
	0 &= \pdv{\pot}{\sigma_y} = \avg{\sigma_y} \Big[ m^2 + \lambda_1 \big( \avg{\sigma_x}^2 + \avg{\sigma_y}^2 \big) + \lambda_2 \avg{\sigma_y}^2 \Big] - h_y.
\end{align}
\label{eq:lsm3f:symmetry_breakers}%
\end{subequations}
Note also that $\pdv{\pot}/{\sigma_a} = -h_a = 0$ for $a \neq \{0,8\}$ and $\pdv{\pot}/{\pi_a}=0$ for all $a$ indeed vanish in the minimum \eqref{eq:lsm:ground_state_3f}.

Things get messier and our computational approach really comes in handy when evaluating the mass-generating second derivatives
\begin{equation}
	\big(m^2_{\sigma\sigma}\big)_{ab} = \pdv{\pot}{\sigma_a, \sigma_b}
	\qquad \text{and} \qquad
	\big(m^2_{\pi\pi}\big)_{ab}       = \pdv{\pot}{\pi_a, \pi_b}
\end{equation}
in the vacuum \eqref{eq:lsm:ground_state_3f}.
All mixed partial derivatives $\pdv{\pot}/{\sigma_a,\pi_b} = 0$ vanish and are therefore left out.
The nonzero entries of the scalar mass matrix are
\begin{equation}
\begin{NiceMatrixBlock}[]
\NiceMatrixOptions{cell-space-limits=1pt}
% smash all fracs to make consistent spacing between row entries
\begin{array}{l @{\,} >{\displaystyle}l}
	\begin{NiceArray}{c}
	\big(m^2_{\sigma\sigma}\big)_{00} \\
	\end{NiceArray}
	& = m^2 + \smash{\frac{\lambda_1}{3}} \big(4 \sqrt{2} \avg{\sigma_x} \avg{\sigma_y} + 7 \avg{\sigma_x}^2 + 5 \avg{\sigma_y}^2\big) + \lambda_2 \big(\avg{\sigma_x}^2 + \avg{\sigma_y}^2\big), \\
	\begin{NiceArray}{c}
	\big(m^2_{\sigma\sigma}\big)_{11} \\
	\big(m^2_{\sigma\sigma}\big)_{22} \\
	\big(m^2_{\sigma\sigma}\big)_{33} \\
	\end{NiceArray} \custombracketr{\}}{0.9cm}
	& = m^2 + \lambda_1 \big(\avg{\sigma_x}^2 + \avg{\sigma_y}^2\big) + \smash{\frac32} \lambda_2 \avg{\sigma_x}^2, \\
	\begin{NiceArray}{c}
	\big(m^2_{\sigma\sigma}\big)_{44} \\
	\big(m^2_{\sigma\sigma}\big)_{55} \\
	\big(m^2_{\sigma\sigma}\big)_{66} \\
	\big(m^2_{\sigma\sigma}\big)_{77} \\
	\end{NiceArray} \custombracketr{\}}{1.2cm}
	& = m^2 + \lambda_1 \big(\avg{\sigma_x}^2 + \avg{\sigma_y}^2\big) + \smash{\frac{\lambda_2}{2}} \big(\sqrt{2} \avg{\sigma_x} \avg{\sigma_y} + \avg{\sigma_x}^2 + 2 \avg{\sigma_y}^2\big), \\
	\begin{NiceArray}{c}
	\big(m^2_{\sigma\sigma}\big)_{88} \\
	\end{NiceArray}
	&= m^2 - \smash{\frac{\lambda_1}{3}} \big(4 \sqrt{2} \avg{\sigma_x} \avg{\sigma_y} - 5 \avg{\sigma_x}^2 - 7 \avg{\sigma_y}^2\big) + \smash{\frac{\lambda_2}{2}} \big(\avg{\sigma_x}^2 + 4 \avg{\sigma_y}^2\big), \\
	\begin{NiceArray}{c}
	\big(m^2_{\sigma\sigma}\big)_{08} \\
	\big(m^2_{\sigma\sigma}\big)_{80} \\
	\end{NiceArray} \custombracketr{\}}{0.7cm}
	& = \smash{\frac23} \lambda_1 \big(\sqrt{2} \avg{\sigma_x}^2 - \sqrt{2} \avg{\sigma_y}^2 - \avg{\sigma_x} \avg{\sigma_y}\big) + \smash{\frac{\lambda_2}{\sqrt{2}}} \big(\avg{\sigma_x}^2 - 2 \avg{\sigma_y}^2\big),\\[0.1cm]
\end{array}
\end{NiceMatrixBlock}
\label{eq:lsm3f:mass_sigma_sigma}
\end{equation}
and the nonzero entries of the pseudoscalar mass matrix are
\begin{equation}
\begin{NiceMatrixBlock}[]
\NiceMatrixOptions{cell-space-limits=1pt}
% smash all fracs to make consistent spacing between row entries
\begin{array}{l @{\,} >{\displaystyle}l}
	\begin{NiceArray}{c}
	\big(m^2_{\pi\pi}\big)_{00} \\
	\end{NiceArray}
	&= m^2 + \lambda_1 \big(\avg{\sigma_x}^2 + \avg{\sigma_y}^2\big) + \smash{\frac{\lambda_2}{3}} \big(\avg{\sigma_x}^2 + \avg{\sigma_y}^2\big), \\
	\begin{NiceArray}{c}
	\big(m^2_{\pi\pi}\big)_{11} \\
	\big(m^2_{\pi\pi}\big)_{22} \\
	\big(m^2_{\pi\pi}\big)_{33} \\
	\end{NiceArray} \custombracketr{\}}{0.9cm}
	&= m^2 + \lambda_1 \big(\avg{\sigma_x}^2 + \avg{\sigma_y}^2\big) + \smash{\frac{\lambda_2}{2}} \avg{\sigma_x}^2, \\
	\begin{NiceArray}{c}
	\big(m^2_{\pi\pi}\big)_{44} \\
	\big(m^2_{\pi\pi}\big)_{55} \\
	\big(m^2_{\pi\pi}\big)_{66} \\
	\big(m^2_{\pi\pi}\big)_{77} \\
	\end{NiceArray} \custombracketr{\}}{1.2cm}
	&= m^2 + \lambda_1 \big(\avg{\sigma_x}^2 + \avg{\sigma_y}^2\big) - \smash{\frac{\lambda_2}{2}} \big(\sqrt{2} \avg{\sigma_x} \avg{\sigma_y} - \avg{\sigma_x}^2 - 2 \avg{\sigma_y}^2\big), \\
	\begin{NiceArray}{c}
	\big(m^2_{\pi\pi}\big)_{88} \\
	\end{NiceArray}
	&= m^2 + \lambda_1 \big(\avg{\sigma_x}^2 + \avg{\sigma_y}^2\big) + \smash{\frac{\lambda_2}{6}} \big(\avg{\sigma_x}^2 + 4 \avg{\sigma_y}^2\big), \\ 
	\begin{NiceArray}{c}
	\big(m^2_{\pi\pi}\big)_{08} \\
	\big(m^2_{\pi\pi}\big)_{80} \\
	\end{NiceArray} \custombracketr{\}}{0.7cm}
	&= \smash{\frac{\lambda_2}{6}} \big(\sqrt{2} \avg{\sigma_x}^2 - 2 \sqrt{2} \avg{\sigma_y}^2\big).
\end{array}
\end{NiceMatrixBlock}
\label{eq:lsm3f:mass_pi_pi}
\end{equation}
Notice that both matrices are non-diagonal with nonzero entries in all four corners.
However, it is only fields corresponding to diagonal masses that represent true mass eigenstates of physically propagating particles: \cite{ref:lsm3f_details}
%As we summarizing the identification of different scalar and pseudoscalar meson species made in this reference,
%we will therefore rotate the $0$-$8$ field pairs (or $x$-$y$ pairs) to new bases that diagonalizes the $0$-$8$-sector of the mass matrices.
\begin{itemize}
\item The three diagonal masses $\smash{\big(m^2_{\sigma\sigma}\big)_{11} = \big(m^2_{\sigma\sigma}\big)_{22} = \big(m^2_{\sigma\sigma}\big)_{33}}$ correspond to one degenerate mass $m^2_{a_0}$
      of the three $\sigma_1$, $\sigma_2$ and $\sigma_3$ fields,
      or alternatively of the two charged $a_0^\pm = (\sigma_1 \pm i \sigma_2) / \sqrt{2}$ mesons and the neutral $a_0^0 = \sigma_3$ meson.
      Likewise, $\smash{\big(m^2_{\pi\pi}\big)_{11}} = \smash{\big(m^2_{\pi\pi}\big)_{22}} = \smash{\big(m^2_{\pi\pi}\big)_{33}}$ correspond to one mass $m^2_\pi$
      of the $\pi_1$, $\pi_2$ and $\pi_3$ fields,
      or of the charged $\pi^\pm = (\pi_1 \pm i \pi_2) / \sqrt{2}$ mesons and the neutral $\pi^0 = \pi_3$ meson.
\item The four diagonal masses $\smash{\big(m^2_{\sigma\sigma}\big)_{44} = \big(m^2_{\sigma\sigma}\big)_{55} = \big(m^2_{\sigma\sigma}\big)_{66} = \big(m^2_{\sigma\sigma}\big)_{77}}$ correspond to one degenerate mass $m^2_\kappa$
      of the four $\sigma_4$, $\sigma_5$, $\sigma_6$ and $\sigma_7$ fields,
      or alternatively of the two charged $\kappa^\pm = (\sigma_4 \pm i \sigma_5) / \sqrt{2}$ mesons and the two neutral $\kappa^0 = (\sigma_6 + i \sigma_7) / \sqrt{2}$ and $\bar{\kappa}^0 = (\sigma_6 - i \sigma_7) / \sqrt{2}$ mesons.
      Likewise, $\smash{\big(m^2_{\pi\pi}\big)_{44}} = \smash{\big(m^2_{\pi\pi}\big)_{55}} = \smash{\big(m^2_{\pi\pi}\big)_{66}} = \smash{\big(m^2_{\pi\pi}\big)_{77}}$ correspond to one mass $m^2_K$
      of the $\pi_4$, $\pi_5$, $\pi_6$ and $\pi_7$ fields,
      or of the charged $K^\pm = (\pi_4 \pm i \pi_5) / \sqrt{2}$ mesons and the neutral $K^0 = (\pi_6 + i \pi_7) / \sqrt{2}$ and $\bar{K}^0 = (\pi_6 - i \pi_7) / \sqrt{2}$ mesons.
\item The \emph{diagonalization} of the non-diagonal matrix sector with elements $\smash{\big(m^2_{\sigma\sigma}\big)_{00}}$, $\smash{\big(m^2_{\sigma\sigma}\big)_{88}}$ and $\smash{\big(m^2_{\sigma\sigma}\big)_{08} = \big(m^2_{\sigma\sigma}\big)_{80}}$ correspond to two different masses of the $\sigma$ and $f_0$ mesons.
      Likewise, the diagonalization of $\smash{\big(m^2_{\pi\pi}\big)_{00}}$, $\smash{\big(m^2_{\pi\pi}\big)_{88}}$ and $\smash{\big(m^2_{\pi\pi}\big)_{08} = \big(m^2_{\pi\pi}\big)_{80}}$ correspond to masses of the $\eta$ and $\eta'$ mesons.
      The diagonalizations are achieved by two rotations
      \begin{equation}
          \begin{bmatrix} f_0 \\ \sigma \\ \end{bmatrix} = \begin{bmatrix} \phantom{-} \cos \theta_\sigma & -\sin \theta_\sigma \\ \phantom{-} \sin \theta_\sigma & \phantom{-} \cos \theta_\sigma \\ \end{bmatrix} \begin{bmatrix} \sigma_8 \\ \sigma_0 \\ \end{bmatrix}
          \qquad \text{and} \qquad
          \begin{bmatrix} \eta \\ \eta' \\ \end{bmatrix} = \begin{bmatrix} \phantom{-} \cos \theta_\pi & -\sin \theta_\pi \\ \phantom{-} \sin \theta_\pi & \phantom{-} \cos \theta_\pi \\ \end{bmatrix} \begin{bmatrix} \pi_8 \\ \pi_0 \\ \end{bmatrix}
      \label{eq:lsm3f:diagonalization_transformation}
      \end{equation}
      of the $\sigma_0$, $\sigma_8$, $\pi_0$ and $\pi_8$ fields, parametrized by two mixing angles $\theta_\sigma$ and $\theta_\pi$.
\end{itemize}

To find the diagonalizing pseudoscalar mixing angle $\theta_\pi$,
we invert transformation \eqref{eq:lsm3f:diagonalization_transformation} and use the trigonometric identities $\cos^2 \theta_\pi - \sin^2 \theta_\pi = \cos 2 \theta_\pi$ and $2 \sin\theta_\pi \cos\theta_\pi = \sin 2 \theta_\pi$ to expand
\begin{equation}
\begin{split}
	\smash{\smashoperator{\sum_{a,b=\{0,8\}}}} \big(m^2_{\pi\pi}\big)_{ab} \tilde{\pi}_a \tilde{\pi}_b &= \Big\{ \big(m^2_{\pi\pi}\big)_{00} \sin^2 \theta_\pi + \big(m^2_{\pi\pi}\big)_{88} \cos^2 \theta_\pi + \big(m^2_{\pi\pi}\big)_{08} \sin 2 \theta_\pi \Big\} \, \tilde{\eta}^2 \\
	                                                                                                   &+ \Big\{ \big(m^2_{\pi\pi}\big)_{00} \cos^2 \theta_\pi + \big(m^2_{\pi\pi}\big)_{88} \sin^2 \theta_\pi - \big(m^2_{\pi\pi}\big)_{08} \sin 2 \theta_\pi \Big\} \, \tilde{\eta}'^2 \\
	                                                                                                   &+ \smash{\underbrace{\Big\{ 2 \Big[ \big(m^2_{\pi\pi}\big)_{00} - \big(m^2_{\pi\pi}\big)_{88} \Big] \sin 2 \theta_\pi + 2 \big(m^2_{\pi\pi}\big)_{08} \cos 2 \theta_\pi \Big\}}_{\text{$0$ by demand}}} \, \tilde{\eta} \tilde{\eta}' ,
\end{split}
\label{eq:lsm3f:diagonalization_sum}
\end{equation}
and then require the coefficient of $\tilde{\eta} \tilde{\eta}'$ to vanish.
The scalar angle $\theta_\sigma$ is determined in the same way, only with $m^2_{\pi\pi} \rightarrow m^2_{\sigma\sigma}$, $\eta \rightarrow f_0$ and $\eta' \rightarrow \sigma$.
We can then solve for the mixing angles
\begin{equation}
	\theta_\sigma = \frac12 \arctan \Bigg[ \frac{2\big(m^2_{\sigma\sigma}\big)_{08}}{\big(m^2_{\sigma\sigma}\big)_{88} - \big(m^2_{\sigma\sigma}\big)_{00}} \Bigg]
	\quad \text{and} \quad
	\theta_\pi = \frac12 \arctan \Bigg[ \frac{2\big(m^2_{\pi\pi}\big)_{08}}{\big(m^2_{\pi\pi}\big)_{88} - \big(m^2_{\pi\pi}\big)_{00}} \Bigg],
\label{eq:lsm3f:mixing_angles}
\end{equation}
which in turn fixes the masses of $\eta$ and $\eta'$ as the coefficients of their squares in the sum \eqref{eq:lsm3f:diagonalization_sum}, and analogously for the $f_0$ and $\sigma$ masses.
Explicitly, the diagonalized meson potential \eqref{eq:lsm3f:potential_before_diagonalization} is
\begin{equation}
\begin{split}
	\hspace{-2cm} \pot(\sigma,\pi) = \pot(\avg{\sigma},\avg{\pi}) &+ \frac12 m^2_{f_0} \tilde{f_0}^2  + \frac12 m^2_{\sigma} \tilde{\sigma}^2 \,\, + \frac12 m^2_{a_0} \sum_{\mathrlap{\!\!a_0 = \{a_0^+,a_0^-,a_0^0\}}} \tilde{a}_0^2               \,\, + \, \frac12 m^2_{\kappa} \sum_{\mathrlap{\!\!\kappa = \{\kappa^+,\kappa^-,\kappa^0,\bar{\kappa}^0\}}} \tilde{\kappa}^2 \\
	                                                              &+ \frac12 m^2_{\eta} \tilde{\eta}^2 \,\,\,\,\, + \frac12 m^2_{\eta'} \tilde{\eta}'^2 + \frac12 m^2_{\pi_{\phantom{0}}} \sum_{\mathrlap{\!\!\pi = \{\pi^+,\pi^-,\pi^0\}}} \tilde{\pi}^2 \, + \, \frac12 m^2_{K} \sum_{\mathrlap{\!\!K = \{K^+,K^-,K^0,\bar{K}^0\}}} \tilde{K}^2. \\
\end{split}
\label{eq:lsm3f:potential_after_diagonalization}
\end{equation}
This shows very elaborately how the three couplings $m^2$, $\lambda_1$ and $\lambda_2$,
through the mass matrices \eqref{eq:lsm3f:mass_sigma_sigma} and \eqref{eq:lsm3f:mass_pi_pi} and the mixing angles \eqref{eq:lsm3f:mixing_angles},
generate the eight scalar and pseudoscalar particle masses
\begin{subequations}
\begin{align}
	& m^2_{f_0}   && \hspace{-2.9cm} = \big(m^2_{\sigma\sigma}\big)_{00} \sin^2 \theta_\sigma + \big(m^2_{\sigma\sigma}\big)_{88} \cos^2 \theta_\sigma + \big(m^2_{\sigma\sigma}\big)_{08} \sin 2 \theta_\sigma, \\ % \quad \text{(with $\theta_\sigma$ from \eqref{eq:lsm3f:mixing_angles})} \\
	& m^2_\sigma  && \hspace{-2.9cm} = \big(m^2_{\sigma\sigma}\big)_{00} \cos^2 \theta_\sigma + \big(m^2_{\sigma\sigma}\big)_{88} \sin^2 \theta_\sigma - \big(m^2_{\sigma\sigma}\big)_{08} \sin 2 \theta_\sigma, \\ % \quad \text{(with $\theta_\sigma$ from \eqref{eq:lsm3f:mixing_angles})} \\
	& m^2_{a_0}   && \hspace{-2.9cm} = \big(m^2_{\sigma\sigma}\big)_{11} = \big(m^2_{\sigma\sigma}\big)_{22} = \big(m^2_{\sigma\sigma}\big)_{33}, \\
	& m^2_\kappa  && \hspace{-2.9cm} = \big(m^2_{\sigma\sigma}\big)_{44} = \big(m^2_{\sigma\sigma}\big)_{55} = \big(m^2_{\sigma\sigma}\big)_{66} = \big(m^2_{\sigma\sigma}\big)_{77}, \\
	& m^2_{\eta}  && \hspace{-2.9cm} = \big(m^2_{\pi\pi}\big)_{00} \sin^2 \theta_\pi + \big(m^2_{\pi\pi}\big)_{88} \cos^2 \theta_\pi + \big(m^2_{\pi\pi}\big)_{08} \sin 2 \theta_\pi, \\ % \quad \text{($\theta_\pi$ from \eqref{eq:lsm3f:mixing_angles})} \\
	& m^2_{\eta'} && \hspace{-2.9cm} = \big(m^2_{\pi\pi}\big)_{00} \cos^2 \theta_\pi + \big(m^2_{\pi\pi}\big)_{88} \sin^2 \theta_\pi - \big(m^2_{\pi\pi}\big)_{08} \sin 2 \theta_\pi, \\ % \quad \text{($\theta_\pi$ from \eqref{eq:lsm3f:mixing_angles})} \\
	& m^2_\pi     && \hspace{-2.9cm} = \big(m^2_{\pi\pi}\big)_{11} = \big(m^2_{\pi\pi}\big)_{22} = \big(m^2_{\pi\pi}\big)_{33}, \\
	& m^2_K       && \hspace{-2.9cm} = \big(m^2_{\pi\pi}\big)_{44} = \big(m^2_{\pi\pi}\big)_{55} = \big(m^2_{\pi\pi}\big)_{66} = \big(m^2_{\pi\pi}\big)_{77}.
\end{align}%
\label{eq:lsm:mass_system_3f}%
\end{subequations}%
These relations will come in handy when we fit the parameters of the model in \cref{sec:lsm3f:parameter_fit}.


\section{Grand potential}

Let us calculate the grand potential $\Omega$ in the same way as in \cref{sec:lsm:grand_potential},
integrating over one fermion loop while
neglecting pion condensation by keeping $\avg{\pi}=0$
and using the mean-field approximation for $\avg{\sigma_x}$ and $\avg{\sigma_y}$,
determining them in retrospect according to the self-consistency equations
\begin{equation}
	\pdv{\Omega}{\avg{\sigma_x}} =
	\pdv{\Omega}{\avg{\sigma_y}} = 0.
\end{equation}
As explained in \cref{chap:lsm2f},
this is inconsistent in terms of the number of loops considered for the fermionic and bosonic fields,
but consistent in the large-$N_c$ approximation scheme.

The calculation proceeds more or less identically up to the non-renormalized grand potential \eqref{eq:lsm:grand_potential_nonrenormalized},
only with a different vacuum potential $\pot(\avg{\sigma},0)$, $N_f=3$ flavors and $m_u = m_d \neq m_s$.
In addition, instead of using one common renormalization scale,
we operate with two separate scales $\Lambda_x$ and $\Lambda_y$ for the non-strange and strange quarks.
We comment on this in the next section.
Before removing the vacuum divergence, we then have the divergent grand potential
\begin{equation}
\begin{split}
	\Omega(\avg{\sigma},\mu) &= \pot(\avg{\sigma},0) + \frac{N_c m_x^4}{8 \pi^2} \Bigg[ \frac{1}{\epsilon} + \frac{3}{2} + \log\bigg(\frac{\Lambda_x^2}{m_x^2}\bigg) \Bigg] + \frac{N_c m_y^4}{16 \pi^2} \Bigg[ \frac{1}{\epsilon} + \frac{3}{2} + \log\bigg(\frac{\Lambda_y^2}{m_y^2}\bigg) \Bigg] \\ % + N_c \smashoperator{\sum_{\vphantom{\big|}f=\{u,d,s\}}} \frac{m_f^4}{16 \pi^2} \left[ \frac{1}{\epsilon} + \frac{3}{2} + \log\left(\frac{{\Lambda}^2}{m_f^2}\right) \right] \\
	                         &-\smashoperator{\sum_{\vphantom{\big|} f=\{u,d,s\}}} \frac{N_c}{24 \pi^2} \left[ \left( 2 \mu_f^2 - 5 m_f^2 \right) \mu_f \sqrt{\mu_f^2 - m_f^2} + 3 m_f^4 \asinh \left( \sqrt{\frac{\mu_{\smash{f}}^2}{m_f^2}-1} \right) \right]. \\
\end{split}
\end{equation}
Once again, we see that the divergent term
\begin{equation}
	N_c \frac{2 m_x^4 + m_y^4}{16 \pi^2 \epsilon} =
	\frac{N_c g^4}{16 \pi^2} \frac{\avg{\sigma_x}^4 + 2 \avg{\sigma_y}^4}{8 \epsilon}
\end{equation}
can be removed by shifting the $\lambda_2$ coupling in the vacuum meson potential \eqref{eq:lsm:potential_tree_3f} to
\begin{equation}
	\lambda_2 \rightarrow \lambda_2 + \delta\lambda_2 \qquad \text{with the counterterm} \qquad \delta\lambda_2 = -\frac{N_c g^4}{16 \pi^2 \epsilon} .
\end{equation}
Adding electrons, we obtain the natural generalization of the two-flavor grand potential \eqref{eq:lsm:grand_potential},
\begin{equation}
\begin{split}
	\Omega(\avg{\sigma},\vec{\mu}) &= \pot(\avg{\sigma},0) + N_c \frac{m_x^4}{8 \pi^2} \Bigg[ \frac{3}{2} + \log\bigg(\frac{\Lambda_x^2}{m_x^2}\bigg) \Bigg] + N_c \frac{m_y^4}{16 \pi^2} \Bigg[ \frac{3}{2} + \log\bigg(\frac{\Lambda_y^2}{m_y^2}\bigg) \Bigg] \\ % + N_c \smashoperator{\sum_{\vphantom{\big|}f=\{u,d,s\}}} \frac{m_f^4}{16 \pi^2} \left[ \frac{1}{\epsilon} + \frac{3}{2} + \log\left(\frac{{\Lambda}^2}{m_f^2}\right) \right] \\
	                               &-\smashoperator{\sum_{\vphantom{\big|} f=\{u,d,s\}}} \frac{N_c}{24 \pi^2} \left[ \left( 2 \mu_f^2 - 5 m_f^2 \right) \mu_f \sqrt{\mu_f^2 - m_f^2} + 3 m_f^4 \asinh \left( \sqrt{\frac{\mu_{\smash{f}}^2}{m_f^2}-1} \right) \right] \\
	                               &-\phantom{\sum} \, \frac{1}{24 \pi^2} \left[ \left( 2 \mu_e^2 - 5 m_e^2 \right) \mu_e \sqrt{\mu_e^2 - m_e^2} \, + 3 m_e^4 \asinh \left( \sqrt{\frac{\mu_{\smash{e}}^2}{m_e^2}-1} \right) \right].
\end{split}
\label{eq:lsm3f:grand_potential}
\end{equation}

\section{Parameter fitting at tree-level}
\label{sec:lsm3f:parameter_fit}

We are now in position to fit the six model parameters $g$, $m^2$, $\lambda_1$, $\lambda_2$, $h_x$ and $h_y$ in vacuum,
where the pion and kaon decay constants $f_\pi$ and $f_K$ are related to the nonzero expectation values by
\cite{ref:lsm3f_details}
\begin{equation}
	%f_\pi = \sqrt{\frac23} \avg{\sigma_0} + \frac{\avg{\sigma_8}}{\sqrt{3}}
	%\quad \text{and} \quad
	%f_K = \sqrt{\frac23} \avg{\sigma_0} - \frac{\avg{\sigma_8}}{\sqrt{12}},
	%\quad \text{or} \quad
	\avg{\sigma_x} = f_\pi
	\qquad \text{and} \qquad
	\avg{\sigma_y} = \sqrt{2} f_K - \frac{f_\pi}{\sqrt{2}}.
\label{eq:lsm3f:decay_constants}
\end{equation}
With vacuum measurements of the two decay constants and one of the quark masses \eqref{eq:lsm:quark_masses_3f},
we determine the parameter $g$, then use it to predict the other quark masses.
Likewise, we use vacuum measurements of three of the meson masses \eqref{eq:lsm:mass_system_3f} to determine the three parameters $m^2$, $\lambda_1$ and $\lambda_2$,
then use them to predict the remaining meson masses.
It is then straightforward to calculate the two symmetry breakers $h_x$ and $h_y$ from equation \eqref{eq:lsm3f:symmetry_breakers}.

\Cref{tab:lsm3f:parameters} shows values for the fitted and predicted particle masses.
In general there are $8!/3!5! = 56$ possible choices for selecting three of the eight meson masses,
each returning different values for $m^2$, $\lambda_1$ and $\lambda_2$.
We keep using $m_\sigma$ and $m_\pi$ for continuity from \cref{chap:lsm2f}
and include $m_K$ as the third mass, thereby obtaining the unique selection of mesons with lowest mass and energy.
This is the same selection used by \cite{ref:lsm3f} and \cite{ref:lsm3f_details}, for example.

Note that there is a particularly large discrepancy between the modeled and experimental $\eta'$ mass.
This can be improved by including a term $c \, (\tdet{\phi} + \tdet{\phi^\dagger})$ in the Lagrangian that accounts for the anomalous axial $U(1)_A$ symmetry of quantum chromodynamics discussed in \cref{sec:master_intro:qcd}.
How to do so is also shown \cite{ref:lsm3f} and \cite{ref:lsm3f_details},
but we do not include it here in order to keep the two-flavor and three-flavor models as similar and comparable as possible.
In particular, they find that its inclusion drives the chiral transition closer to being a discontinuous phase transition rather than a crossover.

Like in \cref{sec:lsm:parameter_fit},
we keep all parameters fixed except $m_\sigma$, due to its large experimental uncertainty.
As shown in the lasagne of different vacuum potentials in \cref{fig:lsm3f:potential_sigma_mass},
they still admit minima only for $m_\sigma \geq \SI{600}{\mega\electronvolt}$,
so we continue to use $m_\sigma=\{600,700,800\} \, \si{\mega\electronvolt}$.

With two flavors we determined the renormalization scale $\Lambda$ in equation \eqref{eq:lsm:potential_vacuum_minimum} by requiring the minimum to remain at $\avg{\sigma}=f_\pi$ in vacuum.
Now, the natural generalization of this procedure is to determine $\Lambda_x$ and $\Lambda_y$ so that the minimum remains at \eqref{eq:lsm3f:decay_constants} in vacuum, solving
\begin{equation}
	\pdv{\Omega}{\avg{\sigma_x}} = \pdv{\Omega}{\avg{\sigma_y}} = 0,
	\quad \text{so}
	\quad \Lambda_x = \frac{m_x}{\sqrt{e}} = \SI{182.0}{\mega\electronvolt}
	\quad \text{and} \quad
	\Lambda_y = \frac{m_y}{\sqrt{e}} = \SI{260.2}{\mega\electronvolt}.
\end{equation}

This is not the only way to do it.
Instead we could do like \cite{ref:master_berge} and operate with one common renormalization scale $\Lambda = \Lambda_x = \Lambda_y$ in the grand potential,
for example chosen as the flavor-weighted average $\Lambda = (2 \cdot \SI{182.0}{\mega\electronvolt} + \SI{260.2}{\mega\electronvolt}) / 3 = \SI{208.0}{\mega\electronvolt}$ of the two we have found.
In particular, this means that the minimum of $\Omega$ in vacuum is no longer at the minimum of $\pot$.
This is unsatisfying and has surprising consequences --
for example, if $\Omega$ is fit to some vacuum quark mass $m_x = g f_\pi/2$,
\emph{its minimum $\avg{\sigma_x} \neq f_\pi$ moves away} and generates a \emph{different} mass $m_u = g \avg{\sigma_x} / 2$.
Of course, our approach also has its caveats, as it is not apparent how to interpret the presence of two different renormalization scales, let alone one.
But it does not behave in this unpredictable way, and it also permits use of lower values of $m_\sigma$ without removing the minimum,
consistent with the behavior in \cref{chap:lsm2f}.
Ideally, a more consistent calculation such as the one presented in \cref{sec:lsm2f:refinement} should be done anyway,
where one preferable renormalization scale appears, and we even saw that it dropped out in that specific case.
It is hard to say which approach is better,
and this is another reason to at the very least explore the other approach than the one \cite{ref:master_berge} takes.

\begin{table}
\centering
\caption{\label{tab:lsm3f:parameters}%
\textbf{Bold variables} in the left table are used as input to determine the six model parameters in the right table,
which in turn are used to generate the remaining non-bold masses in the left table.
Experimental values are taken from \cite{ref:pdg_review_2021}.
Three different values are used for $m_\sigma$,
generating three different values of $\lambda_1$ and $m^2$.
}
\begin{tabular}{ l r r }
	\toprule
	Variable          & Model                                   & Experimental                            \\
	\midrule
	$f_\pi$           & $\textbf{\SI{93}{\mega\electronvolt}}$  & \SI{92}{}-\SI{93}{\mega\electronvolt}            \\
	$f_K$             & $\textbf{\SI{113}{\mega\electronvolt}}$ & \SI{113}{\mega\electronvolt}                     \\
	\midrule
	$m_u$             & $\textbf{\SI{300}{\mega\electronvolt}}$ & \approx \, \SI{300}{\mega\electronvolt}          \\
	$m_d$             & $\textbf{\SI{300}{\mega\electronvolt}}$ & \approx \, \SI{300}{\mega\electronvolt}          \\
	$m_s$             & $\SI{429}{\mega\electronvolt}$          & \approx \, \SI{500}{\mega\electronvolt}          \\
	\midrule
	$m_{f_0}$         & $\SI{1347}{\mega\electronvolt}$         & \SI{1200}{}-\SI{1500}{\mega\electronvolt}        \\
	$m_\sigma$        & $\textbf{\{600,700,800\}\si{\mega\electronvolt}}$ & \SI{400}{}-\SI{550}{\mega\electronvolt}          \\
	$m_{a_0}$         & $\SI{870}{\mega\electronvolt}$          & \SI{980}{\mega\electronvolt}                     \\
	$m_\kappa$        & $\SI{1141}{\mega\electronvolt}$         & \SI{1414}{\mega\electronvolt}                    \\
	$m_\eta$          & $\SI{636}{\mega\electronvolt}$          & \SI{548}{\mega\electronvolt}                     \\
	$m_{\eta'}$       & $\SI{138}{\mega\electronvolt}$          & \SI{958}{\mega\electronvolt}                     \\
	$m_\pi$           & $\textbf{\SI{138}{\mega\electronvolt}}$ & \SI{138}{\mega\electronvolt}                     \\
	$m_K$             & $\textbf{\SI{496}{\mega\electronvolt}}$ & \SI{496}{\mega\electronvolt}                     \\
	\bottomrule
\end{tabular}
\hfill
\begin{tabular}{ l r }
	\toprule
	Parameter   & Model                                 \\
	\midrule
	$g$         & $\SI{6.45}{}$                         \\
	$\lambda_1$ & $-\{18.2,13.0,6.2\}$                        \\
	$\lambda_2$ & $ \SI{85.3}{}$                        \\
	$m^2$       & $-(\{178,351,492\}\si{\mega\electronvolt})^2$ \\
	$h_x$       & $(\SI{121}{\mega\electronvolt})^3$  \\
	$h_y$       & $(\SI{336}{\mega\electronvolt})^3$  \\
	\\
	\\
	\\
	\\
	\\
	\\
	\\
	\vspace{-2pt} \\ % align with bottomrule of left table
	\bottomrule 
\end{tabular}
\end{table}

\begin{figure}
\centering
\tikzsetnextfilename{3-flavor-potential-vacuum}
\begin{tikzpicture}
\begin{axis}[
	width = 15cm, height = 10cm,
	restrict x to domain=-510:+510, xmin=-500, xmax=+500, xtick distance = 150, minor x tick num=14,
	restrict y to domain=-610:+610, ymin=-600, ymax=+600, ytick distance = 150, minor y tick num=14,
	%ymin=-11, ymax=+6, ytick distance=5, minor y tick num=4,
	xlabel = {$m_x \, / \, \si{\mega\electronvolt}$}, ylabel = {$m_y \, / \, \si{\mega\electronvolt}$}, zlabel = {$\Omega(\avg{\sigma},0) \, / \, f_\pi^4$},
	%legend style = {at={(0.5,1.03)}, anchor=south}, transpose legend, legend columns=2,
	%cycle list/YlOrRd-9,
	grid = major,
	view = {127}{14}, % 127 7 before
	colormap/Greys, colormap/YlOrRd, % load
	colormap/Greys, mesh/interior colormap name=YlOrRd, % set with name (https://tex.stackexchange.com/a/359491)
]

%\pgfplotsset{cycle list shift=+1} % skip weakest line
\addplot3 [surf, thin, point meta=explicit] table [x=Deltax, y=Deltay, z expr={\thisrow{Omega} +  0}, meta expr={\thisrow{Omega}}] {../code/data/LSM3F/potential_vacuum_sigma500.dat};% \addlegendentry{$m_\sigma = \SI{600}{\mega\electronvolt}$};
\addplot3 [surf, thin, point meta=explicit] table [x=Deltax, y=Deltay, z expr={\thisrow{Omega} + 40}, meta expr={\thisrow{Omega}}] {../code/data/LSM3F/potential_vacuum_sigma600.dat};% \addlegendentry{$m_\sigma = \SI{600}{\mega\electronvolt}$};
\addplot3 [surf, thin, point meta=explicit] table [x=Deltax, y=Deltay, z expr={\thisrow{Omega} + 80}, meta expr={\thisrow{Omega}}] {../code/data/LSM3F/potential_vacuum_sigma700.dat};% \addlegendentry{$m_\sigma = \SI{700}{\mega\electronvolt}$};
\addplot3 [surf, thin, point meta=explicit] table [x=Deltax, y=Deltay, z expr={\thisrow{Omega} +120}, meta expr={\thisrow{Omega}}] {../code/data/LSM3F/potential_vacuum_sigma800.dat};% \addlegendentry{$m_\sigma = \SI{800}{\mega\electronvolt}$};

\addplot3 coordinates {(300,429,-33.0)} node [above, xshift=-0.0cm, yshift=-0.0cm, black, font=\footnotesize] {no minimum}; % 500
\addplot3 [every axis plot post/.append style={only marks, mark=*}, mark size=0.5pt] coordinates {(300,429,+7.0)} node [above, font=\footnotesize] {minimum}; % 600
\addplot3 [every axis plot post/.append style={only marks, mark=*}, mark size=0.5pt] coordinates {(300,429,+41.6)} node [above, font=\footnotesize] {minimum}; % 700
\addplot3 [every axis plot post/.append style={only marks, mark=*}, mark size=0.5pt] coordinates {(300,429,+74.7)} node [above, font=\footnotesize] {minimum}; % 800

\addplot3 coordinates {(450,0,0)} node [black, rotate=-30, yshift=+0.5cm] {$m_\sigma = \SI{500}{\mega\electronvolt}$};
\addplot3 coordinates {(450,0,40)} node [black, rotate=-30, yshift=+0.5cm] {$m_\sigma = \SI{600}{\mega\electronvolt}$};
\addplot3 coordinates {(450,0,80)} node [black, rotate=-30, yshift=+0.5cm] {$m_\sigma = \SI{700}{\mega\electronvolt}$};
\addplot3 coordinates {(450,0,120)} node [black, rotate=-30, yshift=+0.5cm] {$m_\sigma = \SI{800}{\mega\electronvolt}$};
\end{axis}
\end{tikzpicture}
\caption{\label{fig:lsm3f:potential_sigma_mass}%
Three-flavor grand potentials \eqref{eq:lsm3f:grand_potential} in vacuum with $\mu=0$ for different $m_\sigma$,
offset by constants into a human-digestible lasagna.
}
\end{figure}

As explained in the consistently fit two-flavor model in \cref{sec:lsm2f:refinement},
our way of fitting parameters at tree-level to a grand potential calculated to one loop is really inconsistent.
However, we know of no similar consistent calculation done with three flavors.
The lesson we learned there was that inconsistently fit $\SI{600}{\mega\electronvolt} \leq m_\sigma \leq \SI{800}{\mega\electronvolt}$
is representative of the physically measured masses $\SI{400}{\mega\electronvolt} \leq m_\sigma \leq \SI{550}{\mega\electronvolt}$,
so we hope that the same remains true with three flavors.

\section{Equation of state}

\begin{figure}
\centering
\tikzsetnextfilename{3-flavor-eos}
\begin{tikzpicture}
\begin{groupplot}[
	group style={group size={1 by 3}, vertical sep=1.9cm},
	width=13cm, height=7cm,
	extra tick style={grid=major, grid style={dashed}},
	minor tick num=9,
	legend style={xshift=-0.2cm, fill=none},
]
\nextgroupplot[
	xlabel={$\mu \, / \, \si{\mega\electronvolt}$},
	ylabel={$\{m_i,\mu_i\} \, / \, \si{\mega\electronvolt}$},
	%xmin=0, xmax=600, ymax=500, xtick distance=100, ytick distance=100, minor x tick num=9,
	xmin=0, xmax=700, xtick distance=100, minor x tick num=9,
	ymin=-20, ymax=850, ytick distance=100, 
	%ymax=600, 
	title={\subcaption{\label{fig:lsm:3-flavor-eos-parametrization}Parametrization of solutions}},
	legend cell align=left, legend pos=north west, legend columns=5, legend transposed,
];
\addplot+ [orange, dotted, semithick, opacity=0.7, forget plot, domain=0:800] {0};
\addplot+ [yellow, dotted, semithick, opacity=0.7, forget plot, domain=0:800] {0};
\addplot+ [red, dotted, semithick, opacity=0.7, forget plot, domain=0:800] {x};
\addplot+ [darkgreen, dotted, semithick, opacity=0.7, forget plot, domain=0:800] {x};
\addplot+ [blue, dotted, semithick, opacity=0.7, forget plot, domain=0:800] {x};

\addplot+ [orange,    solid,          opacity=0.7,            ] table [x expr={(\thisrow{muu}+\thisrow{mud})/2}, y=mu]  {../code/data/LSM3F/eos_sigma_800.dat}; \addlegendentry{$m_x$};
\addplot+ [orange,    densely dashed, opacity=0.7, forget plot] table [x expr={(\thisrow{muu}+\thisrow{mud})/2}, y=mu]  {../code/data/LSM3F/eos_sigma_700.dat};
\addplot+ [yellow,    solid,          opacity=0.7,            ] table [x expr={(\thisrow{muu}+\thisrow{mud})/2}, y=ms]  {../code/data/LSM3F/eos_sigma_800.dat}; \addlegendentry{$m_y$};
\addplot+ [yellow,    densely dashed, opacity=0.7, forget plot] table [x expr={(\thisrow{muu}+\thisrow{mud})/2}, y=ms]  {../code/data/LSM3F/eos_sigma_700.dat};
\addplot+ [red,       solid,          opacity=0.7,            ] table [x expr={(\thisrow{muu}+\thisrow{mud})/2}, y=muu] {../code/data/LSM3F/eos_sigma_800.dat}; \addlegendentry{$\mu_u$};
\addplot+ [red,       densely dashed, opacity=0.7, forget plot] table [x expr={(\thisrow{muu}+\thisrow{mud})/2}, y=muu] {../code/data/LSM3F/eos_sigma_700.dat};
\addplot+ [darkgreen, solid,          opacity=0.7,            ] table [x expr={(\thisrow{muu}+\thisrow{mud})/2}, y=mud] {../code/data/LSM3F/eos_sigma_800.dat}; \addlegendentry{$\mu_d=\mu_s$};
\addplot+ [darkgreen, densely dashed, opacity=0.7, forget plot] table [x expr={(\thisrow{muu}+\thisrow{mud})/2}, y=mud] {../code/data/LSM3F/eos_sigma_700.dat};
\addplot+ [darkgreen, solid,          opacity=0.7, forget plot] table [x expr={(\thisrow{muu}+\thisrow{mud})/2}, y=mus] {../code/data/LSM3F/eos_sigma_800.dat}; %\addlegendentry{$\mu_s$};
\addplot+ [darkgreen, densely dashed, opacity=0.7, forget plot] table [x expr={(\thisrow{muu}+\thisrow{mud})/2}, y=mus] {../code/data/LSM3F/eos_sigma_700.dat};
\addplot+ [blue,      solid,          opacity=0.7,            ] table [x expr={(\thisrow{muu}+\thisrow{mud})/2}, y=mue] {../code/data/LSM3F/eos_sigma_800.dat}; \addlegendentry{$\mu_e$};
\addplot+ [blue,      densely dashed, opacity=0.7, forget plot] table [x expr={(\thisrow{muu}+\thisrow{mud})/2}, y=mue] {../code/data/LSM3F/eos_sigma_700.dat};

\addplot+ [black, solid] {-100}; \addlegendentry{$m_\sigma=\SI{800}{\mega\electronvolt}$};
\addplot+ [black, densely dashed] {-100}; \addlegendentry{$m_\sigma=\SI{700}{\mega\electronvolt}$};
\addplot+ [black, dotted] {-100}; \addlegendentry{massless};
\addlegendimage{empty legend}; % ???
\addlegendimage{empty legend}; % ???
\addplot+ {-50}; \addlegendimage{empty legend} \addlegendentry{}
\addplot+ {-50}; \addlegendimage{empty legend} \addlegendentry{}

\nextgroupplot[
	xlabel={$\mu \, / \, \si{\mega\electronvolt}$}, ylabel={$n_i \, / \, (1/\si{\femto\meter\cubed})$},
	xmin=0, xmax=700, xtick distance=100, minor x tick num=9,
	ymin=-0.01, ymax=4.0, ytick distance=1.0, minor y tick num=9, restrict y to domain=-10:10,
	title={\subcaption{\label{fig:lsm:3-flavor-eos-density}Particle number densities}},
	legend cell align=left, legend pos=north west, legend columns=4, legend transposed, reverse legend,
];
%\coordinate (zoomplot) at (232, 1.32);
%\draw [draw=none, fill=gray!50] (250, -0.005) rectangle (350, 0.02);
%\draw [-Latex, gray] (300,0.03) to [out=90, in=0] (232, 0.75);
%\addplot+ [red,       densely dashed, semithick, opacity=0.4, forget plot, domain=0:600] {nq(muu(x))*nconv};
%\addplot+ [darkgreen, densely dashed, semithick, opacity=0.4, forget plot, domain=0:600] {nq(mud(x))*nconv};
%\addplot+ [blue,      densely dashed, semithick, opacity=0.4, forget plot, domain=0:600] {ne(mue(x))*nconv};
\addplot+ [black] {-1}; \addlegendimage{empty legend}; \addlegendentry{\vphantom{$m = \SI{0}{\mega\electronvolt}$}};
\addplot+ [black, dotted] {-1}; \addlegendentry{massless};
\addplot+ [black, densely dashed] {-1}; \addlegendentry{$m_\sigma=\SI{700}{\mega\electronvolt}$};
\addplot+ [black, solid] {-1}; \addlegendentry{$m_\sigma=\SI{800}{\mega\electronvolt}$};
\addplot+ [blue,      densely dashed, opacity=0.7, forget plot] table [x expr={(\thisrow{muu}+\thisrow{mud})/2}, y=ne] {../code/data/LSM3F/eos_sigma_700.dat};
\addplot+ [blue,      solid,          opacity=0.7,            ] table [x expr={(\thisrow{muu}+\thisrow{mud})/2}, y=ne] {../code/data/LSM3F/eos_sigma_800.dat}; \addlegendentry{$n_e$};
\addplot+ [purple,    densely dashed, opacity=0.7, forget plot] table [x expr={(\thisrow{muu}+\thisrow{mud})/2}, y=ns] {../code/data/LSM3F/eos_sigma_700.dat};
\addplot+ [purple,    solid,          opacity=0.7,            ] table [x expr={(\thisrow{muu}+\thisrow{mud})/2}, y=ns] {../code/data/LSM3F/eos_sigma_800.dat}; \addlegendentry{$n_s$};
\addplot+ [darkgreen, densely dashed, opacity=0.7, forget plot] table [x expr={(\thisrow{muu}+\thisrow{mud})/2}, y=nd] {../code/data/LSM3F/eos_sigma_700.dat};
\addplot+ [darkgreen, solid,          opacity=0.7,            ] table [x expr={(\thisrow{muu}+\thisrow{mud})/2}, y=nd] {../code/data/LSM3F/eos_sigma_800.dat}; \addlegendentry{$n_d$};
\addplot+ [red,       densely dashed, opacity=0.7, forget plot] table [x expr={(\thisrow{muu}+\thisrow{mud})/2}, y=nu] {../code/data/LSM3F/eos_sigma_700.dat};
\addplot+ [red,       solid,          opacity=0.7,            ] table [x expr={(\thisrow{muu}+\thisrow{mud})/2}, y=nu] {../code/data/LSM3F/eos_sigma_800.dat}; \addlegendentry{$n_u$};

\nextgroupplot[
	xlabel={$P        \, / \, (\si{\giga\electronvolt\per\femto\meter\cubed})$},
	ylabel={$\epsilon \, / \, (\si{\giga\electronvolt\per\femto\meter\cubed})$},
	xmin=-0.005, xmax=0.8, ymin=0, ymax=3.5, xtick distance=0.1, ytick distance=1.0, minor y tick num=9, restrict y to domain=-10:+10,
	title={\subcaption{\label{fig:lsm:3-flavor-eos-eos}Equation of state}},
	legend cell align=left, legend pos=north west,
];
%\addplot+ [black!50!white, densely dashed, semithick, opacity=0.7, forget plot] table [x=P, y expr={3*\thisrow{P}+4*0.07774628475613433}] {../code/data/LSM2F/eos.dat}; % +4*P0, since both ϵ and P modified by P0
%\addplot+ [black, densely dashed, opacity=0.4, domain=0:0.1, forget plot] {3*x}; % +4*P0, since both ϵ and P modified by P0
\addplot+ [black, solid] table [x=P,y=epsilon] {../code/data/LSM3F/eos_sigma_800.dat}; \addlegendentry{$m_\sigma=\SI{800}{\mega\electronvolt}$};
\addplot+ [gray, densely dashed] table [x=Porg,y=epsilonorg] {../code/data/LSM3F/eos_sigma_700.dat}; \addlegendentry{$m_\sigma=\SI{700}{\mega\electronvolt}$ (uncorrected)};
\addplot+ [black, densely dashed] table [x=P,y=epsilon] {../code/data/LSM3F/eos_sigma_700.dat}; \addlegendentry{$m_\sigma=\SI{700}{\mega\electronvolt}$ (corrected)};
\addplot+ [black, dotted, domain=0:3.0] {3*x}; \addlegendentry{massless};
\coordinate (zoomplot) at (0.55, 0.0);
\end{groupplot}
\node [anchor=south west] at (zoomplot) {
	\begin{tikzpicture}[trim axis left, trim axis right] % use axes as bounding box
	\begin{axis}[
		width=4cm, height=4cm,
		axis background/.style={fill=white},
		xmin=-0.002, xmax=+0.01, xtick distance=0.01, restrict x to domain=-0.03:+0.03,
		ymin=0, ymax=0.3, restrict y to domain=-1:+1,
	]
	\addplot+ [black, solid, opacity=0.7] table [x=P,y=epsilon] {../code/data/LSM3F/eos_sigma_800.dat};
	\addplot+ [gray, densely dashed, opacity=0.7] table [x=Porg,y=epsilonorg] {../code/data/LSM3F/eos_sigma_700.dat};
	\addplot+ [black, densely dashed, opacity=0.7] table [x=P,y=epsilon] {../code/data/LSM3F/eos_sigma_700.dat};
	\addplot+ [black, dotted, opacity=0.7, domain=0:0.05] {3*x};
	\end{axis}
	\end{tikzpicture}
};
\end{tikzpicture}
\caption{\label{fig:lsm:3-flavor-eos}%
Properties of electrically charge neutral three-flavor quark matter in $\beta$-equilibrium parametrized by the common quark chemical potential $\mu = (\mu_u+\mu_d)/2$.
Upper panel \subref{fig:lsm:3-flavor-eos-parametrization} shows solutions to equation \eqref{eq:lsm3f:minsys},
middle panel \subref{fig:lsm:3-flavor-eos-density} the corresponding particle number densities \eqref{eq:lsm3f:particle_densities} and
lower panel \subref{fig:lsm:3-flavor-eos-eos} the corresponding equation of state $\epsilon(P)$ corrected with the Maxwell construction.
Dotted lines show the massless solutions \eqref{eq:mit:densities_massless}, \eqref{eq:mit:eos_ur} and \eqref{eq:mit:chemical_potentials_massless_3f}.
}
\end{figure}

\begin{figure}
\centering
\tikzsetnextfilename{3-flavor-density-ratios}
\begin{tikzpicture}
\begin{axis}[
	width=12cm, height=5cm,
	xlabel={$n_B \, / \, n_\text{sat}$}, ylabel={$n_i \, / \, n_B$},
	xmin=0, xmax=50, xtick distance=5, minor x tick num=4, restrict x to domain=0.01:100,
	ymin=-0.05, ymax=2.05, ytick distance=0.5, minor y tick num=4,
	declare function={sat=0.165;},
	legend cell align=left, legend columns=4, legend transposed,
]
\addplot [blue] table [x expr={(\thisrow{nu}+\thisrow{nd}+\thisrow{ns})/sat/3}, y expr={3*\thisrow{ne}/(\thisrow{nu}+\thisrow{nd}+\thisrow{ns})}] {../code/data/LSM3F/eos.dat};
\addlegendentry{$n_e$};
\addplot [red] table [x expr={(\thisrow{nu}+\thisrow{nd}+\thisrow{ns})/sat/3}, y expr={3*\thisrow{nu}/(\thisrow{nu}+\thisrow{nd}+\thisrow{ns})}] {../code/data/LSM3F/eos.dat};
\addlegendentry{$n_u$};
\addplot [darkgreen] table [x expr={(\thisrow{nu}+\thisrow{nd}+\thisrow{ns})/sat/3}, y expr={3*\thisrow{nd}/(\thisrow{nu}+\thisrow{nd}+\thisrow{ns})}] {../code/data/LSM3F/eos.dat};
\addlegendentry{$n_d$};
\addplot [purple] table [x expr={(\thisrow{nu}+\thisrow{nd}+\thisrow{ns})/sat/3}, y expr={3*\thisrow{ns}/(\thisrow{nu}+\thisrow{nd}+\thisrow{ns})}] {../code/data/LSM3F/eos.dat};
\addlegendentry{$n_s$};
\end{axis}
\end{tikzpicture}
\caption{\label{fig:lsm3f:3-flavor-density-ratios}%
Fractions of each particle density $n_i$ to the baryon density $n_B = (n_u+n_d+n_s)/3$
in units of the nuclear saturation density $n_\text{sat} = \SI{0.165}{\per\femto\meter\cubed}$
for the charge-neutral three-flavor quark matter in \cref{fig:lsm:3-flavor-eos}.
}
\end{figure}

To find the equation of state we will follow the same procedure as in \cref{chap:lsm2f}.
The generalization of the system of equations \eqref{eq:lsm:minsys} with strange quarks that we need to solve,
subject to charge neutrality \eqref{eq:lsm:charge_neutrality} and chemical equilibrium \eqref{eq:lsm:chemical_equilibrium},
is
\begin{subequations}
\begin{align}
	0 &= \pdv{\Omega}{\avg{\sigma_x}} , \label{eq:lsm3f:minsys_minx} \\
	0 &= \pdv{\Omega}{\avg{\sigma_y}} , \label{eq:lsm3f:minsys_miny} \\
	0 &= 2 \Big(\mu_u^2-m_u^2\Big)^\frac32 - \Big(\mu_d^2-m_d^2\Big)^\frac32 - \Big(\mu_s^2-m_s^2\Big)^\frac32 - \Big(\mu_e^2-m_e^2\Big)^\frac32 \label{eq:lsm:minsys3f_charge}, \\
	\mu_d &= \mu_u + \mu_e, \\
	\mu_s &= \mu_d .
\end{align}%
\label{eq:lsm3f:minsys}%
\end{subequations}%
This is a system of five equations for the six unknowns
$\avg{\sigma_x}$, $\avg{\sigma_y}$, $\mu_u$, $\mu_d$, $\mu_s$ and $\mu_e$.
Like in \cref{chap:lsm2f},
we parametrize solutions with $\avg{\sigma_x}$,
calculate the particle densities
\begin{equation}
	n_f = -\pdv{\Omega}{\mu_f} = \frac{N_c}{3 \pi^2} \Big( \mu_f^2 - m_f^2 \Big)^{\frac32}
	\qquad \text{and} \qquad
	n_e = -\pdv{\Omega}{\mu_e} = \frac{  1}{3 \pi^2} \Big( \mu_e^2 - m_e^2 \Big)^{\frac32},
\label{eq:lsm3f:particle_densities}
\end{equation}%
the pressure \eqref{eq:master_intro:pressure} and energy density \eqref{eq:master_intro:energy_density}, and finally eliminate the free variable to get the equation of state $\epsilon(P)$.
The numerical implementation in \cref{sec:num:qstars2f} yields
the solutions, particle densities and equation of state shown in \cref{fig:lsm:3-flavor-eos}:
\begin{itemize}
\item There is a non-strange crossover (for $m_\sigma \geq \SI{800}{\mega\electronvolt}$)
      or discontinuous phase transition (for $m_\sigma < \SI{800}{\mega\electronvolt}$)
      at $\mu=\SI{300}{\mega\electronvolt}$, like in the two-flavor case in \cref{fig:lsm:2-flavor-eos}.
      Even the strange minimum $\avg{\sigma_y}$ moves a bit during this transition
      due to the cross term $\lambda_1 \avg{\sigma_x}^2 \avg{\sigma_y}^2 / 2$ in the meson potential \eqref{eq:lsm:potential3f} changing.
      Accordingly, the equations of state for the two-flavor and three-flavors models are quite similar up to $P \leq \SI{0.05}{\giga\electronvolt\per\femto\meter\cubed}$.
\item There is a strange crossover at $\mu = \SI{429}{\mega\electronvolt}$ for all values of $m_\sigma$.
      It is never a phase transition, is significantly slower than the non-strange transition,
      and the strange quark retains a noticeable mass when the non-strange quarks have lost theirs.
      During this crossover, the equation of state gently ramps up from the two-flavor level to a second plateau
      and softens compared to the two-flavor equation of state.
\item Whereas the up and down quark densities become nonzero at their vacuum mass $\mu = \SI{300}{\mega\electronvolt}$,
      the strange quark density does so at a value $\mu < \SI{429}{\mega\electronvolt}$ \emph{below} its vacuum mass.
      The explanation (for $m_\sigma=\SI{800}{\mega\electronvolt}$) is that $m_x$ remains at its vacuum value up to $\mu = \SI{300}{\mega\electronvolt}$,
      beyond which $m_x$ and $m_y$ \emph{both} fall.
      By the time it intersects $\mu_s$ at $\mu = \SI{375}{\mega\electronvolt}$,
      the effective strange mass has fallen to $m_y = \SI{410}{\mega\electronvolt}$.
\item In the massless limit $\avg{\sigma_x} \rightarrow \avg{\sigma_y} \rightarrow 0$, the ultra-relativistic solution \eqref{eq:mit:chemical_potentials_massless_3f} is recovered.
\item After spreading before the strange crossover, $\mu_u$ and $\mu_d$ find back to each other in the ultra-relativistic limit,
      so the isospin chemical potential $\mu_I=(\mu_u-\mu_d)/2$ never exceeds half the pion mass.
      In contrast to the two-flavor case, neglecting pion condensation is therefore a consistent assumption with three flavors.
\item As confirmed by \cref{fig:lsm3f:3-flavor-density-ratios}, each quark density converges towards a third of the total baryon density as the density increases.
      At low densities the strange quark is absent and there are twice as many down quarks as up quarks, as in the two-flavor case,
      and the strange quark enters as the density exceeds $4 \, n_\text{sat}$.
      The fraction of up quarks is the same at all densities.
\item After the strange crossover, the strange quark provides charge neutrality and wipes out the already small electron density,
      consistent with the ultra-relativistic solution \eqref{eq:mit:chemical_potentials_massless_3f}.
\item Like before, the equation of state always satisfies the microscopic stability criteria \eqref{eq:nstars:stability_speed_of_sound} and \eqref{eq:nstars:stability_pressure_energy_density}.
\end{itemize}

Once again, we modify the equation of state with bag constants $B$ according to the shift \eqref{eq:mit:bag_shift}
and determine upper bounds for them by solving the three-flavor inequality \eqref{eq:mit:bag_stability}, owing to the strange matter hypothesis.
This yields the upper bag constant bounds
\begin{subequations}
\begin{align}
	%\text{no solution!} \Big(\text{with } m_\sigma = \SI{700}{\mega\electronvolt}\Big), \label{eq:lsm3f:bag_upper_bound_700} \\
	%B(f_\pi)^\frac14 < \SI{46.5}{\mega\electronvolt}, \quad \text{or} \quad B(0)^\frac14 > \SI{156.5}{\mega\electronvolt} \qquad \Big(\text{with } m_\sigma = \SI{800}{\mega\electronvolt}\Big). \label{eq:lsm3f:bag_upper_bound_800}
	B \leq (\SI{112.0}{\mega\electronvolt})^4           \quad \Big(\text{or } B-\pot_0 \leq (\SI{226.4}{\mega\electronvolt})^4\Big) \qquad \Big(m_\sigma = \SI{600}{\mega\electronvolt}\Big), \label{eq:lsm3f:bag_upper_bound_600} \\
	B \leq \phantom{0}(\SI{68.2}{\mega\electronvolt})^4 \quad \Big(\text{or } B-\pot_0 \leq (\SI{231.4}{\mega\electronvolt})^4\Big) \qquad \Big(m_\sigma = \SI{700}{\mega\electronvolt}\Big), \label{eq:lsm3f:bag_upper_bound_700} \\
	B \leq \phantom{0}(\SI{27.0}{\mega\electronvolt})^4 \quad \Big(\text{or } B-\pot_0 \leq (\SI{241.3}{\mega\electronvolt})^4\Big) \qquad \Big(m_\sigma = \SI{800}{\mega\electronvolt}\Big). \label{eq:lsm3f:bag_upper_bound_800}
\end{align}%
\label{eq:lsm3f:bag_upper_bound}%
\end{subequations}%
These are so close to the lower bounds \eqref{eq:lsm:bag_lower_bound} that the inequalities practically become equalities,
and the bag constant is more or less fixed \emph{if} the strange matter hypothesis holds, but can take greater values if it does not.
Like before, we focus on the lowest bag constants since they generate greater maximum masses.
Note that if one uses a single renormalization scale $\Lambda$ and let the vacuum quark masses move from their fit values,
like \cite{ref:master_berge},
one obtains the rather different bounds $B^\frac14 \leq \{6.3,48.2\} \, \si{\mega\electronvolt}$ for $m_\sigma=\{700,800\}\,\si{\mega\electronvolt}$,
whereas the potential is broken for $m_\sigma=\SI{600}{\mega\electronvolt}$.
These $m_\sigma=\{600,700\}\,\si{\mega\electronvolt}$-results
are impossible to reconcile with the lower bag bounds \eqref{eq:lsm:bag_lower_bound}.
This is one of the reasons for why we operate with two renormalization scales here.

\pgfplotsset{
	mesh line legend/.style={legend image code/.code=\meshlinelegend#1},
}
%% Code for the coloured line legend
%% adapted from https://tex.stackexchange.com/a/59075 with pgfplots manual "line legend" size (0.6cm, 0.1cm)
\makeatletter
\long\def\meshlinelegend#1{%
    \scope[%
        #1,
        /pgfplots/mesh/rows=1,
        /pgfplots/mesh/cols=4,
        /pgfplots/mesh/num points=,
        /tikz/x={(0.6cm,0cm)}, %/tikz/x={(0.44237cm,0cm)},
        /tikz/y={(0cm,0.1cm)}, %/tikz/y={(0cm,0.23932cm)},
        /tikz/z={(0.0cm,0cm)},
        scale=1.0, %scale=0.4,
    ]
    \let\pgfplots@metamax=\pgfutil@empty
    \pgfplots@curplot@threedimtrue

    \pgfplotsplothandlermesh
    \pgfplotstreamstart

    \def\simplecoordinate(##1,##2,##3){%
        \pgfmathparse{1000*(##3)}%
        \pgfmathfloatparsenumber\pgfmathresult
        \let\pgfplots@current@point@meta=\pgfmathresult
        \pgfplotstreampoint{\pgfqpointxyz@orig{##1}{##2}{##3}}%
    }%

    \simplecoordinate(0,0,0)
    \simplecoordinate(0.125,0,0.125)
    \simplecoordinate(0.25,0,0.25)
    \simplecoordinate(0.375,0,0.375)
    \simplecoordinate(0.5,0,0.5)
    \simplecoordinate(0.625,0,0.625)
    \simplecoordinate(0.75,0,0.75)
    \simplecoordinate(0.875,0,0.875)
    \simplecoordinate(1,0,1)

    \pgfplotstreamend
    \pgfusepath{stroke}
    \endscope
}%
\makeatother
%% End code for the coloured line legend

\begin{figure}[p]
\centering
\tikzsetnextfilename{3-flavor-mass-radius}
\begin{tikzpicture}
\begin{groupplot}[
	group style={group size={3 by 1}, vertical sep=0cm, horizontal sep=0.3cm},
	width=6cm, height=6cm,
	xmin=5, xmax=20, ymin=0.5, ymax=2.5, xtick distance=5, ytick distance=0.5, minor tick num=4, grid=major,
	point meta=explicit, point meta min=33, point meta max=36,
	%colorbar horizontal, colormap name=plasmarev, colorbar style={xlabel=$\log_{10} (P_c \, / \, \si{\pascal})$, xtick distance=1, minor x tick num=9, at={(0.5,1.03)}, anchor=south, xticklabel pos=upper},
	/tikz/declare function={
		e0 = 4.266500881855304e+37;
	},
	legend columns=2, legend style={anchor=north, at={(0.5, 0.96)}},
]
\tikzset{
	Bpin/.style={gray, sloped, allow upside down=true, rotate=180, yshift=+0.4cm, font=\small},
}
\nextgroupplot[
	xlabel={$R \, / \, \si{\kilo\meter}$ },
	ylabel={$M \, / \, M_\odot$}, %title={Mass-radius diagram for 2-flavor quark stars }, title style={yshift=2.0cm},
	title = {$m_\sigma = \SI{600}{\mega\electronvolt}$ \\ $B^\frac14 = \{111,131,151\} \, \si{\mega\electronvolt}$},
];
\addplot+ [solid, gray, opacity=0.5] table [x=R, y=M, meta expr={log10(\thisrow{P}*e0)}] {../code/data/LSM2F/stars_sigma_600_B14_111.dat}; % node [Bpin, pos=0.920] {$B = (\SI{27}{\mega\electronvolt})^4$};
\addplot+ [solid, gray, opacity=0.5, forget plot] table [x=R, y=M, meta expr={log10(\thisrow{P}*e0)}] {../code/data/LSM2F/stars_sigma_600_B14_131.dat}; % node [Bpin, pos=0.920] {$B = (\SI{27}{\mega\electronvolt})^4$};
\addplot+ [solid, gray, opacity=0.5, forget plot] table [x=R, y=M, meta expr={log10(\thisrow{P}*e0)}] {../code/data/LSM2F/stars_sigma_600_B14_151.dat}; % node [Bpin, pos=0.920] {$B = (\SI{27}{\mega\electronvolt})^4$};
\addlegendentry{$N_f \!=\! 2$};
\addplot+ [solid, mesh, mesh line legend] table [x=R, y=M, meta expr={log10(\thisrow{P}*e0)}] {../code/data/LSM3F/stars_sigma_600_B14_111.dat}; % node [Bpin, pos=0.920] {$B = (\SI{27}{\mega\electronvolt})^4$};
\addplot+ [solid, mesh, forget plot] table [x=R, y=M, meta expr={log10(\thisrow{P}*e0)}] {../code/data/LSM3F/stars_sigma_600_B14_131.dat}; % node [Bpin, pos=0.920] {$B = (\SI{27}{\mega\electronvolt})^4$};
\addplot+ [solid, mesh, forget plot] table [x=R, y=M, meta expr={log10(\thisrow{P}*e0)}] {../code/data/LSM3F/stars_sigma_600_B14_151.dat}; % node [Bpin, pos=0.920] {$B = (\SI{27}{\mega\electronvolt})^4$};
\addlegendentry{$N_f \!=\! 3$};

\nextgroupplot[
	xlabel={$R \, / \, \si{\kilo\meter}$},
	yticklabels={,,},
	title = {$m_\sigma = \SI{700}{\mega\electronvolt}$ \\ $B^\frac14 = \{68,88,108\} \, \si{\mega\electronvolt}$},
	colorbar horizontal, colormap name=plasmarev, colorbar style={width=11cm, ylabel=$\log_{10} (P_c \, / \, \si{\pascal})$, ylabel style={rotate=-90}, xtick distance=1, minor x tick num=9, at={(0.5,-0.3)}, anchor=north, xticklabel pos=lower},
];
\addplot+ [solid, gray, opacity=0.5] table [x=R, y=M, meta expr={log10(\thisrow{P}*e0)}] {../code/data/LSM2F/stars_sigma_700_B14_68.dat}; % node [Bpin, pos=0.920] {$B = (\SI{27}{\mega\electronvolt})^4$};
\addplot+ [solid, gray, opacity=0.5, forget plot] table [x=R, y=M, meta expr={log10(\thisrow{P}*e0)}] {../code/data/LSM2F/stars_sigma_700_B14_88.dat}; % node [Bpin, pos=0.920] {$B = (\SI{27}{\mega\electronvolt})^4$};
\addplot+ [solid, gray, opacity=0.5, forget plot] table [x=R, y=M, meta expr={log10(\thisrow{P}*e0)}] {../code/data/LSM2F/stars_sigma_700_B14_108.dat}; % node [Bpin, pos=0.920] {$B = (\SI{27}{\mega\electronvolt})^4$};
\addlegendentry{$N_f=2$};
\addplot+ [solid, mesh, mesh line legend] table [x=R, y=M, meta expr={log10(\thisrow{P}*e0)}] {../code/data/LSM3F/stars_sigma_700_B14_68.dat};  % node [Bpin, pos=0.920] {$B = (\SI{27}{\mega\electronvolt})^4$};
\addplot+ [solid, mesh, forget plot] table [x=R, y=M, meta expr={log10(\thisrow{P}*e0)}] {../code/data/LSM3F/stars_sigma_700_B14_88.dat};  % node [Bpin, pos=0.920] {$B = (\SI{27}{\mega\electronvolt})^4$};
\addplot+ [solid, mesh, forget plot] table [x=R, y=M, meta expr={log10(\thisrow{P}*e0)}] {../code/data/LSM3F/stars_sigma_700_B14_108.dat}; % node [Bpin, pos=0.920] {$B = (\SI{27}{\mega\electronvolt})^4$};
\addlegendentry{$N_f=3$};

\nextgroupplot[
	xlabel={$R \, / \, \si{\kilo\meter}$},
	yticklabels={,,},
	title = {$m_\sigma = \SI{800}{\mega\electronvolt}$ \\ $B^\frac14 = \{27,47,67\} \, \si{\mega\electronvolt}$},
];
\addplot+ [solid, gray, opacity=0.5] table [x=R, y=M, meta expr={log10(\thisrow{P}*e0)}] {../code/data/LSM2F/stars_sigma_800_B14_27.dat}; % node [Bpin, pos=0.920] {$B = (\SI{27}{\mega\electronvolt})^4$};
\addplot+ [solid, gray, opacity=0.5, forget plot] table [x=R, y=M, meta expr={log10(\thisrow{P}*e0)}] {../code/data/LSM2F/stars_sigma_800_B14_47.dat}; % node [Bpin, pos=0.920] {$B = (\SI{27}{\mega\electronvolt})^4$};
\addplot+ [solid, gray, opacity=0.5, forget plot] table [x=R, y=M, meta expr={log10(\thisrow{P}*e0)}] {../code/data/LSM2F/stars_sigma_800_B14_67.dat}; % node [Bpin, pos=0.920] {$B = (\SI{27}{\mega\electronvolt})^4$};
\addlegendentry{$N_f=2$};
\addplot+ [solid, mesh, mesh line legend] table [x=R, y=M, meta expr={log10(\thisrow{P}*e0)}] {../code/data/LSM3F/stars_sigma_800_B14_27.dat}; % node [Bpin, pos=0.920] {$B = (\SI{27}{\mega\electronvolt})^4$};
\addplot+ [solid, mesh, forget plot] table [x=R, y=M, meta expr={log10(\thisrow{P}*e0)}] {../code/data/LSM3F/stars_sigma_800_B14_47.dat}; % node [Bpin, pos=0.920] {$B = (\SI{27}{\mega\electronvolt})^4$};
\addplot+ [solid, mesh, forget plot] table [x=R, y=M, meta expr={log10(\thisrow{P}*e0)}] {../code/data/LSM3F/stars_sigma_800_B14_67.dat}; % node [Bpin, pos=0.920] {$B = (\SI{27}{\mega\electronvolt})^4$};
\addlegendentry{$N_f=3$};
\node[scale=0.75] at (11.565, 1.630) {\goldenstar};

\end{groupplot}
\node [anchor=south, yshift=+1.5cm] at (group c2r1.north) {Three-flavor quark-meson model quark stars};
\end{tikzpicture}
\caption{\label{fig:lsm:3-flavor-mass-radius}%
	Mass-radius solutions of the Tolman-Oppenheimer-Volkoff equations \eqref{eq:tov:tovsys} parametrized by the central pressure $P_c$, obtained with the equations of state for three-flavor quark matter in \cref{fig:lsm:3-flavor-eos-eos} modified by a range of bag constants above the bounds \eqref{eq:lsm:bag_lower_bound}.
}

\bigskip

\tikzsetnextfilename{3-flavor-extreme-star}
\begin{tikzpicture}
\begin{groupplot}[
	group style={group size={2 by 2}, horizontal sep=1.2cm, vertical sep=0.4cm},
	width=8cm, height=6cm,
	ylabel style={yshift=-0.2cm},
	enlargelimits=false, xtick distance=1.0, minor xtick={0,0.1,...,11.6},
	legend cell align=left,
]
\nextgroupplot[
	xticklabels={,,},
	ymax=1.5, ytick distance=0.5, minor y tick num=4,
	ylabel={$\{\epsilon,P\} \, / \, (\si{\giga\electronvolt\per\femto\meter\cubed})$},
];
\addplot+ [blue] table [x=r, y=epsilon] {../code/data/LSM3F/star_sigma_800_B14_27_Pc_0.0009376.dat}; \addlegendentry{$\epsilon$};
\addplot+ [red] table [x=r, y=P] {../code/data/LSM3F/star_sigma_800_B14_27_Pc_0.0009376.dat}; \addlegendentry{$P$};
\nextgroupplot[
	ylabel=$m \, / \, M_\odot$,
	xticklabels={,,},
	ymax=2, ytick distance=0.5, minor y tick num=4,
];
\addplot+ [black] table [x=r, y=m] {../code/data/LSM3F/star_sigma_800_B14_27_Pc_0.0009376.dat};
\nextgroupplot[
	xlabel=$r \, / \, \si{\kilo\meter}$,
	ylabel=$\mu \, / \, \si{\mega\electronvolt}$,
	ymin=300, ymax=500, ytick distance=50, minor y tick num=4,
];
\addplot+ [black] table [x=r, y=muQ] {../code/data/LSM3F/star_sigma_800_B14_27_Pc_0.0009376.dat};
\nextgroupplot[
	xlabel=$r \, / \, \si{\kilo\meter}$,
	ylabel=$n_i \, / \, n_\text{sat}$,
	ymax=15, ytick distance=5, minor y tick num=4,
];
\addplot+ [red] table [x=r, y=nu] {../code/data/LSM3F/star_sigma_800_B14_27_Pc_0.0009376.dat}; \addlegendentry{$n_u$};
\addplot+ [darkgreen] table [x=r, y=nd] {../code/data/LSM3F/star_sigma_800_B14_27_Pc_0.0009376.dat}; \addlegendentry{$n_d$};
\addplot+ [purple] table [x=r, y=ns] {../code/data/LSM3F/star_sigma_800_B14_27_Pc_0.0009376.dat}; \addlegendentry{$n_s$};
\addplot+ [blue] table [x=r, y=ne] {../code/data/LSM3F/star_sigma_800_B14_27_Pc_0.0009376.dat}; \addlegendentry{$n_e$};
\addplot+ [black, dashed, forget plot] table [x=r, y expr={(\thisrow{nu}+\thisrow{nd}+\thisrow{ns})/3}] {../code/data/LSM3F/star_sigma_800_B14_27_Pc_0.0009376.dat};
% cheat legend
\addplot+ [draw=none, black, solid] table [x=r, y expr={(\thisrow{nu}+\thisrow{nd}+\thisrow{ns})/3}] {../code/data/LSM3F/star_sigma_800_B14_27_Pc_0.0009376.dat}; \addlegendentry{$n_B$};
\end{groupplot}
\node (title) at ($(group c1r1.north)!0.5!(group c2r1.north)$) [above, yshift=\pgfkeysvalueof{/pgfplots/every axis title shift}] {\goldenstar Maximum mass star ($m_\sigma=\SI{800}{\mega\electronvolt}$, $B^\frac14 = \SI{27}{\mega\electronvolt}$, $P_c=10^{34.602} \, \si{\pascal}$)};
\end{tikzpicture}
\caption{\label{fig:lsm:3-flavor-star}%
	Radial profiles for the
	pressure $P$,
	energy density $\epsilon$,
	cumulative mass $m$,
	quark chemical potential $\mu$,
	particle densities $n_i$
	and baryon density $n_B = (n_u+n_d)/3$
	for the maximum mass three-flavor quark star \goldenstar in \cref{fig:lsm:3-flavor-mass-radius}.
}

\end{figure}

\section{Stellar solutions}

Solving the Tolman-Oppenheimer-Volkoff equations \eqref{eq:master_intro:tov} like in \cref{chap:lsm2f} with the same bag constants used there,
we obtain the mass-radius solutions displayed in \cref{fig:lsm:3-flavor-mass-radius}:
\begin{itemize}
\item For stars with sufficiently low central pressure,
      the quark chemical potential $\mu$ inside does not reach large enough levels to activate the strange quark.
      Then the three-flavor equation of state tracks the two-flavor one almost perfectly, and so does the mass-radius curve.
      It is somewhat surprising that the correspondence is so good, given that the two models use different vacuum potentials.
\item For stars with greater central pressure, $\mu$ reaches sufficient levels and the strange quark is present in the core of the star.
      Exactly at the threshold, we see that the mass-radius curve grows apart from the two-flavor curve into a new branch with lower stellar masses.
      This is caused by the softening of the equation of state as it ramps up to another plateau during the strange crossover.
\item The masses and radii are generally comparable to the two-flavor results, except that the mass is lowered for stars with strange quarks.
      For stars that just satisfy the lower bag bound and hence respect instability of two-flavor quark matter with respect to hadronic matter,
      we find masses $1.6 \, M_\odot\leq M \leq 1.8 \, M_\odot$ and radii $\SI{11}{\kilo\meter} \leq R \leq \SI{12}{\kilo\meter}$ depending on the precise mass $m_\sigma$.
      Greater bag constants violate the strange matter hypothesis and result in only lighter and smaller stars, like before.
\end{itemize}

In \cref{fig:lsm:3-flavor-star} we take a more detailed look at the maximum mass star that has no phase transition and respects the strange matter hypothesis:
\begin{itemize}
\item Apart from the presence of strange quarks, many features are similar to that of the two-flavor star we discussed on page \pageref{list:lsm:2-flavor-star-discussion}.
\item The star is composed of three-flavor quark matter including strange quarks for $r \leq \SI{6.5}{\kilo\meter}$ and two-flavor quark matter out to the surface.
\item With a maximum chemical potential $\mu=\SI{465}{\mega\electronvolt}$, the strange quark has a significant mass at the center compared to the up and down quarks, as seen from \cref{fig:lsm:3-flavor-eos-parametrization}.
      Accordingly, the star is far from realizing ultra-relativistic three-flavor quark matter with equal proportions of each quark.
\item The maximum central chemical potential is comparable in the two-flavor and three-flavor models.
      If we assert that this carries over to a four-flavor model that also includes the charm quark with a constituent mass $m_c \approx \SI{1550}{\mega\electronvolt}$,
      we would expect it to be present only for extreme central pressures and chemical potentials that lie far beyond the maximum mass star.
      For physical and stable stars, three-flavor models are therefore the most complicated ones we need to study.
%\item If the core of this star is used to model the core of a hybrid neutron star with a quark model for $n_B > 4 n_\text{sat}$,
      %this core would extend for $R'=\SI{5}{\kilo\meter}$, weigh in at $M' = m(R') = 0.4 \, M_\odot$, have a central pressure $P_c'=p(R')=\SI{0.12}{\giga\electronvolt\per\femto\meter\cubed}$ and be entirely composed of three-flavor quark matter.
\end{itemize}

\section{Summary}

In this chapter we have studied quark stars modeled with the three-flavor quark-meson model.
We found that the addition of the strange quark lowered the maximum masses to $1.6 \, M_\odot \leq M \leq 1.8 \, M_\odot$
with corresponding radii $\SI{11}{\kilo\meter} \leq R \leq \SI{12}{\kilo\meter}$,
using $\SI{600}{\mega\electronvolt} \leq m_\sigma \leq \SI{800}{\mega\electronvolt}$ and the lowest stability-respecting bag constants.
If the strange matter hypothesis is violated, one can use greater bag constants and produce arbitrarily smaller and less massive stars.
The three-flavor stars are less massive than their two-flavor counterparts.
Only close to the maximum mass star does the pressure become sufficiently high
for the baryon density $n_B$ to exceed $4 n_\text{sat}$
and activate the presence of the strange quark.
As with the two-flavor model, we saw that chiral symmetry restoration happens in a crossover for $m_\sigma \geq \SI{800}{\mega\electronvolt}$,
but a first-order phase transition for $m_\sigma < \SI{800}{\mega\electronvolt}$.

By fitting parameters at tree-level, it was again impossible to use physical values of $m_\sigma$ without breaking the grand potential.
Ideally, a consistent calculation like the one presented in \cref{sec:lsm2f:refinement}
should be carried out also for the three-flavor model,
where one takes loop effects into account when fitting the parameters of the grand potential.
Unfortunately, no such calculation has been done yet,
and we rely on the lesson learned in \cref{sec:lsm2f:refinement}:
that inconsistently fit $\SI{600}{\mega\electronvolt} \leq m_\sigma \leq \SI{800}{\mega\electronvolt}$
roughly correspond to physically measured masses $\SI{400}{\mega\electronvolt} \leq m_\sigma \leq \SI{550}{\mega\electronvolt}$.
