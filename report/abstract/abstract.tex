\chapter*{Abstract}
\addcontentsline{toc}{chapter}{Abstract} % but still display in TOC (see https://tex.stackexchange.com/a/222961)

General relativity and quantum field theory are indispensable theories for the study of compact stars, composed of subatomic particles with extreme density.
In this thesis, we solve the Tolman-Oppenheimer-Volkoff system of equations for a cold Fermi gas composed of free neutrons, producing a mass-radius curve for neutron stars whose stability is then analyzed.
First, we derive the Tolman-Oppenheimer-Volkoff system from the Einstein field equations in a radially symmetric metric for a perfect fluid in equilibrium.
Second, we present thermal field theory and use it to calculate the path integral for the partition function of a free Fermi gas described by the Dirac Lagrangian.
Next, we merge these two results by calculating the equation of state that relates the energy density and pressure from the partition function, then numerically integrate the Tolman-Oppenheimer-Volkoff equations with this equation of state to find a mass-radius curve for cold neutron stars parametrized by their central pressures.
Finally, we use perturbation theory on the initial equilibrium analysis of general relativity to find a Sturm-Liouville differential equation that determines normal, radial vibration modes of stars outside equilibrium, then solve it with the shooting method to analyze the stability of the stars on the mass-radius curve.
Along the way, we review general relativity, discover the Buchdal limit for stars, take a particularly detailed look on fermionic coherent states and demonstrate the technique of Matsubara energy summation.
\TODO{should I omit the preceeding sentence?}
We find that our mass-radius curve reproduces the upper mass limit of 0.71 solar masses for neutron stars originally calculated by Oppenheimer and Volkoff in 1939.
Likewise, our quantitative stability analysis confirms the correctness of a set of qualitative rules based on curvature and extrema in the mass-radius diagram.
Unfortunately the model is too simple, as many heavier neutron stars around 2 solar masses have been observed.
Nevertheless, the content of this thesis establishes a broad and solid base platform for a fundamental understanding of compact stars and continued study of more advanced stellar models.

\TODO{må jeg ha på norsk også?}

\chapter*{Sammendrag}
\addcontentsline{toc}{chapter}{Sammendrag} % but still display in TOC (see https://tex.stackexchange.com/a/222961)
