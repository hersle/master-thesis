\chapter*{Sammendrag}%

\subsection*{Del \ref{part:project} (prosjektoppgave): Grunnelementer innen kompakte stjerner}

Generell relativitetsteori og kvantefeltteori er uvurderlige for å studere kompakte stjerner som består av subatomære partikler med ekstrem tetthet.
I dette prosjektet løser vi Tolman-Oppenheimer-Volkoff-ligningene for en kald Fermigass som består av frie nøytroner,
noe som produserer en masse-radius kurve for ideelle nøytronstjerner parametrisert med sentraltrykk,
og til slutt analyserer vi stabiliteten deres.
Først utleder vi Tolman-Oppenheimer-Volkoff-ligningene fra Einsteins feltlikninger
i en radielt symmetrisk metrikk for et perfekt fluid i likevekt.
Deretter presenterer vi termisk feltteori og bruker det
til å uttrykke partisjonsfunksjonen for en fri Fermigass som et veiintegral.
Så kombinerer vi disse to resultatene ved å numerisk integrere Tolman-Oppenheimer-Volkoff-ligningene
med tilstandsligningen som følger fra partisjonsfunksjonen, og genererer slik masse-radius-kurven.
Til slutt bruker vi perturbasjonsteori på den innledende likevektsanalysen av generell relativitetsteori
for å finne et Sturm-Liouville-problem som bestemmer normale vibrasjonsmoder for stjernene utenfor likevekt,
og løser det med skytemetoden for å analysere stabilitet deres.
Vår masse-radius-kurve reproduserer den øvre massegrensen på 0.71 solmasser
for nøytronstjerner opprinnelig beregnet av Oppenheimer og Volkoff i 1939.
På samme måte bekrefter vår kvantitative stabilitetsanalyse korrektheten
av et sett med kvalitative regler basert på krumning og ekstremalpunkter i masse-radius-diagrammet.
Mange observasjoner har blitt gjort av nøytronstjerner rundt 2 solmasser,
så modellen er for enkel for å beskrive ekte nøytronstjerner.
Uansett etablerer dette prosjektet en bred og solid grunnplattform
som man kan studere mer avanserte modeller for kompakte stjerner fra.

\subsection*{\textls[-10]{Del \ref{part:master} (masteroppgave): Kvark- og hybridstjerner med kvark-meson-modellen}}
% squeeze slightly-too-long title (https://tex.stackexchange.com/questions/522584/avoid-line-break-in-title)

Ifølge kvantekromodynamikk bryter innesperrede kvarker fri fra hadroner til en tilstand av uavgrenset kvarkmaterie ved høy tetthet.
Nylige observasjoner av de massive $2 M_\odot$-pulsarene PSR J1614$-$2230, PSR J0348$+$0432 og PSR J0740$+$6620
antyder at tettheten i nøytronstjerner kan nå tilstrekkelig høye nivåer for dannelse av små kjerner med uavgrenset kvarkmaterie
i hva som da kalles en hybridstjerne.
Til og med rene kvarkstjerner som består utelukkende av uavgrenset kvarkmaterie har blitt foreslått.
Etter en gjennomgang av den konvensjonelle MIT-pose-modellen,
modellerer vi kvarkstjerner ved hjelp av den effektive kvark-meson-modellen for kvantekromodynamikk,
der vi beregner den frie energien med én fermionløkke i midlere felt-tilnærmingen for bosoner, % Helmholz' ? se PC Hemmer.
noe som er konsistent i $1/N_c$-ekspansjonen.
Vi finner maksimumsmasser på henholdsvis $M \leq 2.0 \, M_\odot$ og $M \leq 1.8 \, M_\odot$ med
to og tre kvarktyper i modellen.
Spesielt sliter vi med å tilpasse målte masser av modellens $\sigma$-meson til den frie energien på trenivå,
men løser opp i dette ved hjelp av nylig arbeid av Adhikari og andre
som tilpasser parameterene på én-løkke-nivå på en konsistent måte.
Til slutt setter vi sammen hybridstjerner ved å knyte kvark-meson-modellen
sammen med den hadronske Akmal-Pandharipande-Ravenhall tilstandsligningen.
Dette genererer korte grener av stabile hybridstjerner med rimelige maksimumsmasser $1.9 M_\odot \leq M \leq 2.1 M_\odot$
og små kjerner med to og tre kvarktyper som utgjør henholdsvis $0.12 M_\odot$ og $0.02 M_\odot$.
En diskontinuerlig faseovergang destabiliserer stjerner med tyngre kvarkkjerner.
Resultatene stemmer godt overens med annet arbeid
som benytter variasjoner av kvark-meson-modellen
og Nambu-Jona-Lasinio-modellen.%
\tikzexternaldisable%
\begin{tikzpicture}[remember picture, overlay]%
\node[anchor=north east,inner sep=0pt, yshift=-2cm] at (current page text area.north east) {\includesvg[height=1.5cm]{figures/flag-norway.svg}};%
\end{tikzpicture}%
\tikzexternalenable%

\chapter{Abstract}
%\addcontentsline{toc}{chapter}{Abstract} % but still display in TOC (see https://tex.stackexchange.com/a/222961)

\subsection*{\Cref{part:project} (project thesis): Preliminaries to compact stars}

General relativity and quantum field theory are indispensable for studying compact stars
that are composed of subatomic particles with extreme density.
In this project, we solve the Tolman-Oppenheimer-Volkoff equations for a cold Fermi gas composed of free neutrons,
producing a mass-radius curve for ideal neutron stars parametrized by central pressure, then finally analyze their stability.
First, we derive the Tolman-Oppenheimer-Volkoff equations from the Einstein field equations
in a radially symmetric metric for a perfect fluid in equilibrium.
Second, we present thermal field theory and use it
to express the partition function of a free Fermi gas as a path integral.
Next, we combine these two results by numerically integrating the Tolman-Oppenheimer-Volkoff equations
with the equation of state that follows from the partition function, yielding the mass-radius curve.
Finally, we apply perturbation theory to the initial equilibrium analysis of general relativity
to find a Sturm-Liouville problem that determines normal radial vibration modes of stars out of equilibrium,
then solve it with the shooting method to analyze the stability of the stars.
%Along the way, we review general relativity, discover the Buchdal limit for stars, take a particularly detailed look on fermionic coherent states and demonstrate the technique of Matsubara energy summation.
%\TODO{should I omit the preceeding sentence?}
Our mass-radius curve reproduces the upper mass limit of 0.71 solar masses
for neutron stars originally calculated by Oppenheimer and Volkoff in 1939.
Likewise, our quantitative stability analysis confirms the correctness
of a set of qualitative rules based on curvature and extrema in the mass-radius diagram.
Many observations have been made of neutron stars around 2 solar masses,
so the model is too simple for describing real neutron stars.
Nevertheless, this project establishes a broad and solid base platform
from which one can continue to study more advanced models for compact stars.
%for a fundamental understanding of compact stars and continued study of more advanced stellar models.


\subsection*{\Cref{part:master} (master thesis): Quark and hybrid stars with the quark-meson model}

According to quantum chromodynamics,
hadron-confined quarks break free into a state of deconfined quark matter at high density.
Recent observations of the massive $2 M_\odot$-pulsars PSR J1614$-$2230, PSR J0348$+$0432 and PSR J0740$+$6620
suggest that the density in neutron stars could reach sufficiently high levels
for formation of small cores of deconfined quark matter in what is then referred to as hybrid stars.
Even pure quark stars consisting only of deconfined quark matter have been hypothesized.
After reviewing the conventional MIT bag model,
we model quark stars using the effective quark-meson model of quantum chromodynamics,
calculating its grand potential to one fermion loop in the mean-field approximation for bosons,
which is consistent in large-$N_c$ approximation scheme.
We find maximum masses $M \leq 2.0 \, M_\odot$ and $M \leq 1.8 \, M_\odot$ with the two-flavor and three-flavor models, respectively.
In particular, we struggle to fit measured masses 
of the model's $\sigma$ meson to its grand potential at tree-level,
but resolve this using recent work of Adhikari and others who consistently fit parameters at one loop level.
Finally, we assemble hybrid stars by joining the quark-meson model with the hadronic Akmal-Pandharipande-Ravenhall equation of state.
This generates short branches of stable hybrid stars with plausible maximum masses $1.9 M_\odot \leq M \leq 2.1 M_\odot$
and small two-flavor and three-flavor quark cores around only $0.12 M_\odot$ and $0.02 M_\odot$, respectively.
A discontinuous phase transition destabilizes stars with heavier quark cores.
The results agree with other work using variations of the quark-meson model and the Nambu-Jona-Lasinio model.%
\tikzexternaldisable%
\begin{tikzpicture}[remember picture, overlay]%
\node[anchor=north east,inner sep=0pt, yshift=-2cm] at (current page text area.north east) {\includesvg[height=1.5cm]{figures/flag-usa.svg}};%
\end{tikzpicture}%
\tikzexternalenable%
% absolute page positioning of tikz figures:
% https://tex.stackexchange.com/questions/124067/how-can-i-put-a-picture-alongside-the-chapter-title

\TODO{JO read Norwegian abstract?}
