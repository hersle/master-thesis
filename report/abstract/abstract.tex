\chapter{Abstract}
%\addcontentsline{toc}{chapter}{Abstract} % but still display in TOC (see https://tex.stackexchange.com/a/222961)

General relativity and quantum field theory are indispensable for studying compact stars that are composed of subatomic particles with extreme density.
In this project, we solve the Tolman-Oppenheimer-Volkoff equations for a cold Fermi gas composed of free neutrons, producing a mass-radius curve for ideal neutron stars parametrized by central pressure, then finally analyze their stability.
First, we derive the Tolman-Oppenheimer-Volkoff equations in a radially symmetric metric for a perfect fluid in equilibrium.
Second, we present thermal field theory and use it to express the partition function of a free Fermi gas as a path integral.
Next, we combine these two results by numerically integrating the Tolman-Oppenheimer-Volkoff equations with the equation of state that follows from the partition function, yielding the mass-radius curve.
Finally, we apply perturbation theory to the initial equilibrium analysis of general relativity to find a Sturm-Liouville problem that determines normal radial vibration modes of stars out of equilibrium, then solve it with the shooting method to analyze the stability of the stars.
%Along the way, we review general relativity, discover the Buchdal limit for stars, take a particularly detailed look on fermionic coherent states and demonstrate the technique of Matsubara energy summation.
%\TODO{should I omit the preceeding sentence?}
Our mass-radius curve reproduces the upper mass limit of 0.71 solar masses for neutron stars originally calculated by Oppenheimer and Volkoff in 1939.
Likewise, our quantitative stability analysis confirms the correctness of a set of qualitative rules based on curvature and extrema in the mass-radius diagram.
Many observations have been made of neutron stars around 2 solar masses, so the model is too simple for describing real neutron stars.
Nevertheless, this project establishes a broad and solid base platform for a fundamental understanding of compact stars and continued study of more advanced stellar models.


%\chapter*{Sammendrag}
%\addcontentsline{toc}{chapter}{Sammendrag} % but still display in TOC (see https://tex.stackexchange.com/a/222961)
%\TODO{må jeg ha på norsk også?}
