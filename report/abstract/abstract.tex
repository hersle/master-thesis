\chapter*{Abstract}
\addcontentsline{toc}{chapter}{Abstract} % but still display in TOC (see https://tex.stackexchange.com/a/222961)

General relativity and quantum field theory are indispensable theories for the study of compact stars, composed of subatomic particles with extreme density.
In this thesis, we solve the Tolman-Oppenheimer-Volkoff system of equations for a cold Fermi gas composed of free neutrons, producing a mass-radius curve for neutron stars whose stability is analyzed.
First, we derive the Tolman-Oppenheimer-Volkoff system from the Einstein field equations in a radially symmetric metric for a perfect fluid in equilibrium.
Second, we present thermal field theory and use it to calculate the path integral for the partition function of a free Fermi gas described by the Dirac Lagrangian.
Third, we calculate the equation of state that relates the energy density and pressure from the partition function, then numerically integrate Tolman-Oppenheimer-Volkoff equations to find the mass-radius curve for neutron stars parametrized by their central pressures.
Finally, we use perturbation theory on the initial equilibrium analysis of general relativity to find a Sturm-Liouville differential equation that determines normal, radial vibration modes of stars outside equilibrium, then solve it with the shooting method to analyze the stability of the stars on the curve.
In addition, we review every required aspect of general relativity, relativistic fluid mechanics and Matsubara frequency summation. \TODO{drop preceeding sentence?}
Our mass-radius curve reproduces the upper mass limit of 0.71 solar masses for neutron stars originally calculated by Oppenheimer and Volkoff in 1939, and our quantitative stability analysis confirms the correctness of qualitative rules based on curvature and extrema in the mass-radius diagram.
Although observed neutron stars above 2 solar masses exceed this limit, the content of this thesis nevertheless establishes a broad and general base for continued study of more advanced stellar models.

\chapter*{Sammendrag}
\addcontentsline{toc}{chapter}{Sammendrag} % but still display in TOC (see https://tex.stackexchange.com/a/222961)

\TODO{do I need Norwegian abstract in project thesis, too?}
